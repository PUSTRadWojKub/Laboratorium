
\chapter{Dob�r parametr�w cyfrowego algorytmu PID oraz algorytmu DMC}



TODO: Wstawi� przebiegi z strojenia PIDA i DMC.



Skoki:

$0 \rightarrow 0.5 \rightarrow 1 \rightarrow -0.1 \rightarrow 5$

\section{Regulator PID}

\begin{figure}[h!]
	\centering
	% This file was created by matlab2tikz.
%
\definecolor{mycolor1}{rgb}{0.00000,0.44700,0.74100}%
%
\begin{tikzpicture}

\begin{axis}[%
width=4.521in,
height=3.566in,
at={(0.758in,0.481in)},
scale only axis,
xmin=0,
xmax=1200,
ymin=-0.2,
ymax=1,
axis background/.style={fill=white},
title style={font=\bfseries},
title={Sygna� wej�ciowy},
xmajorgrids,
ymajorgrids,
legend style={legend cell align=left, align=left, draw=white!15!black}
]
\addplot[const plot, color=mycolor1] table[row sep=crcr] {%
1	0\\
2	0\\
3	0\\
4	0\\
5	0\\
6	0\\
7	0\\
8	0\\
9	0\\
10	0\\
11	0\\
12	0\\
13	0\\
14	0\\
15	0\\
16	0\\
17	0\\
18	0\\
19	0\\
20	0\\
21	0\\
22	0\\
23	0\\
24	0\\
25	0\\
26	0\\
27	0\\
28	0\\
29	0\\
30	0\\
31	0\\
32	0\\
33	0\\
34	0\\
35	0\\
36	0\\
37	0\\
38	0\\
39	0\\
40	0\\
41	0\\
42	0\\
43	0\\
44	0\\
45	0\\
46	0\\
47	0\\
48	0\\
49	0\\
50	0\\
51	0\\
52	0\\
53	0\\
54	0\\
55	0\\
56	0\\
57	0\\
58	0\\
59	0\\
60	0\\
61	0\\
62	0\\
63	0\\
64	0\\
65	0\\
66	0\\
67	0\\
68	0\\
69	0\\
70	0\\
71	0\\
72	0\\
73	0\\
74	0\\
75	0\\
76	0\\
77	0\\
78	0\\
79	0\\
80	0\\
81	0\\
82	0\\
83	0\\
84	0\\
85	0\\
86	0\\
87	0\\
88	0\\
89	0\\
90	0\\
91	0\\
92	0\\
93	0\\
94	0\\
95	0\\
96	0\\
97	0\\
98	0\\
99	0\\
100	0.129489130434783\\
101	0.0639673913043479\\
102	0.0699456521739131\\
103	0.0759239130434783\\
104	0.0819021739130436\\
105	0.0850060251941894\\
106	0.0881420289832941\\
107	0.0928772310171366\\
108	0.0976200298547019\\
109	0.102308493378265\\
110	0.106996952831214\\
111	0.11170068996586\\
112	0.116337898498037\\
113	0.1208699121837\\
114	0.12529893919621\\
115	0.129628390851833\\
116	0.133860307361911\\
117	0.137998812965618\\
118	0.142049755078541\\
119	0.146018110980667\\
120	0.149907349988545\\
121	0.15371980340029\\
122	0.157456935634643\\
123	0.16111951443379\\
124	0.164707835385482\\
125	0.168221959279082\\
126	0.171661886525023\\
127	0.175027665869129\\
128	0.178319457869087\\
129	0.18153756482025\\
130	0.184682434892395\\
131	0.187754649852732\\
132	0.190754904995458\\
133	0.193683987036782\\
134	0.196542753346684\\
135	0.199332114364576\\
136	0.202053019924869\\
137	0.204706449392238\\
138	0.207293405029981\\
139	0.209814907839697\\
140	0.212271995103854\\
141	0.21466571895517\\
142	0.216997145439471\\
143	0.219267353695075\\
144	0.221477435015453\\
145	0.223628491679162\\
146	0.225721635517611\\
147	0.227757986248304\\
148	0.229738669633097\\
149	0.231664815533279\\
150	0.233537555932018\\
151	0.235358022985034\\
152	0.237127347146703\\
153	0.238846655404215\\
154	0.240517069638975\\
155	0.242139705123364\\
156	0.243715669152631\\
157	0.245246059806115\\
158	0.246731964828744\\
159	0.248174460622466\\
160	0.249574611337212\\
161	0.250933468051903\\
162	0.252252068037258\\
163	0.253531434093596\\
164	0.254772573958126\\
165	0.25597647977735\\
166	0.257144127641026\\
167	0.258276477174718\\
168	0.259374471188323\\
169	0.260439035378069\\
170	0.261471078079582\\
171	0.262471490069486\\
172	0.263441144412986\\
173	0.264380896354757\\
174	0.265291583250413\\
175	0.266174024535811\\
176	0.267029021731435\\
177	0.267857358479176\\
178	0.268659800608891\\
179	0.269437096232231\\
180	0.270189975861359\\
181	0.270919152550305\\
182	0.271625322056833\\
183	0.272309163022852\\
184	0.27297133717151\\
185	0.273612489519239\\
186	0.274233248601149\\
187	0.274834226708237\\
188	0.275416020135034\\
189	0.275979209436341\\
190	0.276524359691846\\
191	0.277052020777452\\
192	0.277562727642261\\
193	0.278057000590194\\
194	0.278535345565334\\
195	0.278998254440114\\
196	0.279446205305553\\
197	0.279879662762808\\
198	0.280299078215355\\
199	0.280704890161167\\
200	0.281097524484332\\
201	0.281477394745577\\
202	0.281844902471223\\
203	0.282200437440145\\
204	0.28254437796834\\
205	0.282877091190754\\
206	0.283198933340057\\
207	0.28351025002207\\
208	0.283811376487613\\
209	0.284102637900538\\
210	0.284384349601751\\
211	0.284656817369063\\
212	0.284920337672727\\
213	0.285175197926509\\
214	0.285421676734229\\
215	0.28566004413165\\
216	0.285890561823666\\
217	0.286113483416729\\
218	0.286329054646471\\
219	0.28653751360051\\
220	0.286739090936408\\
221	0.286934010094788\\
222	0.287122487507619\\
223	0.287304732801681\\
224	0.287480948997235\\
225	0.287651332701938\\
226	0.287816074300024\\
227	0.287975358136825\\
228	0.288129362698656\\
229	0.28827826078813\\
230	0.288422219694973\\
231	0.28856140136238\\
232	0.288695962549006\\
233	0.288826054986638\\
234	0.288951825533638\\
235	0.28907341632422\\
236	0.289190964913647\\
237	0.289304604419413\\
238	0.28941446365851\\
239	0.289520667280828\\
240	0.289623335898805\\
241	0.28972258621338\\
242	0.289818531136339\\
243	0.289911279909141\\
244	0.290000938218295\\
245	0.290087608307374\\
246	0.290171389085751\\
247	0.290252376234115\\
248	0.290330662306883\\
249	0.290406336831542\\
250	0.29047948640504\\
251	0.290550194787266\\
252	0.29061854299172\\
253	0.290684609373438\\
254	0.290748469714237\\
255	0.290810197305364\\
256	0.290869863027615\\
257	0.29092753542899\\
258	0.290983280799952\\
259	0.291037163246367\\
260	0.291089244760171\\
261	0.291139585287846\\
262	0.291188242796759\\
263	0.291235273339421\\
264	0.291280731115743\\
265	0.291324668533317\\
266	0.291367136265816\\
267	0.291408183309539\\
268	0.291447857038166\\
269	0.291486203255784\\
270	0.291523266248219\\
271	0.291559088832739\\
272	0.291593712406172\\
273	0.291627176991479\\
274	0.291659521282846\\
275	0.291690782689321\\
276	0.291720997377054\\
277	0.29175020031018\\
278	0.291778425290379\\
279	0.291805704995161\\
280	0.291832071014921\\
281	0.291857553888786\\
282	0.291882183139305\\
283	0.291905987306014\\
284	0.291928993977908\\
285	0.291951229824864\\
286	0.291972720628034\\
287	0.29199349130925\\
288	0.292013565959472\\
289	0.292032967866306\\
290	0.292051719540622\\
291	0.292069842742302\\
292	0.292087358505143\\
293	0.292104287160949\\
294	0.292120648362828\\
295	0.292136461107728\\
296	0.292151743758227\\
297	0.292166514063614\\
298	0.292180789180274\\
299	0.292194585691399\\
300	0.421697050060834\\
301	0.356188197781952\\
302	0.362178913395474\\
303	0.368169211376283\\
304	0.374159105715303\\
305	0.372892493963607\\
306	0.372295070383575\\
307	0.373059752229073\\
308	0.373749689664232\\
309	0.374343839396379\\
310	0.375292883310857\\
311	0.376695134127159\\
312	0.378401318972688\\
313	0.38032636894848\\
314	0.382418735742349\\
315	0.38460310200348\\
316	0.386795441572586\\
317	0.388929325320799\\
318	0.390957883062496\\
319	0.392849356968156\\
320	0.394586275006457\\
321	0.396164662162311\\
322	0.397591175806098\\
323	0.398879653863428\\
324	0.40004816648039\\
325	0.401116548687554\\
326	0.402104339178901\\
327	0.403029238635006\\
328	0.403906137500434\\
329	0.404746646623301\\
330	0.405559033181329\\
331	0.406348468123741\\
332	0.40711748885506\\
333	0.407866581531724\\
334	0.408594799501105\\
335	0.409300352688101\\
336	0.409981121643875\\
337	0.410635068045054\\
338	0.411260529871408\\
339	0.41185640307621\\
340	0.41242222154455\\
341	0.412958153493533\\
342	0.413464935575684\\
343	0.413943766338285\\
344	0.414396179007899\\
345	0.414823910491235\\
346	0.415228779645017\\
347	0.415612583804855\\
348	0.415977018694345\\
349	0.416323623457873\\
350	0.416653749853697\\
351	0.416968552684857\\
352	0.417268997326588\\
353	0.417555879659451\\
354	0.41782985372453\\
355	0.41809146284573\\
356	0.418341170674084\\
357	0.418579389468284\\
358	0.418806503820743\\
359	0.419022888879829\\
360	0.419228922843319\\
361	0.419424994068218\\
362	0.419611503543539\\
363	0.419788863709687\\
364	0.419957494699006\\
365	0.420117819043795\\
366	0.420270255781906\\
367	0.420415214717677\\
368	0.420553091396412\\
369	0.420684263148594\\
370	0.420809086374264\\
371	0.420927895081533\\
372	0.421041000573021\\
373	0.421148692092439\\
374	0.421251238198589\\
375	0.421348888621095\\
376	0.421441876364747\\
377	0.42153041986008\\
378	0.421614724999782\\
379	0.421694986946999\\
380	0.421771391647516\\
381	0.421844117018949\\
382	0.421913333823968\\
383	0.421979206259804\\
384	0.422041892312679\\
385	0.422101543933924\\
386	0.422158307095659\\
387	0.422212321779602\\
388	0.422263721944439\\
389	0.422312635507011\\
390	0.422359184361598\\
391	0.422403484451156\\
392	0.422445645895135\\
393	0.422485773171199\\
394	0.422523965342807\\
395	0.422560316321378\\
396	0.422594915150327\\
397	0.422627846298378\\
398	0.422659189950835\\
399	0.42268902228954\\
400	0.422717415754664\\
401	0.42274443928403\\
402	0.422770158527965\\
403	0.422794636039727\\
404	0.422817931443076\\
405	0.422840101579655\\
406	0.422861200639436\\
407	0.422881280277697\\
408	0.422900389721844\\
409	0.422918575871025\\
410	0.422935883390935\\
411	0.422952354805612\\
412	0.422968030587444\\
413	0.422982949246022\\
414	0.422997147416059\\
415	0.423010659944224\\
416	0.423023519974506\\
417	0.423035759031617\\
418	0.423047407101886\\
419	0.423058492711157\\
420	0.423069042999267\\
421	0.423079083790834\\
422	0.423088639662187\\
423	0.423097734004417\\
424	0.42310638908262\\
425	0.423114626091524\\
426	0.423122465207721\\
427	0.423129925638791\\
428	0.423137025669602\\
429	0.423143782706069\\
430	0.423150213316645\\
431	0.423156333271759\\
432	0.423162157581418\\
433	0.423167700531115\\
434	0.423172975716172\\
435	0.423177996074608\\
436	0.42318277391861\\
437	0.423187320964635\\
438	0.4231916483622\\
439	0.423195766721374\\
440	0.423199686139014\\
441	0.423203416223756\\
442	0.423206966119811\\
443	0.42321034452959\\
444	0.42321355973521\\
445	0.423216619618914\\
446	0.42321953168247\\
447	0.423222303065589\\
448	0.423224940563421\\
449	0.423227450643176\\
450	0.423229839459921\\
451	0.423232112871611\\
452	0.423234276453378\\
453	0.42323633551114\\
454	0.423238295094553\\
455	0.423240160009353\\
456	0.423241934829101\\
457	0.423243623906386\\
458	0.423245231383479\\
459	0.423246761202499\\
460	0.423248217115079\\
461	0.423249602691577\\
462	0.423250921329843\\
463	0.423252176263567\\
464	0.423253370570213\\
465	0.423254507178581\\
466	0.423255588875993\\
467	0.423256618315134\\
468	0.423257598020558\\
469	0.423258530394878\\
470	0.423259417724657\\
471	0.423260262186011\\
472	0.423261065849944\\
473	0.423261830687419\\
474	0.423262558574189\\
475	0.423263251295394\\
476	0.423263910549936\\
477	0.423264537954643\\
478	0.42326513504823\\
479	0.423265703295077\\
480	0.423266244088812\\
481	0.423266758755738\\
482	0.423267248558078\\
483	0.423267714697075\\
484	0.423268158315938\\
485	0.423268580502648\\
486	0.423268982292621\\
487	0.423269364671253\\
488	0.423269728576337\\
489	0.423270074900358\\
490	0.423270404492688\\
491	0.423270718161664\\
492	0.423271016676572\\
493	0.423271300769535\\
494	0.423271571137305\\
495	0.423271828442975\\
496	0.423272073317602\\
497	0.423272306361756\\
498	0.423272528146994\\
499	0.423272739217258\\
500	0.13839685313369\\
501	0.282544870388946\\
502	0.269392878408399\\
503	0.256240877638253\\
504	0.243088868503152\\
505	0.256346461023395\\
506	0.259562273744173\\
507	0.258921181281677\\
508	0.257678184917609\\
509	0.255969976527486\\
510	0.252498252069723\\
511	0.247614723687926\\
512	0.24203263555099\\
513	0.236108948088065\\
514	0.230061705455387\\
515	0.224118131318575\\
516	0.218460350820959\\
517	0.213176686235692\\
518	0.208284554656645\\
519	0.203758957294509\\
520	0.199547464843832\\
521	0.195583160528487\\
522	0.191798464774016\\
523	0.188135136955187\\
524	0.184548989373171\\
525	0.181010839578391\\
526	0.177505109821535\\
527	0.174026974117913\\
528	0.170578908506384\\
529	0.167167394171675\\
530	0.163800228143477\\
531	0.160484618080794\\
532	0.157226059088906\\
533	0.15402788164269\\
534	0.150891298942444\\
535	0.147815766086157\\
536	0.144799482697682\\
537	0.141839909320911\\
538	0.1389342126357\\
539	0.136079596525406\\
540	0.133273509758109\\
541	0.130513743997456\\
542	0.12779844814171\\
543	0.125126088503105\\
544	0.122495381765798\\
545	0.119905221696568\\
546	0.117354613484605\\
547	0.114842622955747\\
548	0.112368342658974\\
549	0.109930873294113\\
550	0.10752931705825\\
551	0.105162778910121\\
552	0.102830372067601\\
553	0.100531224856911\\
554	0.09826448699403\\
555	0.0960293342730185\\
556	0.0938249713380063\\
557	0.0916506326838724\\
558	0.0895055822803725\\
559	0.0873891122925563\\
560	0.0853005413341761\\
561	0.0832392125941185\\
562	0.0812044920607704\\
563	0.0791957669634732\\
564	0.0772124444683739\\
565	0.075253950612283\\
566	0.0733197294300635\\
567	0.0714092422226047\\
568	0.0695219669166549\\
569	0.0676573974784826\\
570	0.0658150433557937\\
571	0.0639944289335867\\
572	0.0621950929981929\\
573	0.0604165882092941\\
574	0.0586584805825806\\
575	0.056920348986596\\
576	0.0552017846569477\\
577	0.0535023907300864\\
578	0.051821781797739\\
579	0.0501595834820912\\
580	0.048515432031101\\
581	0.0468889739328892\\
582	0.0452798655479673\\
583	0.0436877727580481\\
584	0.0421123706302797\\
585	0.0405533430958789\\
586	0.0390103826422873\\
587	0.0374831900180991\\
588	0.0359714739501117\\
589	0.0344749508719212\\
590	0.0329933446635356\\
591	0.0315263864015151\\
592	0.0300738141191679\\
593	0.028635372576353\\
594	0.0272108130384543\\
595	0.0257998930641114\\
596	0.0244023763013046\\
597	0.0230180322914138\\
598	0.0216466362808879\\
599	0.0202879690401783\\
600	0.0189418166896113\\
601	0.0176079705318863\\
602	0.0162862268909093\\
603	0.0149763869566793\\
604	0.0136782566359655\\
605	0.0123916464085228\\
606	0.011116371188606\\
607	0.0098522501915573\\
608	0.00859910680524874\\
609	0.00735676846617556\\
610	0.00612506654000326\\
611	0.00490383620638261\\
612	0.00369291634785438\\
613	0.00249214944267473\\
614	0.00130138146139969\\
615	0.000120461767075069\\
616	-0.00105075698111514\\
617	-0.00221241892088142\\
618	-0.00336466507667629\\
619	-0.00450763344648378\\
620	-0.00564145908567262\\
621	-0.00676627418802916\\
622	-0.00788220816408078\\
623	-0.00898938771681579\\
624	-0.0100879369149009\\
625	-0.011177977263493\\
626	-0.0122596277727378\\
627	-0.0133330050240437\\
628	-0.014398223234215\\
629	-0.015455394317526\\
630	-0.0165046279458131\\
631	-0.0175460316066592\\
632	-0.0185797106597403\\
633	-0.0196057683914043\\
634	-0.0206243060675444\\
635	-0.0216354229848323\\
636	-0.0226392165203681\\
637	-0.0236357821798062\\
638	-0.0246252136440118\\
639	-0.0256076028142989\\
640	-0.0265830398563033\\
641	-0.0275516132425356\\
642	-0.0285134097936638\\
643	-0.0294685147185675\\
644	-0.0304170116532088\\
645	-0.0313589826983589\\
646	-0.0322945084562216\\
647	-0.0332236680659901\\
648	-0.0341465392383751\\
649	-0.0350631982891375\\
650	-0.0359737201716613\\
651	-0.0368781785085969\\
652	-0.0377766456226076\\
653	-0.0386691925662482\\
654	-0.0395558891510054\\
655	-0.0404368039755262\\
656	-0.0413120044530627\\
657	-0.0421815568381572\\
658	-0.0430455262525943\\
659	-0.0439039767106409\\
660	-0.0447569711436005\\
661	-0.0456045714237001\\
662	-0.0464468383873347\\
663	-0.047283831857686\\
664	-0.0481156106667384\\
665	-0.0489422326767091\\
666	-0.0497637548009118\\
667	-0.0505802330240701\\
668	-0.0513917224221003\\
669	-0.052198277181377\\
670	-0.0529999506174993\\
671	-0.0537967951935727\\
672	-0.0545888625380199\\
673	-0.0553762034619376\\
674	-0.0561588679760092\\
675	-0.0569369053069907\\
676	-0.0577103639137797\\
677	-0.0584792915030807\\
678	-0.0592437350446794\\
679	-0.0600037407863365\\
680	-0.0607593542683125\\
681	-0.0615106203375347\\
682	-0.0622575831614163\\
683	-0.0630002862413374\\
684	-0.0637387724257988\\
685	-0.0644730839232565\\
686	-0.0652032623146472\\
687	-0.065929348565613\\
688	-0.0666513830384342\\
689	-0.0673694055036773\\
690	-0.0680834551515687\\
691	-0.0687935706030983\\
692	-0.0694997899208636\\
693	-0.0702021506196599\\
694	-0.070900689676824\\
695	-0.0715954435423376\\
696	-0.0722864481486987\\
697	-0.0729737389205647\\
698	-0.0736573507841758\\
699	-0.0743373181765636\\
700	1\\
701	0.331005481020572\\
702	0.391314506047544\\
703	0.451627042512734\\
704	0.511943058294149\\
705	0.545400499158448\\
706	0.575107172752457\\
707	0.617113893917823\\
708	0.655935856983303\\
709	0.690956751551389\\
710	0.723246913038594\\
711	0.753447824584058\\
712	0.780774170047431\\
713	0.80491749118972\\
714	0.825988538997652\\
715	0.844109378738789\\
716	0.859378757830073\\
717	0.871957653310637\\
718	0.882077278491303\\
719	0.89000268792274\\
720	0.896017189926392\\
721	0.900418134319925\\
722	0.903508649740414\\
723	0.905586313983817\\
724	0.906933439793083\\
725	0.90780929754767\\
726	0.908443401052647\\
727	0.909030043837952\\
728	0.9097246082153\\
729	0.910641825346597\\
730	0.911855946482796\\
731	0.913402740802021\\
732	0.915283155914313\\
733	0.917468354576157\\
734	0.919905747125355\\
735	0.922525593234856\\
736	0.925247738778327\\
737	0.927988080299894\\
738	0.930664408804972\\
739	0.933201367181982\\
740	0.935534348542844\\
741	0.937612254618828\\
742	0.939399115631502\\
743	0.940874640443221\\
744	0.942033815997038\\
745	0.9428857084\\
746	0.943451636478059\\
747	0.943762894945225\\
748	0.943858201226673\\
749	0.943781029837014\\
750	0.943576982901202\\
751	0.94329132627199\\
752	0.942966798721374\\
753	0.942641777634166\\
754	0.942348859234741\\
755	0.942113885433396\\
756	0.941955423821687\\
757	0.941884683238943\\
758	0.941905825797182\\
759	0.942016618360488\\
760	0.942209353119411\\
761	0.942471958674825\\
762	0.942789220160174\\
763	0.943144029184305\\
764	0.943518591180499\\
765	0.943895528203944\\
766	0.944258828237082\\
767	0.944594606477941\\
768	0.944891658786771\\
769	0.945141801478688\\
770	0.945340004203645\\
771	0.945484333205615\\
772	0.94557573047897\\
773	0.945617660117878\\
774	0.945615656522315\\
775	0.94557681024249\\
776	0.945509226359276\\
777	0.945421487714825\\
778	0.945322151361059\\
779	0.9452193016359\\
780	0.945120177661743\\
781	0.945030887133097\\
782	0.944956212346913\\
783	0.944899508827172\\
784	0.944862691863003\\
785	0.944846302025867\\
786	0.944849637408329\\
787	0.944870938023929\\
788	0.944907606550214\\
789	0.944956449348548\\
790	0.945013922363703\\
791	0.945076367957238\\
792	0.945140230793233\\
793	0.945202243386954\\
794	0.945259574656236\\
795	0.945309937599983\\
796	0.945351654905114\\
797	0.945383683715819\\
798	0.945405602881487\\
799	0.945417567659315\\
800	0.945420238044396\\
801	0.945414687624082\\
802	0.945402300121435\\
803	0.945384660643715\\
804	0.945363448142281\\
805	0.945340334788643\\
806	0.945316896953461\\
807	0.945294541319659\\
808	0.945274448444353\\
809	0.945257534878914\\
810	0.945244433825372\\
811	0.945235493303416\\
812	0.945230789965515\\
813	0.945230156055276\\
814	0.945233216569615\\
815	0.945239433459357\\
816	0.945248153674862\\
817	0.945258658012684\\
818	0.945270208018078\\
819	0.945282088613364\\
820	0.945293644617629\\
821	0.945304309862981\\
822	0.945313628161781\\
823	0.945321265906954\\
824	0.945327016567163\\
825	0.945330797749674\\
826	0.945332641831519\\
827	0.945332681395851\\
828	0.945331130852643\\
829	0.945328265674211\\
830	0.945324400643919\\
831	0.945319868412084\\
832	0.945314999490343\\
833	0.945310104610001\\
834	0.945305460137106\\
835	0.945301296992873\\
836	0.945297793287052\\
837	0.945295070646715\\
838	0.945293194024004\\
839	0.945292174601704\\
840	0.945291975289929\\
841	0.945292518223463\\
842	0.945293693626923\\
843	0.945295369411909\\
844	0.945297400902066\\
845	0.945299640143153\\
846	0.945301944338977\\
847	0.94530418305334\\
848	0.945306243925834\\
849	0.945308036758572\\
850	0.945309495935618\\
851	0.945310581231817\\
852	0.945311277148901\\
853	0.945311590981196\\
854	0.945311549859501\\
855	0.945311197049158\\
856	0.945310587787755\\
857	0.94530978494068\\
858	0.945308854731286\\
859	0.94530786276938\\
860	0.94530687056032\\
861	0.945305932630316\\
862	0.945305094354754\\
863	0.945304390528351\\
864	0.945303844671252\\
865	0.945303469025841\\
866	0.945303265166595\\
867	0.945303225120727\\
868	0.945303332881134\\
869	0.945303566185199\\
870	0.94530389843279\\
871	0.945304300623523\\
872	0.945304743205828\\
873	0.945305197747262\\
874	0.945305638355452\\
875	0.94530604280057\\
876	0.945306393311984\\
877	0.945306677042557\\
878	0.945306886212841\\
879	0.945307017963441\\
880	0.945307073956495\\
881	0.945307059776242\\
882	0.945306984183923\\
883	0.94530685828395\\
884	0.945306694656681\\
885	0.945306506508715\\
886	0.94530630688492\\
887	0.945306107978064\\
888	0.945305920562584\\
889	0.945305753569259\\
890	0.945305613808039\\
891	0.945305505837372\\
892	0.945305431970626\\
893	0.945305392403804\\
894	0.945305385443909\\
895	0.945305407814237\\
896	0.945305455011283\\
897	0.945305521688082\\
898	0.945305602040141\\
899	0.945305690172707\\
900	0.945305780431505\\
901	0.945305867683096\\
902	0.945305947535291\\
903	0.945306016492407\\
904	0.945306072044272\\
905	0.945306112691618\\
906	0.945306137913643\\
907	0.945306148086049\\
908	0.945306144359579\\
909	0.945306128510115\\
910	0.945306102771704\\
911	0.945306069663498\\
912	0.94530603182071\\
913	0.945305991838322\\
914	0.945305952134607\\
915	0.945305914839635\\
916	0.945305881712023\\
917	0.945305854085262\\
918	0.945305832843194\\
919	0.945305818422708\\
920	0.945305810840415\\
921	0.945305809739159\\
922	0.945305814449602\\
923	0.945305824061829\\
924	0.945305837501954\\
925	0.945305853608996\\
926	0.945305871207823\\
927	0.945305889174633\\
928	0.945305906492265\\
929	0.945305922293467\\
930	0.945305935891143\\
931	0.945305946795388\\
932	0.945305954717891\\
933	0.945305959564872\\
934	0.94530596142025\\
935	0.945305960521048\\
936	0.945305957227242\\
937	0.945305951988336\\
938	0.945305945308833\\
939	0.945305937714615\\
940	0.945305929721948\\
941	0.945305921810504\\
942	0.945305914401422\\
943	0.945305907841015\\
944	0.945305902390389\\
945	0.94530589822086\\
946	0.945305895414768\\
947	0.94530589397104\\
948	0.945305893814656\\
949	0.945305894809065\\
950	0.945305896770552\\
951	0.945305899483541\\
952	0.945305902715908\\
953	0.945305906233467\\
954	0.945305909812933\\
955	0.945305913252827\\
956	0.945305916381973\\
957	0.945305919065386\\
958	0.945305921207527\\
959	0.945305922753047\\
960	0.945305923685266\\
961	0.94530592402272\\
962	0.945305923814176\\
963	0.945305923132581\\
964	0.945305922068365\\
965	0.945305920722554\\
966	0.945305919200086\\
967	0.945305917603663\\
968	0.945305916028425\\
969	0.945305914557627\\
970	0.945305913259458\\
971	0.945305912185029\\
972	0.945305911367527\\
973	0.945305910822426\\
974	0.945305910548648\\
975	0.945305910530488\\
976	0.945305910740115\\
977	0.945305911140455\\
978	0.945305911688247\\
979	0.945305912337093\\
980	0.945305913040334\\
981	0.945305913753616\\
982	0.94530591443704\\
983	0.945305915056829\\
984	0.945305915586474\\
985	0.945305916007356\\
986	0.945305916308866\\
987	0.945305916488082\\
988	0.945305916549055\\
989	0.94530591650181\\
990	0.945305916361121\\
991	0.945305916145178\\
992	0.945305915874206\\
993	0.945305915569144\\
994	0.945305915250419\\
995	0.945305914936899\\
996	0.945305914645042\\
997	0.945305914388269\\
998	0.945305914176583\\
999	0.945305914016407\\
1000	0.945305913910639\\
1001	0.945305913858893\\
1002	0.945305913857892\\
1003	0.945305913901966\\
1004	0.945305913983639\\
1005	0.94530591409423\\
1006	0.945305914224469\\
1007	0.945305914365059\\
1008	0.945305914507192\\
1009	0.945305914642969\\
1010	0.945305914765726\\
1011	0.945305914870259\\
1012	0.945305914952942\\
1013	0.945305915011745\\
1014	0.945305915046164\\
1015	0.94530591505708\\
1016	0.945305915046549\\
1017	0.945305915017562\\
1018	0.945305914973778\\
1019	0.945305914919244\\
1020	0.945305914858138\\
1021	0.945305914794522\\
1022	0.945305914732138\\
1023	0.945305914674238\\
1024	0.945305914623463\\
1025	0.945305914581771\\
1026	0.945305914550403\\
1027	0.9453059145299\\
1028	0.94530591452015\\
1029	0.945305914520473\\
1030	0.945305914529714\\
1031	0.945305914546364\\
1032	0.945305914568685\\
1033	0.945305914594822\\
1034	0.945305914622925\\
1035	0.945305914651244\\
1036	0.945305914678216\\
1037	0.945305914702526\\
1038	0.945305914723154\\
1039	0.945305914739393\\
1040	0.945305914750856\\
1041	0.945305914757458\\
1042	0.945305914759388\\
1043	0.945305914757069\\
1044	0.945305914751105\\
1045	0.945305914742233\\
1046	0.945305914731262\\
1047	0.945305914719025\\
1048	0.94530591470633\\
1049	0.94530591469392\\
1050	0.945305914682436\\
1051	0.945305914672398\\
1052	0.945305914664189\\
1053	0.945305914658049\\
1054	0.945305914654078\\
1055	0.945305914652248\\
1056	0.945305914652416\\
1057	0.945305914654348\\
1058	0.94530591465774\\
1059	0.945305914662244\\
1060	0.945305914667488\\
1061	0.945305914673105\\
1062	0.945305914678747\\
1063	0.945305914684104\\
1064	0.945305914688918\\
1065	0.945305914692988\\
1066	0.945305914696176\\
1067	0.94530591469841\\
1068	0.945305914699674\\
1069	0.94530591470001\\
1070	0.945305914699504\\
1071	0.945305914698279\\
1072	0.945305914696482\\
1073	0.945305914694275\\
1074	0.945305914691825\\
1075	0.945305914689292\\
1076	0.945305914686823\\
1077	0.945305914684545\\
1078	0.945305914682561\\
1079	0.945305914680945\\
1080	0.945305914679744\\
1081	0.945305914678975\\
1082	0.945305914678633\\
1083	0.945305914678688\\
1084	0.94530591467909\\
1085	0.945305914679781\\
1086	0.94530591468069\\
1087	0.945305914681742\\
1088	0.945305914682864\\
1089	0.945305914683988\\
1090	0.945305914685051\\
1091	0.945305914686004\\
1092	0.945305914686807\\
1093	0.945305914687433\\
1094	0.945305914687868\\
1095	0.94530591468811\\
1096	0.945305914688167\\
1097	0.945305914688058\\
1098	0.945305914687806\\
1099	0.945305914687442\\
1100	0.945305914686998\\
1101	0.945305914686508\\
1102	0.945305914686003\\
1103	0.945305914685512\\
1104	0.94530591468506\\
1105	0.945305914684669\\
1106	0.94530591468435\\
1107	0.945305914684115\\
1108	0.945305914683967\\
1109	0.945305914683903\\
1110	0.945305914683918\\
1111	0.945305914684002\\
1112	0.945305914684143\\
1113	0.945305914684326\\
1114	0.945305914684537\\
1115	0.945305914684761\\
1116	0.945305914684985\\
1117	0.945305914685196\\
1118	0.945305914685385\\
1119	0.945305914685542\\
1120	0.945305914685665\\
1121	0.94530591468575\\
1122	0.945305914685796\\
1123	0.945305914685806\\
1124	0.945305914685782\\
1125	0.945305914685731\\
1126	0.945305914685658\\
1127	0.945305914685569\\
1128	0.945305914685471\\
1129	0.94530591468537\\
1130	0.945305914685272\\
1131	0.945305914685183\\
1132	0.945305914685105\\
1133	0.945305914685043\\
1134	0.945305914684997\\
1135	0.945305914684969\\
1136	0.945305914684957\\
1137	0.945305914684961\\
1138	0.945305914684979\\
1139	0.945305914685007\\
1140	0.945305914685044\\
1141	0.945305914685086\\
1142	0.945305914685131\\
1143	0.945305914685175\\
1144	0.945305914685216\\
1145	0.945305914685253\\
1146	0.945305914685284\\
1147	0.945305914685308\\
1148	0.945305914685324\\
1149	0.945305914685334\\
1150	0.945305914685335\\
1151	0.945305914685331\\
1152	0.94530591468532\\
1153	0.945305914685305\\
1154	0.945305914685288\\
1155	0.945305914685268\\
1156	0.945305914685248\\
1157	0.945305914685229\\
1158	0.94530591468521\\
1159	0.945305914685195\\
1160	0.945305914685183\\
1161	0.945305914685174\\
1162	0.945305914685169\\
1163	0.945305914685167\\
1164	0.945305914685168\\
1165	0.945305914685172\\
1166	0.945305914685179\\
1167	0.945305914685186\\
1168	0.945305914685194\\
1169	0.945305914685203\\
1170	0.945305914685212\\
1171	0.94530591468522\\
1172	0.945305914685227\\
1173	0.945305914685233\\
1174	0.945305914685237\\
1175	0.94530591468524\\
1176	0.945305914685241\\
1177	0.945305914685241\\
1178	0.945305914685239\\
1179	0.945305914685236\\
1180	0.945305914685233\\
1181	0.94530591468523\\
1182	0.945305914685226\\
1183	0.945305914685222\\
1184	0.945305914685219\\
1185	0.945305914685215\\
1186	0.945305914685212\\
1187	0.94530591468521\\
1188	0.945305914685209\\
1189	0.945305914685208\\
1190	0.945305914685208\\
1191	0.945305914685209\\
1192	0.94530591468521\\
1193	0.945305914685211\\
1194	0.945305914685213\\
1195	0.945305914685215\\
1196	0.945305914685217\\
1197	0.945305914685218\\
1198	0.94530591468522\\
1199	0.945305914685221\\
1200	0.945305914685222\\
};
\addlegendentry{$u(k)$}

\end{axis}
\end{tikzpicture}%
	\caption{Regulator PID - sygna� steruj�cy}
	\label{pid_we}
\end{figure}

\begin{figure}[h!]
	\centering
	% This file was created by matlab2tikz.
%
\definecolor{mycolor1}{rgb}{0.00000,0.44700,0.74100}%
\definecolor{mycolor2}{rgb}{0.85000,0.32500,0.09800}%
%
\begin{tikzpicture}

\begin{axis}[%
width=4.521in,
height=3.566in,
at={(0.758in,0.481in)},
scale only axis,
xmin=0,
xmax=1200,
ymin=-1,
ymax=6,
axis background/.style={fill=white},
title style={font=\bfseries},
title={Sygna� wyj�ciowy i zadany},
xmajorgrids,
ymajorgrids,
legend style={legend cell align=left, align=left, draw=white!15!black}
]
\addplot [color=mycolor1]
  table[row sep=crcr]{%
1	0\\
2	0\\
3	0\\
4	0\\
5	0\\
6	0\\
7	0\\
8	0\\
9	0\\
10	0\\
11	0\\
12	0\\
13	0\\
14	0\\
15	0\\
16	0\\
17	0\\
18	0\\
19	0\\
20	0\\
21	0\\
22	0\\
23	0\\
24	0\\
25	0\\
26	0\\
27	0\\
28	0\\
29	0\\
30	0\\
31	0\\
32	0\\
33	0\\
34	0\\
35	0\\
36	0\\
37	0\\
38	0\\
39	0\\
40	0\\
41	0\\
42	0\\
43	0\\
44	0\\
45	0\\
46	0\\
47	0\\
48	0\\
49	0\\
50	0\\
51	0\\
52	0\\
53	0\\
54	0\\
55	0\\
56	0\\
57	0\\
58	0\\
59	0\\
60	0\\
61	0\\
62	0\\
63	0\\
64	0\\
65	0\\
66	0\\
67	0\\
68	0\\
69	0\\
70	0\\
71	0\\
72	0\\
73	0\\
74	0\\
75	0\\
76	0\\
77	0\\
78	0\\
79	0\\
80	0\\
81	0\\
82	0\\
83	0\\
84	0\\
85	0\\
86	0\\
87	0\\
88	0\\
89	0\\
90	0\\
91	0\\
92	0\\
93	0\\
94	0\\
95	0\\
96	0\\
97	0\\
98	0\\
99	0\\
100	0\\
101	0\\
102	0\\
103	0\\
104	0\\
105	0.011099038115277\\
106	0.0276900586357233\\
107	0.0403725825320625\\
108	0.0502820910747874\\
109	0.058412727775787\\
110	0.0651857650549656\\
111	0.0708374867244441\\
112	0.0758660114422493\\
113	0.0807244601755672\\
114	0.0856623582781202\\
115	0.0908006459056687\\
116	0.0961889730526776\\
117	0.101827286074044\\
118	0.107681435198474\\
119	0.113703380307476\\
120	0.119845452111013\\
121	0.126066781140001\\
122	0.132335486751135\\
123	0.138628813623977\\
124	0.144931918878619\\
125	0.151235920547062\\
126	0.157535871303451\\
127	0.163829038182472\\
128	0.170113605355455\\
129	0.17638779994665\\
130	0.18264939137883\\
131	0.188895483463697\\
132	0.195122507828997\\
133	0.201326336677152\\
134	0.207502451067755\\
135	0.213646119795288\\
136	0.21975256076136\\
137	0.225817070520202\\
138	0.231835117785491\\
139	0.237802403178097\\
140	0.24371489099925\\
141	0.249568820125198\\
142	0.255360700964936\\
143	0.261087304412874\\
144	0.266745647342117\\
145	0.272332977755931\\
146	0.27784676144834\\
147	0.283284671019413\\
148	0.288644577373596\\
149	0.293924543379972\\
150	0.299122819145014\\
151	0.304237838284941\\
152	0.309268214630861\\
153	0.314212738907781\\
154	0.319070375061172\\
155	0.323840256035862\\
156	0.328521678925603\\
157	0.333114099499714\\
158	0.337617126173994\\
159	0.342030513528692\\
160	0.346354155491264\\
161	0.350588078301121\\
162	0.354732433362925\\
163	0.358787490078729\\
164	0.362753628731032\\
165	0.366631333471103\\
166	0.370421185451462\\
167	0.374123856128759\\
168	0.377740100753844\\
169	0.381270752059212\\
170	0.384716714149748\\
171	0.388078956600314\\
172	0.391358508762545\\
173	0.39455645428276\\
174	0.397673925832855\\
175	0.400712100055985\\
176	0.403672192728716\\
177	0.406555454140982\\
178	0.409363164694618\\
179	0.412096630720484\\
180	0.414757180513313\\
181	0.417346160582431\\
182	0.41986493211555\\
183	0.422314867651891\\
184	0.424697347960063\\
185	0.427013759115402\\
186	0.429265489770862\\
187	0.431453928615116\\
188	0.433580462011115\\
189	0.435646471808196\\
190	0.437653333320618\\
191	0.439602413465375\\
192	0.441495069052133\\
193	0.443332645218179\\
194	0.445116474001339\\
195	0.446847873043985\\
196	0.448528144421316\\
197	0.450158573587319\\
198	0.451740428431924\\
199	0.453274958443081\\
200	0.454763393967643\\
201	0.45620694556515\\
202	0.457606803448794\\
203	0.458964137008045\\
204	0.46028009440764\\
205	0.461555802257826\\
206	0.462792365350987\\
207	0.463990866459957\\
208	0.465152366193594\\
209	0.466277902905324\\
210	0.46736849265065\\
211	0.468425129189767\\
212	0.469448784031657\\
213	0.470440406516217\\
214	0.471400923931175\\
215	0.47233124166072\\
216	0.473232243362961\\
217	0.474104791173492\\
218	0.474949725932503\\
219	0.475767867433057\\
220	0.476560014688253\\
221	0.477326946215213\\
222	0.478069420333886\\
223	0.478788175478866\\
224	0.479483930522498\\
225	0.480157385107685\\
226	0.480809219988923\\
227	0.481440097380176\\
228	0.482050661308352\\
229	0.482641537971167\\
230	0.483213336098347\\
231	0.483766647315149\\
232	0.484302046507292\\
233	0.484820092186452\\
234	0.48532132685555\\
235	0.485806277373136\\
236	0.48627545531621\\
237	0.486729357340919\\
238	0.487168465540583\\
239	0.487593247800589\\
240	0.488004158149706\\
241	0.488401637107458\\
242	0.488786112027199\\
243	0.489157997434587\\
244	0.489517695361197\\
245	0.489865595673034\\
246	0.490202076393742\\
247	0.490527504022339\\
248	0.490842233845333\\
249	0.491146610243081\\
250	0.491440966990315\\
251	0.491725627550732\\
252	0.492000905365608\\
253	0.492267104136374\\
254	0.492524518101148\\
255	0.492773432305195\\
256	0.493014122865324\\
257	0.493246857228236\\
258	0.493471894422849\\
259	0.493689485306632\\
260	0.493899872805994\\
261	0.494103292150781\\
262	0.494299971102944\\
263	0.494490130179433\\
264	0.494673982869409\\
265	0.494851735845828\\
266	0.495023589171498\\
267	0.495189736499685\\
268	0.49535036526936\\
269	0.495505656895177\\
270	0.49565578695228\\
271	0.495800925356035\\
272	0.495941236536776\\
273	0.496076879609683\\
274	0.496208008539876\\
275	0.496334772302828\\
276	0.496457315040207\\
277	0.496575776211244\\
278	0.496690290739724\\
279	0.496800989156706\\
280	0.496907997739074\\
281	0.497011438644012\\
282	0.497111430039513\\
283	0.497208086231008\\
284	0.497301517784225\\
285	0.497391831644357\\
286	0.497479131251655\\
287	0.497563516653525\\
288	0.497645084613219\\
289	0.497723928715227\\
290	0.497800139467438\\
291	0.497873804400176\\
292	0.497945008162181\\
293	0.498013832613636\\
294	0.498080356916307\\
295	0.498144657620884\\
296	0.498206808751604\\
297	0.498266881888234\\
298	0.498324946245479\\
299	0.498381068749903\\
300	0.498435314114432\\
301	0.498487744910498\\
302	0.498538421637915\\
303	0.498587402792527\\
304	0.498634744931722\\
305	0.526698746229228\\
306	0.566353088107042\\
307	0.605316967694089\\
308	0.642389321062218\\
309	0.677075539998887\\
310	0.707472449350021\\
311	0.732347584498124\\
312	0.751851736654251\\
313	0.766644614924771\\
314	0.777507046095499\\
315	0.785342465291044\\
316	0.791113925996098\\
317	0.795704990950537\\
318	0.799839014633019\\
319	0.804059139712752\\
320	0.80872876333154\\
321	0.814043138817375\\
322	0.82005460595261\\
323	0.826706449057157\\
324	0.833868011768485\\
325	0.841367027256849\\
326	0.849017350121723\\
327	0.856640862776585\\
328	0.864082954936987\\
329	0.871221797996159\\
330	0.877972200000358\\
331	0.884285087353039\\
332	0.890143764539399\\
333	0.895558105587719\\
334	0.900557732109257\\
335	0.905185058894851\\
336	0.909488878355762\\
337	0.913518938530697\\
338	0.917321765327184\\
339	0.920937802959724\\
340	0.924399807020444\\
341	0.927732326049821\\
342	0.930952048627582\\
343	0.934068769293385\\
344	0.93708673140419\\
345	0.940006130737621\\
346	0.942824602645251\\
347	0.94553856095978\\
348	0.948144302951599\\
349	0.950638837080408\\
350	0.953020426176117\\
351	0.955288866404554\\
352	0.957445541446987\\
353	0.95949330215559\\
354	0.961436225580588\\
355	0.963279305127732\\
356	0.965028117293984\\
357	0.966688501528409\\
358	0.968266279706896\\
359	0.969767031683839\\
360	0.971195934290882\\
361	0.972557663590286\\
362	0.973856354473527\\
363	0.975095607896403\\
364	0.976278534041283\\
365	0.977407819242157\\
366	0.978485805267093\\
367	0.979514571163985\\
368	0.98049600998777\\
369	0.981431895028417\\
370	0.982323932395235\\
371	0.983173798797247\\
372	0.983983164972476\\
373	0.984753706403901\\
374	0.985487103711959\\
375	0.986185035468488\\
376	0.986849166197844\\
377	0.987481132095901\\
378	0.988082526589176\\
379	0.988654887353108\\
380	0.98919968587885\\
381	0.989718320176091\\
382	0.990212110764484\\
383	0.990682299761507\\
384	0.991130052629913\\
385	0.991556462001582\\
386	0.99196255293637\\
387	0.992349288988807\\
388	0.99271757852348\\
389	0.993068280822891\\
390	0.993402211651557\\
391	0.993720148062449\\
392	0.994022832344616\\
393	0.994310975106273\\
394	0.994585257561052\\
395	0.99484633313519\\
396	0.995094828541104\\
397	0.995331344470942\\
398	0.995556456056135\\
399	0.995770713220252\\
400	0.995974641026741\\
401	0.996168740094612\\
402	0.996353487126879\\
403	0.99652933557122\\
404	0.996696716411364\\
405	0.996856039072119\\
406	0.997007692410728\\
407	0.997152045762149\\
408	0.997289450005053\\
409	0.997420238617887\\
410	0.997544728699271\\
411	0.997663221933183\\
412	0.997776005486037\\
413	0.997883352829173\\
414	0.997985524485799\\
415	0.998082768705938\\
416	0.998175322076108\\
417	0.998263410072417\\
418	0.998347247566541\\
419	0.998427039293922\\
420	0.998502980292664\\
421	0.998575256320254\\
422	0.998644044253694\\
423	0.998709512476985\\
424	0.998771821258411\\
425	0.998831123118777\\
426	0.998887563190724\\
427	0.998941279568558\\
428	0.99899240364758\\
429	0.999041060451754\\
430	0.999087368948599\\
431	0.999131442350368\\
432	0.999173388400885\\
433	0.999213309647736\\
434	0.999251303699859\\
435	0.999287463470854\\
436	0.999321877408616\\
437	0.99935462971204\\
438	0.999385800535706\\
439	0.999415466183436\\
440	0.999443699291674\\
441	0.999470569003538\\
442	0.999496141134335\\
443	0.999520478329217\\
444	0.999543640213575\\
445	0.999565683536623\\
446	0.999586662308577\\
447	0.999606627931737\\
448	0.999625629325711\\
449	0.999643713047001\\
450	0.999660923403126\\
451	0.999677302561451\\
452	0.999692890652896\\
453	0.999707725870674\\
454	0.999721844564256\\
455	0.999735281328738\\
456	0.999748069089799\\
457	0.999760239184469\\
458	0.999771821437888\\
459	0.999782844236279\\
460	0.999793334596322\\
461	0.999803318231132\\
462	0.999812819613016\\
463	0.999821862033195\\
464	0.999830467658643\\
465	0.999838657586203\\
466	0.999846451894112\\
467	0.999853869691071\\
468	0.999860929162977\\
469	0.999867647617425\\
470	0.999874041526091\\
471	0.999880126565085\\
472	0.999885917653375\\
473	0.999891428989359\\
474	0.999896674085686\\
475	0.999901665802389\\
476	0.999906416378416\\
477	0.999910937461641\\
478	0.999915240137418\\
479	0.999919334955746\\
480	0.999923231957128\\
481	0.999926940697168\\
482	0.99993047026998\\
483	0.999933829330469\\
484	0.999937026115528\\
485	0.999940068464219\\
486	0.999942963836973\\
487	0.999945719333871\\
488	0.999948341712038\\
489	0.999950837402207\\
490	0.999953212524472\\
491	0.999955472903295\\
492	0.999957624081781\\
493	0.99995967133527\\
494	0.99996161968427\\
495	0.999963473906762\\
496	0.999965238549924\\
497	0.999966917941274\\
498	0.999968516199284\\
499	0.999970037243474\\
500	0.999971484804028\\
501	0.999972862430937\\
502	0.999974173502702\\
503	0.999975421234618\\
504	0.999976608686658\\
505	0.898001574719617\\
506	0.783200934688832\\
507	0.681511278772896\\
508	0.594080075281499\\
509	0.520354824544862\\
510	0.464410853438658\\
511	0.426319431818142\\
512	0.402541625671858\\
513	0.389084169140741\\
514	0.382423690589083\\
515	0.379423501934379\\
516	0.377479385075131\\
517	0.374763571126575\\
518	0.370235195768808\\
519	0.363499736009054\\
520	0.35464369165637\\
521	0.344071116179407\\
522	0.332345311539057\\
523	0.320055444620181\\
524	0.307723500495215\\
525	0.295752329239734\\
526	0.284407867157555\\
527	0.273826657038642\\
528	0.264039295047274\\
529	0.255000999063174\\
530	0.246622351940615\\
531	0.238795694385329\\
532	0.231414877489186\\
533	0.224387845369991\\
534	0.21764274942252\\
535	0.211129004611332\\
536	0.204814950214089\\
537	0.198683690488465\\
538	0.192728397218584\\
539	0.186947972764123\\
540	0.181343588831778\\
541	0.175916292065691\\
542	0.170665632332699\\
543	0.165589128166596\\
544	0.160682324256569\\
545	0.155939197313173\\
546	0.15135270610142\\
547	0.146915338249558\\
548	0.142619564883047\\
549	0.138458164135379\\
550	0.134424411298736\\
551	0.130512155984389\\
552	0.126715816985542\\
553	0.123030326715694\\
554	0.119451052561729\\
555	0.115973715272448\\
556	0.112594316877571\\
557	0.10930908403664\\
558	0.106114427838638\\
559	0.103006918030683\\
560	0.0999832682071197\\
561	0.0970403282246917\\
562	0.0941750805797676\\
563	0.0913846383022963\\
564	0.0886662428021787\\
565	0.0860172608697698\\
566	0.0834351806005124\\
567	0.0809176063708178\\
568	0.0784622531673264\\
569	0.0760669406130858\\
570	0.0737295869930655\\
571	0.0714482035013205\\
572	0.0692208888445327\\
573	0.0670458242606749\\
574	0.0649212689557838\\
575	0.0628455559274983\\
576	0.0608170881279212\\
577	0.0588343349154343\\
578	0.0568958287501319\\
579	0.0550001620962133\\
580	0.0531459845039322\\
581	0.05133199985175\\
582	0.0495569637353974\\
583	0.0478196809945635\\
584	0.0461190033702144\\
585	0.0444538272865956\\
586	0.0428230917522671\\
587	0.0412257763744528\\
588	0.0396608994808244\\
589	0.0381275163427421\\
590	0.036624717493999\\
591	0.0351516271392782\\
592	0.033707401646792\\
593	0.0322912281199036\\
594	0.0309023230428886\\
595	0.0295399309963494\\
596	0.0282033234381346\\
597	0.0268917975459214\\
598	0.025604675117898\\
599	0.0243413015282276\\
600	0.0231010447341965\\
601	0.0218832943321486\\
602	0.0206874606594876\\
603	0.0195129739401977\\
604	0.0183592834714856\\
605	0.0172258568492965\\
606	0.0161121792305856\\
607	0.0150177526303622\\
608	0.0139420952516352\\
609	0.0128847408465063\\
610	0.0118452381067606\\
611	0.0108231500823994\\
612	0.0098180536266561\\
613	0.00882953886611921\\
614	0.00785720869466567\\
615	0.0069006782899855\\
616	0.00595957465154607\\
617	0.00503353615891149\\
618	0.00412221214939351\\
619	0.00322526251406811\\
620	0.00234235731124588\\
621	0.00147317639653514\\
622	0.000617409068684301\\
623	-0.00022524626956547\\
624	-0.001055082436344\\
625	-0.00187238378191974\\
626	-0.00267742648702158\\
627	-0.00347047884883022\\
628	-0.00425180155522179\\
629	-0.00502164794781537\\
630	-0.0057802642743472\\
631	-0.0065278899308667\\
632	-0.00726475769422404\\
633	-0.00799109394529425\\
634	-0.00870711888336028\\
635	-0.00941304673205564\\
636	-0.0101090859372469\\
637	-0.0107954393572174\\
638	-0.0114723044454941\\
639	-0.0121398734266452\\
640	-0.0127983334653554\\
641	-0.013447866829076\\
642	-0.0140886510445271\\
643	-0.0147208590483194\\
644	-0.0153446593319482\\
645	-0.0159602160814011\\
646	-0.0165676893116072\\
647	-0.0171672349959488\\
648	-0.0177590051910399\\
649	-0.0183431481569733\\
650	-0.0189198084732228\\
651	-0.0194891271503805\\
652	-0.020051241737902\\
653	-0.0206062864280221\\
654	-0.0211543921559972\\
655	-0.0216956866968235\\
656	-0.0222302947585734\\
657	-0.0227583380724847\\
658	-0.0232799354799335\\
659	-0.0237952030164137\\
660	-0.0243042539926415\\
661	-0.0248071990728973\\
662	-0.0253041463507149\\
663	-0.0257952014220179\\
664	-0.0262804674558054\\
665	-0.0267600452624792\\
666	-0.0272340333599035\\
667	-0.0277025280372834\\
668	-0.0281656234169451\\
669	-0.0286234115140969\\
670	-0.0290759822946462\\
671	-0.0295234237311461\\
672	-0.0299658218569407\\
673	-0.0304032608185749\\
674	-0.0308358229265343\\
675	-0.0312635887043746\\
676	-0.0316866369363006\\
677	-0.03210504471325\\
678	-0.0325188874775365\\
679	-0.0329282390661043\\
680	-0.0333331717524424\\
681	-0.0337337562872083\\
682	-0.0341300619376047\\
683	-0.0345221565255551\\
684	-0.0349101064647189\\
685	-0.0352939767963877\\
686	-0.0356738312243012\\
687	-0.0360497321484195\\
688	-0.036421740697689\\
689	-0.0367899167618357\\
690	-0.0371543190222187\\
691	-0.0375150049817771\\
692	-0.0378720309940992\\
693	-0.0382254522916456\\
694	-0.0385753230131523\\
695	-0.0389216962302436\\
696	-0.0392646239732793\\
697	-0.0396041572564616\\
698	-0.0399403461022272\\
699	-0.0402732395649475\\
700	-0.0406028857539583\\
701	-0.0409293318559438\\
702	-0.0412526241566914\\
703	-0.0415728080622425\\
704	-0.0418899281194545\\
705	0.0615190507248006\\
706	0.231911210517837\\
707	0.383928274878367\\
708	0.531078228472625\\
709	0.683424255413877\\
710	0.842150134397384\\
711	1.00513799026439\\
712	1.17405399210545\\
713	1.35073551581125\\
714	1.53541094014397\\
715	1.72736597213333\\
716	1.92548881923763\\
717	2.12825924193262\\
718	2.33373044978238\\
719	2.53967933117324\\
720	2.74376235119135\\
721	2.94362345842054\\
722	3.13698615059547\\
723	3.32174438538961\\
724	3.49604241857061\\
725	3.65833736319747\\
726	3.80744521468476\\
727	3.94257079269891\\
728	4.06332057485578\\
729	4.16969795778346\\
730	4.26208165423711\\
731	4.34118872271334\\
732	4.40802439834323\\
733	4.46382163806117\\
734	4.50997393258545\\
735	4.54796530414776\\
736	4.57930146043551\\
737	4.60544581725824\\
738	4.62776355993047\\
739	4.64747614853063\\
740	4.66562777981749\\
741	4.68306440134104\\
742	4.70042502159298\\
743	4.71814434139505\\
744	4.73646518605342\\
745	4.75545885794942\\
746	4.77505134601028\\
747	4.79505329821956\\
748	4.81519175525516\\
749	4.83514182567523\\
750	4.85455672688812\\
751	4.87309489740215\\
752	4.89044318577607\\
753	4.90633542608148\\
754	4.92056600769465\\
755	4.93299833007199\\
756	4.94356829307016\\
757	4.95228320301892\\
758	4.95921666715923\\
759	4.96450019794621\\
760	4.96831234918737\\
761	4.97086625527838\\
762	4.97239644293368\\
763	4.97314573494014\\
764	4.97335297382006\\
765	4.9732421687057\\
766	4.97301352181954\\
767	4.97283663310156\\
768	4.97284602381767\\
769	4.97313897229435\\
770	4.97377552521718\\
771	4.97478044187682\\
772	4.97614674962303\\
773	4.97784053764217\\
774	4.9798065921442\\
775	4.98197447678342\\
776	4.98426468425036\\
777	4.98659452442703\\
778	4.98888346697051\\
779	4.99105771733352\\
780	4.99305387087718\\
781	4.99482155604007\\
782	4.99632504108531\\
783	4.99754383684949\\
784	4.99847237782333\\
785	4.9991189040746\\
786	4.99950369588578\\
787	4.99965683107248\\
788	4.99961564194792\\
789	4.99942204555271\\
790	4.99911990832122\\
791	4.99875258643857\\
792	4.99836075765144\\
793	4.99798063125329\\
794	4.99764259238622\\
795	4.99737030657305\\
796	4.99718028216859\\
797	4.99708186354879\\
798	4.99707760733679\\
799	4.99716397842906\\
800	4.99733229230425\\
801	4.99756982501962\\
802	4.99786101208943\\
803	4.99818866153718\\
804	4.9985351140922\\
805	4.99888329393132\\
806	4.99921760566958\\
807	4.99952464660393\\
808	4.99979371668687\\
809	5.00001712160906\\
810	5.00019027607528\\
811	5.00031162437104\\
812	5.00038240328725\\
813	5.00040627819482\\
814	5.00038888647781\\
815	5.00033732371399\\
816	5.0002596071173\\
817	5.00016414810384\\
818	5.00005926175315\\
819	4.99995273579309\\
820	4.9998514759385\\
821	4.99976123835134\\
822	4.99968645402444\\
823	4.99963014433807\\
824	4.99959392215453\\
825	4.99957806879233\\
826	4.99958167417996\\
827	4.99960282548372\\
828	4.99963882852641\\
829	4.9996864462997\\
830	4.99974213971372\\
831	4.99980229728103\\
832	4.99986344253187\\
833	4.99992241042939\\
834	4.99997648671974\\
835	5.00002350684706\\
836	5.00006191363791\\
837	5.00009077528694\\
838	5.00010976715836\\
839	5.00011912248606\\
840	5.00011955817144\\
841	5.00011218253151\\
842	5.000098392058\\
843	5.0000797640486\\
844	5.00005795142207\\
845	5.00003458519783\\
846	5.0000111890852\\
847	4.99998910946743\\
848	4.99996946285722\\
849	4.99995310171566\\
850	4.99994059842707\\
851	4.99993224625782\\
852	4.99992807533554\\
853	4.99992788109003\\
854	4.99993126220741\\
855	4.99993766496506\\
856	4.99994643082031\\
857	4.99995684430223\\
858	4.99996817857151\\
859	4.99997973643743\\
860	4.99999088511687\\
861	5.00000108355381\\
862	5.000009901655\\
863	5.00001703130948\\
864	5.00002228952098\\
865	5.00002561437346\\
866	5.00002705485861\\
867	5.00002675581126\\
868	5.00002493932437\\
869	5.00002188405146\\
870	5.00001790376079\\
871	5.00001332639202\\
872	5.00000847469779\\
873	5.00000364934444\\
874	4.99999911511373\\
875	4.99999509060625\\
876	4.99999174161198\\
877	4.99998917809505\\
878	4.99998745454936\\
879	4.99998657332606\\
880	4.99998649041691\\
881	4.99998712310271\\
882	4.99998835884075\\
883	4.99999006476884\\
884	4.99999209723979\\
885	4.99999431086511\\
886	4.99999656663159\\
887	4.99999873875448\\
888	5.00000072003724\\
889	5.00000242561521\\
890	5.000003795062\\
891	5.00000479292884\\
892	5.00000540786441\\
893	5.00000565052324\\
894	5.00000555051293\\
895	5.0000051526547\\
896	5.00000451283806\\
897	5.00000369374065\\
898	5.00000276066125\\
899	5.0000017776798\\
900	5.0000008043162\\
901	4.9999998928134\\
902	4.99999908612207\\
903	4.99999841661733\\
904	4.99999790553468\\
905	4.99999756307488\\
906	4.99999738909646\\
907	4.99999737429215\\
908	4.99999750173059\\
909	4.99999774863848\\
910	4.999998088299\\
911	4.99999849195046\\
912	4.99999893058178\\
913	4.99999937653885\\
914	4.99999980487598\\
915	5.00000019440746\\
916	5.00000052843618\\
917	5.0000007951559\\
918	5.00000098774229\\
919	5.00000110416279\\
920	5.00000114674736\\
921	5.0000011215705\\
922	5.0000010376993\\
923	5.00000090636363\\
924	5.00000074010226\\
925	5.00000055193408\\
926	5.00000035459671\\
927	5.00000015988617\\
928	4.99999997812219\\
929	4.99999981775402\\
930	4.9999996851124\\
931	4.99999958430457\\
932	4.99999951724199\\
933	4.99999948378426\\
934	4.99999948197836\\
935	4.99999950836936\\
936	4.99999955835777\\
937	4.99999962657877\\
938	4.99999970728022\\
939	4.99999979467921\\
940	4.99999988327987\\
941	4.99999996813991\\
942	5.00000004507688\\
943	5.00000011080993\\
944	5.00000016303652\\
945	5.00000020044731\\
946	5.0000002226854\\
947	5.00000023025836\\
948	5.00000022441319\\
949	5.00000020698521\\
950	5.00000018023195\\
951	5.00000014666291\\
952	5.00000010887473\\
953	5.00000006940024\\
954	5.00000003057806\\
955	4.99999999444733\\
956	4.99999996267072\\
957	4.99999993648652\\
958	4.99999991668918\\
959	4.99999990363622\\
960	4.99999989727809\\
961	4.99999989720678\\
962	4.99999990271842\\
963	4.99999991288494\\
964	4.99999992662981\\
965	4.99999994280329\\
966	4.99999996025321\\
967	4.99999997788787\\
968	4.99999999472859\\
969	5.00000000995013\\
970	5.00000002290828\\
971	5.00000003315447\\
972	5.00000004043808\\
973	5.00000004469774\\
974	5.00000004604333\\
975	5.00000004473072\\
976	5.00000004113136\\
977	5.00000003569911\\
978	5.00000002893622\\
979	5.00000002136062\\
980	5.00000001347597\\
981	5.00000000574594\\
982	4.99999999857344\\
983	4.99999999228566\\
984	4.99999998712476\\
985	4.99999998324432\\
986	4.99999998071101\\
987	4.99999997951072\\
988	4.99999997955846\\
989	4.999999980711\\
990	4.99999998278118\\
991	4.99999998555306\\
992	4.9999999887969\\
993	4.99999999228323\\
994	4.99999999579524\\
995	4.9999999991392\\
996	5.00000000215233\\
997	5.0000000047082\\
998	5.0000000067195\\
999	5.00000000813835\\
1000	5.00000000895446\\
1001	5.0000000091915\\
1002	5.00000000890189\\
1003	5.00000000816081\\
1004	5.00000000705947\\
1005	5.0000000056984\\
1006	5.00000000418087\\
1007	5.00000000260706\\
1008	5.00000000106886\\
1009	4.99999999964588\\
1010	4.9999999984025\\
1011	4.99999999738606\\
1012	4.99999999662623\\
1013	4.99999999613537\\
1014	4.99999999590986\\
1015	4.99999999593202\\
1016	4.99999999617273\\
1017	4.99999999659427\\
1018	4.99999999715333\\
1019	4.99999999780403\\
1020	4.99999999850065\\
1021	4.99999999920017\\
1022	4.99999999986424\\
1023	5.00000000046076\\
1024	5.00000000096492\\
1025	5.00000000135975\\
1026	5.0000000016361\\
1027	5.00000000179233\\
1028	5.00000000183349\\
1029	5.00000000177033\\
1030	5.00000000161803\\
1031	5.00000000139495\\
1032	5.00000000112118\\
1033	5.00000000081732\\
1034	5.0000000005033\\
1035	5.00000000019731\\
1036	4.9999999999151\\
1037	4.99999999966932\\
1038	4.99999999946922\\
1039	4.99999999932054\\
1040	4.99999999922555\\
1041	4.99999999918336\\
1042	4.99999999919031\\
1043	4.99999999924049\\
1044	4.99999999932629\\
1045	4.99999999943903\\
1046	4.99999999956954\\
1047	4.99999999970873\\
1048	4.99999999984805\\
1049	4.99999999997992\\
1050	5.000000000098\\
1051	5.00000000019743\\
1052	5.00000000027492\\
1053	5.00000000032872\\
1054	5.00000000035859\\
1055	5.00000000036559\\
1056	5.00000000035193\\
1057	5.00000000032068\\
1058	5.00000000027552\\
1059	5.00000000022048\\
1060	5.00000000015965\\
1061	5.00000000009701\\
1062	5.00000000003615\\
1063	4.9999999999802\\
1064	4.99999999993162\\
1065	4.99999999989225\\
1066	4.99999999986317\\
1067	4.9999999998448\\
1068	4.99999999983694\\
1069	4.99999999983884\\
1070	4.99999999984927\\
1071	4.99999999986673\\
1072	4.99999999988946\\
1073	4.99999999991563\\
1074	4.99999999994344\\
1075	4.99999999997119\\
1076	4.99999999999738\\
1077	5.00000000002075\\
1078	5.00000000004036\\
1079	5.00000000005556\\
1080	5.00000000006603\\
1081	5.00000000007173\\
1082	5.00000000007289\\
1083	5.00000000006995\\
1084	5.00000000006354\\
1085	5.00000000005441\\
1086	5.00000000004334\\
1087	5.00000000003117\\
1088	5.00000000001867\\
1089	5.00000000000657\\
1090	4.99999999999548\\
1091	4.99999999998589\\
1092	4.99999999997814\\
1093	4.99999999997245\\
1094	4.99999999996891\\
1095	4.99999999996745\\
1096	4.99999999996793\\
1097	4.99999999997009\\
1098	4.99999999997364\\
1099	4.99999999997822\\
1100	4.99999999998347\\
1101	4.99999999998903\\
1102	4.99999999999455\\
1103	4.99999999999975\\
1104	5.00000000000437\\
1105	5.00000000000824\\
1106	5.00000000001122\\
1107	5.00000000001326\\
1108	5.00000000001434\\
1109	5.00000000001452\\
1110	5.0000000000139\\
1111	5.00000000001258\\
1112	5.00000000001074\\
1113	5.00000000000851\\
1114	5.00000000000608\\
1115	5.00000000000359\\
1116	5.00000000000118\\
1117	4.99999999999898\\
1118	4.99999999999709\\
1119	4.99999999999557\\
1120	4.99999999999446\\
1121	4.99999999999378\\
1122	4.99999999999351\\
1123	4.99999999999362\\
1124	4.99999999999407\\
1125	4.99999999999479\\
1126	4.99999999999571\\
1127	4.99999999999676\\
1128	4.99999999999787\\
1129	4.99999999999897\\
1130	5\\
1131	5.00000000000091\\
1132	5.00000000000168\\
1133	5.00000000000226\\
1134	5.00000000000266\\
1135	5.00000000000286\\
1136	5.00000000000289\\
1137	5.00000000000275\\
1138	5.00000000000248\\
1139	5.00000000000211\\
1140	5.00000000000166\\
1141	5.00000000000118\\
1142	5.00000000000068\\
1143	5.00000000000021\\
1144	4.99999999999978\\
1145	4.99999999999941\\
1146	4.99999999999911\\
1147	4.99999999999889\\
1148	4.99999999999876\\
1149	4.99999999999871\\
1150	4.99999999999874\\
1151	4.99999999999883\\
1152	4.99999999999897\\
1153	4.99999999999915\\
1154	4.99999999999936\\
1155	4.99999999999959\\
1156	4.9999999999998\\
1157	5.00000000000001\\
1158	5.00000000000019\\
1159	5.00000000000034\\
1160	5.00000000000046\\
1161	5.00000000000053\\
1162	5.00000000000057\\
1163	5.00000000000057\\
1164	5.00000000000054\\
1165	5.00000000000048\\
1166	5.00000000000041\\
1167	5.00000000000031\\
1168	5.00000000000022\\
1169	5.00000000000012\\
1170	5.00000000000002\\
1171	4.99999999999994\\
1172	4.99999999999987\\
1173	4.99999999999982\\
1174	4.99999999999978\\
1175	4.99999999999976\\
1176	4.99999999999975\\
1177	4.99999999999976\\
1178	4.99999999999979\\
1179	4.99999999999982\\
1180	4.99999999999986\\
1181	4.9999999999999\\
1182	4.99999999999994\\
1183	4.99999999999998\\
1184	5.00000000000002\\
1185	5.00000000000005\\
1186	5.00000000000008\\
1187	5.0000000000001\\
1188	5.00000000000011\\
1189	5.00000000000011\\
1190	5.00000000000011\\
1191	5.0000000000001\\
1192	5.00000000000009\\
1193	5.00000000000007\\
1194	5.00000000000005\\
1195	5.00000000000003\\
1196	5.00000000000001\\
1197	4.99999999999999\\
1198	4.99999999999998\\
1199	4.99999999999997\\
1200	4.99999999999996\\
};
\addlegendentry{$y(k)$}

\addplot[const plot, color=mycolor2, dashed] table[row sep=crcr] {%
1	0\\
2	0\\
3	0\\
4	0\\
5	0\\
6	0\\
7	0\\
8	0\\
9	0\\
10	0\\
11	0\\
12	0\\
13	0\\
14	0\\
15	0\\
16	0\\
17	0\\
18	0\\
19	0\\
20	0\\
21	0\\
22	0\\
23	0\\
24	0\\
25	0\\
26	0\\
27	0\\
28	0\\
29	0\\
30	0\\
31	0\\
32	0\\
33	0\\
34	0\\
35	0\\
36	0\\
37	0\\
38	0\\
39	0\\
40	0\\
41	0\\
42	0\\
43	0\\
44	0\\
45	0\\
46	0\\
47	0\\
48	0\\
49	0\\
50	0\\
51	0\\
52	0\\
53	0\\
54	0\\
55	0\\
56	0\\
57	0\\
58	0\\
59	0\\
60	0\\
61	0\\
62	0\\
63	0\\
64	0\\
65	0\\
66	0\\
67	0\\
68	0\\
69	0\\
70	0\\
71	0\\
72	0\\
73	0\\
74	0\\
75	0\\
76	0\\
77	0\\
78	0\\
79	0\\
80	0\\
81	0\\
82	0\\
83	0\\
84	0\\
85	0\\
86	0\\
87	0\\
88	0\\
89	0\\
90	0\\
91	0\\
92	0\\
93	0\\
94	0\\
95	0\\
96	0\\
97	0\\
98	0\\
99	0\\
100	0.5\\
101	0.5\\
102	0.5\\
103	0.5\\
104	0.5\\
105	0.5\\
106	0.5\\
107	0.5\\
108	0.5\\
109	0.5\\
110	0.5\\
111	0.5\\
112	0.5\\
113	0.5\\
114	0.5\\
115	0.5\\
116	0.5\\
117	0.5\\
118	0.5\\
119	0.5\\
120	0.5\\
121	0.5\\
122	0.5\\
123	0.5\\
124	0.5\\
125	0.5\\
126	0.5\\
127	0.5\\
128	0.5\\
129	0.5\\
130	0.5\\
131	0.5\\
132	0.5\\
133	0.5\\
134	0.5\\
135	0.5\\
136	0.5\\
137	0.5\\
138	0.5\\
139	0.5\\
140	0.5\\
141	0.5\\
142	0.5\\
143	0.5\\
144	0.5\\
145	0.5\\
146	0.5\\
147	0.5\\
148	0.5\\
149	0.5\\
150	0.5\\
151	0.5\\
152	0.5\\
153	0.5\\
154	0.5\\
155	0.5\\
156	0.5\\
157	0.5\\
158	0.5\\
159	0.5\\
160	0.5\\
161	0.5\\
162	0.5\\
163	0.5\\
164	0.5\\
165	0.5\\
166	0.5\\
167	0.5\\
168	0.5\\
169	0.5\\
170	0.5\\
171	0.5\\
172	0.5\\
173	0.5\\
174	0.5\\
175	0.5\\
176	0.5\\
177	0.5\\
178	0.5\\
179	0.5\\
180	0.5\\
181	0.5\\
182	0.5\\
183	0.5\\
184	0.5\\
185	0.5\\
186	0.5\\
187	0.5\\
188	0.5\\
189	0.5\\
190	0.5\\
191	0.5\\
192	0.5\\
193	0.5\\
194	0.5\\
195	0.5\\
196	0.5\\
197	0.5\\
198	0.5\\
199	0.5\\
200	0.5\\
201	0.5\\
202	0.5\\
203	0.5\\
204	0.5\\
205	0.5\\
206	0.5\\
207	0.5\\
208	0.5\\
209	0.5\\
210	0.5\\
211	0.5\\
212	0.5\\
213	0.5\\
214	0.5\\
215	0.5\\
216	0.5\\
217	0.5\\
218	0.5\\
219	0.5\\
220	0.5\\
221	0.5\\
222	0.5\\
223	0.5\\
224	0.5\\
225	0.5\\
226	0.5\\
227	0.5\\
228	0.5\\
229	0.5\\
230	0.5\\
231	0.5\\
232	0.5\\
233	0.5\\
234	0.5\\
235	0.5\\
236	0.5\\
237	0.5\\
238	0.5\\
239	0.5\\
240	0.5\\
241	0.5\\
242	0.5\\
243	0.5\\
244	0.5\\
245	0.5\\
246	0.5\\
247	0.5\\
248	0.5\\
249	0.5\\
250	0.5\\
251	0.5\\
252	0.5\\
253	0.5\\
254	0.5\\
255	0.5\\
256	0.5\\
257	0.5\\
258	0.5\\
259	0.5\\
260	0.5\\
261	0.5\\
262	0.5\\
263	0.5\\
264	0.5\\
265	0.5\\
266	0.5\\
267	0.5\\
268	0.5\\
269	0.5\\
270	0.5\\
271	0.5\\
272	0.5\\
273	0.5\\
274	0.5\\
275	0.5\\
276	0.5\\
277	0.5\\
278	0.5\\
279	0.5\\
280	0.5\\
281	0.5\\
282	0.5\\
283	0.5\\
284	0.5\\
285	0.5\\
286	0.5\\
287	0.5\\
288	0.5\\
289	0.5\\
290	0.5\\
291	0.5\\
292	0.5\\
293	0.5\\
294	0.5\\
295	0.5\\
296	0.5\\
297	0.5\\
298	0.5\\
299	0.5\\
300	1\\
301	1\\
302	1\\
303	1\\
304	1\\
305	1\\
306	1\\
307	1\\
308	1\\
309	1\\
310	1\\
311	1\\
312	1\\
313	1\\
314	1\\
315	1\\
316	1\\
317	1\\
318	1\\
319	1\\
320	1\\
321	1\\
322	1\\
323	1\\
324	1\\
325	1\\
326	1\\
327	1\\
328	1\\
329	1\\
330	1\\
331	1\\
332	1\\
333	1\\
334	1\\
335	1\\
336	1\\
337	1\\
338	1\\
339	1\\
340	1\\
341	1\\
342	1\\
343	1\\
344	1\\
345	1\\
346	1\\
347	1\\
348	1\\
349	1\\
350	1\\
351	1\\
352	1\\
353	1\\
354	1\\
355	1\\
356	1\\
357	1\\
358	1\\
359	1\\
360	1\\
361	1\\
362	1\\
363	1\\
364	1\\
365	1\\
366	1\\
367	1\\
368	1\\
369	1\\
370	1\\
371	1\\
372	1\\
373	1\\
374	1\\
375	1\\
376	1\\
377	1\\
378	1\\
379	1\\
380	1\\
381	1\\
382	1\\
383	1\\
384	1\\
385	1\\
386	1\\
387	1\\
388	1\\
389	1\\
390	1\\
391	1\\
392	1\\
393	1\\
394	1\\
395	1\\
396	1\\
397	1\\
398	1\\
399	1\\
400	1\\
401	1\\
402	1\\
403	1\\
404	1\\
405	1\\
406	1\\
407	1\\
408	1\\
409	1\\
410	1\\
411	1\\
412	1\\
413	1\\
414	1\\
415	1\\
416	1\\
417	1\\
418	1\\
419	1\\
420	1\\
421	1\\
422	1\\
423	1\\
424	1\\
425	1\\
426	1\\
427	1\\
428	1\\
429	1\\
430	1\\
431	1\\
432	1\\
433	1\\
434	1\\
435	1\\
436	1\\
437	1\\
438	1\\
439	1\\
440	1\\
441	1\\
442	1\\
443	1\\
444	1\\
445	1\\
446	1\\
447	1\\
448	1\\
449	1\\
450	1\\
451	1\\
452	1\\
453	1\\
454	1\\
455	1\\
456	1\\
457	1\\
458	1\\
459	1\\
460	1\\
461	1\\
462	1\\
463	1\\
464	1\\
465	1\\
466	1\\
467	1\\
468	1\\
469	1\\
470	1\\
471	1\\
472	1\\
473	1\\
474	1\\
475	1\\
476	1\\
477	1\\
478	1\\
479	1\\
480	1\\
481	1\\
482	1\\
483	1\\
484	1\\
485	1\\
486	1\\
487	1\\
488	1\\
489	1\\
490	1\\
491	1\\
492	1\\
493	1\\
494	1\\
495	1\\
496	1\\
497	1\\
498	1\\
499	1\\
500	-0.1\\
501	-0.1\\
502	-0.1\\
503	-0.1\\
504	-0.1\\
505	-0.1\\
506	-0.1\\
507	-0.1\\
508	-0.1\\
509	-0.1\\
510	-0.1\\
511	-0.1\\
512	-0.1\\
513	-0.1\\
514	-0.1\\
515	-0.1\\
516	-0.1\\
517	-0.1\\
518	-0.1\\
519	-0.1\\
520	-0.1\\
521	-0.1\\
522	-0.1\\
523	-0.1\\
524	-0.1\\
525	-0.1\\
526	-0.1\\
527	-0.1\\
528	-0.1\\
529	-0.1\\
530	-0.1\\
531	-0.1\\
532	-0.1\\
533	-0.1\\
534	-0.1\\
535	-0.1\\
536	-0.1\\
537	-0.1\\
538	-0.1\\
539	-0.1\\
540	-0.1\\
541	-0.1\\
542	-0.1\\
543	-0.1\\
544	-0.1\\
545	-0.1\\
546	-0.1\\
547	-0.1\\
548	-0.1\\
549	-0.1\\
550	-0.1\\
551	-0.1\\
552	-0.1\\
553	-0.1\\
554	-0.1\\
555	-0.1\\
556	-0.1\\
557	-0.1\\
558	-0.1\\
559	-0.1\\
560	-0.1\\
561	-0.1\\
562	-0.1\\
563	-0.1\\
564	-0.1\\
565	-0.1\\
566	-0.1\\
567	-0.1\\
568	-0.1\\
569	-0.1\\
570	-0.1\\
571	-0.1\\
572	-0.1\\
573	-0.1\\
574	-0.1\\
575	-0.1\\
576	-0.1\\
577	-0.1\\
578	-0.1\\
579	-0.1\\
580	-0.1\\
581	-0.1\\
582	-0.1\\
583	-0.1\\
584	-0.1\\
585	-0.1\\
586	-0.1\\
587	-0.1\\
588	-0.1\\
589	-0.1\\
590	-0.1\\
591	-0.1\\
592	-0.1\\
593	-0.1\\
594	-0.1\\
595	-0.1\\
596	-0.1\\
597	-0.1\\
598	-0.1\\
599	-0.1\\
600	-0.1\\
601	-0.1\\
602	-0.1\\
603	-0.1\\
604	-0.1\\
605	-0.1\\
606	-0.1\\
607	-0.1\\
608	-0.1\\
609	-0.1\\
610	-0.1\\
611	-0.1\\
612	-0.1\\
613	-0.1\\
614	-0.1\\
615	-0.1\\
616	-0.1\\
617	-0.1\\
618	-0.1\\
619	-0.1\\
620	-0.1\\
621	-0.1\\
622	-0.1\\
623	-0.1\\
624	-0.1\\
625	-0.1\\
626	-0.1\\
627	-0.1\\
628	-0.1\\
629	-0.1\\
630	-0.1\\
631	-0.1\\
632	-0.1\\
633	-0.1\\
634	-0.1\\
635	-0.1\\
636	-0.1\\
637	-0.1\\
638	-0.1\\
639	-0.1\\
640	-0.1\\
641	-0.1\\
642	-0.1\\
643	-0.1\\
644	-0.1\\
645	-0.1\\
646	-0.1\\
647	-0.1\\
648	-0.1\\
649	-0.1\\
650	-0.1\\
651	-0.1\\
652	-0.1\\
653	-0.1\\
654	-0.1\\
655	-0.1\\
656	-0.1\\
657	-0.1\\
658	-0.1\\
659	-0.1\\
660	-0.1\\
661	-0.1\\
662	-0.1\\
663	-0.1\\
664	-0.1\\
665	-0.1\\
666	-0.1\\
667	-0.1\\
668	-0.1\\
669	-0.1\\
670	-0.1\\
671	-0.1\\
672	-0.1\\
673	-0.1\\
674	-0.1\\
675	-0.1\\
676	-0.1\\
677	-0.1\\
678	-0.1\\
679	-0.1\\
680	-0.1\\
681	-0.1\\
682	-0.1\\
683	-0.1\\
684	-0.1\\
685	-0.1\\
686	-0.1\\
687	-0.1\\
688	-0.1\\
689	-0.1\\
690	-0.1\\
691	-0.1\\
692	-0.1\\
693	-0.1\\
694	-0.1\\
695	-0.1\\
696	-0.1\\
697	-0.1\\
698	-0.1\\
699	-0.1\\
700	5\\
701	5\\
702	5\\
703	5\\
704	5\\
705	5\\
706	5\\
707	5\\
708	5\\
709	5\\
710	5\\
711	5\\
712	5\\
713	5\\
714	5\\
715	5\\
716	5\\
717	5\\
718	5\\
719	5\\
720	5\\
721	5\\
722	5\\
723	5\\
724	5\\
725	5\\
726	5\\
727	5\\
728	5\\
729	5\\
730	5\\
731	5\\
732	5\\
733	5\\
734	5\\
735	5\\
736	5\\
737	5\\
738	5\\
739	5\\
740	5\\
741	5\\
742	5\\
743	5\\
744	5\\
745	5\\
746	5\\
747	5\\
748	5\\
749	5\\
750	5\\
751	5\\
752	5\\
753	5\\
754	5\\
755	5\\
756	5\\
757	5\\
758	5\\
759	5\\
760	5\\
761	5\\
762	5\\
763	5\\
764	5\\
765	5\\
766	5\\
767	5\\
768	5\\
769	5\\
770	5\\
771	5\\
772	5\\
773	5\\
774	5\\
775	5\\
776	5\\
777	5\\
778	5\\
779	5\\
780	5\\
781	5\\
782	5\\
783	5\\
784	5\\
785	5\\
786	5\\
787	5\\
788	5\\
789	5\\
790	5\\
791	5\\
792	5\\
793	5\\
794	5\\
795	5\\
796	5\\
797	5\\
798	5\\
799	5\\
800	5\\
801	5\\
802	5\\
803	5\\
804	5\\
805	5\\
806	5\\
807	5\\
808	5\\
809	5\\
810	5\\
811	5\\
812	5\\
813	5\\
814	5\\
815	5\\
816	5\\
817	5\\
818	5\\
819	5\\
820	5\\
821	5\\
822	5\\
823	5\\
824	5\\
825	5\\
826	5\\
827	5\\
828	5\\
829	5\\
830	5\\
831	5\\
832	5\\
833	5\\
834	5\\
835	5\\
836	5\\
837	5\\
838	5\\
839	5\\
840	5\\
841	5\\
842	5\\
843	5\\
844	5\\
845	5\\
846	5\\
847	5\\
848	5\\
849	5\\
850	5\\
851	5\\
852	5\\
853	5\\
854	5\\
855	5\\
856	5\\
857	5\\
858	5\\
859	5\\
860	5\\
861	5\\
862	5\\
863	5\\
864	5\\
865	5\\
866	5\\
867	5\\
868	5\\
869	5\\
870	5\\
871	5\\
872	5\\
873	5\\
874	5\\
875	5\\
876	5\\
877	5\\
878	5\\
879	5\\
880	5\\
881	5\\
882	5\\
883	5\\
884	5\\
885	5\\
886	5\\
887	5\\
888	5\\
889	5\\
890	5\\
891	5\\
892	5\\
893	5\\
894	5\\
895	5\\
896	5\\
897	5\\
898	5\\
899	5\\
900	5\\
901	5\\
902	5\\
903	5\\
904	5\\
905	5\\
906	5\\
907	5\\
908	5\\
909	5\\
910	5\\
911	5\\
912	5\\
913	5\\
914	5\\
915	5\\
916	5\\
917	5\\
918	5\\
919	5\\
920	5\\
921	5\\
922	5\\
923	5\\
924	5\\
925	5\\
926	5\\
927	5\\
928	5\\
929	5\\
930	5\\
931	5\\
932	5\\
933	5\\
934	5\\
935	5\\
936	5\\
937	5\\
938	5\\
939	5\\
940	5\\
941	5\\
942	5\\
943	5\\
944	5\\
945	5\\
946	5\\
947	5\\
948	5\\
949	5\\
950	5\\
951	5\\
952	5\\
953	5\\
954	5\\
955	5\\
956	5\\
957	5\\
958	5\\
959	5\\
960	5\\
961	5\\
962	5\\
963	5\\
964	5\\
965	5\\
966	5\\
967	5\\
968	5\\
969	5\\
970	5\\
971	5\\
972	5\\
973	5\\
974	5\\
975	5\\
976	5\\
977	5\\
978	5\\
979	5\\
980	5\\
981	5\\
982	5\\
983	5\\
984	5\\
985	5\\
986	5\\
987	5\\
988	5\\
989	5\\
990	5\\
991	5\\
992	5\\
993	5\\
994	5\\
995	5\\
996	5\\
997	5\\
998	5\\
999	5\\
1000	5\\
1001	5\\
1002	5\\
1003	5\\
1004	5\\
1005	5\\
1006	5\\
1007	5\\
1008	5\\
1009	5\\
1010	5\\
1011	5\\
1012	5\\
1013	5\\
1014	5\\
1015	5\\
1016	5\\
1017	5\\
1018	5\\
1019	5\\
1020	5\\
1021	5\\
1022	5\\
1023	5\\
1024	5\\
1025	5\\
1026	5\\
1027	5\\
1028	5\\
1029	5\\
1030	5\\
1031	5\\
1032	5\\
1033	5\\
1034	5\\
1035	5\\
1036	5\\
1037	5\\
1038	5\\
1039	5\\
1040	5\\
1041	5\\
1042	5\\
1043	5\\
1044	5\\
1045	5\\
1046	5\\
1047	5\\
1048	5\\
1049	5\\
1050	5\\
1051	5\\
1052	5\\
1053	5\\
1054	5\\
1055	5\\
1056	5\\
1057	5\\
1058	5\\
1059	5\\
1060	5\\
1061	5\\
1062	5\\
1063	5\\
1064	5\\
1065	5\\
1066	5\\
1067	5\\
1068	5\\
1069	5\\
1070	5\\
1071	5\\
1072	5\\
1073	5\\
1074	5\\
1075	5\\
1076	5\\
1077	5\\
1078	5\\
1079	5\\
1080	5\\
1081	5\\
1082	5\\
1083	5\\
1084	5\\
1085	5\\
1086	5\\
1087	5\\
1088	5\\
1089	5\\
1090	5\\
1091	5\\
1092	5\\
1093	5\\
1094	5\\
1095	5\\
1096	5\\
1097	5\\
1098	5\\
1099	5\\
1100	5\\
1101	5\\
1102	5\\
1103	5\\
1104	5\\
1105	5\\
1106	5\\
1107	5\\
1108	5\\
1109	5\\
1110	5\\
1111	5\\
1112	5\\
1113	5\\
1114	5\\
1115	5\\
1116	5\\
1117	5\\
1118	5\\
1119	5\\
1120	5\\
1121	5\\
1122	5\\
1123	5\\
1124	5\\
1125	5\\
1126	5\\
1127	5\\
1128	5\\
1129	5\\
1130	5\\
1131	5\\
1132	5\\
1133	5\\
1134	5\\
1135	5\\
1136	5\\
1137	5\\
1138	5\\
1139	5\\
1140	5\\
1141	5\\
1142	5\\
1143	5\\
1144	5\\
1145	5\\
1146	5\\
1147	5\\
1148	5\\
1149	5\\
1150	5\\
1151	5\\
1152	5\\
1153	5\\
1154	5\\
1155	5\\
1156	5\\
1157	5\\
1158	5\\
1159	5\\
1160	5\\
1161	5\\
1162	5\\
1163	5\\
1164	5\\
1165	5\\
1166	5\\
1167	5\\
1168	5\\
1169	5\\
1170	5\\
1171	5\\
1172	5\\
1173	5\\
1174	5\\
1175	5\\
1176	5\\
1177	5\\
1178	5\\
1179	5\\
1180	5\\
1181	5\\
1182	5\\
1183	5\\
1184	5\\
1185	5\\
1186	5\\
1187	5\\
1188	5\\
1189	5\\
1190	5\\
1191	5\\
1192	5\\
1193	5\\
1194	5\\
1195	5\\
1196	5\\
1197	5\\
1198	5\\
1199	5\\
1200	5\\
};
\addlegendentry{$y^{zad}(k)$}

\end{axis}
\end{tikzpicture}%
	\caption{Regulator PID - sygna� wyj�ciowy i zadany}
	\label{pid_wy}
\end{figure}

K = 0.11; 
Ti = 4.6;
Td = 0.65;

Error = 403.7229

\section{Regulator DMC}

\begin{figure}[h!]
	\centering
	% This file was created by matlab2tikz.
%
\definecolor{mycolor1}{rgb}{0.00000,0.44700,0.74100}%
%
\begin{tikzpicture}

\begin{axis}[%
width=4.521in,
height=3.566in,
at={(0.758in,0.481in)},
scale only axis,
xmin=0,
xmax=1200,
ymin=-0.4,
ymax=1,
axis background/.style={fill=white},
title style={font=\bfseries},
title={Sygna� wej�ciowy},
xmajorgrids,
ymajorgrids,
legend style={legend cell align=left, align=left, draw=white!15!black}
]
\addplot[const plot, color=mycolor1] table[row sep=crcr] {%
1	0\\
2	0\\
3	0\\
4	0\\
5	0\\
6	0\\
7	0\\
8	0\\
9	0\\
10	0\\
11	0\\
12	0\\
13	0\\
14	0\\
15	0\\
16	0\\
17	0\\
18	0\\
19	0\\
20	0\\
21	0\\
22	0\\
23	0\\
24	0\\
25	0\\
26	0\\
27	0\\
28	0\\
29	0\\
30	0\\
31	0\\
32	0\\
33	0\\
34	0\\
35	0\\
36	0\\
37	0\\
38	0\\
39	0\\
40	0\\
41	0\\
42	0\\
43	0\\
44	0\\
45	0\\
46	0\\
47	0\\
48	0\\
49	0\\
50	0\\
51	0\\
52	0\\
53	0\\
54	0\\
55	0\\
56	0\\
57	0\\
58	0\\
59	0\\
60	0\\
61	0\\
62	0\\
63	0\\
64	0\\
65	0\\
66	0\\
67	0\\
68	0\\
69	0\\
70	0\\
71	0\\
72	0\\
73	0\\
74	0\\
75	0\\
76	0\\
77	0\\
78	0\\
79	0\\
80	0\\
81	0\\
82	0\\
83	0\\
84	0\\
85	0\\
86	0\\
87	0\\
88	0\\
89	0\\
90	0\\
91	0\\
92	0\\
93	0\\
94	0\\
95	0\\
96	0\\
97	0\\
98	0\\
99	0\\
100	0.271608910960138\\
101	0.304880006344421\\
102	0.233744609790547\\
103	0.143566177431679\\
104	0.0758910909592878\\
105	0.0438935983283275\\
106	0.0465445160954397\\
107	0.0779241310362272\\
108	0.129235523792262\\
109	0.185475565566606\\
110	0.228445231607837\\
111	0.246361826759644\\
112	0.239361774645266\\
113	0.216923276482842\\
114	0.191341782389126\\
115	0.172429832484453\\
116	0.165504857844794\\
117	0.171703069919188\\
118	0.188865278609023\\
119	0.212238684705006\\
120	0.235533267215669\\
121	0.252848953056968\\
122	0.260762980439574\\
123	0.259310858666829\\
124	0.251404272508386\\
125	0.241291707899799\\
126	0.232996511308167\\
127	0.229300917171883\\
128	0.231336206733494\\
129	0.238592479376439\\
130	0.249209848407211\\
131	0.260536650266524\\
132	0.269913861857778\\
133	0.275473556972248\\
134	0.276643646350157\\
135	0.274160916074511\\
136	0.269662608557182\\
137	0.265080692024238\\
138	0.262077179675435\\
139	0.26166199225267\\
140	0.264031074984499\\
141	0.268608702330971\\
142	0.274264332334984\\
143	0.279657694127291\\
144	0.283627336693347\\
145	0.285506170134811\\
146	0.28526377458734\\
147	0.283443112955042\\
148	0.280938753014115\\
149	0.278709607613465\\
150	0.277517462803658\\
151	0.27775430450321\\
152	0.279384440300557\\
153	0.281997011052777\\
154	0.284969506791977\\
155	0.287591525323457\\
156	0.289312038898044\\
157	0.289882936929202\\
158	0.289391397155163\\
159	0.28819319275738\\
160	0.286786531649499\\
161	0.285665054031009\\
162	0.285189226301114\\
163	0.285505963545187\\
164	0.286531591259898\\
165	0.287996502549644\\
166	0.289534768547357\\
167	0.29079163805896\\
168	0.291518574458715\\
169	0.291630540629165\\
170	0.291212813807017\\
171	0.290480530082045\\
172	0.289707513977544\\
173	0.289147398162689\\
174	0.288968833710998\\
175	0.289219874813142\\
176	0.289827412273128\\
177	0.290628316415925\\
178	0.291421428710418\\
179	0.29202500822099\\
180	0.292323766071367\\
181	0.292293540622091\\
182	0.291998887872569\\
183	0.291566959742596\\
184	0.291147310243731\\
185	0.290869970258735\\
186	0.290813059537393\\
187	0.290987355365515\\
188	0.291340077652004\\
189	0.291775052372979\\
190	0.292182481986536\\
191	0.292469588865916\\
192	0.292583841920752\\
193	0.292523163415427\\
194	0.29233158614558\\
195	0.292082983447428\\
196	0.291858475758204\\
197	0.291724201143505\\
198	0.291715290346283\\
199	0.291829623115527\\
200	0.292032033205116\\
201	0.29226685690996\\
202	0.292474734973797\\
203	0.292608811389172\\
204	0.29264605287661\\
205	0.292591108386127\\
206	0.292472390189149\\
207	0.292332196108975\\
208	0.292214113971159\\
209	0.292151327415225\\
210	0.292158839315405\\
211	0.292231274750184\\
212	0.292346296019811\\
213	0.292472188609252\\
214	0.292577214100134\\
215	0.292638075908106\\
216	0.292645321720568\\
217	0.292604530046604\\
218	0.292533371825282\\
219	0.292455737160898\\
220	0.292394783443064\\
221	0.292366860093051\\
222	0.292377834633465\\
223	0.292422557739981\\
224	0.292487304346595\\
225	0.292554257614114\\
226	0.292606649771172\\
227	0.29263312910721\\
228	0.292630263062477\\
229	0.292602689021633\\
230	0.292561100784519\\
231	0.292518818872497\\
232	0.292487996617864\\
233	0.292476506323187\\
234	0.292486265701121\\
235	0.292513308397127\\
236	0.292549410984925\\
237	0.292584694412929\\
238	0.29261041476544\\
239	0.292621182516244\\
240	0.292616074233777\\
241	0.292598447552743\\
242	0.292574634526338\\
243	0.292551970487174\\
244	0.292536747687467\\
245	0.292532645089046\\
246	0.292540005136304\\
247	0.292556067383712\\
248	0.292576004301105\\
249	0.292594406530578\\
250	0.292606779300942\\
251	0.292610650769109\\
252	0.292606034742933\\
253	0.292595187080676\\
254	0.292581789129145\\
255	0.292569831174902\\
256	0.292562522082141\\
257	0.292561512482816\\
258	0.29256660754944\\
259	0.292575998221679\\
260	0.292586899242575\\
261	0.292596385040991\\
262	0.292602182033852\\
263	0.292603210775367\\
264	0.292599757888451\\
265	0.292593267882279\\
266	0.292585847110307\\
267	0.292579639771536\\
268	0.292576254343862\\
269	0.292576388073798\\
270	0.29257973015213\\
271	0.292585142313412\\
272	0.29259104198225\\
273	0.292595866480912\\
274	0.292598486990159\\
275	0.292598467005255\\
276	0.29259611158187\\
277	0.2925923145383\\
278	0.29258826384907\\
279	0.292585097347821\\
280	0.292583605129806\\
281	0.292584053266801\\
282	0.292586163952898\\
283	0.292589242206064\\
284	0.292592401240707\\
285	0.292594817021537\\
286	0.292595941468765\\
287	0.292595621605793\\
288	0.292594102064433\\
289	0.292591921582406\\
290	0.292589741298991\\
291	0.292588157130495\\
292	0.292587547670113\\
293	0.292587994619724\\
294	0.292589289885699\\
295	0.292591019056438\\
296	0.292592691663478\\
297	0.292593879051412\\
298	0.292594322449821\\
299	0.292593985387444\\
300	0.564201952858439\\
301	0.597471816222266\\
302	0.526335262421144\\
303	0.436156055309443\\
304	0.368480752290675\\
305	0.320889531074754\\
306	0.292231196583087\\
307	0.285611918765307\\
308	0.304807014247697\\
309	0.347445120725973\\
310	0.403301859395939\\
311	0.457378491631481\\
312	0.495192031160626\\
313	0.507864973604745\\
314	0.494915731272062\\
315	0.463258890667967\\
316	0.423478359600749\\
317	0.385956730088587\\
318	0.358577497550019\\
319	0.346080952846186\\
320	0.350193590636683\\
321	0.369655731730651\\
322	0.400006652592385\\
323	0.433877328484123\\
324	0.462628154291133\\
325	0.479075335682073\\
326	0.47989658160132\\
327	0.46639677789022\\
328	0.443461134379499\\
329	0.417558710998058\\
330	0.394854423988286\\
331	0.380007571040479\\
332	0.375641894302692\\
333	0.382174205154913\\
334	0.397782042558656\\
335	0.418596385977778\\
336	0.439427250661523\\
337	0.455097907124439\\
338	0.461971179759316\\
339	0.458977237603912\\
340	0.447698260150194\\
341	0.431597160042817\\
342	0.414856369081778\\
343	0.401308903711551\\
344	0.393710879485171\\
345	0.393358902969297\\
346	0.399956639187096\\
347	0.411693136045763\\
348	0.425596614195027\\
349	0.438218510598106\\
350	0.446535806289439\\
351	0.44877309836596\\
352	0.444822860885773\\
353	0.436125223892085\\
354	0.42514348591643\\
355	0.414580562451691\\
356	0.406790214630326\\
357	0.403333984871334\\
358	0.4047318791394\\
359	0.41040321971459\\
360	0.418815771268907\\
361	0.427843318833558\\
362	0.435288363833098\\
363	0.439444542288886\\
364	0.439521219743078\\
365	0.435788720729022\\
366	0.429416708273571\\
367	0.422095743018429\\
368	0.415586145476984\\
369	0.411319026601287\\
370	0.410119488633073\\
371	0.412075462463819\\
372	0.416555619641361\\
373	0.422373639027639\\
374	0.428078997475895\\
375	0.432317119124511\\
376	0.434164116969281\\
377	0.433336378144511\\
378	0.430216348933445\\
379	0.425702627065846\\
380	0.420948033849008\\
381	0.41706960134768\\
382	0.414901245151027\\
383	0.414832199674602\\
384	0.416749987696016\\
385	0.42009101482558\\
386	0.423988432406328\\
387	0.427488341541182\\
388	0.429784045203168\\
389	0.430406454879313\\
390	0.429318775827044\\
391	0.426894439516038\\
392	0.42379465030305\\
393	0.420788691865783\\
394	0.41856773515116\\
395	0.417594396433363\\
396	0.41801484608052\\
397	0.419645214523457\\
398	0.422031108870171\\
399	0.424566305907751\\
400	0.426643508348325\\
401	0.427800197373757\\
402	0.427822163436927\\
403	0.426779214398796\\
404	0.42498815832664\\
405	0.42291893867688\\
406	0.421072909701929\\
407	0.419864789805866\\
408	0.419534084233288\\
409	0.420102077260773\\
410	0.421380133796185\\
411	0.423025022625837\\
412	0.424627514117302\\
413	0.425812804120302\\
414	0.426327975598951\\
415	0.426095052727739\\
416	0.425218054480788\\
417	0.423945517942813\\
418	0.422601499037123\\
419	0.421504491933403\\
420	0.420894135941089\\
421	0.420881394614081\\
422	0.421431271941348\\
423	0.422379755369261\\
424	0.423479504922776\\
425	0.424462678033613\\
426	0.425105424085953\\
427	0.425278361929206\\
428	0.424971325112042\\
429	0.424287838604181\\
430	0.423412857601041\\
431	0.422563686029842\\
432	0.421937058585987\\
433	0.421664888314817\\
434	0.421787993282659\\
435	0.422252415953114\\
436	0.422927783473743\\
437	0.423642367363293\\
438	0.424225905910045\\
439	0.424549683155462\\
440	0.424554367647204\\
441	0.424259563953216\\
442	0.423753945302834\\
443	0.42316972993943\\
444	0.42264875636535\\
445	0.422308745709146\\
446	0.422217512556974\\
447	0.422380434213904\\
448	0.422743149337458\\
449	0.423207952243203\\
450	0.42365933851365\\
451	0.423992240377251\\
452	0.424136121491818\\
453	0.424069410370681\\
454	0.423821404157679\\
455	0.423462016744261\\
456	0.423082664554475\\
457	0.422773438815954\\
458	0.422602159492851\\
459	0.422600018260725\\
460	0.422756662610608\\
461	0.423025244591242\\
462	0.423335665356458\\
463	0.423612459300549\\
464	0.42379285609823\\
465	0.423840739417721\\
466	0.423753433211885\\
467	0.423560156309409\\
468	0.423313067078706\\
469	0.423073524566609\\
470	0.422897125282841\\
471	0.422821088862716\\
472	0.422856757210629\\
473	0.422988604797817\\
474	0.423179576229761\\
475	0.423381123279736\\
476	0.423545314806202\\
477	0.423636050397848\\
478	0.423636785380333\\
479	0.423553157062465\\
480	0.423410219734564\\
481	0.423245299558945\\
482	0.423098446916822\\
483	0.423002885878168\\
484	0.422977688211264\\
485	0.423024227610553\\
486	0.423126992714544\\
487	0.423258294102951\\
488	0.423385525795846\\
489	0.4234791241567\\
490	0.423519310997573\\
491	0.423500107892665\\
492	0.423429848266568\\
493	0.423328293962069\\
494	0.423221259291115\\
495	0.423134170688714\\
496	0.423086140133251\\
497	0.42308590157155\\
498	0.423130435539987\\
499	0.423206429533358\\
500	-0.174245550348479\\
501	-0.247363989555006\\
502	-0.0908154544144246\\
503	0.107590335114895\\
504	0.256450648799475\\
505	0.425727238978213\\
506	0.509145414237166\\
507	0.459753702913273\\
508	0.325759663983721\\
509	0.176237097069495\\
510	0.0590546724313124\\
511	-0.00751874075304287\\
512	-0.0233514338414766\\
513	0.00751064915956598\\
514	0.0793959372725004\\
515	0.170470660295219\\
516	0.241511795506509\\
517	0.262066517793305\\
518	0.22971794654976\\
519	0.165653713112203\\
520	0.097716200103104\\
521	0.0465433776039063\\
522	0.0216161270192322\\
523	0.0238138748085138\\
524	0.04827008726178\\
525	0.084506818887405\\
526	0.117460579187908\\
527	0.133538477853578\\
528	0.127543720332547\\
529	0.104192791081688\\
530	0.0739847272501344\\
531	0.0474736387481421\\
532	0.0314986823523363\\
533	0.0281608184268729\\
534	0.0353980580044841\\
535	0.0481307438121266\\
536	0.0599631056102644\\
537	0.06555853203243\\
538	0.0627152167391168\\
539	0.0528685908073296\\
540	0.0398584037553596\\
541	0.0279320742064401\\
542	0.0200987147295524\\
543	0.0173667509013313\\
544	0.0188134828728324\\
545	0.0222238675948646\\
546	0.0250482221016389\\
547	0.0253825981723245\\
548	0.0226001598328966\\
549	0.0173834972660065\\
550	0.0112217271787022\\
551	0.00568486711716915\\
552	0.00181915225251616\\
553	-0.000132704241077135\\
554	-0.000712939334804229\\
555	-0.000781502449028908\\
556	-0.00122669372148321\\
557	-0.00263581787235245\\
558	-0.00511400078622649\\
559	-0.00831788480792534\\
560	-0.0116880364085612\\
561	-0.0147063644466128\\
562	-0.0170780729356474\\
563	-0.0187964706426714\\
564	-0.020090681693178\\
565	-0.0212952505458322\\
566	-0.0227030435395258\\
567	-0.0244604846500332\\
568	-0.0265392835325307\\
569	-0.028783219200504\\
570	-0.0309964080609458\\
571	-0.0330277309977009\\
572	-0.0348190716665982\\
573	-0.0364076240786227\\
574	-0.0378917540233869\\
575	-0.0393807916054472\\
576	-0.0409507798895632\\
577	-0.0426219198797829\\
578	-0.0443623887112558\\
579	-0.0461120630691303\\
580	-0.0478127206235587\\
581	-0.0494309926476679\\
582	-0.0509662280964843\\
583	-0.0524435276389347\\
584	-0.0538981514315897\\
585	-0.0553594180657081\\
586	-0.0568405203257564\\
587	-0.0583369630311892\\
588	-0.0598324046881955\\
589	-0.0613080206576399\\
590	-0.0627508631757236\\
591	-0.0641579862066795\\
592	-0.0655355239811995\\
593	-0.0668942153160036\\
594	-0.0682440333191885\\
595	-0.0695903470185669\\
596	-0.0709328501876119\\
597	-0.0722670762195151\\
598	-0.0735873022198167\\
599	-0.0748893376774059\\
600	-0.0761720588707548\\
601	-0.077437309405426\\
602	-0.0786885562886898\\
603	-0.0799291414613634\\
604	-0.0811609644837275\\
605	-0.0823840680019299\\
606	-0.0835971089787345\\
607	-0.0847983231832316\\
608	-0.0859864682535948\\
609	-0.0871613304384486\\
610	-0.0883236850301453\\
611	-0.0894748358491924\\
612	-0.0906160150926744\\
613	-0.0917479319781276\\
614	-0.0928706399309439\\
615	-0.0939837154445929\\
616	-0.0950866073578459\\
617	-0.0961789710799821\\
618	-0.0972608460518824\\
619	-0.0983326301716403\\
620	-0.0993949015429531\\
621	-0.10044819424584\\
622	-0.101492834385404\\
623	-0.102528896375234\\
624	-0.103556275721909\\
625	-0.104574824496037\\
626	-0.105584478752204\\
627	-0.106585324326282\\
628	-0.10757758440216\\
629	-0.108561549122808\\
630	-0.109537488545377\\
631	-0.110505589525841\\
632	-0.111465939158027\\
633	-0.11241855314501\\
634	-0.113363428494909\\
635	-0.114300593354103\\
636	-0.115230133153058\\
637	-0.116152186379477\\
638	-0.117066917613094\\
639	-0.11797448373752\\
640	-0.118875009098275\\
641	-0.11976857850974\\
642	-0.120655247674985\\
643	-0.121535063237949\\
644	-0.122408082080883\\
645	-0.123274381805696\\
646	-0.124134059656767\\
647	-0.124987222608924\\
648	-0.12583397458839\\
649	-0.12667440685443\\
650	-0.127508595038616\\
651	-0.128336602808842\\
652	-0.129158489303988\\
653	-0.129974316439035\\
654	-0.130784153005676\\
655	-0.131588074468345\\
656	-0.132386159410571\\
657	-0.133178484827927\\
658	-0.133965122520668\\
659	-0.134746137920651\\
660	-0.135521591379469\\
661	-0.136291540885484\\
662	-0.137056044771962\\
663	-0.137815163268467\\
664	-0.138568958474529\\
665	-0.139317493094793\\
666	-0.140060828737089\\
667	-0.140799024601144\\
668	-0.141532137051789\\
669	-0.14226022009108\\
670	-0.142983326354081\\
671	-0.143701508103143\\
672	-0.144414817801647\\
673	-0.14512330811336\\
674	-0.145827031451572\\
675	-0.146526039370937\\
676	-0.147220382103533\\
677	-0.147910108417667\\
678	-0.148595265802571\\
679	-0.149275900840675\\
680	-0.149952059575718\\
681	-0.150623787725145\\
682	-0.151291130682809\\
683	-0.151954133359345\\
684	-0.152612839968108\\
685	-0.153267293866271\\
686	-0.15391753751468\\
687	-0.154563612555739\\
688	-0.155205559957532\\
689	-0.15584342015393\\
690	-0.156477233126121\\
691	-0.157107038407154\\
692	-0.157732875028015\\
693	-0.158354781445245\\
694	-0.158972795490008\\
695	-0.159586954361086\\
696	-0.160197294660656\\
697	-0.160803852453267\\
698	-0.16140666332217\\
699	-0.162005762403389\\
700	1\\
701	1\\
702	1\\
703	1\\
704	1\\
705	1\\
706	1\\
707	1\\
708	1\\
709	0.9398667517128\\
710	0.841439516175099\\
711	0.775173738341158\\
712	0.756383552345406\\
713	0.773623864624258\\
714	0.813739226664772\\
715	0.871425952170987\\
716	0.941387623613947\\
717	1\\
718	1\\
719	1\\
720	1\\
721	1\\
722	1\\
723	1\\
724	1\\
725	0.998828294043221\\
726	0.960620919761496\\
727	0.918790195673961\\
728	0.888889666120585\\
729	0.874199403897741\\
730	0.87169197779149\\
731	0.880437731035768\\
732	0.899999537877643\\
733	0.927736314348617\\
734	0.958421170970729\\
735	0.986010730952504\\
736	1\\
737	1\\
738	1\\
739	0.998217581105041\\
740	0.979816753547418\\
741	0.953938550119128\\
742	0.928908763443949\\
743	0.90958660702363\\
744	0.897555636711213\\
745	0.893820102530574\\
746	0.89886255409845\\
747	0.911893854122536\\
748	0.930534372300105\\
749	0.951292249320807\\
750	0.970345680653414\\
751	0.984305202212335\\
752	0.990854585807635\\
753	0.989174403286987\\
754	0.980120070930699\\
755	0.965637669030918\\
756	0.948546332004465\\
757	0.931905396144884\\
758	0.918470291174189\\
759	0.910288932433824\\
760	0.908470436232074\\
761	0.91307504075303\\
762	0.923117705559713\\
763	0.936705472026775\\
764	0.951350037984179\\
765	0.964418831020532\\
766	0.973633865545873\\
767	0.977509524478111\\
768	0.975615642885484\\
769	0.968599814633136\\
770	0.957981261355369\\
771	0.945795765803875\\
772	0.934195378950244\\
773	0.925096890884778\\
774	0.919898352540696\\
775	0.919297335270997\\
776	0.923212998473491\\
777	0.93081535015656\\
778	0.940665846100796\\
779	0.950965336891285\\
780	0.959877119326002\\
781	0.965876063526053\\
782	0.968028772676419\\
783	0.966143632734549\\
784	0.960762544300297\\
785	0.953010407727563\\
786	0.944350380082168\\
787	0.936306678952715\\
788	0.93020883024813\\
789	0.926992307653387\\
790	0.927072316544606\\
791	0.930299556860504\\
792	0.936000320379451\\
793	0.943096472736252\\
794	0.950296427459902\\
795	0.956331874509395\\
796	0.96018960830338\\
797	0.961289696878027\\
798	0.959572492366641\\
799	0.955480361628269\\
800	0.9498459051347\\
801	0.943718857119799\\
802	0.938171696917836\\
803	0.934119810497368\\
804	0.932181537182608\\
805	0.932592951556602\\
806	0.935184701117013\\
807	0.939422645141443\\
808	0.944507580377792\\
809	0.949520230305133\\
810	0.953587460165873\\
811	0.956038188796224\\
812	0.956517417928254\\
813	0.955035819947481\\
814	0.951947939984273\\
815	0.947868703008395\\
816	0.943549993227991\\
817	0.939743728402537\\
818	0.937075760199596\\
819	0.935949017640896\\
820	0.936487597798419\\
821	0.938527505488162\\
822	0.941654219336061\\
823	0.945281272496992\\
824	0.948757343411735\\
825	0.951483208759414\\
826	0.953016656735869\\
827	0.953145144172222\\
828	0.951912924549633\\
829	0.94959969012578\\
830	0.946658182302103\\
831	0.943625683475274\\
832	0.941027317780193\\
833	0.939288022693781\\
834	0.938666425666564\\
835	0.939219249286085\\
836	0.940800146765774\\
837	0.943092153948142\\
838	0.945668107201447\\
839	0.9480686984587\\
840	0.949884268678395\\
841	0.950825261144819\\
842	0.950768337732585\\
843	0.949770316705123\\
844	0.948048974773665\\
845	0.945936414337117\\
846	0.943815418204192\\
847	0.942051178782125\\
848	0.940930177695698\\
849	0.940615602613331\\
850	0.941125385321565\\
851	0.942335297344478\\
852	0.944005798803779\\
853	0.945827690622345\\
854	0.947478463232747\\
855	0.948679219965537\\
856	0.949241857538788\\
857	0.949098125737374\\
858	0.948305941687306\\
859	0.94703297352212\\
860	0.945521844105721\\
861	0.944044345746701\\
862	0.94285332766372\\
863	0.942140501385545\\
864	0.942006738441258\\
865	0.942449015431351\\
866	0.943365392370504\\
867	0.944576572283175\\
868	0.945859953801335\\
869	0.946990010655981\\
870	0.94777773690054\\
871	0.94810212737466\\
872	0.947928287547732\\
873	0.94730947192141\\
874	0.946373503227558\\
875	0.94529688695181\\
876	0.944271913156279\\
877	0.943472844183122\\
878	0.943026957534759\\
879	0.942994995154599\\
880	0.943363780193992\\
881	0.944051693604725\\
882	0.944925612737658\\
883	0.945826059143944\\
884	0.946595961689205\\
885	0.947107879488302\\
886	0.94728490995878\\
887	0.947111797733998\\
888	0.946634698999061\\
889	0.945950212921267\\
890	0.945186199141957\\
891	0.944478191648728\\
892	0.943945716735468\\
893	0.943672541241335\\
894	0.94369396734483\\
895	0.943992962998674\\
896	0.944505392814875\\
897	0.945133106525441\\
898	0.94576236191151\\
899	0.946284208636855\\
900	0.9466132000508\\
901	0.946701200508362\\
902	0.946544050189908\\
903	0.946180234515694\\
904	0.945682187892789\\
905	0.945142139639644\\
906	0.944655252493438\\
907	0.944303098400907\\
908	0.944140270548635\\
909	0.944186243144087\\
910	0.944423608881229\\
911	0.944802713912101\\
912	0.945251639226088\\
913	0.945689606743039\\
914	0.946041358714641\\
915	0.946249965479019\\
916	0.946285880630577\\
917	0.946150814652986\\
918	0.945875985338532\\
919	0.945515325247324\\
920	0.945135085104681\\
921	0.944801819076658\\
922	0.944570901615948\\
923	0.94447751282933\\
924	0.944531508448741\\
925	0.944716867825661\\
926	0.944995612760818\\
927	0.945315338290422\\
928	0.945618911693699\\
929	0.94585457278145\\
930	0.94598466150908\\
931	0.945991507124879\\
932	0.945879574527621\\
933	0.945673665807771\\
934	0.945413680157558\\
935	0.945147012188397\\
936	0.944920020881574\\
937	0.944770083608139\\
938	0.944719568950591\\
939	0.944772668176182\\
940	0.944915496046942\\
941	0.945119299032218\\
942	0.945346086420424\\
943	0.945555611183282\\
944	0.945712435299002\\
945	0.945791848533298\\
946	0.945783659980914\\
947	0.945693297043895\\
948	0.945540145079121\\
949	0.945353547505365\\
950	0.945167272922719\\
951	0.945013479859863\\
952	0.944917242925376\\
953	0.944892553859838\\
954	0.944940414464389\\
955	0.945049253076293\\
956	0.945197489629062\\
957	0.94535771400489\\
958	0.945501686834614\\
959	0.945605261666309\\
960	0.945652377924711\\
961	0.945637471992462\\
962	0.945565958856428\\
963	0.945452789880569\\
964	0.945319427524437\\
965	0.945189836187753\\
966	0.945086228686508\\
967	0.945025313034303\\
968	0.945015661697536\\
969	0.945056603521658\\
970	0.945138759509632\\
971	0.945246056687431\\
972	0.94535880763705\\
973	0.945457277065475\\
974	0.945525096827322\\
975	0.945551944187655\\
976	0.945535051662165\\
977	0.945479339575937\\
978	0.945396211956028\\
979	0.945301286863566\\
980	0.945211503977757\\
981	0.945142138328635\\
982	0.945104239523725\\
983	0.945102917712492\\
984	0.945136732363876\\
985	0.945198238943179\\
986	0.945275547236138\\
987	0.945354577378341\\
988	0.945421592838087\\
989	0.945465559818779\\
990	0.945479932208932\\
991	0.945463578670182\\
992	0.945420730321231\\
993	0.945360003232798\\
994	0.945292707938042\\
995	0.945230771542333\\
996	0.94518464937849\\
997	0.945161586964329\\
998	0.945164515670488\\
999	0.945191743338529\\
1000	0.945237457025505\\
1001	0.945292914523448\\
1002	0.94534808806017\\
1003	0.945393455902774\\
1004	0.94542162534753\\
1005	0.945428513764752\\
1006	0.945413903228984\\
1007	0.945381301316363\\
1008	0.945337164096173\\
1009	0.94528964524366\\
1010	0.945247109479692\\
1011	0.945216678008774\\
1012	0.945203055501954\\
1013	0.945207827999156\\
1014	0.945229331263316\\
1015	0.945263086343386\\
1016	0.945302701510408\\
1017	0.945341063780464\\
1018	0.94537160106787\\
1019	0.945389393519853\\
1020	0.945391948582098\\
1021	0.945379520958638\\
1022	0.945354942906358\\
1023	0.94532301672338\\
1024	0.945289594671005\\
1025	0.945260519773822\\
1026	0.945240616862963\\
1027	0.945232905724065\\
1028	0.94523816194465\\
1029	0.945254885347177\\
1030	0.945279662903347\\
1031	0.945307845580354\\
1032	0.945334407998911\\
1033	0.945354834067833\\
1034	0.945365874293354\\
1035	0.945366049608719\\
1036	0.945355826120648\\
1037	0.945337445597162\\
1038	0.945314456808832\\
1039	0.945291042501363\\
1040	0.945271267708018\\
1041	0.945258382859558\\
1042	0.945254299506309\\
1043	0.945259321252115\\
1044	0.945272164690756\\
1045	0.945290253521949\\
1046	0.945310222641147\\
1047	0.945328535590257\\
1048	0.945342103518511\\
1049	0.945348798582286\\
1050	0.945347777813956\\
1051	0.945339570099736\\
1052	0.94532592206085\\
1053	0.94530944053487\\
1054	0.945293102789286\\
1055	0.945279725172963\\
1056	0.945271483898789\\
1057	0.945269568328456\\
1058	0.945274020566166\\
1059	0.945283780552344\\
1060	0.945296919505185\\
1061	0.945311012817945\\
1062	0.945323581645397\\
1063	0.945332523742804\\
1064	0.945336459573714\\
1065	0.945334937712286\\
1066	0.945328470451297\\
1067	0.945318401224589\\
1068	0.945306634463507\\
1069	0.945295280880709\\
1070	0.945286283349168\\
1071	0.945281088895361\\
1072	0.945280421349747\\
1073	0.945284189317562\\
1074	0.9452915391863\\
1075	0.945301037401508\\
1076	0.945310944634821\\
1077	0.945319530289406\\
1078	0.945325371155364\\
1079	0.945327583317683\\
1080	0.945325950274184\\
1081	0.94532092985649\\
1082	0.945313544299044\\
1083	0.945305177804082\\
1084	0.945297320821628\\
1085	0.945291307669157\\
1086	0.945288093124487\\
1087	0.945288104800607\\
1088	0.945291193338801\\
1089	0.945296684557727\\
1090	0.945303519899514\\
1091	0.945310456883962\\
1092	0.945316292200178\\
1093	0.945320067851462\\
1094	0.945321225498582\\
1095	0.945319684688889\\
1096	0.945315834926575\\
1097	0.945310446893192\\
1098	0.945304521859662\\
1099	0.945299108140094\\
1100	0.945295117809335\\
1101	0.945293175337224\\
1102	0.945293522829437\\
1103	0.945295995658249\\
1104	0.94530006947945\\
1105	0.945304967255625\\
1106	0.945309805058133\\
1107	0.945313749686148\\
1108	0.945316160326179\\
1109	0.945316690498813\\
1110	0.945315334486566\\
1111	0.945312412754949\\
1112	0.94530850167462\\
1113	0.945304322236199\\
1114	0.945300608863007\\
1115	0.945297981898965\\
1116	0.945296845634171\\
1117	0.945297328318547\\
1118	0.945299272608862\\
1119	0.9453022757969\\
1120	0.94530577059662\\
1121	0.94530913068123\\
1122	0.94531178160498\\
1123	0.945313297696272\\
1124	0.945313468818118\\
1125	0.945312326834553\\
1126	0.945310129049838\\
1127	0.945307303448625\\
1128	0.945304366945928\\
1129	0.945301832006742\\
1130	0.945300118301721\\
1131	0.945299484432567\\
1132	0.945299990606031\\
1133	0.945301497291739\\
1134	0.945303698452729\\
1135	0.945306182029042\\
1136	0.945308505979878\\
1137	0.945310276034646\\
1138	0.945311211639454\\
1139	0.945311189241974\\
1140	0.945310256469543\\
1141	0.945308616080041\\
1142	0.945306583840071\\
1143	0.945304528800618\\
1144	0.945302807094631\\
1145	0.945301700991409\\
1146	0.945301373497811\\
1147	0.945301845639879\\
1148	0.945302999314194\\
1149	0.945304604048732\\
1150	0.945306361957841\\
1151	0.945307962293126\\
1152	0.94530913572611\\
1153	0.945309698994978\\
1154	0.945309582640277\\
1155	0.945308837809104\\
1156	0.945307621908129\\
1157	0.945306166550998\\
1158	0.945304734147715\\
1159	0.945303571156005\\
1160	0.94530286622473\\
1161	0.945302720237956\\
1162	0.945303132891866\\
1163	0.945304007368679\\
1164	0.945305171473595\\
1165	0.945306410830919\\
1166	0.94530750785285\\
1167	0.945308279483801\\
1168	0.945308607254052\\
1169	0.945308454800688\\
1170	0.945307870400589\\
1171	0.945306974762423\\
1172	0.945305936860595\\
1173	0.945304942533249\\
1174	0.945304161600919\\
1175	0.945303719254896\\
1176	0.945303676463934\\
1177	0.945304022372768\\
1178	0.945304679458151\\
1179	0.945305519968946\\
1180	0.945306390294339\\
1181	0.945307138687587\\
1182	0.945307641401583\\
1183	0.945307822792658\\
1184	0.945307666194191\\
1185	0.945307214102682\\
1186	0.945306558132324\\
1187	0.945305820941267\\
1188	0.945305133619375\\
1189	0.945304612651988\\
1190	0.945304340459467\\
1191	0.945304352711685\\
1192	0.94530463429989\\
1193	0.945305124266549\\
1194	0.945305728432256\\
1195	0.945306337186415\\
1196	0.945306845131835\\
1197	0.945307169104106\\
1198	0.94530726152634\\
1199	0.945307117004665\\
1200	0.945306771333532\\
};
\addlegendentry{$u(k)$}

\end{axis}
\end{tikzpicture}%
	\caption{Regulator DMC - sygna� steruj�cy}
	\label{dmc_we}
\end{figure}

\begin{figure}[h!]
	\centering
	% This file was created by matlab2tikz.
%
\definecolor{mycolor1}{rgb}{0.00000,0.44700,0.74100}%
\definecolor{mycolor2}{rgb}{0.85000,0.32500,0.09800}%
%
\begin{tikzpicture}

\begin{axis}[%
width=4.521in,
height=3.566in,
at={(0.758in,0.481in)},
scale only axis,
xmin=0,
xmax=1200,
ymin=-1,
ymax=6,
axis background/.style={fill=white},
title style={font=\bfseries},
title={Sygna� wyj�ciowy i zadany},
xmajorgrids,
ymajorgrids,
legend style={legend cell align=left, align=left, draw=white!15!black}
]
\addplot [color=mycolor1]
  table[row sep=crcr]{%
1	0\\
2	0\\
3	0\\
4	0\\
5	0\\
6	0\\
7	0\\
8	0\\
9	0\\
10	0\\
11	0\\
12	0\\
13	0\\
14	0\\
15	0\\
16	0\\
17	0\\
18	0\\
19	0\\
20	0\\
21	0\\
22	0\\
23	0\\
24	0\\
25	0\\
26	0\\
27	0\\
28	0\\
29	0\\
30	0\\
31	0\\
32	0\\
33	0\\
34	0\\
35	0\\
36	0\\
37	0\\
38	0\\
39	0\\
40	0\\
41	0\\
42	0\\
43	0\\
44	0\\
45	0\\
46	0\\
47	0\\
48	0\\
49	0\\
50	0\\
51	0\\
52	0\\
53	0\\
54	0\\
55	0\\
56	0\\
57	0\\
58	0\\
59	0\\
60	0\\
61	0\\
62	0\\
63	0\\
64	0\\
65	0\\
66	0\\
67	0\\
68	0\\
69	0\\
70	0\\
71	0\\
72	0\\
73	0\\
74	0\\
75	0\\
76	0\\
77	0\\
78	0\\
79	0\\
80	0\\
81	0\\
82	0\\
83	0\\
84	0\\
85	0\\
86	0\\
87	0\\
88	0\\
89	0\\
90	0\\
91	0\\
92	0\\
93	0\\
94	0\\
95	0\\
96	0\\
97	0\\
98	0\\
99	0\\
100	0\\
101	0\\
102	0\\
103	0\\
104	0\\
105	0.0255546180250559\\
106	0.089179822861266\\
107	0.158649189170292\\
108	0.199491307988689\\
109	0.200841210360201\\
110	0.174614773504775\\
111	0.139471315463205\\
112	0.110282789192583\\
113	0.0964578193711743\\
114	0.102896667010028\\
115	0.129507698518275\\
116	0.170219346808983\\
117	0.214198514776656\\
118	0.249891590753321\\
119	0.269843684145842\\
120	0.273266801917211\\
121	0.265044018244447\\
122	0.252656391994882\\
123	0.243296764496907\\
124	0.242130782135911\\
125	0.251471750186699\\
126	0.270527146283781\\
127	0.295757694662264\\
128	0.321986722108447\\
129	0.344071313592871\\
130	0.35854764753592\\
131	0.36457382661784\\
132	0.363842273748182\\
133	0.359670566622778\\
134	0.355789344970869\\
135	0.355279909760357\\
136	0.359876990172354\\
137	0.369685142573476\\
138	0.383303991937412\\
139	0.39832490200554\\
140	0.412080530455148\\
141	0.422432437549104\\
142	0.428345824526071\\
143	0.430074606420593\\
144	0.428937786908346\\
145	0.426819005626939\\
146	0.425586318113868\\
147	0.426602789095128\\
148	0.430430692186628\\
149	0.436768956870236\\
150	0.444614864182144\\
151	0.452596746816846\\
152	0.459382997012717\\
153	0.464049311672288\\
154	0.466299067847866\\
155	0.466483067490298\\
156	0.46543448084437\\
157	0.464190868345021\\
158	0.463696145999733\\
159	0.464568312435129\\
160	0.466967685874922\\
161	0.470596532238675\\
162	0.474823310546293\\
163	0.47888669887292\\
164	0.48211625854889\\
165	0.484107736839722\\
166	0.484808448126537\\
167	0.484497386668944\\
168	0.483676645693965\\
169	0.482914260400757\\
170	0.482686820067257\\
171	0.483263421146251\\
172	0.484656225956467\\
173	0.486643272921561\\
174	0.488850790382294\\
175	0.490868313418082\\
176	0.492363182691277\\
177	0.493163417393802\\
178	0.493289055354536\\
179	0.492928307814718\\
180	0.492370772812907\\
181	0.491920459774818\\
182	0.491814071777438\\
183	0.492165477724752\\
184	0.492948100406044\\
185	0.494016029939252\\
186	0.495154698851698\\
187	0.496145072685539\\
188	0.496823023641302\\
189	0.497118333868734\\
190	0.497064675385903\\
191	0.496780695656945\\
192	0.496430239061771\\
193	0.496174526859638\\
194	0.496129844967767\\
195	0.496341340663131\\
196	0.49677819844639\\
197	0.497349381042283\\
198	0.497933838150163\\
199	0.498415828398337\\
200	0.498715459981054\\
201	0.498806713506667\\
202	0.498719301176433\\
203	0.498525442142097\\
204	0.498316598491634\\
205	0.498177388807887\\
206	0.498163892490929\\
207	0.49829164877645\\
208	0.49853556751164\\
209	0.498840623059207\\
210	0.499139470790827\\
211	0.499371651086055\\
212	0.499499112504264\\
213	0.499514258103947\\
214	0.49943908252773\\
215	0.499316499578411\\
216	0.499196944809015\\
217	0.499124288412286\\
218	0.499124868498019\\
219	0.499202245812986\\
220	0.499338512132508\\
221	0.499501172598806\\
222	0.49965324071747\\
223	0.499763554725854\\
224	0.499814546463513\\
225	0.499805635981421\\
226	0.499751769499096\\
227	0.499677963500775\\
228	0.499611699178811\\
229	0.499575401428606\\
230	0.499580992623106\\
231	0.499627764713861\\
232	0.499703812329649\\
233	0.499790302360206\\
234	0.499867172462783\\
235	0.49991860546378\\
236	0.499936843902514\\
237	0.499923488884257\\
238	0.499888180940002\\
239	0.499845268712827\\
240	0.499809548220106\\
241	0.499792296568862\\
242	0.499798625700167\\
243	0.499826730973948\\
244	0.499869050586498\\
245	0.499914843005816\\
246	0.499953360351182\\
247	0.499976716124486\\
248	0.499981713655409\\
249	0.499970249069544\\
250	0.499948319668755\\
251	0.499924037051207\\
252	0.499905270336867\\
253	0.499897582251628\\
254	0.499902978346165\\
255	0.499919722944307\\
256	0.499943167421825\\
257	0.499967272692548\\
258	0.499986353942027\\
259	0.499996562037359\\
260	0.499996732941602\\
261	0.499988440299594\\
262	0.499975314437652\\
263	0.499961880133973\\
264	0.499952269031519\\
265	0.499949161296957\\
266	0.499953215708146\\
267	0.499963092358727\\
268	0.499976004551028\\
269	0.499988601907912\\
270	0.499997917795882\\
271	0.500002122833289\\
272	0.500000902806716\\
273	0.499995396836604\\
274	0.499987754863359\\
275	0.499980469081626\\
276	0.499975679074943\\
277	0.499974638005058\\
278	0.499977466201986\\
279	0.499983229834332\\
280	0.499990292256492\\
281	0.499996818196123\\
282	0.50000128189213\\
283	0.50000284364536\\
284	0.500001507363906\\
285	0.49999803841237\\
286	0.499993686846716\\
287	0.499989808516462\\
288	0.499987494712628\\
289	0.499987308032478\\
290	0.499989184415045\\
291	0.499992511302556\\
292	0.499996344082685\\
293	0.49999968985689\\
294	0.500001776542581\\
295	0.500002237154689\\
296	0.500001168466794\\
297	0.499999060619534\\
298	0.499996628905084\\
299	0.499994601950163\\
300	0.499993526860643\\
301	0.499993641508923\\
302	0.499994841425163\\
303	0.499996740933476\\
304	0.499998803144922\\
305	0.554261979402173\\
306	0.672009715963775\\
307	0.812321116167983\\
308	0.927314305683796\\
309	0.98829595736509\\
310	0.989381175453351\\
311	0.941709943189057\\
312	0.866588921426706\\
313	0.788491061893066\\
314	0.72880499606715\\
315	0.70215860245955\\
316	0.714975141944849\\
317	0.765010474416171\\
318	0.841768834039199\\
319	0.928461773423488\\
320	1.0057998169816\\
321	1.05702400749615\\
322	1.0727073022996\\
323	1.05334258182321\\
324	1.0084080274874\\
325	0.952558480888262\\
326	0.90107742874781\\
327	0.866369867617273\\
328	0.855942130268596\\
329	0.871514632335339\\
330	0.909004758514017\\
331	0.959461723414419\\
332	1.01106263778726\\
333	1.05195884627025\\
334	1.07332927806911\\
335	1.07168558411281\\
336	1.04955137304\\
337	1.0142768320776\\
338	0.975613174072886\\
339	0.94308235582501\\
340	0.923909000439816\\
341	0.921770165770325\\
342	0.936329652185974\\
343	0.963510082567116\\
344	0.996490240273851\\
345	1.02732686468711\\
346	1.04890055497194\\
347	1.05669012168387\\
348	1.0498287751688\\
349	1.03109526742222\\
350	1.00589979411037\\
351	0.980699365587698\\
352	0.961385848421937\\
353	0.952035832125273\\
354	0.954190162016669\\
355	0.966695975676385\\
356	0.986094017895239\\
357	1.00748699425708\\
358	1.02573334651563\\
359	1.03670625182845\\
360	1.0382668815602\\
361	1.03068481649998\\
362	1.01640510802623\\
363	0.999275183947228\\
364	0.983504798061712\\
365	0.972662072813352\\
366	0.968926794021833\\
367	0.972712167646387\\
368	0.982684651076019\\
369	0.996155443262793\\
370	1.00976073243369\\
371	1.02028383835606\\
372	1.02542250195098\\
373	1.02430234859117\\
374	1.01760322178709\\
375	1.00728394221749\\
376	0.996013028783924\\
377	0.986483732703402\\
378	0.980791683679953\\
379	0.980004682222944\\
380	0.983992429445718\\
381	0.991529163597987\\
382	1.00063446998799\\
383	1.00907274238724\\
384	1.01489461671842\\
385	1.01688824562243\\
386	1.01482819095445\\
387	1.00946597211954\\
388	1.00228126034432\\
389	0.995077250115093\\
390	0.989534774463727\\
391	0.98683335798966\\
392	0.987416388216769\\
393	0.990938001571529\\
394	0.996390420415094\\
395	1.00237514464132\\
396	1.00745193279432\\
397	1.01048161399634\\
398	1.01087980390734\\
399	1.00872212902927\\
400	1.00468264088537\\
401	0.999832386154072\\
402	0.995359041577861\\
403	0.992281547294063\\
404	0.991225989416417\\
405	0.992307644944302\\
406	0.995137118454863\\
407	0.998941356954971\\
408	1.00276647569158\\
409	1.00571218903786\\
410	1.00714121922674\\
411	1.00681429038779\\
412	1.00492155282051\\
413	1.0020090792844\\
414	0.998825793377598\\
415	0.996133815014506\\
416	0.994529823561586\\
417	0.994317663721701\\
418	0.995457117238241\\
419	0.99759508636075\\
420	1.00016710921023\\
421	1.00254217618453\\
422	1.004174985402\\
423	1.00472924346299\\
424	1.00414397864074\\
425	1.00263017045444\\
426	1.00060301421159\\
427	0.998570611859383\\
428	0.997008729596186\\
429	0.996251951963235\\
430	0.996425113026388\\
431	0.997427826874108\\
432	0.998972152021238\\
433	1.00066153207751\\
434	1.00209040502391\\
435	1.00294003279533\\
436	1.00304808880686\\
437	1.00243693241306\\
438	1.00129635918653\\
439	0.999927950233215\\
440	0.998666903749027\\
441	0.997801361560081\\
442	0.997508222309281\\
443	0.997819130931819\\
444	0.998622439625715\\
445	0.999698379549717\\
446	1.00077727921914\\
447	1.00160591179504\\
448	1.00200592580439\\
449	1.00191102015624\\
450	1.00137533820582\\
451	1.00055291831535\\
452	0.999654991232078\\
453	0.998896714070551\\
454	0.998446565378676\\
455	0.998390014557977\\
456	0.9987149298785\\
457	0.999320675717145\\
458	1.000047284242\\
459	1.0007166763318\\
460	1.00117556528134\\
461	1.00132986156618\\
462	1.00116298360242\\
463	1.00073475547476\\
464	1.00016240469681\\
465	0.999589327246769\\
466	0.999149786020664\\
467	0.998938082982961\\
468	0.998989095255465\\
469	0.999273952117297\\
470	0.999710877028446\\
471	1.00018770864652\\
472	1.00059011531282\\
473	1.0008285426374\\
474	1.00085764103432\\
475	1.00068406904758\\
476	1.00036157350531\\
477	0.999975331050286\\
478	0.999619941218858\\
479	0.999376656004565\\
480	0.99929522471199\\
481	0.999384287914374\\
482	0.999612002366166\\
483	0.999916092724733\\
484	1.00022038488659\\
485	1.00045354594776\\
486	1.00056549933974\\
487	1.0005378058496\\
488	1.00038595218724\\
489	1.00015352834381\\
490	0.999900189493347\\
491	0.999686635999708\\
492	0.999560327993607\\
493	0.999545228686255\\
494	0.999637709942508\\
495	0.999809177148922\\
496	1.00001436893791\\
497	1.00020302969059\\
498	1.0003320079014\\
499	1.00037491643266\\
500	1.00032724167764\\
501	1.00020599248116\\
502	1.00004432321477\\
503	0.999882719593564\\
504	0.999759038668175\\
505	0.761348643770406\\
506	0.479270961498893\\
507	0.259941628090773\\
508	0.119444419623205\\
509	0.0602710996751683\\
510	0.0837378177774088\\
511	0.183804441897338\\
512	0.325162665775031\\
513	0.446600036703045\\
514	0.489437124060102\\
515	0.441372296463523\\
516	0.340523931761248\\
517	0.232527409838086\\
518	0.144636546411752\\
519	0.0883084031175471\\
520	0.0685490045786759\\
521	0.0854486575432241\\
522	0.129198036404845\\
523	0.179099957485947\\
524	0.21145968456152\\
525	0.212702731507232\\
526	0.186165214048674\\
527	0.145793486043818\\
528	0.105768650380518\\
529	0.0755369236672918\\
530	0.0597525421714071\\
531	0.0590229134152958\\
532	0.0698256060786453\\
533	0.0850184095729995\\
534	0.0964181682460026\\
535	0.0985475497650855\\
536	0.0908406785270111\\
537	0.0767668473844113\\
538	0.0612142221865442\\
539	0.0483152010286678\\
540	0.0404630427216413\\
541	0.0380986146293368\\
542	0.0398775409378608\\
543	0.0432576884854804\\
544	0.0455383561658664\\
545	0.0449599198139545\\
546	0.0412624454284385\\
547	0.0354594309065187\\
548	0.0291297107152644\\
549	0.0237057639646459\\
550	0.0200269548893616\\
551	0.0181907888275778\\
552	0.0176550165663144\\
553	0.0175331886452564\\
554	0.0169801783223861\\
555	0.0155095875256475\\
556	0.0131110459579382\\
557	0.0101498562679531\\
558	0.00714256133543193\\
559	0.00452775416986634\\
560	0.0025261612311071\\
561	0.00111084284780897\\
562	6.50748959414304e-05\\
563	-0.000903233465631736\\
564	-0.00203758215969057\\
565	-0.003450248309165\\
566	-0.00510818497916652\\
567	-0.00687820746461784\\
568	-0.00859973945993115\\
569	-0.0101511508608353\\
570	-0.011487035489512\\
571	-0.0126394410452555\\
572	-0.0136899635903845\\
573	-0.0147286987755106\\
574	-0.015818336415303\\
575	-0.0169765422397694\\
576	-0.0181797534651735\\
577	-0.0193820366616059\\
578	-0.0205381318994253\\
579	-0.0216208897773671\\
580	-0.0226278355490758\\
581	-0.0235768577656587\\
582	-0.0244950421521984\\
583	-0.0254064049352577\\
584	-0.0263235346581318\\
585	-0.0272455924402563\\
586	-0.0281620550541002\\
587	-0.029059397615985\\
588	-0.0299273729648613\\
589	-0.0307625001783109\\
590	-0.0315680506329217\\
591	-0.0323513639920995\\
592	-0.033120177236231\\
593	-0.0338796500220117\\
594	-0.0346311115582503\\
595	-0.0353726354084812\\
596	-0.0361007876061018\\
597	-0.0368125648903063\\
598	-0.0375066878156758\\
599	-0.0381838723513717\\
600	-0.0388462114100334\\
601	-0.0394961321857479\\
602	-0.0401354593558071\\
603	-0.0407649459290612\\
604	-0.0413843562423537\\
605	-0.0419929376613\\
606	-0.0425899935683085\\
607	-0.0431752935638714\\
608	-0.0437491854204157\\
609	-0.0443124302258607\\
610	-0.0448658947519856\\
611	-0.045410265361674\\
612	-0.0459459012408677\\
613	-0.0464728586862678\\
614	-0.0469910374448842\\
615	-0.0475003586092902\\
616	-0.0480008907546304\\
617	-0.048492881302861\\
618	-0.0489766999964749\\
619	-0.0494527379423776\\
620	-0.0499213158436056\\
621	-0.050382639630361\\
622	-0.0508368129828178\\
623	-0.0512838895470161\\
624	-0.0517239340142623\\
625	-0.0521570637447704\\
626	-0.0525834568314606\\
627	-0.0530033299107\\
628	-0.0534169014929029\\
629	-0.053824359721977\\
630	-0.0542258477786433\\
631	-0.0546214698483037\\
632	-0.0550113110747959\\
633	-0.0553954601765874\\
634	-0.0557740244989269\\
635	-0.0561471325581415\\
636	-0.0565149254887106\\
637	-0.0568775432905981\\
638	-0.0572351128446425\\
639	-0.057587742536783\\
640	-0.0579355245483773\\
641	-0.058278542376095\\
642	-0.0586168793825675\\
643	-0.0589506245637273\\
644	-0.0592798736606539\\
645	-0.0596047261008083\\
646	-0.0599252799342595\\
647	-0.0602416273521828\\
648	-0.0605538526099311\\
649	-0.0608620327890593\\
650	-0.0611662405374496\\
651	-0.0614665472568245\\
652	-0.0617630253225498\\
653	-0.0620557486154043\\
654	-0.0623447915068087\\
655	-0.0626302270669452\\
656	-0.0629121254387157\\
657	-0.0631905530575592\\
658	-0.0634655728983202\\
659	-0.0637372454560174\\
660	-0.0640056299144546\\
661	-0.0642707849876829\\
662	-0.0645327691632547\\
663	-0.0647916403857004\\
664	-0.0650474554484393\\
665	-0.0653002694307196\\
666	-0.0655501354270209\\
667	-0.0657971046400127\\
668	-0.0660412267377928\\
669	-0.0662825502832835\\
670	-0.0665211230521259\\
671	-0.066756992140615\\
672	-0.0669902038748875\\
673	-0.0672208036147873\\
674	-0.0674488355710906\\
675	-0.067674342724005\\
676	-0.0678973668687939\\
677	-0.0681179487541008\\
678	-0.0683361282455403\\
679	-0.0685519444499065\\
680	-0.0687654357653264\\
681	-0.0689766398613633\\
682	-0.0691855936220363\\
683	-0.0693923330935272\\
684	-0.0695968934673728\\
685	-0.0697993091079854\\
686	-0.0699996136121299\\
687	-0.0701978398764953\\
688	-0.0703940201506745\\
689	-0.0705881860635968\\
690	-0.0707803686251391\\
691	-0.070970598214764\\
692	-0.0711589045719683\\
693	-0.0713453167992777\\
694	-0.0715298633806782\\
695	-0.0717125722109072\\
696	-0.0718934706270716\\
697	-0.0720725854346182\\
698	-0.0722499429235754\\
699	-0.0724255688758545\\
700	-0.0725994885679323\\
701	-0.072771726774185\\
702	-0.07294230777462\\
703	-0.07311125536793\\
704	-0.0732785928881473\\
705	0.0228017576110979\\
706	0.297462105272302\\
707	0.688422528421905\\
708	1.1472582634235\\
709	1.6371537154252\\
710	2.13088114704575\\
711	2.60901346078425\\
712	3.05836906314471\\
713	3.47067877894596\\
714	3.82453594338728\\
715	4.0859007379816\\
716	4.23797633021483\\
717	4.29412530194627\\
718	4.28646201437514\\
719	4.25105127304574\\
720	4.22058208231943\\
721	4.22123516863395\\
722	4.26686345536559\\
723	4.34840376635634\\
724	4.45235620122141\\
725	4.5683278602906\\
726	4.68851189278142\\
727	4.80722261420268\\
728	4.92048757632061\\
729	5.02569510332531\\
730	5.12093892783622\\
731	5.19349341108611\\
732	5.22965684199184\\
733	5.22439739471398\\
734	5.18228543218744\\
735	5.11389773562542\\
736	5.03281878231112\\
737	4.95362815160295\\
738	4.88978181357273\\
739	4.85138878090491\\
740	4.84361663131828\\
741	4.8645739684538\\
742	4.90647488938501\\
743	4.96165599096169\\
744	5.02366631804499\\
745	5.08243875818493\\
746	5.12675870070098\\
747	5.14849500366105\\
748	5.14459265564274\\
749	5.1169047614642\\
750	5.0712539330752\\
751	5.01623476273016\\
752	4.96167538089347\\
753	4.91686671192504\\
754	4.8889547703238\\
755	4.8818391657525\\
756	4.89571006679761\\
757	4.92721081097158\\
758	4.97015014710835\\
759	5.0166708576325\\
760	5.05860580976607\\
761	5.08887507831602\\
762	5.10269401245566\\
763	5.09835226026038\\
764	5.07740628372069\\
765	5.0442595046214\\
766	5.00523769907383\\
767	4.96736824744025\\
768	4.93711364276851\\
769	4.9192931295355\\
770	4.91635978145294\\
771	4.92811555818987\\
772	4.95187377752969\\
773	4.98301915836918\\
774	5.01586457127498\\
775	5.04466121958996\\
776	5.06459218515926\\
777	5.07257740268156\\
778	5.06775076561503\\
779	5.05152492399578\\
780	5.02724064003747\\
781	4.99948354689512\\
782	4.97321547278145\\
783	4.9528924738166\\
784	4.94172632742694\\
785	4.94120234262792\\
786	4.95091492389532\\
787	4.96872653525193\\
788	4.99120697753463\\
789	5.01427102586355\\
790	5.03390463134807\\
791	5.04685592441009\\
792	5.05117197120837\\
793	5.04648863826692\\
794	5.03402692816254\\
795	5.01630614103835\\
796	4.99663947858327\\
797	4.97851655003914\\
798	4.96499008817867\\
799	4.9581731089988\\
800	4.95892502057099\\
801	4.96676666579996\\
802	4.98002456277341\\
803	4.99616925932952\\
804	5.01228445125239\\
805	5.02558449034862\\
806	5.03389097547539\\
807	5.03598671572347\\
808	5.03178778415408\\
809	5.02230855791097\\
810	5.00943398105944\\
811	4.99554906985765\\
812	4.98309974333828\\
813	4.97416671029873\\
814	4.97012556048801\\
815	4.9714455779443\\
816	4.97765279557058\\
817	4.98745457875125\\
818	4.99899740189032\\
819	5.01020932201875\\
820	5.01916630943873\\
821	5.02441895024713\\
822	5.02522402062099\\
823	5.02164331325171\\
824	5.01449698644897\\
825	5.00518576546655\\
826	4.9954198347944\\
827	4.98690730051149\\
828	4.9810590835331\\
829	4.97876034640781\\
830	4.98024367440033\\
831	4.98507986688474\\
832	4.99228190026892\\
833	5.00049935671404\\
834	5.00826671865247\\
835	5.01426121647144\\
836	5.01752555974786\\
837	5.01761811764451\\
838	5.01466690556756\\
839	5.0093216638993\\
840	5.00261690414574\\
841	4.99577430064944\\
842	4.98998221515018\\
843	4.98619199647429\\
844	4.9849653217851\\
845	4.98639596818588\\
846	4.99011545883052\\
847	4.99537752237306\\
848	5.00120345481607\\
849	5.00656107130274\\
850	5.0105452825014\\
851	5.01252911116769\\
852	5.01226007290645\\
853	5.00988725361984\\
854	5.00591722619836\\
855	5.00110970628833\\
856	4.99633408767563\\
857	4.99241385616315\\
858	4.98998652356267\\
859	4.98940244714662\\
860	4.9906778969316\\
861	4.99350768577809\\
862	4.99733237128092\\
863	5.00144609907551\\
864	5.0051248968129\\
865	5.0077525397863\\
866	5.00892236158262\\
867	5.00849832618607\\
868	5.00662641675913\\
869	5.00369649560731\\
870	5.00026353376193\\
871	4.99694386992467\\
872	4.99430576659654\\
873	4.99277351270166\\
874	4.99256093884798\\
875	4.99364431205841\\
876	4.9957773485373\\
877	4.99854380279174\\
878	5.00143693195405\\
879	5.00395102790975\\
880	5.00566874884008\\
881	5.006329339968\\
882	5.00586672293856\\
883	5.00441212526321\\
884	5.00226235518054\\
885	4.99982080997039\\
886	4.99752275817486\\
887	4.99575861929212\\
888	4.99480861561176\\
889	4.99479951581452\\
890	4.99568984535841\\
891	4.99728475281867\\
892	4.99927660715813\\
893	5.00130319355373\\
894	5.00301272290505\\
895	5.00412415293617\\
896	5.00447259759345\\
897	5.00403260252835\\
898	5.00291621672772\\
899	5.00134732181139\\
900	4.99961776420364\\
901	4.99803375397546\\
902	4.99686228035917\\
903	4.99628680917387\\
904	4.99637946257284\\
905	4.99709369021724\\
906	4.99827773319577\\
907	4.99970560875577\\
908	5.00111949343307\\
909	5.00227569688159\\
910	5.00298613572454\\
911	5.00314833532522\\
912	5.00275926934771\\
913	5.00191135614666\\
914	5.0007721109833\\
915	4.99955173206815\\
916	4.99846479918514\\
917	4.99769299160134\\
918	4.99735522115282\\
919	4.99748998522931\\
920	4.99805240850705\\
921	4.99892580132105\\
922	4.99994507903506\\
923	5.00092747103856\\
924	5.00170489757531\\
925	5.00215235032928\\
926	5.00220754780059\\
927	5.00187885214241\\
928	5.00124060282647\\
929	5.00041724614001\\
930	4.99955952367927\\
931	4.99881721036884\\
932	4.9983132788585\\
933	4.99812388799954\\
934	4.99826737770342\\
935	4.99870374960943\\
936	4.99934423934946\\
937	5.00006886895811\\
938	5.00074859099875\\
939	5.00126799654892\\
940	5.00154463974661\\
941	5.00154178965001\\
942	5.00127269827984\\
943	5.00079602907918\\
944	5.00020363936792\\
945	4.99960318226326\\
946	4.99909877752635\\
947	4.99877318309657\\
948	4.99867447797501\\
949	4.9988093463967\\
950	4.99914381706023\\
951	4.99961098728568\\
952	5.00012408070025\\
953	5.00059234322158\\
954	5.00093690478121\\
955	5.0011038681388\\
956	5.00107248707207\\
957	5.00085724114537\\
958	5.00050372479299\\
959	5.00007934398703\\
960	4.99966066801385\\
961	4.9993197778412\\
962	4.99911201726354\\
963	4.99906719819714\\
964	4.99918561933579\\
965	4.99943936384293\\
966	4.99977841010593\\
967	5.00014028124093\\
968	5.00046140671936\\
969	5.00068815642079\\
970	5.00078565548949\\
971	5.00074295668407\\
972	5.00057384089817\\
973	5.00031330613945\\
974	5.00001054899386\\
975	4.99971981303558\\
976	4.99949078445138\\
977	4.99936021566544\\
978	4.99934616754932\\
979	4.99944574413371\\
980	4.99963655061438\\
981	4.99988145251056\\
982	5.00013566367596\\
983	5.00035483322221\\
984	5.0005026891214\\
985	5.00055693785529\\
986	5.00051247951352\\
987	5.00038150446763\\
988	5.00019059691787\\
989	4.99997548282196\\
990	4.99977443810428\\
991	4.99962155805615\\
992	4.99954105743781\\
993	4.99954353867643\\
994	4.99962478239324\\
995	4.99976715354126\\
996	4.99994326072377\\
997	5.00012113394284\\
998	5.00026995733293\\
999	5.00036534127672\\
1000	5.00039324383435\\
1001	5.00035192513765\\
1002	5.00025168643839\\
1003	5.00011253992491\\
1004	4.99996030666129\\
1005	4.99982188909553\\
1006	4.99972057303583\\
1007	4.99967216956933\\
1008	4.99968262495758\\
1009	4.9997474441006\\
1010	4.9998529441387\\
1011	4.9999790380323\\
1012	5.0001029975122\\
1013	5.00020350074282\\
1014	5.00026425241833\\
1015	5.00027657020263\\
1016	5.00024053757361\\
1017	5.00016458913755\\
1018	5.00006367161144\\
1019	4.99995636342159\\
1020	4.99986149839895\\
1021	4.9997948997647\\
1022	4.99976678370498\\
1023	4.99978025045427\\
1024	4.99983107377261\\
1025	4.99990876579852\\
1026	4.99999867506853\\
1027	5.00008470812901\\
1028	5.00015217612013\\
1029	5.00019026880539\\
1030	5.00019374550891\\
1031	5.00016358642667\\
1032	5.00010653915806\\
1033	5.00003368926138\\
1034	4.99995834647939\\
1035	4.99989364323674\\
1036	4.99985027357379\\
1037	4.99983475678244\\
1038	4.9998485017321\\
1039	4.99988779690853\\
1040	4.99994468556911\\
1041	5.0000085343973\\
1042	5.00006799300171\\
1043	5.00011298788591\\
1044	5.00013640478858\\
1045	5.00013518289546\\
1046	5.00011065865353\\
1047	5.00006813400974\\
1048	5.000015779104\\
1049	4.99996308946102\\
1050	4.99991918466369\\
1051	4.99989124974956\\
1052	4.99988338211276\\
1053	4.99989602459559\\
1054	4.99992605591351\\
1055	4.99996749275227\\
1056	5.00001265430213\\
1057	5.00005356692228\\
1058	5.00008335528239\\
1059	5.00009738022475\\
1060	5.0000939382806\\
1061	5.00007442196068\\
1062	5.0000429377331\\
1063	5.00000547247244\\
1064	4.99996877305612\\
1065	4.99993914583049\\
1066	4.99992138705908\\
1067	4.99991802317091\\
1068	4.99992897791687\\
1069	4.99995170442933\\
1070	4.99998173812809\\
1071	5.00001355572481\\
1072	5.00004157793474\\
1073	5.00006113608495\\
1074	5.00006923724582\\
1075	5.00006500486775\\
1076	5.00004973359083\\
1077	5.0000265663713\\
1078	4.99999986696878\\
1079	4.99997441014904\\
1080	4.99995453786569\\
1081	4.99994342878311\\
1082	4.9999426021784\\
1083	4.9999517312278\\
1084	4.99996878380104\\
1085	4.99999045147582\\
1086	5.00001277948496\\
1087	5.00003187957826\\
1088	5.00004459884119\\
1089	5.00004903091472\\
1090	5.00004478847327\\
1091	5.00003300077616\\
1092	5.00001604926956\\
1093	4.99999709889158\\
1094	4.9999795154044\\
1095	4.99996627464433\\
1096	4.99995946613681\\
1097	4.99995997253384\\
1098	4.9999673722283\\
1099	4.99998007177556\\
1100	4.99999563474479\\
1101	5.00001124120927\\
1102	5.00002419251746\\
1103	5.00003237205862\\
1104	5.0000345844157\\
1105	5.00003071985359\\
1106	5.00002172369491\\
1107	5.00000938478541\\
1108	4.99999598786499\\
1109	4.99998389613078\\
1110	4.99997513931276\\
1111	4.9999710781873\\
1112	4.99997220000461\\
1113	4.99997807419641\\
1114	4.99998746869482\\
1115	4.99999859942821\\
1116	5.00000946380249\\
1117	5.00001819671139\\
1118	5.00002338653198\\
1119	5.00002429833113\\
1120	5.00002096995087\\
1121	5.00001417017444\\
1122	5.00000523250079\\
1123	4.99999579893204\\
1124	4.9999875221573\\
1125	4.99998177949394\\
1126	4.99997944747785\\
1127	4.99998077328034\\
1128	4.99998536074598\\
1129	4.99999226822794\\
1130	5.00000019620983\\
1131	5.00000772819483\\
1132	5.00001358080652\\
1133	5.00001681946569\\
1134	5.00001700394796\\
1135	5.00001424187264\\
1136	5.00000914508493\\
1137	5.00000270090346\\
1138	4.99999608436949\\
1139	4.99999044663501\\
1140	4.9999867171448\\
1141	4.99998545316344\\
1142	4.99998676048484\\
1143	4.99999029577295\\
1144	4.99999534640068\\
1145	5.00000097045203\\
1146	5.0000061699489\\
1147	5.00001006585349\\
1148	5.00001204452637\\
1149	5.00001185163692\\
1150	5.00000961969886\\
1151	5.00000582750422\\
1152	5.00000120157582\\
1153	4.99999657931484\\
1154	4.9999927592405\\
1155	4.99999036478879\\
1156	4.99998974458538\\
1157	4.99999092475825\\
1158	4.99999361915303\\
1159	4.99999729304109\\
1160	5.00000126686903\\
1161	5.00000484029413\\
1162	5.00000741414696\\
1163	5.00000858934197\\
1164	5.00000822669769\\
1165	5.00000645911544\\
1166	5.00000365618789\\
1167	5.00000034952574\\
1168	4.99999713349961\\
1169	4.99999455967126\\
1170	4.99999304344383\\
1171	4.99999279850289\\
1172	4.99999380910695\\
1173	4.99999584328639\\
1174	4.99999850280177\\
1175	5.00000129955019\\
1176	5.00000374400956\\
1177	5.00000542988865\\
1178	5.0000061005293\\
1179	5.00000568642058\\
1180	5.00000430866196\\
1181	5.00000224933236\\
1182	4.99999989539661\\
1183	4.99999766705051\\
1184	4.99999594359679\\
1185	4.99999499977266\\
1186	4.99999496305368\\
1187	4.99999579835317\\
1188	4.9999973215099\\
1189	4.99999923791828\\
1190	5.00000119847805\\
1191	5.00000286240477\\
1192	5.00000395573457\\
1193	5.00000431561087\\
1194	5.00000391334997\\
1195	5.00000285326909\\
1196	5.00000134858659\\
1197	4.99999967960555\\
1198	4.99999814221577\\
1199	4.99999699605458\\
1200	4.99999642129962\\
};
\addlegendentry{$y(k)$}

\addplot[const plot, color=mycolor2, dashed] table[row sep=crcr] {%
1	0\\
2	0\\
3	0\\
4	0\\
5	0\\
6	0\\
7	0\\
8	0\\
9	0\\
10	0\\
11	0\\
12	0\\
13	0\\
14	0\\
15	0\\
16	0\\
17	0\\
18	0\\
19	0\\
20	0\\
21	0\\
22	0\\
23	0\\
24	0\\
25	0\\
26	0\\
27	0\\
28	0\\
29	0\\
30	0\\
31	0\\
32	0\\
33	0\\
34	0\\
35	0\\
36	0\\
37	0\\
38	0\\
39	0\\
40	0\\
41	0\\
42	0\\
43	0\\
44	0\\
45	0\\
46	0\\
47	0\\
48	0\\
49	0\\
50	0\\
51	0\\
52	0\\
53	0\\
54	0\\
55	0\\
56	0\\
57	0\\
58	0\\
59	0\\
60	0\\
61	0\\
62	0\\
63	0\\
64	0\\
65	0\\
66	0\\
67	0\\
68	0\\
69	0\\
70	0\\
71	0\\
72	0\\
73	0\\
74	0\\
75	0\\
76	0\\
77	0\\
78	0\\
79	0\\
80	0\\
81	0\\
82	0\\
83	0\\
84	0\\
85	0\\
86	0\\
87	0\\
88	0\\
89	0\\
90	0\\
91	0\\
92	0\\
93	0\\
94	0\\
95	0\\
96	0\\
97	0\\
98	0\\
99	0\\
100	0.5\\
101	0.5\\
102	0.5\\
103	0.5\\
104	0.5\\
105	0.5\\
106	0.5\\
107	0.5\\
108	0.5\\
109	0.5\\
110	0.5\\
111	0.5\\
112	0.5\\
113	0.5\\
114	0.5\\
115	0.5\\
116	0.5\\
117	0.5\\
118	0.5\\
119	0.5\\
120	0.5\\
121	0.5\\
122	0.5\\
123	0.5\\
124	0.5\\
125	0.5\\
126	0.5\\
127	0.5\\
128	0.5\\
129	0.5\\
130	0.5\\
131	0.5\\
132	0.5\\
133	0.5\\
134	0.5\\
135	0.5\\
136	0.5\\
137	0.5\\
138	0.5\\
139	0.5\\
140	0.5\\
141	0.5\\
142	0.5\\
143	0.5\\
144	0.5\\
145	0.5\\
146	0.5\\
147	0.5\\
148	0.5\\
149	0.5\\
150	0.5\\
151	0.5\\
152	0.5\\
153	0.5\\
154	0.5\\
155	0.5\\
156	0.5\\
157	0.5\\
158	0.5\\
159	0.5\\
160	0.5\\
161	0.5\\
162	0.5\\
163	0.5\\
164	0.5\\
165	0.5\\
166	0.5\\
167	0.5\\
168	0.5\\
169	0.5\\
170	0.5\\
171	0.5\\
172	0.5\\
173	0.5\\
174	0.5\\
175	0.5\\
176	0.5\\
177	0.5\\
178	0.5\\
179	0.5\\
180	0.5\\
181	0.5\\
182	0.5\\
183	0.5\\
184	0.5\\
185	0.5\\
186	0.5\\
187	0.5\\
188	0.5\\
189	0.5\\
190	0.5\\
191	0.5\\
192	0.5\\
193	0.5\\
194	0.5\\
195	0.5\\
196	0.5\\
197	0.5\\
198	0.5\\
199	0.5\\
200	0.5\\
201	0.5\\
202	0.5\\
203	0.5\\
204	0.5\\
205	0.5\\
206	0.5\\
207	0.5\\
208	0.5\\
209	0.5\\
210	0.5\\
211	0.5\\
212	0.5\\
213	0.5\\
214	0.5\\
215	0.5\\
216	0.5\\
217	0.5\\
218	0.5\\
219	0.5\\
220	0.5\\
221	0.5\\
222	0.5\\
223	0.5\\
224	0.5\\
225	0.5\\
226	0.5\\
227	0.5\\
228	0.5\\
229	0.5\\
230	0.5\\
231	0.5\\
232	0.5\\
233	0.5\\
234	0.5\\
235	0.5\\
236	0.5\\
237	0.5\\
238	0.5\\
239	0.5\\
240	0.5\\
241	0.5\\
242	0.5\\
243	0.5\\
244	0.5\\
245	0.5\\
246	0.5\\
247	0.5\\
248	0.5\\
249	0.5\\
250	0.5\\
251	0.5\\
252	0.5\\
253	0.5\\
254	0.5\\
255	0.5\\
256	0.5\\
257	0.5\\
258	0.5\\
259	0.5\\
260	0.5\\
261	0.5\\
262	0.5\\
263	0.5\\
264	0.5\\
265	0.5\\
266	0.5\\
267	0.5\\
268	0.5\\
269	0.5\\
270	0.5\\
271	0.5\\
272	0.5\\
273	0.5\\
274	0.5\\
275	0.5\\
276	0.5\\
277	0.5\\
278	0.5\\
279	0.5\\
280	0.5\\
281	0.5\\
282	0.5\\
283	0.5\\
284	0.5\\
285	0.5\\
286	0.5\\
287	0.5\\
288	0.5\\
289	0.5\\
290	0.5\\
291	0.5\\
292	0.5\\
293	0.5\\
294	0.5\\
295	0.5\\
296	0.5\\
297	0.5\\
298	0.5\\
299	0.5\\
300	1\\
301	1\\
302	1\\
303	1\\
304	1\\
305	1\\
306	1\\
307	1\\
308	1\\
309	1\\
310	1\\
311	1\\
312	1\\
313	1\\
314	1\\
315	1\\
316	1\\
317	1\\
318	1\\
319	1\\
320	1\\
321	1\\
322	1\\
323	1\\
324	1\\
325	1\\
326	1\\
327	1\\
328	1\\
329	1\\
330	1\\
331	1\\
332	1\\
333	1\\
334	1\\
335	1\\
336	1\\
337	1\\
338	1\\
339	1\\
340	1\\
341	1\\
342	1\\
343	1\\
344	1\\
345	1\\
346	1\\
347	1\\
348	1\\
349	1\\
350	1\\
351	1\\
352	1\\
353	1\\
354	1\\
355	1\\
356	1\\
357	1\\
358	1\\
359	1\\
360	1\\
361	1\\
362	1\\
363	1\\
364	1\\
365	1\\
366	1\\
367	1\\
368	1\\
369	1\\
370	1\\
371	1\\
372	1\\
373	1\\
374	1\\
375	1\\
376	1\\
377	1\\
378	1\\
379	1\\
380	1\\
381	1\\
382	1\\
383	1\\
384	1\\
385	1\\
386	1\\
387	1\\
388	1\\
389	1\\
390	1\\
391	1\\
392	1\\
393	1\\
394	1\\
395	1\\
396	1\\
397	1\\
398	1\\
399	1\\
400	1\\
401	1\\
402	1\\
403	1\\
404	1\\
405	1\\
406	1\\
407	1\\
408	1\\
409	1\\
410	1\\
411	1\\
412	1\\
413	1\\
414	1\\
415	1\\
416	1\\
417	1\\
418	1\\
419	1\\
420	1\\
421	1\\
422	1\\
423	1\\
424	1\\
425	1\\
426	1\\
427	1\\
428	1\\
429	1\\
430	1\\
431	1\\
432	1\\
433	1\\
434	1\\
435	1\\
436	1\\
437	1\\
438	1\\
439	1\\
440	1\\
441	1\\
442	1\\
443	1\\
444	1\\
445	1\\
446	1\\
447	1\\
448	1\\
449	1\\
450	1\\
451	1\\
452	1\\
453	1\\
454	1\\
455	1\\
456	1\\
457	1\\
458	1\\
459	1\\
460	1\\
461	1\\
462	1\\
463	1\\
464	1\\
465	1\\
466	1\\
467	1\\
468	1\\
469	1\\
470	1\\
471	1\\
472	1\\
473	1\\
474	1\\
475	1\\
476	1\\
477	1\\
478	1\\
479	1\\
480	1\\
481	1\\
482	1\\
483	1\\
484	1\\
485	1\\
486	1\\
487	1\\
488	1\\
489	1\\
490	1\\
491	1\\
492	1\\
493	1\\
494	1\\
495	1\\
496	1\\
497	1\\
498	1\\
499	1\\
500	-0.1\\
501	-0.1\\
502	-0.1\\
503	-0.1\\
504	-0.1\\
505	-0.1\\
506	-0.1\\
507	-0.1\\
508	-0.1\\
509	-0.1\\
510	-0.1\\
511	-0.1\\
512	-0.1\\
513	-0.1\\
514	-0.1\\
515	-0.1\\
516	-0.1\\
517	-0.1\\
518	-0.1\\
519	-0.1\\
520	-0.1\\
521	-0.1\\
522	-0.1\\
523	-0.1\\
524	-0.1\\
525	-0.1\\
526	-0.1\\
527	-0.1\\
528	-0.1\\
529	-0.1\\
530	-0.1\\
531	-0.1\\
532	-0.1\\
533	-0.1\\
534	-0.1\\
535	-0.1\\
536	-0.1\\
537	-0.1\\
538	-0.1\\
539	-0.1\\
540	-0.1\\
541	-0.1\\
542	-0.1\\
543	-0.1\\
544	-0.1\\
545	-0.1\\
546	-0.1\\
547	-0.1\\
548	-0.1\\
549	-0.1\\
550	-0.1\\
551	-0.1\\
552	-0.1\\
553	-0.1\\
554	-0.1\\
555	-0.1\\
556	-0.1\\
557	-0.1\\
558	-0.1\\
559	-0.1\\
560	-0.1\\
561	-0.1\\
562	-0.1\\
563	-0.1\\
564	-0.1\\
565	-0.1\\
566	-0.1\\
567	-0.1\\
568	-0.1\\
569	-0.1\\
570	-0.1\\
571	-0.1\\
572	-0.1\\
573	-0.1\\
574	-0.1\\
575	-0.1\\
576	-0.1\\
577	-0.1\\
578	-0.1\\
579	-0.1\\
580	-0.1\\
581	-0.1\\
582	-0.1\\
583	-0.1\\
584	-0.1\\
585	-0.1\\
586	-0.1\\
587	-0.1\\
588	-0.1\\
589	-0.1\\
590	-0.1\\
591	-0.1\\
592	-0.1\\
593	-0.1\\
594	-0.1\\
595	-0.1\\
596	-0.1\\
597	-0.1\\
598	-0.1\\
599	-0.1\\
600	-0.1\\
601	-0.1\\
602	-0.1\\
603	-0.1\\
604	-0.1\\
605	-0.1\\
606	-0.1\\
607	-0.1\\
608	-0.1\\
609	-0.1\\
610	-0.1\\
611	-0.1\\
612	-0.1\\
613	-0.1\\
614	-0.1\\
615	-0.1\\
616	-0.1\\
617	-0.1\\
618	-0.1\\
619	-0.1\\
620	-0.1\\
621	-0.1\\
622	-0.1\\
623	-0.1\\
624	-0.1\\
625	-0.1\\
626	-0.1\\
627	-0.1\\
628	-0.1\\
629	-0.1\\
630	-0.1\\
631	-0.1\\
632	-0.1\\
633	-0.1\\
634	-0.1\\
635	-0.1\\
636	-0.1\\
637	-0.1\\
638	-0.1\\
639	-0.1\\
640	-0.1\\
641	-0.1\\
642	-0.1\\
643	-0.1\\
644	-0.1\\
645	-0.1\\
646	-0.1\\
647	-0.1\\
648	-0.1\\
649	-0.1\\
650	-0.1\\
651	-0.1\\
652	-0.1\\
653	-0.1\\
654	-0.1\\
655	-0.1\\
656	-0.1\\
657	-0.1\\
658	-0.1\\
659	-0.1\\
660	-0.1\\
661	-0.1\\
662	-0.1\\
663	-0.1\\
664	-0.1\\
665	-0.1\\
666	-0.1\\
667	-0.1\\
668	-0.1\\
669	-0.1\\
670	-0.1\\
671	-0.1\\
672	-0.1\\
673	-0.1\\
674	-0.1\\
675	-0.1\\
676	-0.1\\
677	-0.1\\
678	-0.1\\
679	-0.1\\
680	-0.1\\
681	-0.1\\
682	-0.1\\
683	-0.1\\
684	-0.1\\
685	-0.1\\
686	-0.1\\
687	-0.1\\
688	-0.1\\
689	-0.1\\
690	-0.1\\
691	-0.1\\
692	-0.1\\
693	-0.1\\
694	-0.1\\
695	-0.1\\
696	-0.1\\
697	-0.1\\
698	-0.1\\
699	-0.1\\
700	5\\
701	5\\
702	5\\
703	5\\
704	5\\
705	5\\
706	5\\
707	5\\
708	5\\
709	5\\
710	5\\
711	5\\
712	5\\
713	5\\
714	5\\
715	5\\
716	5\\
717	5\\
718	5\\
719	5\\
720	5\\
721	5\\
722	5\\
723	5\\
724	5\\
725	5\\
726	5\\
727	5\\
728	5\\
729	5\\
730	5\\
731	5\\
732	5\\
733	5\\
734	5\\
735	5\\
736	5\\
737	5\\
738	5\\
739	5\\
740	5\\
741	5\\
742	5\\
743	5\\
744	5\\
745	5\\
746	5\\
747	5\\
748	5\\
749	5\\
750	5\\
751	5\\
752	5\\
753	5\\
754	5\\
755	5\\
756	5\\
757	5\\
758	5\\
759	5\\
760	5\\
761	5\\
762	5\\
763	5\\
764	5\\
765	5\\
766	5\\
767	5\\
768	5\\
769	5\\
770	5\\
771	5\\
772	5\\
773	5\\
774	5\\
775	5\\
776	5\\
777	5\\
778	5\\
779	5\\
780	5\\
781	5\\
782	5\\
783	5\\
784	5\\
785	5\\
786	5\\
787	5\\
788	5\\
789	5\\
790	5\\
791	5\\
792	5\\
793	5\\
794	5\\
795	5\\
796	5\\
797	5\\
798	5\\
799	5\\
800	5\\
801	5\\
802	5\\
803	5\\
804	5\\
805	5\\
806	5\\
807	5\\
808	5\\
809	5\\
810	5\\
811	5\\
812	5\\
813	5\\
814	5\\
815	5\\
816	5\\
817	5\\
818	5\\
819	5\\
820	5\\
821	5\\
822	5\\
823	5\\
824	5\\
825	5\\
826	5\\
827	5\\
828	5\\
829	5\\
830	5\\
831	5\\
832	5\\
833	5\\
834	5\\
835	5\\
836	5\\
837	5\\
838	5\\
839	5\\
840	5\\
841	5\\
842	5\\
843	5\\
844	5\\
845	5\\
846	5\\
847	5\\
848	5\\
849	5\\
850	5\\
851	5\\
852	5\\
853	5\\
854	5\\
855	5\\
856	5\\
857	5\\
858	5\\
859	5\\
860	5\\
861	5\\
862	5\\
863	5\\
864	5\\
865	5\\
866	5\\
867	5\\
868	5\\
869	5\\
870	5\\
871	5\\
872	5\\
873	5\\
874	5\\
875	5\\
876	5\\
877	5\\
878	5\\
879	5\\
880	5\\
881	5\\
882	5\\
883	5\\
884	5\\
885	5\\
886	5\\
887	5\\
888	5\\
889	5\\
890	5\\
891	5\\
892	5\\
893	5\\
894	5\\
895	5\\
896	5\\
897	5\\
898	5\\
899	5\\
900	5\\
901	5\\
902	5\\
903	5\\
904	5\\
905	5\\
906	5\\
907	5\\
908	5\\
909	5\\
910	5\\
911	5\\
912	5\\
913	5\\
914	5\\
915	5\\
916	5\\
917	5\\
918	5\\
919	5\\
920	5\\
921	5\\
922	5\\
923	5\\
924	5\\
925	5\\
926	5\\
927	5\\
928	5\\
929	5\\
930	5\\
931	5\\
932	5\\
933	5\\
934	5\\
935	5\\
936	5\\
937	5\\
938	5\\
939	5\\
940	5\\
941	5\\
942	5\\
943	5\\
944	5\\
945	5\\
946	5\\
947	5\\
948	5\\
949	5\\
950	5\\
951	5\\
952	5\\
953	5\\
954	5\\
955	5\\
956	5\\
957	5\\
958	5\\
959	5\\
960	5\\
961	5\\
962	5\\
963	5\\
964	5\\
965	5\\
966	5\\
967	5\\
968	5\\
969	5\\
970	5\\
971	5\\
972	5\\
973	5\\
974	5\\
975	5\\
976	5\\
977	5\\
978	5\\
979	5\\
980	5\\
981	5\\
982	5\\
983	5\\
984	5\\
985	5\\
986	5\\
987	5\\
988	5\\
989	5\\
990	5\\
991	5\\
992	5\\
993	5\\
994	5\\
995	5\\
996	5\\
997	5\\
998	5\\
999	5\\
1000	5\\
1001	5\\
1002	5\\
1003	5\\
1004	5\\
1005	5\\
1006	5\\
1007	5\\
1008	5\\
1009	5\\
1010	5\\
1011	5\\
1012	5\\
1013	5\\
1014	5\\
1015	5\\
1016	5\\
1017	5\\
1018	5\\
1019	5\\
1020	5\\
1021	5\\
1022	5\\
1023	5\\
1024	5\\
1025	5\\
1026	5\\
1027	5\\
1028	5\\
1029	5\\
1030	5\\
1031	5\\
1032	5\\
1033	5\\
1034	5\\
1035	5\\
1036	5\\
1037	5\\
1038	5\\
1039	5\\
1040	5\\
1041	5\\
1042	5\\
1043	5\\
1044	5\\
1045	5\\
1046	5\\
1047	5\\
1048	5\\
1049	5\\
1050	5\\
1051	5\\
1052	5\\
1053	5\\
1054	5\\
1055	5\\
1056	5\\
1057	5\\
1058	5\\
1059	5\\
1060	5\\
1061	5\\
1062	5\\
1063	5\\
1064	5\\
1065	5\\
1066	5\\
1067	5\\
1068	5\\
1069	5\\
1070	5\\
1071	5\\
1072	5\\
1073	5\\
1074	5\\
1075	5\\
1076	5\\
1077	5\\
1078	5\\
1079	5\\
1080	5\\
1081	5\\
1082	5\\
1083	5\\
1084	5\\
1085	5\\
1086	5\\
1087	5\\
1088	5\\
1089	5\\
1090	5\\
1091	5\\
1092	5\\
1093	5\\
1094	5\\
1095	5\\
1096	5\\
1097	5\\
1098	5\\
1099	5\\
1100	5\\
1101	5\\
1102	5\\
1103	5\\
1104	5\\
1105	5\\
1106	5\\
1107	5\\
1108	5\\
1109	5\\
1110	5\\
1111	5\\
1112	5\\
1113	5\\
1114	5\\
1115	5\\
1116	5\\
1117	5\\
1118	5\\
1119	5\\
1120	5\\
1121	5\\
1122	5\\
1123	5\\
1124	5\\
1125	5\\
1126	5\\
1127	5\\
1128	5\\
1129	5\\
1130	5\\
1131	5\\
1132	5\\
1133	5\\
1134	5\\
1135	5\\
1136	5\\
1137	5\\
1138	5\\
1139	5\\
1140	5\\
1141	5\\
1142	5\\
1143	5\\
1144	5\\
1145	5\\
1146	5\\
1147	5\\
1148	5\\
1149	5\\
1150	5\\
1151	5\\
1152	5\\
1153	5\\
1154	5\\
1155	5\\
1156	5\\
1157	5\\
1158	5\\
1159	5\\
1160	5\\
1161	5\\
1162	5\\
1163	5\\
1164	5\\
1165	5\\
1166	5\\
1167	5\\
1168	5\\
1169	5\\
1170	5\\
1171	5\\
1172	5\\
1173	5\\
1174	5\\
1175	5\\
1176	5\\
1177	5\\
1178	5\\
1179	5\\
1180	5\\
1181	5\\
1182	5\\
1183	5\\
1184	5\\
1185	5\\
1186	5\\
1187	5\\
1188	5\\
1189	5\\
1190	5\\
1191	5\\
1192	5\\
1193	5\\
1194	5\\
1195	5\\
1196	5\\
1197	5\\
1198	5\\
1199	5\\
1200	5\\
};
\addlegendentry{$y^{zad}(k)$}

\end{axis}
\end{tikzpicture}%
	\caption{Regulator DMC - sygna� wyj�ciowy i zadany}
	\label{dmc_wy}
\end{figure}

D = 53;
N = 20; 
Nu = 3;

Error = 264.7306