\chapter{Odpowiedzi skokowe i charakterystyka statyczna}

\section{Wyznaczenie odpowiedzi skokowych toru wej�cie-wyj�cie procesu}
W celu wyznaczenia odpowiedzi skokowych obiekt by� pobudzany, w punkcie pracy, r�nymi skokami sygna�u steruj�cego w chwili $k=16$. Przeprowadzono cztery testy dla r�nych warto�ci skok�w. Uzyskane odpowiedzi skokowe wraz z odpowiadaj�cymi im przebiegami sygna�u sterowania przedstawiono na rys.~\ref{tor_uy}.

\begin{figure}[h!]
\centering
% This file was created by matlab2tikz.
%
\definecolor{mycolor1}{rgb}{0.00000,0.44700,0.74100}%
\definecolor{mycolor2}{rgb}{0.85000,0.32500,0.09800}%
\definecolor{mycolor3}{rgb}{0.92900,0.69400,0.12500}%
\definecolor{mycolor4}{rgb}{0.49400,0.18400,0.55600}%
%
\begin{tikzpicture}

\begin{axis}[%
width=4.521in,
height=1.493in,
at={(0.758in,2.554in)},
scale only axis,
xmin=0,
xmax=600,
ymin=-0.184717123272878,
ymax=6,
axis background/.style={fill=white},
title style={font=\bfseries},
title={Sygna� wyj�ciowy},
xmajorgrids,
ymajorgrids
]
\addplot [color=mycolor1, forget plot]
  table[row sep=crcr]{%
1	0\\
2	0\\
3	0\\
4	0\\
5	0\\
6	0\\
7	0\\
8	0\\
9	0\\
10	0\\
11	0\\
12	0\\
13	0\\
14	0\\
15	0\\
16	0\\
17	0\\
18	0\\
19	0\\
20	0\\
21	-0.0434726666666667\\
22	-0.0956411528837037\\
23	-0.130732401756729\\
24	-0.152307105318603\\
25	-0.165304518198085\\
26	-0.17309619317353\\
27	-0.177761487402132\\
28	-0.180554015185796\\
29	-0.182225428718017\\
30	-0.183225802739724\\
31	-0.183824543658212\\
32	-0.184182899913443\\
33	-0.184397381947923\\
34	-0.184525752917634\\
35	-0.184602585019809\\
36	-0.184648570275768\\
37	-0.184676093195273\\
38	-0.184692566107777\\
39	-0.184702425412357\\
40	-0.184708326365621\\
41	-0.184711858181596\\
42	-0.184713972030584\\
43	-0.184715237203394\\
44	-0.184715994429785\\
45	-0.184716447642028\\
46	-0.184716718896886\\
47	-0.184716881247306\\
48	-0.184716978416658\\
49	-0.184717036574087\\
50	-0.184717071382249\\
51	-0.184717092215497\\
52	-0.184717104684533\\
53	-0.184717112147453\\
54	-0.184717116614131\\
55	-0.18471711928751\\
56	-0.184717120887571\\
57	-0.184717121845233\\
58	-0.184717122418409\\
59	-0.184717122761465\\
60	-0.184717122966789\\
61	-0.184717123089679\\
62	-0.18471712316323\\
63	-0.184717123207252\\
64	-0.1847171232336\\
65	-0.184717123249369\\
66	-0.184717123258808\\
67	-0.184717123264456\\
68	-0.184717123267838\\
69	-0.184717123269861\\
70	-0.184717123271072\\
71	-0.184717123271797\\
72	-0.184717123272231\\
73	-0.184717123272491\\
74	-0.184717123272646\\
75	-0.184717123272739\\
76	-0.184717123272795\\
77	-0.184717123272828\\
78	-0.184717123272848\\
79	-0.18471712327286\\
80	-0.184717123272867\\
81	-0.184717123272871\\
82	-0.184717123272874\\
83	-0.184717123272876\\
84	-0.184717123272876\\
85	-0.184717123272877\\
86	-0.184717123272877\\
87	-0.184717123272878\\
88	-0.184717123272878\\
89	-0.184717123272878\\
90	-0.184717123272878\\
91	-0.184717123272878\\
92	-0.184717123272878\\
93	-0.184717123272878\\
94	-0.184717123272878\\
95	-0.184717123272878\\
96	-0.184717123272878\\
97	-0.184717123272878\\
98	-0.184717123272878\\
99	-0.184717123272878\\
100	-0.184717123272878\\
101	-0.184717123272878\\
102	-0.184717123272878\\
103	-0.184717123272878\\
104	-0.184717123272878\\
105	-0.184717123272878\\
106	-0.184717123272878\\
107	-0.184717123272878\\
108	-0.184717123272878\\
109	-0.184717123272878\\
110	-0.184717123272878\\
111	-0.184717123272878\\
112	-0.184717123272878\\
113	-0.184717123272878\\
114	-0.184717123272878\\
115	-0.184717123272878\\
116	-0.184717123272878\\
117	-0.184717123272878\\
118	-0.184717123272878\\
119	-0.184717123272878\\
120	-0.184717123272878\\
121	-0.184717123272878\\
122	-0.184717123272878\\
123	-0.184717123272878\\
124	-0.184717123272878\\
125	-0.184717123272878\\
126	-0.184717123272878\\
127	-0.184717123272878\\
128	-0.184717123272878\\
129	-0.184717123272878\\
130	-0.184717123272878\\
131	-0.184717123272878\\
132	-0.184717123272878\\
133	-0.184717123272878\\
134	-0.184717123272878\\
135	-0.184717123272878\\
136	-0.184717123272878\\
137	-0.184717123272878\\
138	-0.184717123272878\\
139	-0.184717123272878\\
140	-0.184717123272878\\
141	-0.184717123272878\\
142	-0.184717123272878\\
143	-0.184717123272878\\
144	-0.184717123272878\\
145	-0.184717123272878\\
146	-0.184717123272878\\
147	-0.184717123272878\\
148	-0.184717123272878\\
149	-0.184717123272878\\
150	-0.184717123272878\\
151	-0.184717123272878\\
152	-0.184717123272878\\
153	-0.184717123272878\\
154	-0.184717123272878\\
155	-0.184717123272878\\
156	-0.184717123272878\\
157	-0.184717123272878\\
158	-0.184717123272878\\
159	-0.184717123272878\\
160	-0.184717123272878\\
161	-0.184717123272878\\
162	-0.184717123272878\\
163	-0.184717123272878\\
164	-0.184717123272878\\
165	-0.184717123272878\\
166	-0.184717123272878\\
167	-0.184717123272878\\
168	-0.184717123272878\\
169	-0.184717123272878\\
170	-0.184717123272878\\
171	-0.184717123272878\\
172	-0.184717123272878\\
173	-0.184717123272878\\
174	-0.184717123272878\\
175	-0.184717123272878\\
176	-0.184717123272878\\
177	-0.184717123272878\\
178	-0.184717123272878\\
179	-0.184717123272878\\
180	-0.184717123272878\\
181	-0.184717123272878\\
182	-0.184717123272878\\
183	-0.184717123272878\\
184	-0.184717123272878\\
185	-0.184717123272878\\
186	-0.184717123272878\\
187	-0.184717123272878\\
188	-0.184717123272878\\
189	-0.184717123272878\\
190	-0.184717123272878\\
191	-0.184717123272878\\
192	-0.184717123272878\\
193	-0.184717123272878\\
194	-0.184717123272878\\
195	-0.184717123272878\\
196	-0.184717123272878\\
197	-0.184717123272878\\
198	-0.184717123272878\\
199	-0.184717123272878\\
200	-0.184717123272878\\
201	-0.184717123272878\\
202	-0.184717123272878\\
203	-0.184717123272878\\
204	-0.184717123272878\\
205	-0.184717123272878\\
206	-0.184717123272878\\
207	-0.184717123272878\\
208	-0.184717123272878\\
209	-0.184717123272878\\
210	-0.184717123272878\\
211	-0.184717123272878\\
212	-0.184717123272878\\
213	-0.184717123272878\\
214	-0.184717123272878\\
215	-0.184717123272878\\
216	-0.184717123272878\\
217	-0.184717123272878\\
218	-0.184717123272878\\
219	-0.184717123272878\\
220	-0.184717123272878\\
221	-0.184717123272878\\
222	-0.184717123272878\\
223	-0.184717123272878\\
224	-0.184717123272878\\
225	-0.184717123272878\\
226	-0.184717123272878\\
227	-0.184717123272878\\
228	-0.184717123272878\\
229	-0.184717123272878\\
230	-0.184717123272878\\
231	-0.184717123272878\\
232	-0.184717123272878\\
233	-0.184717123272878\\
234	-0.184717123272878\\
235	-0.184717123272878\\
236	-0.184717123272878\\
237	-0.184717123272878\\
238	-0.184717123272878\\
239	-0.184717123272878\\
240	-0.184717123272878\\
241	-0.184717123272878\\
242	-0.184717123272878\\
243	-0.184717123272878\\
244	-0.184717123272878\\
245	-0.184717123272878\\
246	-0.184717123272878\\
247	-0.184717123272878\\
248	-0.184717123272878\\
249	-0.184717123272878\\
250	-0.184717123272878\\
251	-0.184717123272878\\
252	-0.184717123272878\\
253	-0.184717123272878\\
254	-0.184717123272878\\
255	-0.184717123272878\\
256	-0.184717123272878\\
257	-0.184717123272878\\
258	-0.184717123272878\\
259	-0.184717123272878\\
260	-0.184717123272878\\
261	-0.184717123272878\\
262	-0.184717123272878\\
263	-0.184717123272878\\
264	-0.184717123272878\\
265	-0.184717123272878\\
266	-0.184717123272878\\
267	-0.184717123272878\\
268	-0.184717123272878\\
269	-0.184717123272878\\
270	-0.184717123272878\\
271	-0.184717123272878\\
272	-0.184717123272878\\
273	-0.184717123272878\\
274	-0.184717123272878\\
275	-0.184717123272878\\
276	-0.184717123272878\\
277	-0.184717123272878\\
278	-0.184717123272878\\
279	-0.184717123272878\\
280	-0.184717123272878\\
281	-0.184717123272878\\
282	-0.184717123272878\\
283	-0.184717123272878\\
284	-0.184717123272878\\
285	-0.184717123272878\\
286	-0.184717123272878\\
287	-0.184717123272878\\
288	-0.184717123272878\\
289	-0.184717123272878\\
290	-0.184717123272878\\
291	-0.184717123272878\\
292	-0.184717123272878\\
293	-0.184717123272878\\
294	-0.184717123272878\\
295	-0.184717123272878\\
296	-0.184717123272878\\
297	-0.184717123272878\\
298	-0.184717123272878\\
299	-0.184717123272878\\
300	-0.184717123272878\\
301	-0.184717123272878\\
302	-0.184717123272878\\
303	-0.184717123272878\\
304	-0.184717123272878\\
305	-0.184717123272878\\
306	-0.184717123272878\\
307	-0.184717123272878\\
308	-0.184717123272878\\
309	-0.184717123272878\\
310	-0.184717123272878\\
311	-0.184717123272878\\
312	-0.184717123272878\\
313	-0.184717123272878\\
314	-0.184717123272878\\
315	-0.184717123272878\\
316	-0.184717123272878\\
317	-0.184717123272878\\
318	-0.184717123272878\\
319	-0.184717123272878\\
320	-0.184717123272878\\
321	-0.184717123272878\\
322	-0.184717123272878\\
323	-0.184717123272878\\
324	-0.184717123272878\\
325	-0.184717123272878\\
326	-0.184717123272878\\
327	-0.184717123272878\\
328	-0.184717123272878\\
329	-0.184717123272878\\
330	-0.184717123272878\\
331	-0.184717123272878\\
332	-0.184717123272878\\
333	-0.184717123272878\\
334	-0.184717123272878\\
335	-0.184717123272878\\
336	-0.184717123272878\\
337	-0.184717123272878\\
338	-0.184717123272878\\
339	-0.184717123272878\\
340	-0.184717123272878\\
341	-0.184717123272878\\
342	-0.184717123272878\\
343	-0.184717123272878\\
344	-0.184717123272878\\
345	-0.184717123272878\\
346	-0.184717123272878\\
347	-0.184717123272878\\
348	-0.184717123272878\\
349	-0.184717123272878\\
350	-0.184717123272878\\
351	-0.184717123272878\\
352	-0.184717123272878\\
353	-0.184717123272878\\
354	-0.184717123272878\\
355	-0.184717123272878\\
356	-0.184717123272878\\
357	-0.184717123272878\\
358	-0.184717123272878\\
359	-0.184717123272878\\
360	-0.184717123272878\\
361	-0.184717123272878\\
362	-0.184717123272878\\
363	-0.184717123272878\\
364	-0.184717123272878\\
365	-0.184717123272878\\
366	-0.184717123272878\\
367	-0.184717123272878\\
368	-0.184717123272878\\
369	-0.184717123272878\\
370	-0.184717123272878\\
371	-0.184717123272878\\
372	-0.184717123272878\\
373	-0.184717123272878\\
374	-0.184717123272878\\
375	-0.184717123272878\\
376	-0.184717123272878\\
377	-0.184717123272878\\
378	-0.184717123272878\\
379	-0.184717123272878\\
380	-0.184717123272878\\
381	-0.184717123272878\\
382	-0.184717123272878\\
383	-0.184717123272878\\
384	-0.184717123272878\\
385	-0.184717123272878\\
386	-0.184717123272878\\
387	-0.184717123272878\\
388	-0.184717123272878\\
389	-0.184717123272878\\
390	-0.184717123272878\\
391	-0.184717123272878\\
392	-0.184717123272878\\
393	-0.184717123272878\\
394	-0.184717123272878\\
395	-0.184717123272878\\
396	-0.184717123272878\\
397	-0.184717123272878\\
398	-0.184717123272878\\
399	-0.184717123272878\\
400	-0.184717123272878\\
401	-0.184717123272878\\
402	-0.184717123272878\\
403	-0.184717123272878\\
404	-0.184717123272878\\
405	-0.184717123272878\\
406	-0.184717123272878\\
407	-0.184717123272878\\
408	-0.184717123272878\\
409	-0.184717123272878\\
410	-0.184717123272878\\
411	-0.184717123272878\\
412	-0.184717123272878\\
413	-0.184717123272878\\
414	-0.184717123272878\\
415	-0.184717123272878\\
416	-0.184717123272878\\
417	-0.184717123272878\\
418	-0.184717123272878\\
419	-0.184717123272878\\
420	-0.184717123272878\\
421	-0.184717123272878\\
422	-0.184717123272878\\
423	-0.184717123272878\\
424	-0.184717123272878\\
425	-0.184717123272878\\
426	-0.184717123272878\\
427	-0.184717123272878\\
428	-0.184717123272878\\
429	-0.184717123272878\\
430	-0.184717123272878\\
431	-0.184717123272878\\
432	-0.184717123272878\\
433	-0.184717123272878\\
434	-0.184717123272878\\
435	-0.184717123272878\\
436	-0.184717123272878\\
437	-0.184717123272878\\
438	-0.184717123272878\\
439	-0.184717123272878\\
440	-0.184717123272878\\
441	-0.184717123272878\\
442	-0.184717123272878\\
443	-0.184717123272878\\
444	-0.184717123272878\\
445	-0.184717123272878\\
446	-0.184717123272878\\
447	-0.184717123272878\\
448	-0.184717123272878\\
449	-0.184717123272878\\
450	-0.184717123272878\\
451	-0.184717123272878\\
452	-0.184717123272878\\
453	-0.184717123272878\\
454	-0.184717123272878\\
455	-0.184717123272878\\
456	-0.184717123272878\\
457	-0.184717123272878\\
458	-0.184717123272878\\
459	-0.184717123272878\\
460	-0.184717123272878\\
461	-0.184717123272878\\
462	-0.184717123272878\\
463	-0.184717123272878\\
464	-0.184717123272878\\
465	-0.184717123272878\\
466	-0.184717123272878\\
467	-0.184717123272878\\
468	-0.184717123272878\\
469	-0.184717123272878\\
470	-0.184717123272878\\
471	-0.184717123272878\\
472	-0.184717123272878\\
473	-0.184717123272878\\
474	-0.184717123272878\\
475	-0.184717123272878\\
476	-0.184717123272878\\
477	-0.184717123272878\\
478	-0.184717123272878\\
479	-0.184717123272878\\
480	-0.184717123272878\\
481	-0.184717123272878\\
482	-0.184717123272878\\
483	-0.184717123272878\\
484	-0.184717123272878\\
485	-0.184717123272878\\
486	-0.184717123272878\\
487	-0.184717123272878\\
488	-0.184717123272878\\
489	-0.184717123272878\\
490	-0.184717123272878\\
491	-0.184717123272878\\
492	-0.184717123272878\\
493	-0.184717123272878\\
494	-0.184717123272878\\
495	-0.184717123272878\\
496	-0.184717123272878\\
497	-0.184717123272878\\
498	-0.184717123272878\\
499	-0.184717123272878\\
500	-0.184717123272878\\
501	-0.184717123272878\\
502	-0.184717123272878\\
503	-0.184717123272878\\
504	-0.184717123272878\\
505	-0.184717123272878\\
506	-0.184717123272878\\
507	-0.184717123272878\\
508	-0.184717123272878\\
509	-0.184717123272878\\
510	-0.184717123272878\\
511	-0.184717123272878\\
512	-0.184717123272878\\
513	-0.184717123272878\\
514	-0.184717123272878\\
515	-0.184717123272878\\
516	-0.184717123272878\\
517	-0.184717123272878\\
518	-0.184717123272878\\
519	-0.184717123272878\\
520	-0.184717123272878\\
521	-0.184717123272878\\
522	-0.184717123272878\\
523	-0.184717123272878\\
524	-0.184717123272878\\
525	-0.184717123272878\\
526	-0.184717123272878\\
527	-0.184717123272878\\
528	-0.184717123272878\\
529	-0.184717123272878\\
530	-0.184717123272878\\
531	-0.184717123272878\\
532	-0.184717123272878\\
533	-0.184717123272878\\
534	-0.184717123272878\\
535	-0.184717123272878\\
536	-0.184717123272878\\
537	-0.184717123272878\\
538	-0.184717123272878\\
539	-0.184717123272878\\
540	-0.184717123272878\\
541	-0.184717123272878\\
542	-0.184717123272878\\
543	-0.184717123272878\\
544	-0.184717123272878\\
545	-0.184717123272878\\
546	-0.184717123272878\\
547	-0.184717123272878\\
548	-0.184717123272878\\
549	-0.184717123272878\\
550	-0.184717123272878\\
551	-0.184717123272878\\
552	-0.184717123272878\\
553	-0.184717123272878\\
554	-0.184717123272878\\
555	-0.184717123272878\\
556	-0.184717123272878\\
557	-0.184717123272878\\
558	-0.184717123272878\\
559	-0.184717123272878\\
560	-0.184717123272878\\
561	-0.184717123272878\\
562	-0.184717123272878\\
563	-0.184717123272878\\
564	-0.184717123272878\\
565	-0.184717123272878\\
566	-0.184717123272878\\
567	-0.184717123272878\\
568	-0.184717123272878\\
569	-0.184717123272878\\
570	-0.184717123272878\\
571	-0.184717123272878\\
572	-0.184717123272878\\
573	-0.184717123272878\\
574	-0.184717123272878\\
575	-0.184717123272878\\
576	-0.184717123272878\\
577	-0.184717123272878\\
578	-0.184717123272878\\
579	-0.184717123272878\\
580	-0.184717123272878\\
581	-0.184717123272878\\
582	-0.184717123272878\\
583	-0.184717123272878\\
584	-0.184717123272878\\
585	-0.184717123272878\\
586	-0.184717123272878\\
587	-0.184717123272878\\
588	-0.184717123272878\\
589	-0.184717123272878\\
590	-0.184717123272878\\
591	-0.184717123272878\\
592	-0.184717123272878\\
593	-0.184717123272878\\
594	-0.184717123272878\\
595	-0.184717123272878\\
596	-0.184717123272878\\
597	-0.184717123272878\\
598	-0.184717123272878\\
599	-0.184717123272878\\
600	-0.184717123272878\\
};
\addplot [color=mycolor2, forget plot]
  table[row sep=crcr]{%
1	0\\
2	0\\
3	0\\
4	0\\
5	0\\
6	0\\
7	0\\
8	0\\
9	0\\
10	0\\
11	0\\
12	0\\
13	0\\
14	0\\
15	0\\
16	0\\
17	0\\
18	0\\
19	0\\
20	0\\
21	-0.0252046842105263\\
22	-0.0604556833173869\\
23	-0.0863237400665711\\
24	-0.102730079403429\\
25	-0.112595682984572\\
26	-0.118397283583258\\
27	-0.12177552896831\\
28	-0.123733916018456\\
29	-0.124866893099182\\
30	-0.125521736547209\\
31	-0.125900063183506\\
32	-0.126118592772882\\
33	-0.126244808624652\\
34	-0.126317703880897\\
35	-0.126359803309476\\
36	-0.126384116908441\\
37	-0.126398158634802\\
38	-0.126406268076129\\
39	-0.126410951473261\\
40	-0.126413656246494\\
41	-0.126415218317075\\
42	-0.126416120449741\\
43	-0.12641664145265\\
44	-0.126416942344122\\
45	-0.126417116116045\\
46	-0.126417216473428\\
47	-0.126417274432185\\
48	-0.126417307904733\\
49	-0.126417327235919\\
50	-0.126417338400135\\
51	-0.126417344847733\\
52	-0.126417348571373\\
53	-0.126417350721863\\
54	-0.126417351963822\\
55	-0.126417352681082\\
56	-0.126417353095317\\
57	-0.126417353334547\\
58	-0.126417353472709\\
59	-0.1264173535525\\
60	-0.126417353598581\\
61	-0.126417353625194\\
62	-0.126417353640564\\
63	-0.12641735364944\\
64	-0.126417353654567\\
65	-0.126417353657527\\
66	-0.126417353659237\\
67	-0.126417353660224\\
68	-0.126417353660795\\
69	-0.126417353661124\\
70	-0.126417353661314\\
71	-0.126417353661424\\
72	-0.126417353661488\\
73	-0.126417353661524\\
74	-0.126417353661545\\
75	-0.126417353661558\\
76	-0.126417353661565\\
77	-0.126417353661569\\
78	-0.126417353661571\\
79	-0.126417353661573\\
80	-0.126417353661573\\
81	-0.126417353661574\\
82	-0.126417353661574\\
83	-0.126417353661574\\
84	-0.126417353661574\\
85	-0.126417353661574\\
86	-0.126417353661574\\
87	-0.126417353661574\\
88	-0.126417353661574\\
89	-0.126417353661574\\
90	-0.126417353661574\\
91	-0.126417353661574\\
92	-0.126417353661574\\
93	-0.126417353661574\\
94	-0.126417353661574\\
95	-0.126417353661574\\
96	-0.126417353661574\\
97	-0.126417353661574\\
98	-0.126417353661574\\
99	-0.126417353661574\\
100	-0.126417353661574\\
101	-0.126417353661574\\
102	-0.126417353661574\\
103	-0.126417353661574\\
104	-0.126417353661574\\
105	-0.126417353661574\\
106	-0.126417353661574\\
107	-0.126417353661574\\
108	-0.126417353661574\\
109	-0.126417353661574\\
110	-0.126417353661574\\
111	-0.126417353661574\\
112	-0.126417353661574\\
113	-0.126417353661574\\
114	-0.126417353661574\\
115	-0.126417353661574\\
116	-0.126417353661574\\
117	-0.126417353661574\\
118	-0.126417353661574\\
119	-0.126417353661574\\
120	-0.126417353661574\\
121	-0.126417353661574\\
122	-0.126417353661574\\
123	-0.126417353661574\\
124	-0.126417353661574\\
125	-0.126417353661574\\
126	-0.126417353661574\\
127	-0.126417353661574\\
128	-0.126417353661574\\
129	-0.126417353661574\\
130	-0.126417353661574\\
131	-0.126417353661574\\
132	-0.126417353661574\\
133	-0.126417353661574\\
134	-0.126417353661574\\
135	-0.126417353661574\\
136	-0.126417353661574\\
137	-0.126417353661574\\
138	-0.126417353661574\\
139	-0.126417353661574\\
140	-0.126417353661574\\
141	-0.126417353661574\\
142	-0.126417353661574\\
143	-0.126417353661574\\
144	-0.126417353661574\\
145	-0.126417353661574\\
146	-0.126417353661574\\
147	-0.126417353661574\\
148	-0.126417353661574\\
149	-0.126417353661574\\
150	-0.126417353661574\\
151	-0.126417353661574\\
152	-0.126417353661574\\
153	-0.126417353661574\\
154	-0.126417353661574\\
155	-0.126417353661574\\
156	-0.126417353661574\\
157	-0.126417353661574\\
158	-0.126417353661574\\
159	-0.126417353661574\\
160	-0.126417353661574\\
161	-0.126417353661574\\
162	-0.126417353661574\\
163	-0.126417353661574\\
164	-0.126417353661574\\
165	-0.126417353661574\\
166	-0.126417353661574\\
167	-0.126417353661574\\
168	-0.126417353661574\\
169	-0.126417353661574\\
170	-0.126417353661574\\
171	-0.126417353661574\\
172	-0.126417353661574\\
173	-0.126417353661574\\
174	-0.126417353661574\\
175	-0.126417353661574\\
176	-0.126417353661574\\
177	-0.126417353661574\\
178	-0.126417353661574\\
179	-0.126417353661574\\
180	-0.126417353661574\\
181	-0.126417353661574\\
182	-0.126417353661574\\
183	-0.126417353661574\\
184	-0.126417353661574\\
185	-0.126417353661574\\
186	-0.126417353661574\\
187	-0.126417353661574\\
188	-0.126417353661574\\
189	-0.126417353661574\\
190	-0.126417353661574\\
191	-0.126417353661574\\
192	-0.126417353661574\\
193	-0.126417353661574\\
194	-0.126417353661574\\
195	-0.126417353661574\\
196	-0.126417353661574\\
197	-0.126417353661574\\
198	-0.126417353661574\\
199	-0.126417353661574\\
200	-0.126417353661574\\
201	-0.126417353661574\\
202	-0.126417353661574\\
203	-0.126417353661574\\
204	-0.126417353661574\\
205	-0.126417353661574\\
206	-0.126417353661574\\
207	-0.126417353661574\\
208	-0.126417353661574\\
209	-0.126417353661574\\
210	-0.126417353661574\\
211	-0.126417353661574\\
212	-0.126417353661574\\
213	-0.126417353661574\\
214	-0.126417353661574\\
215	-0.126417353661574\\
216	-0.126417353661574\\
217	-0.126417353661574\\
218	-0.126417353661574\\
219	-0.126417353661574\\
220	-0.126417353661574\\
221	-0.126417353661574\\
222	-0.126417353661574\\
223	-0.126417353661574\\
224	-0.126417353661574\\
225	-0.126417353661574\\
226	-0.126417353661574\\
227	-0.126417353661574\\
228	-0.126417353661574\\
229	-0.126417353661574\\
230	-0.126417353661574\\
231	-0.126417353661574\\
232	-0.126417353661574\\
233	-0.126417353661574\\
234	-0.126417353661574\\
235	-0.126417353661574\\
236	-0.126417353661574\\
237	-0.126417353661574\\
238	-0.126417353661574\\
239	-0.126417353661574\\
240	-0.126417353661574\\
241	-0.126417353661574\\
242	-0.126417353661574\\
243	-0.126417353661574\\
244	-0.126417353661574\\
245	-0.126417353661574\\
246	-0.126417353661574\\
247	-0.126417353661574\\
248	-0.126417353661574\\
249	-0.126417353661574\\
250	-0.126417353661574\\
251	-0.126417353661574\\
252	-0.126417353661574\\
253	-0.126417353661574\\
254	-0.126417353661574\\
255	-0.126417353661574\\
256	-0.126417353661574\\
257	-0.126417353661574\\
258	-0.126417353661574\\
259	-0.126417353661574\\
260	-0.126417353661574\\
261	-0.126417353661574\\
262	-0.126417353661574\\
263	-0.126417353661574\\
264	-0.126417353661574\\
265	-0.126417353661574\\
266	-0.126417353661574\\
267	-0.126417353661574\\
268	-0.126417353661574\\
269	-0.126417353661574\\
270	-0.126417353661574\\
271	-0.126417353661574\\
272	-0.126417353661574\\
273	-0.126417353661574\\
274	-0.126417353661574\\
275	-0.126417353661574\\
276	-0.126417353661574\\
277	-0.126417353661574\\
278	-0.126417353661574\\
279	-0.126417353661574\\
280	-0.126417353661574\\
281	-0.126417353661574\\
282	-0.126417353661574\\
283	-0.126417353661574\\
284	-0.126417353661574\\
285	-0.126417353661574\\
286	-0.126417353661574\\
287	-0.126417353661574\\
288	-0.126417353661574\\
289	-0.126417353661574\\
290	-0.126417353661574\\
291	-0.126417353661574\\
292	-0.126417353661574\\
293	-0.126417353661574\\
294	-0.126417353661574\\
295	-0.126417353661574\\
296	-0.126417353661574\\
297	-0.126417353661574\\
298	-0.126417353661574\\
299	-0.126417353661574\\
300	-0.126417353661574\\
301	-0.126417353661574\\
302	-0.126417353661574\\
303	-0.126417353661574\\
304	-0.126417353661574\\
305	-0.126417353661574\\
306	-0.126417353661574\\
307	-0.126417353661574\\
308	-0.126417353661574\\
309	-0.126417353661574\\
310	-0.126417353661574\\
311	-0.126417353661574\\
312	-0.126417353661574\\
313	-0.126417353661574\\
314	-0.126417353661574\\
315	-0.126417353661574\\
316	-0.126417353661574\\
317	-0.126417353661574\\
318	-0.126417353661574\\
319	-0.126417353661574\\
320	-0.126417353661574\\
321	-0.126417353661574\\
322	-0.126417353661574\\
323	-0.126417353661574\\
324	-0.126417353661574\\
325	-0.126417353661574\\
326	-0.126417353661574\\
327	-0.126417353661574\\
328	-0.126417353661574\\
329	-0.126417353661574\\
330	-0.126417353661574\\
331	-0.126417353661574\\
332	-0.126417353661574\\
333	-0.126417353661574\\
334	-0.126417353661574\\
335	-0.126417353661574\\
336	-0.126417353661574\\
337	-0.126417353661574\\
338	-0.126417353661574\\
339	-0.126417353661574\\
340	-0.126417353661574\\
341	-0.126417353661574\\
342	-0.126417353661574\\
343	-0.126417353661574\\
344	-0.126417353661574\\
345	-0.126417353661574\\
346	-0.126417353661574\\
347	-0.126417353661574\\
348	-0.126417353661574\\
349	-0.126417353661574\\
350	-0.126417353661574\\
351	-0.126417353661574\\
352	-0.126417353661574\\
353	-0.126417353661574\\
354	-0.126417353661574\\
355	-0.126417353661574\\
356	-0.126417353661574\\
357	-0.126417353661574\\
358	-0.126417353661574\\
359	-0.126417353661574\\
360	-0.126417353661574\\
361	-0.126417353661574\\
362	-0.126417353661574\\
363	-0.126417353661574\\
364	-0.126417353661574\\
365	-0.126417353661574\\
366	-0.126417353661574\\
367	-0.126417353661574\\
368	-0.126417353661574\\
369	-0.126417353661574\\
370	-0.126417353661574\\
371	-0.126417353661574\\
372	-0.126417353661574\\
373	-0.126417353661574\\
374	-0.126417353661574\\
375	-0.126417353661574\\
376	-0.126417353661574\\
377	-0.126417353661574\\
378	-0.126417353661574\\
379	-0.126417353661574\\
380	-0.126417353661574\\
381	-0.126417353661574\\
382	-0.126417353661574\\
383	-0.126417353661574\\
384	-0.126417353661574\\
385	-0.126417353661574\\
386	-0.126417353661574\\
387	-0.126417353661574\\
388	-0.126417353661574\\
389	-0.126417353661574\\
390	-0.126417353661574\\
391	-0.126417353661574\\
392	-0.126417353661574\\
393	-0.126417353661574\\
394	-0.126417353661574\\
395	-0.126417353661574\\
396	-0.126417353661574\\
397	-0.126417353661574\\
398	-0.126417353661574\\
399	-0.126417353661574\\
400	-0.126417353661574\\
401	-0.126417353661574\\
402	-0.126417353661574\\
403	-0.126417353661574\\
404	-0.126417353661574\\
405	-0.126417353661574\\
406	-0.126417353661574\\
407	-0.126417353661574\\
408	-0.126417353661574\\
409	-0.126417353661574\\
410	-0.126417353661574\\
411	-0.126417353661574\\
412	-0.126417353661574\\
413	-0.126417353661574\\
414	-0.126417353661574\\
415	-0.126417353661574\\
416	-0.126417353661574\\
417	-0.126417353661574\\
418	-0.126417353661574\\
419	-0.126417353661574\\
420	-0.126417353661574\\
421	-0.126417353661574\\
422	-0.126417353661574\\
423	-0.126417353661574\\
424	-0.126417353661574\\
425	-0.126417353661574\\
426	-0.126417353661574\\
427	-0.126417353661574\\
428	-0.126417353661574\\
429	-0.126417353661574\\
430	-0.126417353661574\\
431	-0.126417353661574\\
432	-0.126417353661574\\
433	-0.126417353661574\\
434	-0.126417353661574\\
435	-0.126417353661574\\
436	-0.126417353661574\\
437	-0.126417353661574\\
438	-0.126417353661574\\
439	-0.126417353661574\\
440	-0.126417353661574\\
441	-0.126417353661574\\
442	-0.126417353661574\\
443	-0.126417353661574\\
444	-0.126417353661574\\
445	-0.126417353661574\\
446	-0.126417353661574\\
447	-0.126417353661574\\
448	-0.126417353661574\\
449	-0.126417353661574\\
450	-0.126417353661574\\
451	-0.126417353661574\\
452	-0.126417353661574\\
453	-0.126417353661574\\
454	-0.126417353661574\\
455	-0.126417353661574\\
456	-0.126417353661574\\
457	-0.126417353661574\\
458	-0.126417353661574\\
459	-0.126417353661574\\
460	-0.126417353661574\\
461	-0.126417353661574\\
462	-0.126417353661574\\
463	-0.126417353661574\\
464	-0.126417353661574\\
465	-0.126417353661574\\
466	-0.126417353661574\\
467	-0.126417353661574\\
468	-0.126417353661574\\
469	-0.126417353661574\\
470	-0.126417353661574\\
471	-0.126417353661574\\
472	-0.126417353661574\\
473	-0.126417353661574\\
474	-0.126417353661574\\
475	-0.126417353661574\\
476	-0.126417353661574\\
477	-0.126417353661574\\
478	-0.126417353661574\\
479	-0.126417353661574\\
480	-0.126417353661574\\
481	-0.126417353661574\\
482	-0.126417353661574\\
483	-0.126417353661574\\
484	-0.126417353661574\\
485	-0.126417353661574\\
486	-0.126417353661574\\
487	-0.126417353661574\\
488	-0.126417353661574\\
489	-0.126417353661574\\
490	-0.126417353661574\\
491	-0.126417353661574\\
492	-0.126417353661574\\
493	-0.126417353661574\\
494	-0.126417353661574\\
495	-0.126417353661574\\
496	-0.126417353661574\\
497	-0.126417353661574\\
498	-0.126417353661574\\
499	-0.126417353661574\\
500	-0.126417353661574\\
501	-0.126417353661574\\
502	-0.126417353661574\\
503	-0.126417353661574\\
504	-0.126417353661574\\
505	-0.126417353661574\\
506	-0.126417353661574\\
507	-0.126417353661574\\
508	-0.126417353661574\\
509	-0.126417353661574\\
510	-0.126417353661574\\
511	-0.126417353661574\\
512	-0.126417353661574\\
513	-0.126417353661574\\
514	-0.126417353661574\\
515	-0.126417353661574\\
516	-0.126417353661574\\
517	-0.126417353661574\\
518	-0.126417353661574\\
519	-0.126417353661574\\
520	-0.126417353661574\\
521	-0.126417353661574\\
522	-0.126417353661574\\
523	-0.126417353661574\\
524	-0.126417353661574\\
525	-0.126417353661574\\
526	-0.126417353661574\\
527	-0.126417353661574\\
528	-0.126417353661574\\
529	-0.126417353661574\\
530	-0.126417353661574\\
531	-0.126417353661574\\
532	-0.126417353661574\\
533	-0.126417353661574\\
534	-0.126417353661574\\
535	-0.126417353661574\\
536	-0.126417353661574\\
537	-0.126417353661574\\
538	-0.126417353661574\\
539	-0.126417353661574\\
540	-0.126417353661574\\
541	-0.126417353661574\\
542	-0.126417353661574\\
543	-0.126417353661574\\
544	-0.126417353661574\\
545	-0.126417353661574\\
546	-0.126417353661574\\
547	-0.126417353661574\\
548	-0.126417353661574\\
549	-0.126417353661574\\
550	-0.126417353661574\\
551	-0.126417353661574\\
552	-0.126417353661574\\
553	-0.126417353661574\\
554	-0.126417353661574\\
555	-0.126417353661574\\
556	-0.126417353661574\\
557	-0.126417353661574\\
558	-0.126417353661574\\
559	-0.126417353661574\\
560	-0.126417353661574\\
561	-0.126417353661574\\
562	-0.126417353661574\\
563	-0.126417353661574\\
564	-0.126417353661574\\
565	-0.126417353661574\\
566	-0.126417353661574\\
567	-0.126417353661574\\
568	-0.126417353661574\\
569	-0.126417353661574\\
570	-0.126417353661574\\
571	-0.126417353661574\\
572	-0.126417353661574\\
573	-0.126417353661574\\
574	-0.126417353661574\\
575	-0.126417353661574\\
576	-0.126417353661574\\
577	-0.126417353661574\\
578	-0.126417353661574\\
579	-0.126417353661574\\
580	-0.126417353661574\\
581	-0.126417353661574\\
582	-0.126417353661574\\
583	-0.126417353661574\\
584	-0.126417353661574\\
585	-0.126417353661574\\
586	-0.126417353661574\\
587	-0.126417353661574\\
588	-0.126417353661574\\
589	-0.126417353661574\\
590	-0.126417353661574\\
591	-0.126417353661574\\
592	-0.126417353661574\\
593	-0.126417353661574\\
594	-0.126417353661574\\
595	-0.126417353661574\\
596	-0.126417353661574\\
597	-0.126417353661574\\
598	-0.126417353661574\\
599	-0.126417353661574\\
600	-0.126417353661574\\
};
\addplot [color=mycolor3, forget plot]
  table[row sep=crcr]{%
1	0\\
2	0\\
3	0\\
4	0\\
5	0\\
6	0\\
7	0\\
8	0\\
9	0\\
10	0\\
11	0\\
12	0\\
13	0\\
14	0\\
15	0\\
16	0\\
17	0\\
18	0\\
19	0\\
20	0\\
21	0.0512173157894737\\
22	0.176742791570018\\
23	0.341366642586894\\
24	0.519742077032553\\
25	0.694618438734639\\
26	0.855144855390695\\
27	0.995336597508446\\
28	1.11275218214733\\
29	1.2073975279732\\
30	1.28085193755673\\
31	1.3355971991368\\
32	1.37452372382608\\
33	1.40058469574855\\
34	1.41656933111104\\
35	1.42496841831836\\
36	1.42790850252487\\
37	1.42713476560097\\
38	1.42402640407659\\
39	1.41963184078399\\
40	1.41471425416449\\
41	1.40980059209409\\
42	1.40522943509971\\
43	1.40119480797385\\
44	1.39778435464923\\
45	1.39501124702701\\
46	1.39283985663975\\
47	1.39120563927696\\
48	1.39002992243749\\
49	1.38923039190259\\
50	1.38872808706589\\
51	1.38845166721706\\
52	1.38833962773262\\
53	1.388341044621\\
54	1.38841532121477\\
55	1.38853131072352\\
56	1.38866609814765\\
57	1.38880364742587\\
58	1.38893345552699\\
59	1.38904930413203\\
60	1.38914816044385\\
61	1.38922924994691\\
62	1.3892933038956\\
63	1.38934197122842\\
64	1.38937737690577\\
65	1.38940180496412\\
66	1.3894174836984\\
67	1.38942645138108\\
68	1.38943048306115\\
69	1.38943106170507\\
70	1.38942937984866\\
71	1.38942636075447\\
72	1.38942269064827\\
73	1.38941885584946\\
74	1.38941518047935\\
75	1.38941186193202\\
76	1.38940900245214\\
77	1.38940663602445\\
78	1.38940475038793\\
79	1.38940330439171\\
80	1.38940224115472\\
81	1.38940149761624\\
82	1.38940101110503\\
83	1.38940072353655\\
84	1.38940058379471\\
85	1.38940054878117\\
86	1.38940058353544\\
87	1.38940066074898\\
88	1.38940075992333\\
89	1.38940086635753\\
90	1.38940097009585\\
91	1.38940106492284\\
92	1.38940114745824\\
93	1.38940121637888\\
94	1.38940127177589\\
95	1.3894013146436\\
96	1.38940134648833\\
97	1.38940136904139\\
98	1.38940138405901\\
99	1.3894013931919\\
100	1.38940139790873\\
101	1.38940139945945\\
102	1.38940139886693\\
103	1.3894013969372\\
104	1.3894013942811\\
105	1.38940139134169\\
106	1.38940138842348\\
107	1.38940138572082\\
108	1.38940138334376\\
109	1.38940138134055\\
110	1.38940137971635\\
111	1.38940137844825\\
112	1.38940137749688\\
113	1.38940137681501\\
114	1.38940137635362\\
115	1.38940137606593\\
116	1.38940137590984\\
117	1.38940137584919\\
118	1.38940137585409\\
119	1.38940137590081\\
120	1.3894013759712\\
121	1.38940137605192\\
122	1.38940137613372\\
123	1.38940137621054\\
124	1.38940137627884\\
125	1.38940137633695\\
126	1.38940137638447\\
127	1.38940137642191\\
128	1.38940137645026\\
129	1.38940137647081\\
130	1.38940137648493\\
131	1.38940137649393\\
132	1.38940137649901\\
133	1.38940137650123\\
134	1.38940137650145\\
135	1.38940137650037\\
136	1.38940137649853\\
137	1.38940137649633\\
138	1.38940137649404\\
139	1.38940137649187\\
140	1.38940137648991\\
141	1.38940137648823\\
142	1.38940137648684\\
143	1.38940137648574\\
144	1.38940137648489\\
145	1.38940137648428\\
146	1.38940137648385\\
147	1.38940137648357\\
148	1.3894013764834\\
149	1.38940137648332\\
150	1.38940137648331\\
151	1.38940137648333\\
152	1.38940137648338\\
153	1.38940137648344\\
154	1.3894013764835\\
155	1.38940137648356\\
156	1.38940137648362\\
157	1.38940137648367\\
158	1.38940137648371\\
159	1.38940137648374\\
160	1.38940137648376\\
161	1.38940137648378\\
162	1.3894013764838\\
163	1.3894013764838\\
164	1.38940137648381\\
165	1.38940137648381\\
166	1.38940137648381\\
167	1.38940137648381\\
168	1.38940137648381\\
169	1.38940137648381\\
170	1.38940137648381\\
171	1.38940137648381\\
172	1.3894013764838\\
173	1.3894013764838\\
174	1.3894013764838\\
175	1.3894013764838\\
176	1.3894013764838\\
177	1.3894013764838\\
178	1.3894013764838\\
179	1.3894013764838\\
180	1.3894013764838\\
181	1.3894013764838\\
182	1.3894013764838\\
183	1.3894013764838\\
184	1.3894013764838\\
185	1.3894013764838\\
186	1.3894013764838\\
187	1.3894013764838\\
188	1.3894013764838\\
189	1.3894013764838\\
190	1.3894013764838\\
191	1.3894013764838\\
192	1.3894013764838\\
193	1.3894013764838\\
194	1.3894013764838\\
195	1.3894013764838\\
196	1.3894013764838\\
197	1.3894013764838\\
198	1.3894013764838\\
199	1.3894013764838\\
200	1.3894013764838\\
201	1.3894013764838\\
202	1.3894013764838\\
203	1.3894013764838\\
204	1.3894013764838\\
205	1.3894013764838\\
206	1.3894013764838\\
207	1.3894013764838\\
208	1.3894013764838\\
209	1.3894013764838\\
210	1.3894013764838\\
211	1.3894013764838\\
212	1.3894013764838\\
213	1.3894013764838\\
214	1.3894013764838\\
215	1.3894013764838\\
216	1.3894013764838\\
217	1.3894013764838\\
218	1.3894013764838\\
219	1.3894013764838\\
220	1.3894013764838\\
221	1.3894013764838\\
222	1.3894013764838\\
223	1.3894013764838\\
224	1.3894013764838\\
225	1.3894013764838\\
226	1.3894013764838\\
227	1.3894013764838\\
228	1.3894013764838\\
229	1.3894013764838\\
230	1.3894013764838\\
231	1.3894013764838\\
232	1.3894013764838\\
233	1.3894013764838\\
234	1.3894013764838\\
235	1.3894013764838\\
236	1.3894013764838\\
237	1.3894013764838\\
238	1.3894013764838\\
239	1.3894013764838\\
240	1.3894013764838\\
241	1.3894013764838\\
242	1.3894013764838\\
243	1.3894013764838\\
244	1.3894013764838\\
245	1.3894013764838\\
246	1.3894013764838\\
247	1.3894013764838\\
248	1.3894013764838\\
249	1.3894013764838\\
250	1.3894013764838\\
251	1.3894013764838\\
252	1.3894013764838\\
253	1.3894013764838\\
254	1.3894013764838\\
255	1.3894013764838\\
256	1.3894013764838\\
257	1.3894013764838\\
258	1.3894013764838\\
259	1.3894013764838\\
260	1.3894013764838\\
261	1.3894013764838\\
262	1.3894013764838\\
263	1.3894013764838\\
264	1.3894013764838\\
265	1.3894013764838\\
266	1.3894013764838\\
267	1.3894013764838\\
268	1.3894013764838\\
269	1.3894013764838\\
270	1.3894013764838\\
271	1.3894013764838\\
272	1.3894013764838\\
273	1.3894013764838\\
274	1.3894013764838\\
275	1.3894013764838\\
276	1.3894013764838\\
277	1.3894013764838\\
278	1.3894013764838\\
279	1.3894013764838\\
280	1.3894013764838\\
281	1.3894013764838\\
282	1.3894013764838\\
283	1.3894013764838\\
284	1.3894013764838\\
285	1.3894013764838\\
286	1.3894013764838\\
287	1.3894013764838\\
288	1.3894013764838\\
289	1.3894013764838\\
290	1.3894013764838\\
291	1.3894013764838\\
292	1.3894013764838\\
293	1.3894013764838\\
294	1.3894013764838\\
295	1.3894013764838\\
296	1.3894013764838\\
297	1.3894013764838\\
298	1.3894013764838\\
299	1.3894013764838\\
300	1.3894013764838\\
301	1.3894013764838\\
302	1.3894013764838\\
303	1.3894013764838\\
304	1.3894013764838\\
305	1.3894013764838\\
306	1.3894013764838\\
307	1.3894013764838\\
308	1.3894013764838\\
309	1.3894013764838\\
310	1.3894013764838\\
311	1.3894013764838\\
312	1.3894013764838\\
313	1.3894013764838\\
314	1.3894013764838\\
315	1.3894013764838\\
316	1.3894013764838\\
317	1.3894013764838\\
318	1.3894013764838\\
319	1.3894013764838\\
320	1.3894013764838\\
321	1.3894013764838\\
322	1.3894013764838\\
323	1.3894013764838\\
324	1.3894013764838\\
325	1.3894013764838\\
326	1.3894013764838\\
327	1.3894013764838\\
328	1.3894013764838\\
329	1.3894013764838\\
330	1.3894013764838\\
331	1.3894013764838\\
332	1.3894013764838\\
333	1.3894013764838\\
334	1.3894013764838\\
335	1.3894013764838\\
336	1.3894013764838\\
337	1.3894013764838\\
338	1.3894013764838\\
339	1.3894013764838\\
340	1.3894013764838\\
341	1.3894013764838\\
342	1.3894013764838\\
343	1.3894013764838\\
344	1.3894013764838\\
345	1.3894013764838\\
346	1.3894013764838\\
347	1.3894013764838\\
348	1.3894013764838\\
349	1.3894013764838\\
350	1.3894013764838\\
351	1.3894013764838\\
352	1.3894013764838\\
353	1.3894013764838\\
354	1.3894013764838\\
355	1.3894013764838\\
356	1.3894013764838\\
357	1.3894013764838\\
358	1.3894013764838\\
359	1.3894013764838\\
360	1.3894013764838\\
361	1.3894013764838\\
362	1.3894013764838\\
363	1.3894013764838\\
364	1.3894013764838\\
365	1.3894013764838\\
366	1.3894013764838\\
367	1.3894013764838\\
368	1.3894013764838\\
369	1.3894013764838\\
370	1.3894013764838\\
371	1.3894013764838\\
372	1.3894013764838\\
373	1.3894013764838\\
374	1.3894013764838\\
375	1.3894013764838\\
376	1.3894013764838\\
377	1.3894013764838\\
378	1.3894013764838\\
379	1.3894013764838\\
380	1.3894013764838\\
381	1.3894013764838\\
382	1.3894013764838\\
383	1.3894013764838\\
384	1.3894013764838\\
385	1.3894013764838\\
386	1.3894013764838\\
387	1.3894013764838\\
388	1.3894013764838\\
389	1.3894013764838\\
390	1.3894013764838\\
391	1.3894013764838\\
392	1.3894013764838\\
393	1.3894013764838\\
394	1.3894013764838\\
395	1.3894013764838\\
396	1.3894013764838\\
397	1.3894013764838\\
398	1.3894013764838\\
399	1.3894013764838\\
400	1.3894013764838\\
401	1.3894013764838\\
402	1.3894013764838\\
403	1.3894013764838\\
404	1.3894013764838\\
405	1.3894013764838\\
406	1.3894013764838\\
407	1.3894013764838\\
408	1.3894013764838\\
409	1.3894013764838\\
410	1.3894013764838\\
411	1.3894013764838\\
412	1.3894013764838\\
413	1.3894013764838\\
414	1.3894013764838\\
415	1.3894013764838\\
416	1.3894013764838\\
417	1.3894013764838\\
418	1.3894013764838\\
419	1.3894013764838\\
420	1.3894013764838\\
421	1.3894013764838\\
422	1.3894013764838\\
423	1.3894013764838\\
424	1.3894013764838\\
425	1.3894013764838\\
426	1.3894013764838\\
427	1.3894013764838\\
428	1.3894013764838\\
429	1.3894013764838\\
430	1.3894013764838\\
431	1.3894013764838\\
432	1.3894013764838\\
433	1.3894013764838\\
434	1.3894013764838\\
435	1.3894013764838\\
436	1.3894013764838\\
437	1.3894013764838\\
438	1.3894013764838\\
439	1.3894013764838\\
440	1.3894013764838\\
441	1.3894013764838\\
442	1.3894013764838\\
443	1.3894013764838\\
444	1.3894013764838\\
445	1.3894013764838\\
446	1.3894013764838\\
447	1.3894013764838\\
448	1.3894013764838\\
449	1.3894013764838\\
450	1.3894013764838\\
451	1.3894013764838\\
452	1.3894013764838\\
453	1.3894013764838\\
454	1.3894013764838\\
455	1.3894013764838\\
456	1.3894013764838\\
457	1.3894013764838\\
458	1.3894013764838\\
459	1.3894013764838\\
460	1.3894013764838\\
461	1.3894013764838\\
462	1.3894013764838\\
463	1.3894013764838\\
464	1.3894013764838\\
465	1.3894013764838\\
466	1.3894013764838\\
467	1.3894013764838\\
468	1.3894013764838\\
469	1.3894013764838\\
470	1.3894013764838\\
471	1.3894013764838\\
472	1.3894013764838\\
473	1.3894013764838\\
474	1.3894013764838\\
475	1.3894013764838\\
476	1.3894013764838\\
477	1.3894013764838\\
478	1.3894013764838\\
479	1.3894013764838\\
480	1.3894013764838\\
481	1.3894013764838\\
482	1.3894013764838\\
483	1.3894013764838\\
484	1.3894013764838\\
485	1.3894013764838\\
486	1.3894013764838\\
487	1.3894013764838\\
488	1.3894013764838\\
489	1.3894013764838\\
490	1.3894013764838\\
491	1.3894013764838\\
492	1.3894013764838\\
493	1.3894013764838\\
494	1.3894013764838\\
495	1.3894013764838\\
496	1.3894013764838\\
497	1.3894013764838\\
498	1.3894013764838\\
499	1.3894013764838\\
500	1.3894013764838\\
501	1.3894013764838\\
502	1.3894013764838\\
503	1.3894013764838\\
504	1.3894013764838\\
505	1.3894013764838\\
506	1.3894013764838\\
507	1.3894013764838\\
508	1.3894013764838\\
509	1.3894013764838\\
510	1.3894013764838\\
511	1.3894013764838\\
512	1.3894013764838\\
513	1.3894013764838\\
514	1.3894013764838\\
515	1.3894013764838\\
516	1.3894013764838\\
517	1.3894013764838\\
518	1.3894013764838\\
519	1.3894013764838\\
520	1.3894013764838\\
521	1.3894013764838\\
522	1.3894013764838\\
523	1.3894013764838\\
524	1.3894013764838\\
525	1.3894013764838\\
526	1.3894013764838\\
527	1.3894013764838\\
528	1.3894013764838\\
529	1.3894013764838\\
530	1.3894013764838\\
531	1.3894013764838\\
532	1.3894013764838\\
533	1.3894013764838\\
534	1.3894013764838\\
535	1.3894013764838\\
536	1.3894013764838\\
537	1.3894013764838\\
538	1.3894013764838\\
539	1.3894013764838\\
540	1.3894013764838\\
541	1.3894013764838\\
542	1.3894013764838\\
543	1.3894013764838\\
544	1.3894013764838\\
545	1.3894013764838\\
546	1.3894013764838\\
547	1.3894013764838\\
548	1.3894013764838\\
549	1.3894013764838\\
550	1.3894013764838\\
551	1.3894013764838\\
552	1.3894013764838\\
553	1.3894013764838\\
554	1.3894013764838\\
555	1.3894013764838\\
556	1.3894013764838\\
557	1.3894013764838\\
558	1.3894013764838\\
559	1.3894013764838\\
560	1.3894013764838\\
561	1.3894013764838\\
562	1.3894013764838\\
563	1.3894013764838\\
564	1.3894013764838\\
565	1.3894013764838\\
566	1.3894013764838\\
567	1.3894013764838\\
568	1.3894013764838\\
569	1.3894013764838\\
570	1.3894013764838\\
571	1.3894013764838\\
572	1.3894013764838\\
573	1.3894013764838\\
574	1.3894013764838\\
575	1.3894013764838\\
576	1.3894013764838\\
577	1.3894013764838\\
578	1.3894013764838\\
579	1.3894013764838\\
580	1.3894013764838\\
581	1.3894013764838\\
582	1.3894013764838\\
583	1.3894013764838\\
584	1.3894013764838\\
585	1.3894013764838\\
586	1.3894013764838\\
587	1.3894013764838\\
588	1.3894013764838\\
589	1.3894013764838\\
590	1.3894013764838\\
591	1.3894013764838\\
592	1.3894013764838\\
593	1.3894013764838\\
594	1.3894013764838\\
595	1.3894013764838\\
596	1.3894013764838\\
597	1.3894013764838\\
598	1.3894013764838\\
599	1.3894013764838\\
600	1.3894013764838\\
};
\addplot [color=mycolor4, forget plot]
  table[row sep=crcr]{%
1	0\\
2	0\\
3	0\\
4	0\\
5	0\\
6	0\\
7	0\\
8	0\\
9	0\\
10	0\\
11	0\\
12	0\\
13	0\\
14	0\\
15	0\\
16	0\\
17	0\\
18	0\\
19	0\\
20	0\\
21	0.109371333333333\\
22	0.390235160457037\\
23	0.782149043815645\\
24	1.23818437216075\\
25	1.72272446004133\\
26	2.20948548675422\\
27	2.67976866005456\\
28	3.12094096913245\\
29	3.52513418323437\\
30	3.88814667633855\\
31	4.20852963487684\\
32	4.48683774799088\\
33	4.72502418546716\\
34	4.92596020891683\\
35	5.09306087339802\\
36	5.22999975166667\\
37	5.3404972908916\\
38	5.42816917064799\\
39	5.49642278235639\\
40	5.54839163133954\\
41	5.58689903151007\\
42	5.61444389392049\\
43	5.63320269107431\\
44	5.64504280543934\\
45	5.65154344620621\\
46	5.65402115084907\\
47	5.6535575883391\\
48	5.65102796153735\\
49	5.64712878071068\\
50	5.6424041616784\\
51	5.63727010374404\\
52	5.63203643642345\\
53	5.62692630115048\\
54	5.62209316459023\\
55	5.61763545271073\\
56	5.61360895698774\\
57	5.61003720257722\\
58	5.60691998848982\\
59	5.60424031632074\\
60	5.60196992067473\\
61	5.60007360410411\\
62	5.59851256454601\\
63	5.5972468857682\\
64	5.59623734263165\\
65	5.59544665410232\\
66	5.59484029864577\\
67	5.59438698942628\\
68	5.59405889092848\\
69	5.59383164439937\\
70	5.59368425693725\\
71	5.5935988981168\\
72	5.59356063866525\\
73	5.59355715778524\\
74	5.59357843912396\\
75	5.59361646997539\\
76	5.5936649539271\\
77	5.59371904368582\\
78	5.59377509810302\\
79	5.5938304653516\\
80	5.59388329266539\\
81	5.59393236194847\\
82	5.59397694980373\\
83	5.59401671004799\\
84	5.59405157651074\\
85	5.59408168380406\\
86	5.59410730375935\\
87	5.59412879531786\\
88	5.59414656580856\\
89	5.59416104172743\\
90	5.59417264733009\\
91	5.59418178955235\\
92	5.59418884797202\\
93	5.59419416871343\\
94	5.5941980613708\\
95	5.59420079818368\\
96	5.59420261483821\\
97	5.59420371239025\\
98	5.59420425991157\\
99	5.59420439755035\\
100	5.59420423977139\\
101	5.59420387860355\\
102	5.5942033867719\\
103	5.594202820632\\
104	5.59420222285528\\
105	5.59420162483845\\
106	5.59420104882826\\
107	5.59420050976563\\
108	5.59420001686285\\
109	5.59419957493305\\
110	5.59419918549506\\
111	5.59419884767802\\
112	5.59419855895084\\
113	5.59419831570033\\
114	5.59419811368089\\
115	5.59419794835643\\
116	5.59419781515332\\
117	5.59419770964093\\
118	5.59419762765426\\
119	5.59419756537086\\
120	5.5941975193527\\
121	5.59419748656147\\
122	5.59419746435472\\
123	5.59419745046828\\
124	5.59419744298997\\
125	5.59419744032775\\
126	5.59419744117543\\
127	5.59419744447768\\
128	5.594197449396\\
129	5.59419745527657\\
130	5.59419746162062\\
131	5.59419746805768\\
132	5.59419747432191\\
133	5.59419748023142\\
134	5.59419748567055\\
135	5.59419749057481\\
136	5.59419749491836\\
137	5.59419749870369\\
138	5.59419750195327\\
139	5.59419750470288\\
140	5.5941975069964\\
141	5.59419750888184\\
142	5.5941975104083\\
143	5.59419751162391\\
144	5.59419751257427\\
145	5.59419751330151\\
146	5.59419751384375\\
147	5.59419751423485\\
148	5.59419751450439\\
149	5.59419751467785\\
150	5.59419751477681\\
151	5.59419751481934\\
152	5.59419751482032\\
153	5.59419751479184\\
154	5.59419751474355\\
155	5.59419751468301\\
156	5.59419751461605\\
157	5.59419751454702\\
158	5.59419751447909\\
159	5.59419751441445\\
160	5.59419751435455\\
161	5.59419751430021\\
162	5.59419751425184\\
163	5.59419751420948\\
164	5.59419751417295\\
165	5.59419751414191\\
166	5.59419751411591\\
167	5.59419751409443\\
168	5.59419751407697\\
169	5.59419751406298\\
170	5.59419751405199\\
171	5.59419751404352\\
172	5.59419751403715\\
173	5.5941975140325\\
174	5.59419751402925\\
175	5.59419751402711\\
176	5.59419751402583\\
177	5.59419751402521\\
178	5.59419751402508\\
179	5.5941975140253\\
180	5.59419751402576\\
181	5.59419751402638\\
182	5.59419751402708\\
183	5.59419751402782\\
184	5.59419751402856\\
185	5.59419751402926\\
186	5.59419751402992\\
187	5.59419751403052\\
188	5.59419751403106\\
189	5.59419751403154\\
190	5.59419751403195\\
191	5.5941975140323\\
192	5.59419751403259\\
193	5.59419751403283\\
194	5.59419751403303\\
195	5.59419751403319\\
196	5.59419751403332\\
197	5.59419751403342\\
198	5.5941975140335\\
199	5.59419751403355\\
200	5.59419751403359\\
201	5.59419751403362\\
202	5.59419751403363\\
203	5.59419751403364\\
204	5.59419751403365\\
205	5.59419751403364\\
206	5.59419751403364\\
207	5.59419751403363\\
208	5.59419751403363\\
209	5.59419751403362\\
210	5.59419751403361\\
211	5.5941975140336\\
212	5.59419751403359\\
213	5.59419751403358\\
214	5.59419751403358\\
215	5.59419751403357\\
216	5.59419751403357\\
217	5.59419751403356\\
218	5.59419751403356\\
219	5.59419751403356\\
220	5.59419751403355\\
221	5.59419751403355\\
222	5.59419751403355\\
223	5.59419751403355\\
224	5.59419751403355\\
225	5.59419751403355\\
226	5.59419751403355\\
227	5.59419751403355\\
228	5.59419751403355\\
229	5.59419751403355\\
230	5.59419751403355\\
231	5.59419751403355\\
232	5.59419751403355\\
233	5.59419751403355\\
234	5.59419751403355\\
235	5.59419751403355\\
236	5.59419751403355\\
237	5.59419751403355\\
238	5.59419751403355\\
239	5.59419751403355\\
240	5.59419751403355\\
241	5.59419751403355\\
242	5.59419751403355\\
243	5.59419751403355\\
244	5.59419751403355\\
245	5.59419751403355\\
246	5.59419751403355\\
247	5.59419751403355\\
248	5.59419751403355\\
249	5.59419751403355\\
250	5.59419751403355\\
251	5.59419751403355\\
252	5.59419751403355\\
253	5.59419751403355\\
254	5.59419751403355\\
255	5.59419751403355\\
256	5.59419751403355\\
257	5.59419751403355\\
258	5.59419751403355\\
259	5.59419751403355\\
260	5.59419751403355\\
261	5.59419751403355\\
262	5.59419751403355\\
263	5.59419751403355\\
264	5.59419751403355\\
265	5.59419751403355\\
266	5.59419751403355\\
267	5.59419751403355\\
268	5.59419751403355\\
269	5.59419751403355\\
270	5.59419751403355\\
271	5.59419751403355\\
272	5.59419751403355\\
273	5.59419751403355\\
274	5.59419751403355\\
275	5.59419751403355\\
276	5.59419751403355\\
277	5.59419751403355\\
278	5.59419751403355\\
279	5.59419751403355\\
280	5.59419751403355\\
281	5.59419751403355\\
282	5.59419751403355\\
283	5.59419751403355\\
284	5.59419751403355\\
285	5.59419751403355\\
286	5.59419751403355\\
287	5.59419751403355\\
288	5.59419751403355\\
289	5.59419751403355\\
290	5.59419751403355\\
291	5.59419751403355\\
292	5.59419751403355\\
293	5.59419751403355\\
294	5.59419751403355\\
295	5.59419751403355\\
296	5.59419751403355\\
297	5.59419751403355\\
298	5.59419751403355\\
299	5.59419751403355\\
300	5.59419751403355\\
301	5.59419751403355\\
302	5.59419751403355\\
303	5.59419751403355\\
304	5.59419751403355\\
305	5.59419751403355\\
306	5.59419751403355\\
307	5.59419751403355\\
308	5.59419751403355\\
309	5.59419751403355\\
310	5.59419751403355\\
311	5.59419751403355\\
312	5.59419751403355\\
313	5.59419751403355\\
314	5.59419751403355\\
315	5.59419751403355\\
316	5.59419751403355\\
317	5.59419751403355\\
318	5.59419751403355\\
319	5.59419751403355\\
320	5.59419751403355\\
321	5.59419751403355\\
322	5.59419751403355\\
323	5.59419751403355\\
324	5.59419751403355\\
325	5.59419751403355\\
326	5.59419751403355\\
327	5.59419751403355\\
328	5.59419751403355\\
329	5.59419751403355\\
330	5.59419751403355\\
331	5.59419751403355\\
332	5.59419751403355\\
333	5.59419751403355\\
334	5.59419751403355\\
335	5.59419751403355\\
336	5.59419751403355\\
337	5.59419751403355\\
338	5.59419751403355\\
339	5.59419751403355\\
340	5.59419751403355\\
341	5.59419751403355\\
342	5.59419751403355\\
343	5.59419751403355\\
344	5.59419751403355\\
345	5.59419751403355\\
346	5.59419751403355\\
347	5.59419751403355\\
348	5.59419751403355\\
349	5.59419751403355\\
350	5.59419751403355\\
351	5.59419751403355\\
352	5.59419751403355\\
353	5.59419751403355\\
354	5.59419751403355\\
355	5.59419751403355\\
356	5.59419751403355\\
357	5.59419751403355\\
358	5.59419751403355\\
359	5.59419751403355\\
360	5.59419751403355\\
361	5.59419751403355\\
362	5.59419751403355\\
363	5.59419751403355\\
364	5.59419751403355\\
365	5.59419751403355\\
366	5.59419751403355\\
367	5.59419751403355\\
368	5.59419751403355\\
369	5.59419751403355\\
370	5.59419751403355\\
371	5.59419751403355\\
372	5.59419751403355\\
373	5.59419751403355\\
374	5.59419751403355\\
375	5.59419751403355\\
376	5.59419751403355\\
377	5.59419751403355\\
378	5.59419751403355\\
379	5.59419751403355\\
380	5.59419751403355\\
381	5.59419751403355\\
382	5.59419751403355\\
383	5.59419751403355\\
384	5.59419751403355\\
385	5.59419751403355\\
386	5.59419751403355\\
387	5.59419751403355\\
388	5.59419751403355\\
389	5.59419751403355\\
390	5.59419751403355\\
391	5.59419751403355\\
392	5.59419751403355\\
393	5.59419751403355\\
394	5.59419751403355\\
395	5.59419751403355\\
396	5.59419751403355\\
397	5.59419751403355\\
398	5.59419751403355\\
399	5.59419751403355\\
400	5.59419751403355\\
401	5.59419751403355\\
402	5.59419751403355\\
403	5.59419751403355\\
404	5.59419751403355\\
405	5.59419751403355\\
406	5.59419751403355\\
407	5.59419751403355\\
408	5.59419751403355\\
409	5.59419751403355\\
410	5.59419751403355\\
411	5.59419751403355\\
412	5.59419751403355\\
413	5.59419751403355\\
414	5.59419751403355\\
415	5.59419751403355\\
416	5.59419751403355\\
417	5.59419751403355\\
418	5.59419751403355\\
419	5.59419751403355\\
420	5.59419751403355\\
421	5.59419751403355\\
422	5.59419751403355\\
423	5.59419751403355\\
424	5.59419751403355\\
425	5.59419751403355\\
426	5.59419751403355\\
427	5.59419751403355\\
428	5.59419751403355\\
429	5.59419751403355\\
430	5.59419751403355\\
431	5.59419751403355\\
432	5.59419751403355\\
433	5.59419751403355\\
434	5.59419751403355\\
435	5.59419751403355\\
436	5.59419751403355\\
437	5.59419751403355\\
438	5.59419751403355\\
439	5.59419751403355\\
440	5.59419751403355\\
441	5.59419751403355\\
442	5.59419751403355\\
443	5.59419751403355\\
444	5.59419751403355\\
445	5.59419751403355\\
446	5.59419751403355\\
447	5.59419751403355\\
448	5.59419751403355\\
449	5.59419751403355\\
450	5.59419751403355\\
451	5.59419751403355\\
452	5.59419751403355\\
453	5.59419751403355\\
454	5.59419751403355\\
455	5.59419751403355\\
456	5.59419751403355\\
457	5.59419751403355\\
458	5.59419751403355\\
459	5.59419751403355\\
460	5.59419751403355\\
461	5.59419751403355\\
462	5.59419751403355\\
463	5.59419751403355\\
464	5.59419751403355\\
465	5.59419751403355\\
466	5.59419751403355\\
467	5.59419751403355\\
468	5.59419751403355\\
469	5.59419751403355\\
470	5.59419751403355\\
471	5.59419751403355\\
472	5.59419751403355\\
473	5.59419751403355\\
474	5.59419751403355\\
475	5.59419751403355\\
476	5.59419751403355\\
477	5.59419751403355\\
478	5.59419751403355\\
479	5.59419751403355\\
480	5.59419751403355\\
481	5.59419751403355\\
482	5.59419751403355\\
483	5.59419751403355\\
484	5.59419751403355\\
485	5.59419751403355\\
486	5.59419751403355\\
487	5.59419751403355\\
488	5.59419751403355\\
489	5.59419751403355\\
490	5.59419751403355\\
491	5.59419751403355\\
492	5.59419751403355\\
493	5.59419751403355\\
494	5.59419751403355\\
495	5.59419751403355\\
496	5.59419751403355\\
497	5.59419751403355\\
498	5.59419751403355\\
499	5.59419751403355\\
500	5.59419751403355\\
501	5.59419751403355\\
502	5.59419751403355\\
503	5.59419751403355\\
504	5.59419751403355\\
505	5.59419751403355\\
506	5.59419751403355\\
507	5.59419751403355\\
508	5.59419751403355\\
509	5.59419751403355\\
510	5.59419751403355\\
511	5.59419751403355\\
512	5.59419751403355\\
513	5.59419751403355\\
514	5.59419751403355\\
515	5.59419751403355\\
516	5.59419751403355\\
517	5.59419751403355\\
518	5.59419751403355\\
519	5.59419751403355\\
520	5.59419751403355\\
521	5.59419751403355\\
522	5.59419751403355\\
523	5.59419751403355\\
524	5.59419751403355\\
525	5.59419751403355\\
526	5.59419751403355\\
527	5.59419751403355\\
528	5.59419751403355\\
529	5.59419751403355\\
530	5.59419751403355\\
531	5.59419751403355\\
532	5.59419751403355\\
533	5.59419751403355\\
534	5.59419751403355\\
535	5.59419751403355\\
536	5.59419751403355\\
537	5.59419751403355\\
538	5.59419751403355\\
539	5.59419751403355\\
540	5.59419751403355\\
541	5.59419751403355\\
542	5.59419751403355\\
543	5.59419751403355\\
544	5.59419751403355\\
545	5.59419751403355\\
546	5.59419751403355\\
547	5.59419751403355\\
548	5.59419751403355\\
549	5.59419751403355\\
550	5.59419751403355\\
551	5.59419751403355\\
552	5.59419751403355\\
553	5.59419751403355\\
554	5.59419751403355\\
555	5.59419751403355\\
556	5.59419751403355\\
557	5.59419751403355\\
558	5.59419751403355\\
559	5.59419751403355\\
560	5.59419751403355\\
561	5.59419751403355\\
562	5.59419751403355\\
563	5.59419751403355\\
564	5.59419751403355\\
565	5.59419751403355\\
566	5.59419751403355\\
567	5.59419751403355\\
568	5.59419751403355\\
569	5.59419751403355\\
570	5.59419751403355\\
571	5.59419751403355\\
572	5.59419751403355\\
573	5.59419751403355\\
574	5.59419751403355\\
575	5.59419751403355\\
576	5.59419751403355\\
577	5.59419751403355\\
578	5.59419751403355\\
579	5.59419751403355\\
580	5.59419751403355\\
581	5.59419751403355\\
582	5.59419751403355\\
583	5.59419751403355\\
584	5.59419751403355\\
585	5.59419751403355\\
586	5.59419751403355\\
587	5.59419751403355\\
588	5.59419751403355\\
589	5.59419751403355\\
590	5.59419751403355\\
591	5.59419751403355\\
592	5.59419751403355\\
593	5.59419751403355\\
594	5.59419751403355\\
595	5.59419751403355\\
596	5.59419751403355\\
597	5.59419751403355\\
598	5.59419751403355\\
599	5.59419751403355\\
600	5.59419751403355\\
};
\end{axis}

\begin{axis}[%
width=4.521in,
height=1.493in,
at={(0.758in,0.481in)},
scale only axis,
xmin=0,
xmax=600,
ymin=-1,
ymax=1,
axis background/.style={fill=white},
title style={font=\bfseries},
title={Sygna� wej�ciowy},
xmajorgrids,
ymajorgrids
]
\addplot[const plot, color=mycolor1, forget plot] table[row sep=crcr] {%
1	0\\
2	0\\
3	0\\
4	0\\
5	0\\
6	0\\
7	0\\
8	0\\
9	0\\
10	0\\
11	0\\
12	0\\
13	0\\
14	0\\
15	0\\
16	-1\\
17	-1\\
18	-1\\
19	-1\\
20	-1\\
21	-1\\
22	-1\\
23	-1\\
24	-1\\
25	-1\\
26	-1\\
27	-1\\
28	-1\\
29	-1\\
30	-1\\
31	-1\\
32	-1\\
33	-1\\
34	-1\\
35	-1\\
36	-1\\
37	-1\\
38	-1\\
39	-1\\
40	-1\\
41	-1\\
42	-1\\
43	-1\\
44	-1\\
45	-1\\
46	-1\\
47	-1\\
48	-1\\
49	-1\\
50	-1\\
51	-1\\
52	-1\\
53	-1\\
54	-1\\
55	-1\\
56	-1\\
57	-1\\
58	-1\\
59	-1\\
60	-1\\
61	-1\\
62	-1\\
63	-1\\
64	-1\\
65	-1\\
66	-1\\
67	-1\\
68	-1\\
69	-1\\
70	-1\\
71	-1\\
72	-1\\
73	-1\\
74	-1\\
75	-1\\
76	-1\\
77	-1\\
78	-1\\
79	-1\\
80	-1\\
81	-1\\
82	-1\\
83	-1\\
84	-1\\
85	-1\\
86	-1\\
87	-1\\
88	-1\\
89	-1\\
90	-1\\
91	-1\\
92	-1\\
93	-1\\
94	-1\\
95	-1\\
96	-1\\
97	-1\\
98	-1\\
99	-1\\
100	-1\\
101	-1\\
102	-1\\
103	-1\\
104	-1\\
105	-1\\
106	-1\\
107	-1\\
108	-1\\
109	-1\\
110	-1\\
111	-1\\
112	-1\\
113	-1\\
114	-1\\
115	-1\\
116	-1\\
117	-1\\
118	-1\\
119	-1\\
120	-1\\
121	-1\\
122	-1\\
123	-1\\
124	-1\\
125	-1\\
126	-1\\
127	-1\\
128	-1\\
129	-1\\
130	-1\\
131	-1\\
132	-1\\
133	-1\\
134	-1\\
135	-1\\
136	-1\\
137	-1\\
138	-1\\
139	-1\\
140	-1\\
141	-1\\
142	-1\\
143	-1\\
144	-1\\
145	-1\\
146	-1\\
147	-1\\
148	-1\\
149	-1\\
150	-1\\
151	-1\\
152	-1\\
153	-1\\
154	-1\\
155	-1\\
156	-1\\
157	-1\\
158	-1\\
159	-1\\
160	-1\\
161	-1\\
162	-1\\
163	-1\\
164	-1\\
165	-1\\
166	-1\\
167	-1\\
168	-1\\
169	-1\\
170	-1\\
171	-1\\
172	-1\\
173	-1\\
174	-1\\
175	-1\\
176	-1\\
177	-1\\
178	-1\\
179	-1\\
180	-1\\
181	-1\\
182	-1\\
183	-1\\
184	-1\\
185	-1\\
186	-1\\
187	-1\\
188	-1\\
189	-1\\
190	-1\\
191	-1\\
192	-1\\
193	-1\\
194	-1\\
195	-1\\
196	-1\\
197	-1\\
198	-1\\
199	-1\\
200	-1\\
201	-1\\
202	-1\\
203	-1\\
204	-1\\
205	-1\\
206	-1\\
207	-1\\
208	-1\\
209	-1\\
210	-1\\
211	-1\\
212	-1\\
213	-1\\
214	-1\\
215	-1\\
216	-1\\
217	-1\\
218	-1\\
219	-1\\
220	-1\\
221	-1\\
222	-1\\
223	-1\\
224	-1\\
225	-1\\
226	-1\\
227	-1\\
228	-1\\
229	-1\\
230	-1\\
231	-1\\
232	-1\\
233	-1\\
234	-1\\
235	-1\\
236	-1\\
237	-1\\
238	-1\\
239	-1\\
240	-1\\
241	-1\\
242	-1\\
243	-1\\
244	-1\\
245	-1\\
246	-1\\
247	-1\\
248	-1\\
249	-1\\
250	-1\\
251	-1\\
252	-1\\
253	-1\\
254	-1\\
255	-1\\
256	-1\\
257	-1\\
258	-1\\
259	-1\\
260	-1\\
261	-1\\
262	-1\\
263	-1\\
264	-1\\
265	-1\\
266	-1\\
267	-1\\
268	-1\\
269	-1\\
270	-1\\
271	-1\\
272	-1\\
273	-1\\
274	-1\\
275	-1\\
276	-1\\
277	-1\\
278	-1\\
279	-1\\
280	-1\\
281	-1\\
282	-1\\
283	-1\\
284	-1\\
285	-1\\
286	-1\\
287	-1\\
288	-1\\
289	-1\\
290	-1\\
291	-1\\
292	-1\\
293	-1\\
294	-1\\
295	-1\\
296	-1\\
297	-1\\
298	-1\\
299	-1\\
300	-1\\
301	-1\\
302	-1\\
303	-1\\
304	-1\\
305	-1\\
306	-1\\
307	-1\\
308	-1\\
309	-1\\
310	-1\\
311	-1\\
312	-1\\
313	-1\\
314	-1\\
315	-1\\
316	-1\\
317	-1\\
318	-1\\
319	-1\\
320	-1\\
321	-1\\
322	-1\\
323	-1\\
324	-1\\
325	-1\\
326	-1\\
327	-1\\
328	-1\\
329	-1\\
330	-1\\
331	-1\\
332	-1\\
333	-1\\
334	-1\\
335	-1\\
336	-1\\
337	-1\\
338	-1\\
339	-1\\
340	-1\\
341	-1\\
342	-1\\
343	-1\\
344	-1\\
345	-1\\
346	-1\\
347	-1\\
348	-1\\
349	-1\\
350	-1\\
351	-1\\
352	-1\\
353	-1\\
354	-1\\
355	-1\\
356	-1\\
357	-1\\
358	-1\\
359	-1\\
360	-1\\
361	-1\\
362	-1\\
363	-1\\
364	-1\\
365	-1\\
366	-1\\
367	-1\\
368	-1\\
369	-1\\
370	-1\\
371	-1\\
372	-1\\
373	-1\\
374	-1\\
375	-1\\
376	-1\\
377	-1\\
378	-1\\
379	-1\\
380	-1\\
381	-1\\
382	-1\\
383	-1\\
384	-1\\
385	-1\\
386	-1\\
387	-1\\
388	-1\\
389	-1\\
390	-1\\
391	-1\\
392	-1\\
393	-1\\
394	-1\\
395	-1\\
396	-1\\
397	-1\\
398	-1\\
399	-1\\
400	-1\\
401	-1\\
402	-1\\
403	-1\\
404	-1\\
405	-1\\
406	-1\\
407	-1\\
408	-1\\
409	-1\\
410	-1\\
411	-1\\
412	-1\\
413	-1\\
414	-1\\
415	-1\\
416	-1\\
417	-1\\
418	-1\\
419	-1\\
420	-1\\
421	-1\\
422	-1\\
423	-1\\
424	-1\\
425	-1\\
426	-1\\
427	-1\\
428	-1\\
429	-1\\
430	-1\\
431	-1\\
432	-1\\
433	-1\\
434	-1\\
435	-1\\
436	-1\\
437	-1\\
438	-1\\
439	-1\\
440	-1\\
441	-1\\
442	-1\\
443	-1\\
444	-1\\
445	-1\\
446	-1\\
447	-1\\
448	-1\\
449	-1\\
450	-1\\
451	-1\\
452	-1\\
453	-1\\
454	-1\\
455	-1\\
456	-1\\
457	-1\\
458	-1\\
459	-1\\
460	-1\\
461	-1\\
462	-1\\
463	-1\\
464	-1\\
465	-1\\
466	-1\\
467	-1\\
468	-1\\
469	-1\\
470	-1\\
471	-1\\
472	-1\\
473	-1\\
474	-1\\
475	-1\\
476	-1\\
477	-1\\
478	-1\\
479	-1\\
480	-1\\
481	-1\\
482	-1\\
483	-1\\
484	-1\\
485	-1\\
486	-1\\
487	-1\\
488	-1\\
489	-1\\
490	-1\\
491	-1\\
492	-1\\
493	-1\\
494	-1\\
495	-1\\
496	-1\\
497	-1\\
498	-1\\
499	-1\\
500	-1\\
501	-1\\
502	-1\\
503	-1\\
504	-1\\
505	-1\\
506	-1\\
507	-1\\
508	-1\\
509	-1\\
510	-1\\
511	-1\\
512	-1\\
513	-1\\
514	-1\\
515	-1\\
516	-1\\
517	-1\\
518	-1\\
519	-1\\
520	-1\\
521	-1\\
522	-1\\
523	-1\\
524	-1\\
525	-1\\
526	-1\\
527	-1\\
528	-1\\
529	-1\\
530	-1\\
531	-1\\
532	-1\\
533	-1\\
534	-1\\
535	-1\\
536	-1\\
537	-1\\
538	-1\\
539	-1\\
540	-1\\
541	-1\\
542	-1\\
543	-1\\
544	-1\\
545	-1\\
546	-1\\
547	-1\\
548	-1\\
549	-1\\
550	-1\\
551	-1\\
552	-1\\
553	-1\\
554	-1\\
555	-1\\
556	-1\\
557	-1\\
558	-1\\
559	-1\\
560	-1\\
561	-1\\
562	-1\\
563	-1\\
564	-1\\
565	-1\\
566	-1\\
567	-1\\
568	-1\\
569	-1\\
570	-1\\
571	-1\\
572	-1\\
573	-1\\
574	-1\\
575	-1\\
576	-1\\
577	-1\\
578	-1\\
579	-1\\
580	-1\\
581	-1\\
582	-1\\
583	-1\\
584	-1\\
585	-1\\
586	-1\\
587	-1\\
588	-1\\
589	-1\\
590	-1\\
591	-1\\
592	-1\\
593	-1\\
594	-1\\
595	-1\\
596	-1\\
597	-1\\
598	-1\\
599	-1\\
600	-1\\
};
\addplot[const plot, color=mycolor2, forget plot] table[row sep=crcr] {%
1	0\\
2	0\\
3	0\\
4	0\\
5	0\\
6	0\\
7	0\\
8	0\\
9	0\\
10	0\\
11	0\\
12	0\\
13	0\\
14	0\\
15	0\\
16	-0.5\\
17	-0.5\\
18	-0.5\\
19	-0.5\\
20	-0.5\\
21	-0.5\\
22	-0.5\\
23	-0.5\\
24	-0.5\\
25	-0.5\\
26	-0.5\\
27	-0.5\\
28	-0.5\\
29	-0.5\\
30	-0.5\\
31	-0.5\\
32	-0.5\\
33	-0.5\\
34	-0.5\\
35	-0.5\\
36	-0.5\\
37	-0.5\\
38	-0.5\\
39	-0.5\\
40	-0.5\\
41	-0.5\\
42	-0.5\\
43	-0.5\\
44	-0.5\\
45	-0.5\\
46	-0.5\\
47	-0.5\\
48	-0.5\\
49	-0.5\\
50	-0.5\\
51	-0.5\\
52	-0.5\\
53	-0.5\\
54	-0.5\\
55	-0.5\\
56	-0.5\\
57	-0.5\\
58	-0.5\\
59	-0.5\\
60	-0.5\\
61	-0.5\\
62	-0.5\\
63	-0.5\\
64	-0.5\\
65	-0.5\\
66	-0.5\\
67	-0.5\\
68	-0.5\\
69	-0.5\\
70	-0.5\\
71	-0.5\\
72	-0.5\\
73	-0.5\\
74	-0.5\\
75	-0.5\\
76	-0.5\\
77	-0.5\\
78	-0.5\\
79	-0.5\\
80	-0.5\\
81	-0.5\\
82	-0.5\\
83	-0.5\\
84	-0.5\\
85	-0.5\\
86	-0.5\\
87	-0.5\\
88	-0.5\\
89	-0.5\\
90	-0.5\\
91	-0.5\\
92	-0.5\\
93	-0.5\\
94	-0.5\\
95	-0.5\\
96	-0.5\\
97	-0.5\\
98	-0.5\\
99	-0.5\\
100	-0.5\\
101	-0.5\\
102	-0.5\\
103	-0.5\\
104	-0.5\\
105	-0.5\\
106	-0.5\\
107	-0.5\\
108	-0.5\\
109	-0.5\\
110	-0.5\\
111	-0.5\\
112	-0.5\\
113	-0.5\\
114	-0.5\\
115	-0.5\\
116	-0.5\\
117	-0.5\\
118	-0.5\\
119	-0.5\\
120	-0.5\\
121	-0.5\\
122	-0.5\\
123	-0.5\\
124	-0.5\\
125	-0.5\\
126	-0.5\\
127	-0.5\\
128	-0.5\\
129	-0.5\\
130	-0.5\\
131	-0.5\\
132	-0.5\\
133	-0.5\\
134	-0.5\\
135	-0.5\\
136	-0.5\\
137	-0.5\\
138	-0.5\\
139	-0.5\\
140	-0.5\\
141	-0.5\\
142	-0.5\\
143	-0.5\\
144	-0.5\\
145	-0.5\\
146	-0.5\\
147	-0.5\\
148	-0.5\\
149	-0.5\\
150	-0.5\\
151	-0.5\\
152	-0.5\\
153	-0.5\\
154	-0.5\\
155	-0.5\\
156	-0.5\\
157	-0.5\\
158	-0.5\\
159	-0.5\\
160	-0.5\\
161	-0.5\\
162	-0.5\\
163	-0.5\\
164	-0.5\\
165	-0.5\\
166	-0.5\\
167	-0.5\\
168	-0.5\\
169	-0.5\\
170	-0.5\\
171	-0.5\\
172	-0.5\\
173	-0.5\\
174	-0.5\\
175	-0.5\\
176	-0.5\\
177	-0.5\\
178	-0.5\\
179	-0.5\\
180	-0.5\\
181	-0.5\\
182	-0.5\\
183	-0.5\\
184	-0.5\\
185	-0.5\\
186	-0.5\\
187	-0.5\\
188	-0.5\\
189	-0.5\\
190	-0.5\\
191	-0.5\\
192	-0.5\\
193	-0.5\\
194	-0.5\\
195	-0.5\\
196	-0.5\\
197	-0.5\\
198	-0.5\\
199	-0.5\\
200	-0.5\\
201	-0.5\\
202	-0.5\\
203	-0.5\\
204	-0.5\\
205	-0.5\\
206	-0.5\\
207	-0.5\\
208	-0.5\\
209	-0.5\\
210	-0.5\\
211	-0.5\\
212	-0.5\\
213	-0.5\\
214	-0.5\\
215	-0.5\\
216	-0.5\\
217	-0.5\\
218	-0.5\\
219	-0.5\\
220	-0.5\\
221	-0.5\\
222	-0.5\\
223	-0.5\\
224	-0.5\\
225	-0.5\\
226	-0.5\\
227	-0.5\\
228	-0.5\\
229	-0.5\\
230	-0.5\\
231	-0.5\\
232	-0.5\\
233	-0.5\\
234	-0.5\\
235	-0.5\\
236	-0.5\\
237	-0.5\\
238	-0.5\\
239	-0.5\\
240	-0.5\\
241	-0.5\\
242	-0.5\\
243	-0.5\\
244	-0.5\\
245	-0.5\\
246	-0.5\\
247	-0.5\\
248	-0.5\\
249	-0.5\\
250	-0.5\\
251	-0.5\\
252	-0.5\\
253	-0.5\\
254	-0.5\\
255	-0.5\\
256	-0.5\\
257	-0.5\\
258	-0.5\\
259	-0.5\\
260	-0.5\\
261	-0.5\\
262	-0.5\\
263	-0.5\\
264	-0.5\\
265	-0.5\\
266	-0.5\\
267	-0.5\\
268	-0.5\\
269	-0.5\\
270	-0.5\\
271	-0.5\\
272	-0.5\\
273	-0.5\\
274	-0.5\\
275	-0.5\\
276	-0.5\\
277	-0.5\\
278	-0.5\\
279	-0.5\\
280	-0.5\\
281	-0.5\\
282	-0.5\\
283	-0.5\\
284	-0.5\\
285	-0.5\\
286	-0.5\\
287	-0.5\\
288	-0.5\\
289	-0.5\\
290	-0.5\\
291	-0.5\\
292	-0.5\\
293	-0.5\\
294	-0.5\\
295	-0.5\\
296	-0.5\\
297	-0.5\\
298	-0.5\\
299	-0.5\\
300	-0.5\\
301	-0.5\\
302	-0.5\\
303	-0.5\\
304	-0.5\\
305	-0.5\\
306	-0.5\\
307	-0.5\\
308	-0.5\\
309	-0.5\\
310	-0.5\\
311	-0.5\\
312	-0.5\\
313	-0.5\\
314	-0.5\\
315	-0.5\\
316	-0.5\\
317	-0.5\\
318	-0.5\\
319	-0.5\\
320	-0.5\\
321	-0.5\\
322	-0.5\\
323	-0.5\\
324	-0.5\\
325	-0.5\\
326	-0.5\\
327	-0.5\\
328	-0.5\\
329	-0.5\\
330	-0.5\\
331	-0.5\\
332	-0.5\\
333	-0.5\\
334	-0.5\\
335	-0.5\\
336	-0.5\\
337	-0.5\\
338	-0.5\\
339	-0.5\\
340	-0.5\\
341	-0.5\\
342	-0.5\\
343	-0.5\\
344	-0.5\\
345	-0.5\\
346	-0.5\\
347	-0.5\\
348	-0.5\\
349	-0.5\\
350	-0.5\\
351	-0.5\\
352	-0.5\\
353	-0.5\\
354	-0.5\\
355	-0.5\\
356	-0.5\\
357	-0.5\\
358	-0.5\\
359	-0.5\\
360	-0.5\\
361	-0.5\\
362	-0.5\\
363	-0.5\\
364	-0.5\\
365	-0.5\\
366	-0.5\\
367	-0.5\\
368	-0.5\\
369	-0.5\\
370	-0.5\\
371	-0.5\\
372	-0.5\\
373	-0.5\\
374	-0.5\\
375	-0.5\\
376	-0.5\\
377	-0.5\\
378	-0.5\\
379	-0.5\\
380	-0.5\\
381	-0.5\\
382	-0.5\\
383	-0.5\\
384	-0.5\\
385	-0.5\\
386	-0.5\\
387	-0.5\\
388	-0.5\\
389	-0.5\\
390	-0.5\\
391	-0.5\\
392	-0.5\\
393	-0.5\\
394	-0.5\\
395	-0.5\\
396	-0.5\\
397	-0.5\\
398	-0.5\\
399	-0.5\\
400	-0.5\\
401	-0.5\\
402	-0.5\\
403	-0.5\\
404	-0.5\\
405	-0.5\\
406	-0.5\\
407	-0.5\\
408	-0.5\\
409	-0.5\\
410	-0.5\\
411	-0.5\\
412	-0.5\\
413	-0.5\\
414	-0.5\\
415	-0.5\\
416	-0.5\\
417	-0.5\\
418	-0.5\\
419	-0.5\\
420	-0.5\\
421	-0.5\\
422	-0.5\\
423	-0.5\\
424	-0.5\\
425	-0.5\\
426	-0.5\\
427	-0.5\\
428	-0.5\\
429	-0.5\\
430	-0.5\\
431	-0.5\\
432	-0.5\\
433	-0.5\\
434	-0.5\\
435	-0.5\\
436	-0.5\\
437	-0.5\\
438	-0.5\\
439	-0.5\\
440	-0.5\\
441	-0.5\\
442	-0.5\\
443	-0.5\\
444	-0.5\\
445	-0.5\\
446	-0.5\\
447	-0.5\\
448	-0.5\\
449	-0.5\\
450	-0.5\\
451	-0.5\\
452	-0.5\\
453	-0.5\\
454	-0.5\\
455	-0.5\\
456	-0.5\\
457	-0.5\\
458	-0.5\\
459	-0.5\\
460	-0.5\\
461	-0.5\\
462	-0.5\\
463	-0.5\\
464	-0.5\\
465	-0.5\\
466	-0.5\\
467	-0.5\\
468	-0.5\\
469	-0.5\\
470	-0.5\\
471	-0.5\\
472	-0.5\\
473	-0.5\\
474	-0.5\\
475	-0.5\\
476	-0.5\\
477	-0.5\\
478	-0.5\\
479	-0.5\\
480	-0.5\\
481	-0.5\\
482	-0.5\\
483	-0.5\\
484	-0.5\\
485	-0.5\\
486	-0.5\\
487	-0.5\\
488	-0.5\\
489	-0.5\\
490	-0.5\\
491	-0.5\\
492	-0.5\\
493	-0.5\\
494	-0.5\\
495	-0.5\\
496	-0.5\\
497	-0.5\\
498	-0.5\\
499	-0.5\\
500	-0.5\\
501	-0.5\\
502	-0.5\\
503	-0.5\\
504	-0.5\\
505	-0.5\\
506	-0.5\\
507	-0.5\\
508	-0.5\\
509	-0.5\\
510	-0.5\\
511	-0.5\\
512	-0.5\\
513	-0.5\\
514	-0.5\\
515	-0.5\\
516	-0.5\\
517	-0.5\\
518	-0.5\\
519	-0.5\\
520	-0.5\\
521	-0.5\\
522	-0.5\\
523	-0.5\\
524	-0.5\\
525	-0.5\\
526	-0.5\\
527	-0.5\\
528	-0.5\\
529	-0.5\\
530	-0.5\\
531	-0.5\\
532	-0.5\\
533	-0.5\\
534	-0.5\\
535	-0.5\\
536	-0.5\\
537	-0.5\\
538	-0.5\\
539	-0.5\\
540	-0.5\\
541	-0.5\\
542	-0.5\\
543	-0.5\\
544	-0.5\\
545	-0.5\\
546	-0.5\\
547	-0.5\\
548	-0.5\\
549	-0.5\\
550	-0.5\\
551	-0.5\\
552	-0.5\\
553	-0.5\\
554	-0.5\\
555	-0.5\\
556	-0.5\\
557	-0.5\\
558	-0.5\\
559	-0.5\\
560	-0.5\\
561	-0.5\\
562	-0.5\\
563	-0.5\\
564	-0.5\\
565	-0.5\\
566	-0.5\\
567	-0.5\\
568	-0.5\\
569	-0.5\\
570	-0.5\\
571	-0.5\\
572	-0.5\\
573	-0.5\\
574	-0.5\\
575	-0.5\\
576	-0.5\\
577	-0.5\\
578	-0.5\\
579	-0.5\\
580	-0.5\\
581	-0.5\\
582	-0.5\\
583	-0.5\\
584	-0.5\\
585	-0.5\\
586	-0.5\\
587	-0.5\\
588	-0.5\\
589	-0.5\\
590	-0.5\\
591	-0.5\\
592	-0.5\\
593	-0.5\\
594	-0.5\\
595	-0.5\\
596	-0.5\\
597	-0.5\\
598	-0.5\\
599	-0.5\\
600	-0.5\\
};
\addplot[const plot, color=mycolor3, forget plot] table[row sep=crcr] {%
1	0\\
2	0\\
3	0\\
4	0\\
5	0\\
6	0\\
7	0\\
8	0\\
9	0\\
10	0\\
11	0\\
12	0\\
13	0\\
14	0\\
15	0\\
16	0.5\\
17	0.5\\
18	0.5\\
19	0.5\\
20	0.5\\
21	0.5\\
22	0.5\\
23	0.5\\
24	0.5\\
25	0.5\\
26	0.5\\
27	0.5\\
28	0.5\\
29	0.5\\
30	0.5\\
31	0.5\\
32	0.5\\
33	0.5\\
34	0.5\\
35	0.5\\
36	0.5\\
37	0.5\\
38	0.5\\
39	0.5\\
40	0.5\\
41	0.5\\
42	0.5\\
43	0.5\\
44	0.5\\
45	0.5\\
46	0.5\\
47	0.5\\
48	0.5\\
49	0.5\\
50	0.5\\
51	0.5\\
52	0.5\\
53	0.5\\
54	0.5\\
55	0.5\\
56	0.5\\
57	0.5\\
58	0.5\\
59	0.5\\
60	0.5\\
61	0.5\\
62	0.5\\
63	0.5\\
64	0.5\\
65	0.5\\
66	0.5\\
67	0.5\\
68	0.5\\
69	0.5\\
70	0.5\\
71	0.5\\
72	0.5\\
73	0.5\\
74	0.5\\
75	0.5\\
76	0.5\\
77	0.5\\
78	0.5\\
79	0.5\\
80	0.5\\
81	0.5\\
82	0.5\\
83	0.5\\
84	0.5\\
85	0.5\\
86	0.5\\
87	0.5\\
88	0.5\\
89	0.5\\
90	0.5\\
91	0.5\\
92	0.5\\
93	0.5\\
94	0.5\\
95	0.5\\
96	0.5\\
97	0.5\\
98	0.5\\
99	0.5\\
100	0.5\\
101	0.5\\
102	0.5\\
103	0.5\\
104	0.5\\
105	0.5\\
106	0.5\\
107	0.5\\
108	0.5\\
109	0.5\\
110	0.5\\
111	0.5\\
112	0.5\\
113	0.5\\
114	0.5\\
115	0.5\\
116	0.5\\
117	0.5\\
118	0.5\\
119	0.5\\
120	0.5\\
121	0.5\\
122	0.5\\
123	0.5\\
124	0.5\\
125	0.5\\
126	0.5\\
127	0.5\\
128	0.5\\
129	0.5\\
130	0.5\\
131	0.5\\
132	0.5\\
133	0.5\\
134	0.5\\
135	0.5\\
136	0.5\\
137	0.5\\
138	0.5\\
139	0.5\\
140	0.5\\
141	0.5\\
142	0.5\\
143	0.5\\
144	0.5\\
145	0.5\\
146	0.5\\
147	0.5\\
148	0.5\\
149	0.5\\
150	0.5\\
151	0.5\\
152	0.5\\
153	0.5\\
154	0.5\\
155	0.5\\
156	0.5\\
157	0.5\\
158	0.5\\
159	0.5\\
160	0.5\\
161	0.5\\
162	0.5\\
163	0.5\\
164	0.5\\
165	0.5\\
166	0.5\\
167	0.5\\
168	0.5\\
169	0.5\\
170	0.5\\
171	0.5\\
172	0.5\\
173	0.5\\
174	0.5\\
175	0.5\\
176	0.5\\
177	0.5\\
178	0.5\\
179	0.5\\
180	0.5\\
181	0.5\\
182	0.5\\
183	0.5\\
184	0.5\\
185	0.5\\
186	0.5\\
187	0.5\\
188	0.5\\
189	0.5\\
190	0.5\\
191	0.5\\
192	0.5\\
193	0.5\\
194	0.5\\
195	0.5\\
196	0.5\\
197	0.5\\
198	0.5\\
199	0.5\\
200	0.5\\
201	0.5\\
202	0.5\\
203	0.5\\
204	0.5\\
205	0.5\\
206	0.5\\
207	0.5\\
208	0.5\\
209	0.5\\
210	0.5\\
211	0.5\\
212	0.5\\
213	0.5\\
214	0.5\\
215	0.5\\
216	0.5\\
217	0.5\\
218	0.5\\
219	0.5\\
220	0.5\\
221	0.5\\
222	0.5\\
223	0.5\\
224	0.5\\
225	0.5\\
226	0.5\\
227	0.5\\
228	0.5\\
229	0.5\\
230	0.5\\
231	0.5\\
232	0.5\\
233	0.5\\
234	0.5\\
235	0.5\\
236	0.5\\
237	0.5\\
238	0.5\\
239	0.5\\
240	0.5\\
241	0.5\\
242	0.5\\
243	0.5\\
244	0.5\\
245	0.5\\
246	0.5\\
247	0.5\\
248	0.5\\
249	0.5\\
250	0.5\\
251	0.5\\
252	0.5\\
253	0.5\\
254	0.5\\
255	0.5\\
256	0.5\\
257	0.5\\
258	0.5\\
259	0.5\\
260	0.5\\
261	0.5\\
262	0.5\\
263	0.5\\
264	0.5\\
265	0.5\\
266	0.5\\
267	0.5\\
268	0.5\\
269	0.5\\
270	0.5\\
271	0.5\\
272	0.5\\
273	0.5\\
274	0.5\\
275	0.5\\
276	0.5\\
277	0.5\\
278	0.5\\
279	0.5\\
280	0.5\\
281	0.5\\
282	0.5\\
283	0.5\\
284	0.5\\
285	0.5\\
286	0.5\\
287	0.5\\
288	0.5\\
289	0.5\\
290	0.5\\
291	0.5\\
292	0.5\\
293	0.5\\
294	0.5\\
295	0.5\\
296	0.5\\
297	0.5\\
298	0.5\\
299	0.5\\
300	0.5\\
301	0.5\\
302	0.5\\
303	0.5\\
304	0.5\\
305	0.5\\
306	0.5\\
307	0.5\\
308	0.5\\
309	0.5\\
310	0.5\\
311	0.5\\
312	0.5\\
313	0.5\\
314	0.5\\
315	0.5\\
316	0.5\\
317	0.5\\
318	0.5\\
319	0.5\\
320	0.5\\
321	0.5\\
322	0.5\\
323	0.5\\
324	0.5\\
325	0.5\\
326	0.5\\
327	0.5\\
328	0.5\\
329	0.5\\
330	0.5\\
331	0.5\\
332	0.5\\
333	0.5\\
334	0.5\\
335	0.5\\
336	0.5\\
337	0.5\\
338	0.5\\
339	0.5\\
340	0.5\\
341	0.5\\
342	0.5\\
343	0.5\\
344	0.5\\
345	0.5\\
346	0.5\\
347	0.5\\
348	0.5\\
349	0.5\\
350	0.5\\
351	0.5\\
352	0.5\\
353	0.5\\
354	0.5\\
355	0.5\\
356	0.5\\
357	0.5\\
358	0.5\\
359	0.5\\
360	0.5\\
361	0.5\\
362	0.5\\
363	0.5\\
364	0.5\\
365	0.5\\
366	0.5\\
367	0.5\\
368	0.5\\
369	0.5\\
370	0.5\\
371	0.5\\
372	0.5\\
373	0.5\\
374	0.5\\
375	0.5\\
376	0.5\\
377	0.5\\
378	0.5\\
379	0.5\\
380	0.5\\
381	0.5\\
382	0.5\\
383	0.5\\
384	0.5\\
385	0.5\\
386	0.5\\
387	0.5\\
388	0.5\\
389	0.5\\
390	0.5\\
391	0.5\\
392	0.5\\
393	0.5\\
394	0.5\\
395	0.5\\
396	0.5\\
397	0.5\\
398	0.5\\
399	0.5\\
400	0.5\\
401	0.5\\
402	0.5\\
403	0.5\\
404	0.5\\
405	0.5\\
406	0.5\\
407	0.5\\
408	0.5\\
409	0.5\\
410	0.5\\
411	0.5\\
412	0.5\\
413	0.5\\
414	0.5\\
415	0.5\\
416	0.5\\
417	0.5\\
418	0.5\\
419	0.5\\
420	0.5\\
421	0.5\\
422	0.5\\
423	0.5\\
424	0.5\\
425	0.5\\
426	0.5\\
427	0.5\\
428	0.5\\
429	0.5\\
430	0.5\\
431	0.5\\
432	0.5\\
433	0.5\\
434	0.5\\
435	0.5\\
436	0.5\\
437	0.5\\
438	0.5\\
439	0.5\\
440	0.5\\
441	0.5\\
442	0.5\\
443	0.5\\
444	0.5\\
445	0.5\\
446	0.5\\
447	0.5\\
448	0.5\\
449	0.5\\
450	0.5\\
451	0.5\\
452	0.5\\
453	0.5\\
454	0.5\\
455	0.5\\
456	0.5\\
457	0.5\\
458	0.5\\
459	0.5\\
460	0.5\\
461	0.5\\
462	0.5\\
463	0.5\\
464	0.5\\
465	0.5\\
466	0.5\\
467	0.5\\
468	0.5\\
469	0.5\\
470	0.5\\
471	0.5\\
472	0.5\\
473	0.5\\
474	0.5\\
475	0.5\\
476	0.5\\
477	0.5\\
478	0.5\\
479	0.5\\
480	0.5\\
481	0.5\\
482	0.5\\
483	0.5\\
484	0.5\\
485	0.5\\
486	0.5\\
487	0.5\\
488	0.5\\
489	0.5\\
490	0.5\\
491	0.5\\
492	0.5\\
493	0.5\\
494	0.5\\
495	0.5\\
496	0.5\\
497	0.5\\
498	0.5\\
499	0.5\\
500	0.5\\
501	0.5\\
502	0.5\\
503	0.5\\
504	0.5\\
505	0.5\\
506	0.5\\
507	0.5\\
508	0.5\\
509	0.5\\
510	0.5\\
511	0.5\\
512	0.5\\
513	0.5\\
514	0.5\\
515	0.5\\
516	0.5\\
517	0.5\\
518	0.5\\
519	0.5\\
520	0.5\\
521	0.5\\
522	0.5\\
523	0.5\\
524	0.5\\
525	0.5\\
526	0.5\\
527	0.5\\
528	0.5\\
529	0.5\\
530	0.5\\
531	0.5\\
532	0.5\\
533	0.5\\
534	0.5\\
535	0.5\\
536	0.5\\
537	0.5\\
538	0.5\\
539	0.5\\
540	0.5\\
541	0.5\\
542	0.5\\
543	0.5\\
544	0.5\\
545	0.5\\
546	0.5\\
547	0.5\\
548	0.5\\
549	0.5\\
550	0.5\\
551	0.5\\
552	0.5\\
553	0.5\\
554	0.5\\
555	0.5\\
556	0.5\\
557	0.5\\
558	0.5\\
559	0.5\\
560	0.5\\
561	0.5\\
562	0.5\\
563	0.5\\
564	0.5\\
565	0.5\\
566	0.5\\
567	0.5\\
568	0.5\\
569	0.5\\
570	0.5\\
571	0.5\\
572	0.5\\
573	0.5\\
574	0.5\\
575	0.5\\
576	0.5\\
577	0.5\\
578	0.5\\
579	0.5\\
580	0.5\\
581	0.5\\
582	0.5\\
583	0.5\\
584	0.5\\
585	0.5\\
586	0.5\\
587	0.5\\
588	0.5\\
589	0.5\\
590	0.5\\
591	0.5\\
592	0.5\\
593	0.5\\
594	0.5\\
595	0.5\\
596	0.5\\
597	0.5\\
598	0.5\\
599	0.5\\
600	0.5\\
};
\addplot[const plot, color=mycolor4, forget plot] table[row sep=crcr] {%
1	0\\
2	0\\
3	0\\
4	0\\
5	0\\
6	0\\
7	0\\
8	0\\
9	0\\
10	0\\
11	0\\
12	0\\
13	0\\
14	0\\
15	0\\
16	1\\
17	1\\
18	1\\
19	1\\
20	1\\
21	1\\
22	1\\
23	1\\
24	1\\
25	1\\
26	1\\
27	1\\
28	1\\
29	1\\
30	1\\
31	1\\
32	1\\
33	1\\
34	1\\
35	1\\
36	1\\
37	1\\
38	1\\
39	1\\
40	1\\
41	1\\
42	1\\
43	1\\
44	1\\
45	1\\
46	1\\
47	1\\
48	1\\
49	1\\
50	1\\
51	1\\
52	1\\
53	1\\
54	1\\
55	1\\
56	1\\
57	1\\
58	1\\
59	1\\
60	1\\
61	1\\
62	1\\
63	1\\
64	1\\
65	1\\
66	1\\
67	1\\
68	1\\
69	1\\
70	1\\
71	1\\
72	1\\
73	1\\
74	1\\
75	1\\
76	1\\
77	1\\
78	1\\
79	1\\
80	1\\
81	1\\
82	1\\
83	1\\
84	1\\
85	1\\
86	1\\
87	1\\
88	1\\
89	1\\
90	1\\
91	1\\
92	1\\
93	1\\
94	1\\
95	1\\
96	1\\
97	1\\
98	1\\
99	1\\
100	1\\
101	1\\
102	1\\
103	1\\
104	1\\
105	1\\
106	1\\
107	1\\
108	1\\
109	1\\
110	1\\
111	1\\
112	1\\
113	1\\
114	1\\
115	1\\
116	1\\
117	1\\
118	1\\
119	1\\
120	1\\
121	1\\
122	1\\
123	1\\
124	1\\
125	1\\
126	1\\
127	1\\
128	1\\
129	1\\
130	1\\
131	1\\
132	1\\
133	1\\
134	1\\
135	1\\
136	1\\
137	1\\
138	1\\
139	1\\
140	1\\
141	1\\
142	1\\
143	1\\
144	1\\
145	1\\
146	1\\
147	1\\
148	1\\
149	1\\
150	1\\
151	1\\
152	1\\
153	1\\
154	1\\
155	1\\
156	1\\
157	1\\
158	1\\
159	1\\
160	1\\
161	1\\
162	1\\
163	1\\
164	1\\
165	1\\
166	1\\
167	1\\
168	1\\
169	1\\
170	1\\
171	1\\
172	1\\
173	1\\
174	1\\
175	1\\
176	1\\
177	1\\
178	1\\
179	1\\
180	1\\
181	1\\
182	1\\
183	1\\
184	1\\
185	1\\
186	1\\
187	1\\
188	1\\
189	1\\
190	1\\
191	1\\
192	1\\
193	1\\
194	1\\
195	1\\
196	1\\
197	1\\
198	1\\
199	1\\
200	1\\
201	1\\
202	1\\
203	1\\
204	1\\
205	1\\
206	1\\
207	1\\
208	1\\
209	1\\
210	1\\
211	1\\
212	1\\
213	1\\
214	1\\
215	1\\
216	1\\
217	1\\
218	1\\
219	1\\
220	1\\
221	1\\
222	1\\
223	1\\
224	1\\
225	1\\
226	1\\
227	1\\
228	1\\
229	1\\
230	1\\
231	1\\
232	1\\
233	1\\
234	1\\
235	1\\
236	1\\
237	1\\
238	1\\
239	1\\
240	1\\
241	1\\
242	1\\
243	1\\
244	1\\
245	1\\
246	1\\
247	1\\
248	1\\
249	1\\
250	1\\
251	1\\
252	1\\
253	1\\
254	1\\
255	1\\
256	1\\
257	1\\
258	1\\
259	1\\
260	1\\
261	1\\
262	1\\
263	1\\
264	1\\
265	1\\
266	1\\
267	1\\
268	1\\
269	1\\
270	1\\
271	1\\
272	1\\
273	1\\
274	1\\
275	1\\
276	1\\
277	1\\
278	1\\
279	1\\
280	1\\
281	1\\
282	1\\
283	1\\
284	1\\
285	1\\
286	1\\
287	1\\
288	1\\
289	1\\
290	1\\
291	1\\
292	1\\
293	1\\
294	1\\
295	1\\
296	1\\
297	1\\
298	1\\
299	1\\
300	1\\
301	1\\
302	1\\
303	1\\
304	1\\
305	1\\
306	1\\
307	1\\
308	1\\
309	1\\
310	1\\
311	1\\
312	1\\
313	1\\
314	1\\
315	1\\
316	1\\
317	1\\
318	1\\
319	1\\
320	1\\
321	1\\
322	1\\
323	1\\
324	1\\
325	1\\
326	1\\
327	1\\
328	1\\
329	1\\
330	1\\
331	1\\
332	1\\
333	1\\
334	1\\
335	1\\
336	1\\
337	1\\
338	1\\
339	1\\
340	1\\
341	1\\
342	1\\
343	1\\
344	1\\
345	1\\
346	1\\
347	1\\
348	1\\
349	1\\
350	1\\
351	1\\
352	1\\
353	1\\
354	1\\
355	1\\
356	1\\
357	1\\
358	1\\
359	1\\
360	1\\
361	1\\
362	1\\
363	1\\
364	1\\
365	1\\
366	1\\
367	1\\
368	1\\
369	1\\
370	1\\
371	1\\
372	1\\
373	1\\
374	1\\
375	1\\
376	1\\
377	1\\
378	1\\
379	1\\
380	1\\
381	1\\
382	1\\
383	1\\
384	1\\
385	1\\
386	1\\
387	1\\
388	1\\
389	1\\
390	1\\
391	1\\
392	1\\
393	1\\
394	1\\
395	1\\
396	1\\
397	1\\
398	1\\
399	1\\
400	1\\
401	1\\
402	1\\
403	1\\
404	1\\
405	1\\
406	1\\
407	1\\
408	1\\
409	1\\
410	1\\
411	1\\
412	1\\
413	1\\
414	1\\
415	1\\
416	1\\
417	1\\
418	1\\
419	1\\
420	1\\
421	1\\
422	1\\
423	1\\
424	1\\
425	1\\
426	1\\
427	1\\
428	1\\
429	1\\
430	1\\
431	1\\
432	1\\
433	1\\
434	1\\
435	1\\
436	1\\
437	1\\
438	1\\
439	1\\
440	1\\
441	1\\
442	1\\
443	1\\
444	1\\
445	1\\
446	1\\
447	1\\
448	1\\
449	1\\
450	1\\
451	1\\
452	1\\
453	1\\
454	1\\
455	1\\
456	1\\
457	1\\
458	1\\
459	1\\
460	1\\
461	1\\
462	1\\
463	1\\
464	1\\
465	1\\
466	1\\
467	1\\
468	1\\
469	1\\
470	1\\
471	1\\
472	1\\
473	1\\
474	1\\
475	1\\
476	1\\
477	1\\
478	1\\
479	1\\
480	1\\
481	1\\
482	1\\
483	1\\
484	1\\
485	1\\
486	1\\
487	1\\
488	1\\
489	1\\
490	1\\
491	1\\
492	1\\
493	1\\
494	1\\
495	1\\
496	1\\
497	1\\
498	1\\
499	1\\
500	1\\
501	1\\
502	1\\
503	1\\
504	1\\
505	1\\
506	1\\
507	1\\
508	1\\
509	1\\
510	1\\
511	1\\
512	1\\
513	1\\
514	1\\
515	1\\
516	1\\
517	1\\
518	1\\
519	1\\
520	1\\
521	1\\
522	1\\
523	1\\
524	1\\
525	1\\
526	1\\
527	1\\
528	1\\
529	1\\
530	1\\
531	1\\
532	1\\
533	1\\
534	1\\
535	1\\
536	1\\
537	1\\
538	1\\
539	1\\
540	1\\
541	1\\
542	1\\
543	1\\
544	1\\
545	1\\
546	1\\
547	1\\
548	1\\
549	1\\
550	1\\
551	1\\
552	1\\
553	1\\
554	1\\
555	1\\
556	1\\
557	1\\
558	1\\
559	1\\
560	1\\
561	1\\
562	1\\
563	1\\
564	1\\
565	1\\
566	1\\
567	1\\
568	1\\
569	1\\
570	1\\
571	1\\
572	1\\
573	1\\
574	1\\
575	1\\
576	1\\
577	1\\
578	1\\
579	1\\
580	1\\
581	1\\
582	1\\
583	1\\
584	1\\
585	1\\
586	1\\
587	1\\
588	1\\
589	1\\
590	1\\
591	1\\
592	1\\
593	1\\
594	1\\
595	1\\
596	1\\
597	1\\
598	1\\
599	1\\
600	1\\
};
\end{axis}
\end{tikzpicture}%
\caption{Odpowiedzi skokowe toru wej�cie-wyj�cie procesu}
\label{tor_uy}
\end{figure}

\section{Charakterystyka statyczna}
W celu wyznaczenia charakterystyki statycznej procesu wyznaczono odpowied� uk�adu w stanie ustalonym dla pobudze� r�nymi warto�ciami sygna�u steruj�cego. Zebrane wyniki przedstawiono na rys.~\ref{char_stat}.

\begin{figure}[h!]
\centering
% This file was created by matlab2tikz.
%
\definecolor{mycolor1}{rgb}{0.00000,0.44700,0.74100}%
%
\begin{tikzpicture}

\begin{axis}[%
width=4.521in,
height=3.566in,
at={(0.758in,0.481in)},
scale only axis,
xmin=-1,
xmax=1,
xlabel style={font=\color{white!15!black}},
xlabel={U},
ymin=-1,
ymax=6,
ylabel style={font=\color{white!15!black}},
ylabel={Y},
axis background/.style={fill=white},
title style={font=\bfseries},
title={Charakterystyka statyczna Y(U)}
]
\addplot [color=mycolor1, forget plot]
  table[row sep=crcr]{%
-1	-0.184717123272878\\
-0.99	-0.183522304639754\\
-0.98	-0.182329074656195\\
-0.97	-0.181137372398113\\
-0.96	-0.179947140647028\\
-0.95	-0.178758325993739\\
-0.94	-0.177570878933212\\
-0.93	-0.176384753949267\\
-0.92	-0.175199909587527\\
-0.91	-0.174016308514938\\
-0.9	-0.17283391756403\\
-0.89	-0.17165270775992\\
-0.88	-0.170472654327877\\
-0.87	-0.169293736679087\\
-0.86	-0.168115938372062\\
-0.85	-0.166939247046905\\
-0.84	-0.165763654329414\\
-0.83	-0.164589155701785\\
-0.82	-0.163415750336366\\
-0.81	-0.162243440888675\\
-0.8	-0.161072233245567\\
-0.79	-0.15990213622412\\
-0.78	-0.158733161216472\\
-0.77	-0.157565321775482\\
-0.76	-0.156398633135684\\
-0.75	-0.15523311166362\\
-0.74	-0.154068774231172\\
-0.73	-0.152905637505096\\
-0.72	-0.15174371714543\\
-0.71	-0.150583026904976\\
-0.7	-0.149423577621492\\
-0.69	-0.148265376093675\\
-0.68	-0.147108423831426\\
-0.67	-0.145952715670251\\
-0.66	-0.144798238239021\\
-0.65	-0.143644968269615\\
-0.64	-0.1424928707363\\
-0.63	-0.141341896811923\\
-0.62	-0.14019198162728\\
-0.61	-0.13904304181924\\
-0.6	-0.137894972852373\\
-0.59	-0.136747646098071\\
-0.58	-0.135600905654278\\
-0.57	-0.134454564888136\\
-0.56	-0.133308402683002\\
-0.55	-0.132162159370456\\
-0.54	-0.131015532327112\\
-0.53	-0.129868171215194\\
-0.52	-0.128719672845118\\
-0.51	-0.127569575637525\\
-0.5	-0.126417353661574\\
-0.49	-0.125262410225635\\
-0.48	-0.124104070996036\\
-0.47	-0.122941576619038\\
-0.46	-0.121774074820941\\
-0.45	-0.120600611961007\\
-0.44	-0.119420124011936\\
-0.43	-0.118231426942785\\
-0.42	-0.117033206479673\\
-0.41	-0.115824007220254\\
-0.4	-0.114602221078961\\
-0.39	-0.113366075041269\\
-0.38	-0.112113618206924\\
-0.37	-0.110842708104156\\
-0.36	-0.109550996259405\\
-0.35	-0.108235913010143\\
-0.34	-0.106894651551913\\
-0.33	-0.105524151214906\\
-0.32	-0.104121079970169\\
-0.31	-0.102681816171049\\
-0.3	-0.101202429541689\\
-0.29	-0.0996786614313981\\
-0.28	-0.0981059043615184\\
-0.27	-0.096479180900082\\
-0.26	-0.0947931219090757\\
-0.25	-0.0930419442195595\\
-0.24	-0.091219427801203\\
-0.23	-0.0893188925050176\\
-0.22	-0.0873331744711478\\
-0.21	-0.0852546023074913\\
-0.2	-0.0830749731596015\\
-0.19	-0.0807855288076904\\
-0.18	-0.0783769319424889\\
-0.17	-0.0758392427880986\\
-0.16	-0.0731618962566147\\
-0.15	-0.0703336798360086\\
-0.14	-0.0673427124293024\\
-0.13	-0.064176424379162\\
-0.12	-0.0608215389273943\\
-0.11	-0.0572640553730882\\
-0.1	-0.0534892342059421\\
-0.09	-0.0494815845022437\\
-0.08	-0.0452248538795945\\
-0.07	-0.040702021312345\\
-0.0599999999999999	-0.0358952931123599\\
-0.0499999999999999	-0.030786102378705\\
-0.04	-0.0253551122146744\\
-0.03	-0.0195822230008141\\
-0.02	-0.0134465839978496\\
-0.01	-0.00692660953332257\\
0	0\\
0.01	0.00735623213746833\\
0.02	0.015165731168718\\
0.03	0.023452761681231\\
0.04	0.0322421667551587\\
0.0499999999999999	0.0415593207980987\\
0.0599999999999999	0.0514300770684109\\
0.07	0.0618807100000181\\
0.08	0.072937852509691\\
0.09	0.0846284285386037\\
0.1	0.0969795811523933\\
0.11	0.110018596596796\\
0.12	0.123772824777801\\
0.13	0.138269596704718\\
0.14	0.153536139500013\\
0.15	0.169599489639745\\
0.16	0.186486405141323\\
0.17	0.204223277459664\\
0.18	0.222836043887286\\
0.19	0.242350101277184\\
0.2	0.26279022191854\\
0.21	0.284180472393599\\
0.22	0.306544136228915\\
0.23	0.329903641125518\\
0.24	0.354280491510396\\
0.25	0.379695207096614\\
0.26	0.406167268072171\\
0.27	0.433715067459386\\
0.28	0.462355871098755\\
0.29	0.49210578561533\\
0.3	0.522979734623781\\
0.31	0.554991443322274\\
0.32	0.588153431517368\\
0.33	0.62247701501444\\
0.34	0.657972315202734\\
0.35	0.694648276563131\\
0.36	0.732512691731976\\
0.37	0.771572233667469\\
0.38	0.811832494387691\\
0.39	0.853298029682516\\
0.4	0.89597240914615\\
0.41	0.939858270833747\\
0.42	0.98495737981436\\
0.43	1.03127068987374\\
0.44	1.07879840761355\\
0.45	1.1275400581982\\
0.46	1.17749455201556\\
0.47	1.22866025154285\\
0.48	1.2810350377423\\
0.49	1.33461637535189\\
0.5	1.3894013764838\\
0.51	1.44538686199428\\
0.52	1.50256942014454\\
0.53	1.5609454621293\\
0.54	1.62051127410868\\
0.55	1.68126306543749\\
0.56	1.74319701284409\\
0.57	1.80630930036672\\
0.58	1.87059615490899\\
0.59	1.93605387732643\\
0.6	2.002678869003\\
0.61	2.07046765391946\\
0.62	2.13941689625424\\
0.63	2.20952341359233\\
0.64	2.28078418584743\\
0.65	2.35319636002979\\
0.66	2.42675725101292\\
0.67	2.50146433847161\\
0.68	2.57731526017733\\
0.69	2.65430780184872\\
0.7	2.73243988376257\\
0.71	2.81170954433578\\
0.72	2.89211492089163\\
0.73	2.97365422782414\\
0.74	3.0563257323728\\
0.75	3.14012772821743\\
0.76	3.22505850709788\\
0.77	3.31111632865909\\
0.78	3.3982993887149\\
0.79	3.48660578611857\\
0.8	3.57603348841963\\
0.81	3.66658029648062\\
0.82	3.75824380821956\\
0.83	3.85102138163662\\
0.84	3.94491009727698\\
0.85	4.03990672027457\\
0.86	4.13600766211575\\
0.87	4.23320894225476\\
0.88	4.33150614970886\\
0.89	4.4308944047545\\
0.9	4.53136832084196\\
0.91	4.6329219668413\\
0.92	4.73554882972979\\
0.93	4.83924177782545\\
0.94	4.94399302467115\\
0.95	5.04979409366959\\
0.96	5.15663578356662\\
0.97	5.26450813488143\\
0.98	5.37340039737659\\
0.99	5.4833009986634\\
1	5.59419751403355\\
};
\end{axis}
\end{tikzpicture}%
\caption{Charakterystyka statyczna procesu}
\label{char_stat}
\end{figure}

\section{W�a�ciwo�ci statyczne i dynamiczne}
Na podstawie rys.~\ref{char_stat} mo�na powiedzie�, �e w�a�ciwo�ci statyczne i dynamiczne obiektu s� nieliniowe.

\section{Implementacja}
Do przeprowadzenia eksperymentu wykorzystany zosta� skrypt \verb+zad2.m+.