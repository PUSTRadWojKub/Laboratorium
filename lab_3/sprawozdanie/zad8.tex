\chapter{Dob�r parametr�w \lambda}

W celu poprawy jako�ci regulacji zosta�y dobrane zosta�y parametry $\lambda$ dla regulator�w z 2, 3 oraz 4 regulatorami lokalnymi, z funkcjami przynale�no�ci postaci funkcji Gaussa.

\section{Wyniki symulacji}

\subsection{Liczba regulator�w lokalnych $n_r = 2$}

Parametry regulatora: $\lambda^{1} = 0,2, \lambda^{2} = 0,6$

\begin{figure}[h!]
	\centering
	% This file was created by matlab2tikz.
%
\definecolor{mycolor1}{rgb}{0.00000,0.44700,0.74100}%
\definecolor{mycolor2}{rgb}{0.85000,0.32500,0.09800}%
%
\begin{tikzpicture}[scale=0.8]

\begin{axis}[%
width=4.521in,
height=3.566in,
at={(0.758in,0.481in)},
scale only axis,
xmin=0,
xmax=1200,
xlabel style={font=\color{white!15!black}},
xlabel={$k$},
ymin=-1,
ymax=6,
ylabel style={font=\color{white!15!black}},
ylabel={$y(k)$, $y^{zad}(k)$},
axis background/.style={fill=white},
title style={font=\bfseries},
title={Sygna� wyj�ciowy i zadany},
xmajorgrids,
ymajorgrids,
legend style={at={(0.97,0.03)}, anchor=south east, legend cell align=left, align=left, draw=white!15!black}
]
\addplot [color=mycolor1]
  table[row sep=crcr]{%
1	0\\
2	0\\
3	0\\
4	0\\
5	0\\
6	0\\
7	0\\
8	0\\
9	0\\
10	0\\
11	0\\
12	0\\
13	0\\
14	0\\
15	0\\
16	0\\
17	0\\
18	0\\
19	0\\
20	0\\
21	0\\
22	0\\
23	0\\
24	0\\
25	0\\
26	0\\
27	0\\
28	0\\
29	0\\
30	0\\
31	0\\
32	0\\
33	0\\
34	0\\
35	0\\
36	0\\
37	0\\
38	0\\
39	0\\
40	0\\
41	0\\
42	0\\
43	0\\
44	0\\
45	0\\
46	0\\
47	0\\
48	0\\
49	0\\
50	0\\
51	0\\
52	0\\
53	0\\
54	0\\
55	0\\
56	0\\
57	0\\
58	0\\
59	0\\
60	0\\
61	0\\
62	0\\
63	0\\
64	0\\
65	0\\
66	0\\
67	0\\
68	0\\
69	0\\
70	0\\
71	0\\
72	0\\
73	0\\
74	0\\
75	0\\
76	0\\
77	0\\
78	0\\
79	0\\
80	0\\
81	0\\
82	0\\
83	0\\
84	0\\
85	0\\
86	0\\
87	0\\
88	0\\
89	0\\
90	0\\
91	0\\
92	0\\
93	0\\
94	0\\
95	0\\
96	0\\
97	0\\
98	0\\
99	0\\
100	0\\
101	0\\
102	0\\
103	0\\
104	0\\
105	0.0796751012205679\\
106	0.206605417617173\\
107	0.316915710132445\\
108	0.385979353206721\\
109	0.425549346142723\\
110	0.434622855189415\\
111	0.425244285660089\\
112	0.404546580682\\
113	0.383166805949926\\
114	0.370772559240134\\
115	0.374396463597355\\
116	0.396223952122339\\
117	0.432703513170075\\
118	0.475841903380318\\
119	0.516385091414635\\
120	0.546789687241249\\
121	0.562522165610268\\
122	0.562166087533722\\
123	0.547250663564547\\
124	0.521975050806532\\
125	0.492568292303312\\
126	0.466158034207261\\
127	0.449257430593169\\
128	0.44606471221057\\
129	0.456992040092605\\
130	0.478304285173944\\
131	0.503535890143013\\
132	0.525971835185287\\
133	0.540593428881869\\
134	0.544793995007181\\
135	0.538363530533296\\
136	0.523274440260419\\
137	0.503288156216829\\
138	0.483229953466099\\
139	0.467943199062443\\
140	0.461098758681766\\
141	0.464141706619519\\
142	0.475792596849779\\
143	0.492485620560605\\
144	0.509640596916794\\
145	0.523092704601423\\
146	0.530038916642897\\
147	0.529395847234192\\
148	0.521790348349192\\
149	0.509326649632152\\
150	0.49512923417469\\
151	0.482662713188188\\
152	0.474919252294196\\
153	0.473648523470055\\
154	0.478867979288002\\
155	0.488883229240769\\
156	0.500867053612288\\
157	0.511749641300128\\
158	0.519032507497567\\
159	0.52128806762643\\
160	0.518331050830545\\
161	0.511148425024293\\
162	0.501634841680985\\
163	0.492158237551755\\
164	0.485009661020396\\
165	0.481844141973602\\
166	0.483258173657242\\
167	0.488650373792248\\
168	0.496437121330662\\
169	0.504548944792595\\
170	0.511010503741739\\
171	0.514407904472096\\
172	0.514149260997488\\
173	0.510518850686875\\
174	0.504556839531817\\
175	0.497796306102311\\
176	0.491899045673561\\
177	0.488257274883299\\
178	0.487653329222296\\
179	0.490074579972067\\
180	0.494748398204749\\
181	0.500388900304362\\
182	0.505566085511904\\
183	0.509071725756033\\
184	0.510183957678674\\
185	0.508789307551474\\
186	0.505363426386639\\
187	0.500830622516801\\
188	0.496333036865085\\
189	0.492954585651301\\
190	0.491459889359974\\
191	0.492114133217328\\
192	0.494635874808833\\
193	0.498297532495441\\
194	0.502139019013684\\
195	0.505223793797602\\
196	0.506862828616577\\
197	0.506756207744983\\
198	0.505033398147954\\
199	0.502196366444753\\
200	0.49898420566749\\
201	0.496189394148654\\
202	0.494465973453327\\
203	0.494175064263078\\
204	0.495307629241821\\
205	0.497505205149843\\
206	0.500170250972831\\
207	0.502630327777103\\
208	0.504307309523242\\
209	0.504848364519755\\
210	0.504192647299585\\
211	0.502566288913876\\
212	0.500413036332752\\
213	0.498278883853304\\
214	0.49667737193462\\
215	0.495966949954494\\
216	0.496270292299223\\
217	0.497455713562249\\
218	0.499183920027785\\
219	0.501004521317037\\
220	0.502473430438741\\
221	0.50325945784407\\
222	0.503215361746317\\
223	0.502400506811126\\
224	0.501054337424545\\
225	0.499529871073157\\
226	0.498203944462953\\
227	0.497385561758822\\
228	0.49724424912527\\
229	0.497775675398803\\
230	0.498812144878888\\
231	0.50007329891329\\
232	0.501241582368267\\
233	0.502041537389883\\
234	0.502303110115573\\
235	0.50199533373244\\
236	0.501225086063381\\
237	0.500203732931238\\
238	0.499191103079064\\
239	0.498430726586124\\
240	0.498091933241489\\
241	0.498232610858494\\
242	0.498791118107203\\
243	0.499608128425639\\
244	0.500471278814432\\
245	0.501169944316049\\
246	0.50154587440368\\
247	0.501527792981914\\
248	0.501143039215423\\
249	0.500505200917749\\
250	0.499782116927036\\
251	0.499152731272118\\
252	0.498763463197185\\
253	0.498694565184196\\
254	0.498944261006419\\
255	0.499433788242591\\
256	0.500031018736431\\
257	0.500585689318591\\
258	0.500966791914605\\
259	0.501092880788325\\
260	0.500948555581843\\
261	0.5005841958095\\
262	0.500100096112181\\
263	0.499619629327349\\
264	0.499258337432658\\
265	0.499096513955745\\
266	0.499161736771475\\
267	0.499425145478909\\
268	0.499811679029869\\
269	0.500220968782073\\
270	0.500553103092592\\
271	0.500732667760188\\
272	0.500725383364996\\
273	0.500543867884242\\
274	0.50024186330575\\
275	0.499898988704883\\
276	0.499600159858656\\
277	0.499414856029237\\
278	0.499381221518911\\
279	0.49949861176741\\
280	0.499729965204479\\
281	0.500012882232783\\
282	0.50027619318844\\
283	0.500457641506894\\
284	0.500518333103839\\
285	0.500450691614399\\
286	0.500278434612892\\
287	0.500049065433266\\
288	0.499821101140768\\
289	0.499649372755987\\
290	0.499572023229575\\
291	0.499602249288634\\
292	0.499726538182973\\
293	0.499909473590282\\
294	0.500103565747653\\
295	0.500261417050467\\
296	0.500347133442058\\
297	0.500344279116431\\
298	0.500258682664515\\
299	0.500115738994035\\
300	0.499953176899776\\
301	0.499811278814652\\
302	0.499723036656124\\
303	0.499706614993479\\
304	0.499761816755973\\
305	0.59921351151597\\
306	0.770399779910621\\
307	0.928179995346296\\
308	1.049571419198\\
309	1.13277783892016\\
310	1.17867237718618\\
311	1.18555087860271\\
312	1.15501194515611\\
313	1.09458513272743\\
314	1.01726422209457\\
315	0.939544016601553\\
316	0.878822559008246\\
317	0.849364725696417\\
318	0.857238452596647\\
319	0.897212307044549\\
320	0.955382173084742\\
321	1.01619022279725\\
322	1.06758157468263\\
323	1.10177593193773\\
324	1.11445768986592\\
325	1.10461949032862\\
326	1.07489702354451\\
327	1.03154329153835\\
328	0.983619765900052\\
329	0.94136481717319\\
330	0.913885109164633\\
331	0.906646688013184\\
332	0.919768990744354\\
333	0.948242457351753\\
334	0.984127926507499\\
335	1.0192837489922\\
336	1.04712776851269\\
337	1.06328405976668\\
338	1.06575632882909\\
339	1.05500069754653\\
340	1.03383085443465\\
341	1.00697965123815\\
342	0.980244270899838\\
343	0.959284063126914\\
344	0.948299804017991\\
345	0.948989735067563\\
346	0.960217924161688\\
347	0.978573185476929\\
348	0.999526413374685\\
349	1.01862376190854\\
350	1.03232949488556\\
351	1.03848198144887\\
352	1.03648234062185\\
353	1.02728297806172\\
354	1.01316502077643\\
355	0.997286839026513\\
356	0.983037641376953\\
357	0.973305335460571\\
358	0.969835277612309\\
359	0.972876642581873\\
360	0.981233447900272\\
361	0.99268904930514\\
362	1.00462209397778\\
363	1.01459992958661\\
364	1.02081889153055\\
365	1.02236018745329\\
366	1.01927426210399\\
367	1.01250640337633\\
368	1.00367373632994\\
369	0.994719514120108\\
370	0.987500758843365\\
371	0.983395053218584\\
372	0.983023546657205\\
373	0.986164347497501\\
374	0.991872912255172\\
375	0.99875820682364\\
376	1.00532198374438\\
377	1.01027159052369\\
378	1.01274927215962\\
379	1.01245456477832\\
380	1.00965654735228\\
381	1.00510336160765\\
382	0.999846755093134\\
383	0.995013107286009\\
384	0.991566106408171\\
385	0.990112864527218\\
386	0.990798140243706\\
387	0.993308884417215\\
388	0.996980253162873\\
389	1.00096742189205\\
390	1.00443584094833\\
391	1.00672740080925\\
392	1.00747406924448\\
393	1.006645499463\\
394	1.0045291274337\\
395	1.00165100403017\\
396	0.998654463389764\\
397	0.996161375794069\\
398	0.994644949503871\\
399	0.994341181470567\\
400	0.995216992986651\\
401	0.996998687699199\\
402	0.999249292393736\\
403	1.00147279390472\\
404	1.0032201265508\\
405	1.00417528128217\\
406	1.00420715275989\\
407	1.00338087780581\\
408	1.00192998899599\\
409	1.00019729347404\\
410	0.998557657427355\\
411	0.997339057458997\\
412	0.996758346854202\\
413	0.996884614827548\\
414	0.997636315382807\\
415	0.998810261096758\\
416	1.00013349270029\\
417	1.00132488915574\\
418	1.00215285976871\\
419	1.00247795357182\\
420	1.0022734790065\\
421	1.00162206582709\\
422	1.00069071127292\\
423	0.999690688239659\\
424	0.998831266233964\\
425	0.998277036633494\\
426	0.998117461889989\\
427	0.998354234044881\\
428	0.998907824288026\\
429	0.999640318439328\\
430	1.00038836387995\\
431	1.00099851145936\\
432	1.0013575422064\\
433	1.0014121479338\\
434	1.00117497805477\\
435	1.00071696479571\\
436	1.00014849060638\\
437	0.99959396046558\\
438	0.99916537321133\\
439	0.998940365121572\\
440	0.998948948249594\\
441	0.999171066886232\\
442	0.999544652025716\\
443	0.999981656093744\\
444	1.00038809150482\\
445	1.00068363078598\\
446	1.00081683071572\\
447	1.00077328403811\\
448	1.00057564251499\\
449	1.00027614599272\\
450	0.999943726186577\\
451	0.999648689820776\\
452	0.999448271993429\\
453	0.999375953761871\\
454	0.999436463008337\\
455	0.99960705778845\\
456	0.999844335675194\\
457	1.00009472412473\\
458	1.00030620177998\\
459	1.00043876063737\\
460	1.00047159092039\\
461	1.0004058025201\\
462	1.0002624886535\\
463	1.00007688461343\\
464	0.999890101118536\\
465	0.999740293364002\\
466	0.999655108340793\\
467	0.999646859203163\\
468	0.999711208271249\\
469	0.999829357059112\\
470	0.999973017876783\\
471	1.00011092729615\\
472	1.00021544791608\\
473	1.00026790786629\\
474	1.00026169552277\\
475	1.00020266066161\\
476	1.00010695220975\\
477	0.999996929699134\\
478	0.999896126645862\\
479	0.999824363393012\\
480	0.999793994922008\\
481	0.999807973300733\\
482	0.999859980716857\\
483	0.999936444137898\\
484	1.00001987144313\\
485	1.00009272314481\\
486	1.00014098805263\\
487	1.00015675894137\\
488	1.00013936404841\\
489	1.00009493827039\\
490	1.00003464303141\\
491	0.9999720014682\\
492	0.999919959462502\\
493	0.999888292414267\\
494	0.999881860009778\\
495	0.999899999555276\\
496	0.999937093079695\\
497	0.999984100471528\\
498	1.00003066976909\\
499	1.00006734880603\\
500	1.0000874397482\\
501	1.00008814730235\\
502	1.00007084262996\\
503	1.00004045857462\\
504	1.00000420697314\\
505	0.651551912036594\\
506	0.340887779003685\\
507	0.13357595695711\\
508	0.00633515925326856\\
509	-0.0702880839269912\\
510	-0.116217488863525\\
511	-0.143717208760237\\
512	-0.160177748691603\\
513	-0.169931747097354\\
514	-0.173928053526468\\
515	-0.172209979392575\\
516	-0.166380481578781\\
517	-0.158601767732937\\
518	-0.151124845727572\\
519	-0.145958072502412\\
520	-0.143833260902463\\
521	-0.143833556956132\\
522	-0.144482334253096\\
523	-0.144682773730105\\
524	-0.14395300506085\\
525	-0.142308997863227\\
526	-0.14007081796675\\
527	-0.137669433960472\\
528	-0.135473231241673\\
529	-0.133675089486561\\
530	-0.13227486705664\\
531	-0.131143554112068\\
532	-0.130116823226485\\
533	-0.129068844563652\\
534	-0.127945318329361\\
535	-0.126759568653907\\
536	-0.125566203190606\\
537	-0.124427864707356\\
538	-0.123388176101537\\
539	-0.122459219766988\\
540	-0.121624713915722\\
541	-0.120853321756904\\
542	-0.120113856677982\\
543	-0.119385933192767\\
544	-0.118663482365251\\
545	-0.117952014331636\\
546	-0.117262459120696\\
547	-0.116604809886047\\
548	-0.115983985433431\\
549	-0.115398855382207\\
550	-0.11484390603384\\
551	-0.11431214823332\\
552	-0.113797790463126\\
553	-0.113297722560827\\
554	-0.112811586759948\\
555	-0.112340811190716\\
556	-0.111887260246815\\
557	-0.111452108038196\\
558	-0.111035307645171\\
559	-0.110635704868109\\
560	-0.110251621539028\\
561	-0.109881360621161\\
562	-0.109523624378983\\
563	-0.109177702831702\\
564	-0.108843389924794\\
565	-0.108520735028013\\
566	-0.108209775324448\\
567	-0.107910356850297\\
568	-0.107622086071012\\
569	-0.107344389408122\\
570	-0.107076637607809\\
571	-0.106818267026378\\
572	-0.10656884997966\\
573	-0.106328099578456\\
574	-0.106095822754985\\
575	-0.105871851731078\\
576	-0.105655987017777\\
577	-0.105447972097922\\
578	-0.10524750138643\\
579	-0.105054249371294\\
580	-0.104867903844338\\
581	-0.104688189000501\\
582	-0.10451487176481\\
583	-0.104347752982586\\
584	-0.104186650659827\\
585	-0.104031383736384\\
586	-0.103881762490064\\
587	-0.103737587503883\\
588	-0.103598655286275\\
589	-0.103464766537385\\
590	-0.103335733065763\\
591	-0.103211380923629\\
592	-0.103091549433911\\
593	-0.10297608747057\\
594	-0.10286484907602\\
595	-0.102757690223115\\
596	-0.102654467619778\\
597	-0.102555039430319\\
598	-0.102459267077438\\
599	-0.102367017087673\\
600	-0.102278162196529\\
601	-0.102192581421547\\
602	-0.102110159289275\\
603	-0.102030784683982\\
604	-0.101954349813342\\
605	-0.101880749614424\\
606	-0.101809881670527\\
607	-0.101741646494574\\
608	-0.101675947931461\\
609	-0.101612693451017\\
610	-0.101551794205868\\
611	-0.101493164853083\\
612	-0.101436723231311\\
613	-0.101382390018444\\
614	-0.101330088471239\\
615	-0.101279744290972\\
616	-0.101231285600284\\
617	-0.101184642978333\\
618	-0.101139749493426\\
619	-0.101096540689909\\
620	-0.101054954515833\\
621	-0.101014931205149\\
622	-0.10097641314301\\
623	-0.100939344742715\\
624	-0.10090367235178\\
625	-0.100869344189655\\
626	-0.100836310307528\\
627	-0.10080452255535\\
628	-0.100773934543016\\
629	-0.100744501588985\\
630	-0.100716180656906\\
631	-0.100688930286003\\
632	-0.100662710522567\\
633	-0.100637482858231\\
634	-0.100613210177277\\
635	-0.100589856711761\\
636	-0.100567388001169\\
637	-0.100545770852984\\
638	-0.100524973301731\\
639	-0.100504964565823\\
640	-0.100485715003123\\
641	-0.100467196066917\\
642	-0.100449380263921\\
643	-0.100432241115252\\
644	-0.100415753120391\\
645	-0.100399891723484\\
646	-0.100384633281055\\
647	-0.100369955030357\\
648	-0.100355835057962\\
649	-0.10034225226863\\
650	-0.100329186354813\\
651	-0.100316617767164\\
652	-0.10030452768638\\
653	-0.100292897996436\\
654	-0.100281711259109\\
655	-0.10027095068956\\
656	-0.100260600132724\\
657	-0.100250644040362\\
658	-0.100241067448718\\
659	-0.100231855956802\\
660	-0.100222995705406\\
661	-0.1002144733569\\
662	-0.100206276075846\\
663	-0.100198391510414\\
664	-0.100190807774517\\
665	-0.100183513430608\\
666	-0.100176497473065\\
667	-0.100169749312125\\
668	-0.100163258758357\\
669	-0.100157016007672\\
670	-0.100151011626882\\
671	-0.100145236539798\\
672	-0.100139682013869\\
673	-0.100134339647328\\
674	-0.100129201356812\\
675	-0.100124259365441\\
676	-0.100119506191321\\
677	-0.100114934636456\\
678	-0.100110537776064\\
679	-0.100106308948285\\
680	-0.100102241744267\\
681	-0.100098329998636\\
682	-0.100094567780311\\
683	-0.100090949383675\\
684	-0.100087469320073\\
685	-0.100084122309617\\
686	-0.100080903273303\\
687	-0.10007780732541\\
688	-0.100074829766191\\
689	-0.100071966074831\\
690	-0.10006921190267\\
691	-0.100066563066683\\
692	-0.100064015543205\\
693	-0.100061565461891\\
694	-0.100059209099907\\
695	-0.100056942876327\\
696	-0.100054763346756\\
697	-0.100052667198139\\
698	-0.100050651243777\\
699	-0.100048712418527\\
700	-0.10004684777418\\
701	-0.100045054475015\\
702	-0.100043329793529\\
703	-0.100041671106315\\
704	-0.100040075890107\\
705	-0.0156289425042693\\
706	0.252191429066518\\
707	0.639986006072144\\
708	1.09837562680552\\
709	1.58976785295494\\
710	2.08631267116435\\
711	2.56809326207283\\
712	3.02155035710288\\
713	3.43813047262347\\
714	3.81314284157108\\
715	4.14480655154528\\
716	4.43346773893138\\
717	4.68096626048389\\
718	4.89013172370325\\
719	5.06438982915345\\
720	5.20746144334651\\
721	5.32313851077864\\
722	5.41512269998784\\
723	5.48691446677225\\
724	5.54174194106764\\
725	5.55856010600797\\
726	5.51075375943432\\
727	5.39437303257547\\
728	5.22813083651575\\
729	5.03932493420279\\
730	4.85431251368691\\
731	4.69568268601401\\
732	4.58046883276374\\
733	4.51761549879514\\
734	4.50728099034902\\
735	4.53957619983211\\
736	4.60155969076379\\
737	4.68227236261316\\
738	4.77325750294461\\
739	4.86814440923241\\
740	4.96227611749501\\
741	5.05238217632868\\
742	5.1362954606712\\
743	5.21158053675931\\
744	5.26774639157589\\
745	5.29587568553872\\
746	5.29359207645699\\
747	5.2633799735468\\
748	5.21008149837486\\
749	5.1400316012303\\
750	5.06097154414169\\
751	4.98148819111708\\
752	4.90981127231841\\
753	4.85257169295895\\
754	4.81404706363369\\
755	4.79599824602216\\
756	4.79792893697888\\
757	4.81756164618792\\
758	4.85137754779177\\
759	4.89512829315611\\
760	4.94426153010467\\
761	4.99426603660455\\
762	5.04095506898934\\
763	5.08070497753825\\
764	5.11065873282281\\
765	5.12889356768937\\
766	5.13454202038245\\
767	5.12784853978465\\
768	5.11014114265393\\
769	5.0837014081963\\
770	5.05152837907098\\
771	5.01701245650674\\
772	4.98355918223699\\
773	4.95422043439632\\
774	4.93139262287638\\
775	4.91662515013783\\
776	4.9105543605512\\
777	4.91295039457753\\
778	4.92284673678622\\
779	4.93871751069046\\
780	4.95867331645881\\
781	4.98065499620014\\
782	5.00261271904502\\
783	5.02266362028555\\
784	5.03922413371477\\
785	5.05111364513166\\
786	5.05762546692705\\
787	5.05856069248034\\
788	5.05422116195636\\
789	5.04535996287085\\
790	5.03309186408684\\
791	5.01877103515496\\
792	5.00384859033775\\
793	4.98972639947084\\
794	4.97762468656669\\
795	4.968478554569\\
796	4.96287327258285\\
797	4.9610215043865\\
798	4.96277943661636\\
799	4.96769463002528\\
800	4.97507605667022\\
801	4.98407657003564\\
802	4.99377921241715\\
803	5.0032803704746\\
804	5.01176427883717\\
805	5.01856459163753\\
806	5.0232097552681\\
807	5.02544986598235\\
808	5.0252637440988\\
809	5.02284622663839\\
810	5.01857719927187\\
811	5.01297554649202\\
812	5.00664275505878\\
813	5.00020204793947\\
814	4.99423936883875\\
815	4.98925212679318\\
816	4.98561038874532\\
817	4.98353341341019\\
818	4.98308241012777\\
819	4.9841685457715\\
820	4.98657377917159\\
821	4.98998119725786\\
822	4.99401114392524\\
823	4.9982594669561\\
824	5.00233452721422\\
825	5.00589010237455\\
826	5.00865189888916\\
827	5.01043602418689\\
828	5.0111584543468\\
829	5.01083525362509\\
830	5.0095740417216\\
831	5.00755792065555\\
832	5.0050237016223\\
833	5.00223673782668\\
834	4.999464903447\\
835	4.99695421996194\\
836	4.99490831946161\\
837	4.99347339654062\\
838	4.99272961885863\\
839	4.99268924184679\\
840	4.99330100089527\\
841	4.99445980737091\\
842	4.9960203938447\\
843	4.99781334877555\\
844	4.99966193814648\\
845	5.00139820480062\\
846	5.00287703524552\\
847	5.00398715995606\\
848	5.00465838160393\\
849	5.00486468341656\\
850	5.00462323438842\\
851	5.00398965510544\\
852	5.00305021158663\\
853	5.00191183915543\\
854	5.00069104216066\\
855	4.99950275471086\\
856	4.99845018034268\\
857	4.99761646555071\\
858	4.99705882591964\\
859	4.99680546480701\\
860	4.99685533657093\\
861	4.9971805402496\\
862	4.99773090987797\\
863	4.99844021004005\\
864	4.99923325660967\\
865	5.00003326203305\\
866	5.00076874535733\\
867	5.00137943924703\\
868	5.00182075713388\\
869	5.00206654008834\\
870	5.00210997125871\\
871	5.00196271210316\\
872	5.00165246598896\\
873	5.00121929918031\\
874	5.00071113716119\\
875	5.00017889931169\\
876	4.99967173498121\\
877	4.99923278132121\\
878	4.99889578449977\\
879	4.9986828211997\\
880	4.99860323874366\\
881	4.99865381239549\\
882	4.99882000890189\\
883	4.99907815545505\\
884	4.99939824931494\\
885	4.99974710849707\\
886	5.00009155839975\\
887	5.00040137068436\\
888	5.00065171485551\\
889	5.00082494417711\\
890	5.00091160935318\\
891	5.00091066903209\\
892	5.00082893904159\\
893	5.00067988631519\\
894	5.00048192374421\\
895	5.00025639511773\\
896	5.00002545303552\\
897	4.99981002718187\\
898	4.99962805741469\\
899	4.99949312908648\\
900	4.99941360133428\\
901	4.99939226781352\\
902	4.99942653859808\\
903	4.99950908632823\\
904	4.99962886286723\\
905	4.99977236727163\\
906	4.99992503306899\\
907	5.00007260270934\\
908	5.00020236854977\\
909	5.00030418088762\\
910	5.00037115177211\\
911	5.00040001561302\\
912	5.00039114086938\\
913	5.0003482183817\\
914	5.00027767862112\\
915	5.00018791024225\\
916	5.0000883645293\\
917	4.99998863408832\\
918	4.99989758972843\\
919	4.9998226478839\\
920	4.99976922373521\\
921	4.99974040437185\\
922	4.99973685406201\\
923	4.99975694208137\\
924	4.99979706449283\\
925	4.99985211626439\\
926	4.99991606016935\\
927	4.99998253450573\\
928	5.0000454427402\\
929	5.00009947419408\\
930	5.00014051492529\\
931	5.00016592082457\\
932	5.0001746392915\\
933	5.00016718031221\\
934	5.00014545104709\\
935	5.00011247906823\\
936	5.00007205735463\\
937	5.00002834857837\\
938	4.99998548696679\\
939	4.99994721331069\\
940	4.99991657301178\\
941	4.99989569915502\\
942	4.99988569333318\\
943	4.99988660727383\\
944	4.99989751912117\\
945	4.99991669028943\\
946	4.999941782739\\
947	4.99997011272286\\
948	4.99999891566795\\
949	5.00002559782728\\
950	5.00004795339221\\
951	5.00006433045984\\
952	5.00007373506349\\
953	5.00007586879644\\
954	5.000071101793\\
955	5.00006038843145\\
956	5.00004513764295\\
957	5.00002705282999\\
958	5.00000795794678\\
959	4.9999896262531\\
960	4.99997362674758\\
961	4.99996120056166\\
962	4.99995317598771\\
963	4.9999499267135\\
964	4.99995137364369\\
965	4.99995702678736\\
966	4.99996606040389\\
967	4.99997741217147\\
968	4.99998989572248\\
969	5.00000231552584\\
970	5.00001357373736\\
971	5.0000227601522\\
972	5.00002921857538\\
973	5.00003258553356\\
974	5.00003280001949\\
975	5.00003008563198\\
976	5.00002490882018\\
977	5.00001791877729\\
978	5.00000987573213\\
979	5.000001574899\\
980	4.99999377317326\\
981	4.99998712487163\\
982	4.99998213152768\\
983	4.99997910911935\\
984	4.99997817429637\\
985	4.99997924936777\\
986	4.99998208416166\\
987	4.99998629151662\\
988	4.99999139220387\\
989	4.99999686456511\\
990	5.00000219409612\\
991	5.00000691857819\\
992	5.0000106650932\\
993	5.00001317626109\\
994	5.00001432419967\\
995	5.00001411191224\\
996	5.00001266294709\\
997	5.00001020114968\\
998	5.00000702306736\\
999	5.00000346602237\\
1000	4.9999998750216\\
1001	4.99999657153075\\
1002	4.99999382674231\\
1003	4.99999184136531\\
1004	4.99999073322934\\
1005	4.999990533202\\
1006	4.99999118914235\\
1007	4.99999257692045\\
1008	4.99999451698064\\
1009	4.99999679455098\\
1010	4.99999918142378\\
1011	5.00000145725247\\
1012	5.00000342851108\\
1013	5.00000494361301\\
1014	5.00000590314226\\
1015	5.00000626466503\\
1016	5.00000604211149\\
1017	5.00000530020053\\
1018	5.00000414478407\\
1019	5.0000027102822\\
1020	5.00000114554882\\
1021	4.99999959954363\\
1022	4.99999820809742\\
1023	4.99999708286201\\
1024	4.99999630325754\\
1025	4.99999591190166\\
1026	4.99999591365829\\
1027	4.99999627811191\\
1028	4.99999694498383\\
1029	4.99999783178319\\
1030	4.99999884284117\\
1031	4.99999987881993\\
1032	5.0000008458153\\
1033	5.00000166327638\\
1034	5.00000227012988\\
1035	5.00000262870364\\
1036	5.00000272627031\\
1037	5.00000257425741\\
1038	5.00000220537336\\
1039	5.00000166906524\\
1040	5.00000102583989\\
1041	5.00000034104007\\
1042	4.99999967867038\\
1043	4.9999990958171\\
1044	4.99999863811201\\
1045	4.99999833656229\\
1046	4.9999982059231\\
1047	4.99999824463842\\
1048	4.99999843623479\\
1049	4.99999875193222\\
1050	4.99999915414694\\
1051	4.99999960050618\\
1052	5.00000004797911\\
1053	5.00000045674825\\
1054	5.00000079349758\\
1055	5.0000010338703\\
1056	5.00000116394181\\
1057	5.00000118065315\\
1058	5.00000109124625\\
1059	5.00000091182796\\
1060	5.0000006652576\\
1061	5.00000037859746\\
1062	5.00000008038639\\
1063	4.9999997979921\\
1064	4.99999955527134\\
1065	4.99999937072182\\
1066	4.99999925625231\\
1067	4.99999921663239\\
1068	4.99999924961853\\
1069	4.99999934669306\\
1070	4.99999949430347\\
1071	4.99999967545329\\
1072	4.99999987147629\\
1073	5.00000006382248\\
1074	5.00000023569618\\
1075	5.00000037341231\\
1076	5.00000046737211\\
1077	5.00000051260082\\
1078	5.00000050883354\\
1079	5.00000046017617\\
1080	5.00000037440442\\
1081	5.00000026199092\\
1082	5.00000013496775\\
1083	5.00000000573804\\
1084	4.99999988594613\\
1085	4.99999978550206\\
1086	4.99999971183513\\
1087	4.9999996694252\\
1088	4.99999965963195\\
1089	4.99999968081429\\
1090	4.99999972870711\\
1091	4.99999979700193\\
1092	4.99999987806434\\
1093	4.99999996371392\\
1094	5.00000004599263\\
1095	5.00000011785428\\
1096	5.00000017372002\\
1097	5.00000020986072\\
1098	5.00000022458569\\
1099	5.00000021823594\\
1100	5.00000019299746\\
1101	5.00000015256515\\
1102	5.00000010169859\\
1103	5.00000004571745\\
1104	4.99999998998606\\
1105	4.99999993943368\\
1106	4.99999989815028\\
1107	4.999999869088\\
1108	4.99999985388662\\
1109	4.99999985282897\\
1110	4.99999986492026\\
1111	4.99999988807477\\
1112	4.99999991938499\\
1113	4.99999995544307\\
1114	4.9999999926819\\
1115	5.0000000277041\\
1116	5.00000005757063\\
1117	5.00000008002643\\
1118	5.00000009364807\\
1119	5.00000009790605\\
1120	5.00000009314306\\
1121	5.00000008047639\\
1122	5.0000000616391\\
1123	5.00000003877876\\
1124	5.00000001423484\\
1125	4.99999999031622\\
1126	4.99999996909855\\
1127	4.99999995225786\\
1128	4.99999994095256\\
1129	4.99999993576038\\
1130	4.99999993667179\\
1131	4.99999994313606\\
1132	4.99999995415177\\
1133	4.9999999683905\\
1134	4.99999998433988\\
1135	5.00000000045205\\
1136	5.00000001528388\\
1137	5.00000002761696\\
1138	5.00000003654857\\
1139	5.00000004154748\\
1140	5.00000004247255\\
1141	5.00000003955521\\
1142	5.00000003335026\\
1143	5.00000002466171\\
1144	5.0000000144522\\
1145	5.00000000374533\\
1146	4.99999999353005\\
1147	4.99999998467555\\
1148	4.99999997786329\\
1149	4.99999997354096\\
1150	4.99999997190066\\
1151	4.99999997288161\\
1152	4.99999997619493\\
1153	4.9999999813669\\
1154	4.99999998779522\\
1155	4.99999999481241\\
1156	5.00000000175008\\
1157	5.00000000799833\\
1158	5.00000001305545\\
1159	5.0000000165641\\
1160	5.00000001833192\\
1161	5.00000001833594\\
1162	5.00000001671158\\
1163	5.00000001372841\\
1164	5.0000000097559\\
1165	5.00000000522292\\
1166	5.00000000057512\\
1167	4.99999999623401\\
1168	4.99999999256144\\
1169	4.99999998983204\\
1170	4.99999998821556\\
1171	4.99999998776986\\
1172	4.99999998844441\\
1173	4.99999999009307\\
1174	4.99999999249448\\
1175	4.99999999537747\\
1176	4.99999999844898\\
1177	5.00000000142179\\
1178	5.00000000403952\\
1179	5.00000000609708\\
1180	5.0000000074549\\
1181	5.00000000804626\\
1182	5.00000000787767\\
1183	5.00000000702259\\
1184	5.00000000560978\\
1185	5.00000000380762\\
1186	5.00000000180609\\
1187	4.99999999979827\\
1188	4.99999999796291\\
1189	4.9999999964497\\
1190	4.9999999953682\\
1191	4.99999999478121\\
1192	4.99999999470279\\
1193	4.99999999510076\\
1194	4.9999999959031\\
1195	4.99999999700745\\
1196	4.9999999982925\\
1197	4.99999999963026\\
1198	5.00000000089786\\
1199	5.00000000198818\\
1200	5.00000000281804\\
};
\addlegendentry{$y(k)$}

\addplot[const plot, color=mycolor2, dashed] table[row sep=crcr] {%
1	0\\
2	0\\
3	0\\
4	0\\
5	0\\
6	0\\
7	0\\
8	0\\
9	0\\
10	0\\
11	0\\
12	0\\
13	0\\
14	0\\
15	0\\
16	0\\
17	0\\
18	0\\
19	0\\
20	0\\
21	0\\
22	0\\
23	0\\
24	0\\
25	0\\
26	0\\
27	0\\
28	0\\
29	0\\
30	0\\
31	0\\
32	0\\
33	0\\
34	0\\
35	0\\
36	0\\
37	0\\
38	0\\
39	0\\
40	0\\
41	0\\
42	0\\
43	0\\
44	0\\
45	0\\
46	0\\
47	0\\
48	0\\
49	0\\
50	0\\
51	0\\
52	0\\
53	0\\
54	0\\
55	0\\
56	0\\
57	0\\
58	0\\
59	0\\
60	0\\
61	0\\
62	0\\
63	0\\
64	0\\
65	0\\
66	0\\
67	0\\
68	0\\
69	0\\
70	0\\
71	0\\
72	0\\
73	0\\
74	0\\
75	0\\
76	0\\
77	0\\
78	0\\
79	0\\
80	0\\
81	0\\
82	0\\
83	0\\
84	0\\
85	0\\
86	0\\
87	0\\
88	0\\
89	0\\
90	0\\
91	0\\
92	0\\
93	0\\
94	0\\
95	0\\
96	0\\
97	0\\
98	0\\
99	0\\
100	0.5\\
101	0.5\\
102	0.5\\
103	0.5\\
104	0.5\\
105	0.5\\
106	0.5\\
107	0.5\\
108	0.5\\
109	0.5\\
110	0.5\\
111	0.5\\
112	0.5\\
113	0.5\\
114	0.5\\
115	0.5\\
116	0.5\\
117	0.5\\
118	0.5\\
119	0.5\\
120	0.5\\
121	0.5\\
122	0.5\\
123	0.5\\
124	0.5\\
125	0.5\\
126	0.5\\
127	0.5\\
128	0.5\\
129	0.5\\
130	0.5\\
131	0.5\\
132	0.5\\
133	0.5\\
134	0.5\\
135	0.5\\
136	0.5\\
137	0.5\\
138	0.5\\
139	0.5\\
140	0.5\\
141	0.5\\
142	0.5\\
143	0.5\\
144	0.5\\
145	0.5\\
146	0.5\\
147	0.5\\
148	0.5\\
149	0.5\\
150	0.5\\
151	0.5\\
152	0.5\\
153	0.5\\
154	0.5\\
155	0.5\\
156	0.5\\
157	0.5\\
158	0.5\\
159	0.5\\
160	0.5\\
161	0.5\\
162	0.5\\
163	0.5\\
164	0.5\\
165	0.5\\
166	0.5\\
167	0.5\\
168	0.5\\
169	0.5\\
170	0.5\\
171	0.5\\
172	0.5\\
173	0.5\\
174	0.5\\
175	0.5\\
176	0.5\\
177	0.5\\
178	0.5\\
179	0.5\\
180	0.5\\
181	0.5\\
182	0.5\\
183	0.5\\
184	0.5\\
185	0.5\\
186	0.5\\
187	0.5\\
188	0.5\\
189	0.5\\
190	0.5\\
191	0.5\\
192	0.5\\
193	0.5\\
194	0.5\\
195	0.5\\
196	0.5\\
197	0.5\\
198	0.5\\
199	0.5\\
200	0.5\\
201	0.5\\
202	0.5\\
203	0.5\\
204	0.5\\
205	0.5\\
206	0.5\\
207	0.5\\
208	0.5\\
209	0.5\\
210	0.5\\
211	0.5\\
212	0.5\\
213	0.5\\
214	0.5\\
215	0.5\\
216	0.5\\
217	0.5\\
218	0.5\\
219	0.5\\
220	0.5\\
221	0.5\\
222	0.5\\
223	0.5\\
224	0.5\\
225	0.5\\
226	0.5\\
227	0.5\\
228	0.5\\
229	0.5\\
230	0.5\\
231	0.5\\
232	0.5\\
233	0.5\\
234	0.5\\
235	0.5\\
236	0.5\\
237	0.5\\
238	0.5\\
239	0.5\\
240	0.5\\
241	0.5\\
242	0.5\\
243	0.5\\
244	0.5\\
245	0.5\\
246	0.5\\
247	0.5\\
248	0.5\\
249	0.5\\
250	0.5\\
251	0.5\\
252	0.5\\
253	0.5\\
254	0.5\\
255	0.5\\
256	0.5\\
257	0.5\\
258	0.5\\
259	0.5\\
260	0.5\\
261	0.5\\
262	0.5\\
263	0.5\\
264	0.5\\
265	0.5\\
266	0.5\\
267	0.5\\
268	0.5\\
269	0.5\\
270	0.5\\
271	0.5\\
272	0.5\\
273	0.5\\
274	0.5\\
275	0.5\\
276	0.5\\
277	0.5\\
278	0.5\\
279	0.5\\
280	0.5\\
281	0.5\\
282	0.5\\
283	0.5\\
284	0.5\\
285	0.5\\
286	0.5\\
287	0.5\\
288	0.5\\
289	0.5\\
290	0.5\\
291	0.5\\
292	0.5\\
293	0.5\\
294	0.5\\
295	0.5\\
296	0.5\\
297	0.5\\
298	0.5\\
299	0.5\\
300	1\\
301	1\\
302	1\\
303	1\\
304	1\\
305	1\\
306	1\\
307	1\\
308	1\\
309	1\\
310	1\\
311	1\\
312	1\\
313	1\\
314	1\\
315	1\\
316	1\\
317	1\\
318	1\\
319	1\\
320	1\\
321	1\\
322	1\\
323	1\\
324	1\\
325	1\\
326	1\\
327	1\\
328	1\\
329	1\\
330	1\\
331	1\\
332	1\\
333	1\\
334	1\\
335	1\\
336	1\\
337	1\\
338	1\\
339	1\\
340	1\\
341	1\\
342	1\\
343	1\\
344	1\\
345	1\\
346	1\\
347	1\\
348	1\\
349	1\\
350	1\\
351	1\\
352	1\\
353	1\\
354	1\\
355	1\\
356	1\\
357	1\\
358	1\\
359	1\\
360	1\\
361	1\\
362	1\\
363	1\\
364	1\\
365	1\\
366	1\\
367	1\\
368	1\\
369	1\\
370	1\\
371	1\\
372	1\\
373	1\\
374	1\\
375	1\\
376	1\\
377	1\\
378	1\\
379	1\\
380	1\\
381	1\\
382	1\\
383	1\\
384	1\\
385	1\\
386	1\\
387	1\\
388	1\\
389	1\\
390	1\\
391	1\\
392	1\\
393	1\\
394	1\\
395	1\\
396	1\\
397	1\\
398	1\\
399	1\\
400	1\\
401	1\\
402	1\\
403	1\\
404	1\\
405	1\\
406	1\\
407	1\\
408	1\\
409	1\\
410	1\\
411	1\\
412	1\\
413	1\\
414	1\\
415	1\\
416	1\\
417	1\\
418	1\\
419	1\\
420	1\\
421	1\\
422	1\\
423	1\\
424	1\\
425	1\\
426	1\\
427	1\\
428	1\\
429	1\\
430	1\\
431	1\\
432	1\\
433	1\\
434	1\\
435	1\\
436	1\\
437	1\\
438	1\\
439	1\\
440	1\\
441	1\\
442	1\\
443	1\\
444	1\\
445	1\\
446	1\\
447	1\\
448	1\\
449	1\\
450	1\\
451	1\\
452	1\\
453	1\\
454	1\\
455	1\\
456	1\\
457	1\\
458	1\\
459	1\\
460	1\\
461	1\\
462	1\\
463	1\\
464	1\\
465	1\\
466	1\\
467	1\\
468	1\\
469	1\\
470	1\\
471	1\\
472	1\\
473	1\\
474	1\\
475	1\\
476	1\\
477	1\\
478	1\\
479	1\\
480	1\\
481	1\\
482	1\\
483	1\\
484	1\\
485	1\\
486	1\\
487	1\\
488	1\\
489	1\\
490	1\\
491	1\\
492	1\\
493	1\\
494	1\\
495	1\\
496	1\\
497	1\\
498	1\\
499	1\\
500	-0.1\\
501	-0.1\\
502	-0.1\\
503	-0.1\\
504	-0.1\\
505	-0.1\\
506	-0.1\\
507	-0.1\\
508	-0.1\\
509	-0.1\\
510	-0.1\\
511	-0.1\\
512	-0.1\\
513	-0.1\\
514	-0.1\\
515	-0.1\\
516	-0.1\\
517	-0.1\\
518	-0.1\\
519	-0.1\\
520	-0.1\\
521	-0.1\\
522	-0.1\\
523	-0.1\\
524	-0.1\\
525	-0.1\\
526	-0.1\\
527	-0.1\\
528	-0.1\\
529	-0.1\\
530	-0.1\\
531	-0.1\\
532	-0.1\\
533	-0.1\\
534	-0.1\\
535	-0.1\\
536	-0.1\\
537	-0.1\\
538	-0.1\\
539	-0.1\\
540	-0.1\\
541	-0.1\\
542	-0.1\\
543	-0.1\\
544	-0.1\\
545	-0.1\\
546	-0.1\\
547	-0.1\\
548	-0.1\\
549	-0.1\\
550	-0.1\\
551	-0.1\\
552	-0.1\\
553	-0.1\\
554	-0.1\\
555	-0.1\\
556	-0.1\\
557	-0.1\\
558	-0.1\\
559	-0.1\\
560	-0.1\\
561	-0.1\\
562	-0.1\\
563	-0.1\\
564	-0.1\\
565	-0.1\\
566	-0.1\\
567	-0.1\\
568	-0.1\\
569	-0.1\\
570	-0.1\\
571	-0.1\\
572	-0.1\\
573	-0.1\\
574	-0.1\\
575	-0.1\\
576	-0.1\\
577	-0.1\\
578	-0.1\\
579	-0.1\\
580	-0.1\\
581	-0.1\\
582	-0.1\\
583	-0.1\\
584	-0.1\\
585	-0.1\\
586	-0.1\\
587	-0.1\\
588	-0.1\\
589	-0.1\\
590	-0.1\\
591	-0.1\\
592	-0.1\\
593	-0.1\\
594	-0.1\\
595	-0.1\\
596	-0.1\\
597	-0.1\\
598	-0.1\\
599	-0.1\\
600	-0.1\\
601	-0.1\\
602	-0.1\\
603	-0.1\\
604	-0.1\\
605	-0.1\\
606	-0.1\\
607	-0.1\\
608	-0.1\\
609	-0.1\\
610	-0.1\\
611	-0.1\\
612	-0.1\\
613	-0.1\\
614	-0.1\\
615	-0.1\\
616	-0.1\\
617	-0.1\\
618	-0.1\\
619	-0.1\\
620	-0.1\\
621	-0.1\\
622	-0.1\\
623	-0.1\\
624	-0.1\\
625	-0.1\\
626	-0.1\\
627	-0.1\\
628	-0.1\\
629	-0.1\\
630	-0.1\\
631	-0.1\\
632	-0.1\\
633	-0.1\\
634	-0.1\\
635	-0.1\\
636	-0.1\\
637	-0.1\\
638	-0.1\\
639	-0.1\\
640	-0.1\\
641	-0.1\\
642	-0.1\\
643	-0.1\\
644	-0.1\\
645	-0.1\\
646	-0.1\\
647	-0.1\\
648	-0.1\\
649	-0.1\\
650	-0.1\\
651	-0.1\\
652	-0.1\\
653	-0.1\\
654	-0.1\\
655	-0.1\\
656	-0.1\\
657	-0.1\\
658	-0.1\\
659	-0.1\\
660	-0.1\\
661	-0.1\\
662	-0.1\\
663	-0.1\\
664	-0.1\\
665	-0.1\\
666	-0.1\\
667	-0.1\\
668	-0.1\\
669	-0.1\\
670	-0.1\\
671	-0.1\\
672	-0.1\\
673	-0.1\\
674	-0.1\\
675	-0.1\\
676	-0.1\\
677	-0.1\\
678	-0.1\\
679	-0.1\\
680	-0.1\\
681	-0.1\\
682	-0.1\\
683	-0.1\\
684	-0.1\\
685	-0.1\\
686	-0.1\\
687	-0.1\\
688	-0.1\\
689	-0.1\\
690	-0.1\\
691	-0.1\\
692	-0.1\\
693	-0.1\\
694	-0.1\\
695	-0.1\\
696	-0.1\\
697	-0.1\\
698	-0.1\\
699	-0.1\\
700	5\\
701	5\\
702	5\\
703	5\\
704	5\\
705	5\\
706	5\\
707	5\\
708	5\\
709	5\\
710	5\\
711	5\\
712	5\\
713	5\\
714	5\\
715	5\\
716	5\\
717	5\\
718	5\\
719	5\\
720	5\\
721	5\\
722	5\\
723	5\\
724	5\\
725	5\\
726	5\\
727	5\\
728	5\\
729	5\\
730	5\\
731	5\\
732	5\\
733	5\\
734	5\\
735	5\\
736	5\\
737	5\\
738	5\\
739	5\\
740	5\\
741	5\\
742	5\\
743	5\\
744	5\\
745	5\\
746	5\\
747	5\\
748	5\\
749	5\\
750	5\\
751	5\\
752	5\\
753	5\\
754	5\\
755	5\\
756	5\\
757	5\\
758	5\\
759	5\\
760	5\\
761	5\\
762	5\\
763	5\\
764	5\\
765	5\\
766	5\\
767	5\\
768	5\\
769	5\\
770	5\\
771	5\\
772	5\\
773	5\\
774	5\\
775	5\\
776	5\\
777	5\\
778	5\\
779	5\\
780	5\\
781	5\\
782	5\\
783	5\\
784	5\\
785	5\\
786	5\\
787	5\\
788	5\\
789	5\\
790	5\\
791	5\\
792	5\\
793	5\\
794	5\\
795	5\\
796	5\\
797	5\\
798	5\\
799	5\\
800	5\\
801	5\\
802	5\\
803	5\\
804	5\\
805	5\\
806	5\\
807	5\\
808	5\\
809	5\\
810	5\\
811	5\\
812	5\\
813	5\\
814	5\\
815	5\\
816	5\\
817	5\\
818	5\\
819	5\\
820	5\\
821	5\\
822	5\\
823	5\\
824	5\\
825	5\\
826	5\\
827	5\\
828	5\\
829	5\\
830	5\\
831	5\\
832	5\\
833	5\\
834	5\\
835	5\\
836	5\\
837	5\\
838	5\\
839	5\\
840	5\\
841	5\\
842	5\\
843	5\\
844	5\\
845	5\\
846	5\\
847	5\\
848	5\\
849	5\\
850	5\\
851	5\\
852	5\\
853	5\\
854	5\\
855	5\\
856	5\\
857	5\\
858	5\\
859	5\\
860	5\\
861	5\\
862	5\\
863	5\\
864	5\\
865	5\\
866	5\\
867	5\\
868	5\\
869	5\\
870	5\\
871	5\\
872	5\\
873	5\\
874	5\\
875	5\\
876	5\\
877	5\\
878	5\\
879	5\\
880	5\\
881	5\\
882	5\\
883	5\\
884	5\\
885	5\\
886	5\\
887	5\\
888	5\\
889	5\\
890	5\\
891	5\\
892	5\\
893	5\\
894	5\\
895	5\\
896	5\\
897	5\\
898	5\\
899	5\\
900	5\\
901	5\\
902	5\\
903	5\\
904	5\\
905	5\\
906	5\\
907	5\\
908	5\\
909	5\\
910	5\\
911	5\\
912	5\\
913	5\\
914	5\\
915	5\\
916	5\\
917	5\\
918	5\\
919	5\\
920	5\\
921	5\\
922	5\\
923	5\\
924	5\\
925	5\\
926	5\\
927	5\\
928	5\\
929	5\\
930	5\\
931	5\\
932	5\\
933	5\\
934	5\\
935	5\\
936	5\\
937	5\\
938	5\\
939	5\\
940	5\\
941	5\\
942	5\\
943	5\\
944	5\\
945	5\\
946	5\\
947	5\\
948	5\\
949	5\\
950	5\\
951	5\\
952	5\\
953	5\\
954	5\\
955	5\\
956	5\\
957	5\\
958	5\\
959	5\\
960	5\\
961	5\\
962	5\\
963	5\\
964	5\\
965	5\\
966	5\\
967	5\\
968	5\\
969	5\\
970	5\\
971	5\\
972	5\\
973	5\\
974	5\\
975	5\\
976	5\\
977	5\\
978	5\\
979	5\\
980	5\\
981	5\\
982	5\\
983	5\\
984	5\\
985	5\\
986	5\\
987	5\\
988	5\\
989	5\\
990	5\\
991	5\\
992	5\\
993	5\\
994	5\\
995	5\\
996	5\\
997	5\\
998	5\\
999	5\\
1000	5\\
1001	5\\
1002	5\\
1003	5\\
1004	5\\
1005	5\\
1006	5\\
1007	5\\
1008	5\\
1009	5\\
1010	5\\
1011	5\\
1012	5\\
1013	5\\
1014	5\\
1015	5\\
1016	5\\
1017	5\\
1018	5\\
1019	5\\
1020	5\\
1021	5\\
1022	5\\
1023	5\\
1024	5\\
1025	5\\
1026	5\\
1027	5\\
1028	5\\
1029	5\\
1030	5\\
1031	5\\
1032	5\\
1033	5\\
1034	5\\
1035	5\\
1036	5\\
1037	5\\
1038	5\\
1039	5\\
1040	5\\
1041	5\\
1042	5\\
1043	5\\
1044	5\\
1045	5\\
1046	5\\
1047	5\\
1048	5\\
1049	5\\
1050	5\\
1051	5\\
1052	5\\
1053	5\\
1054	5\\
1055	5\\
1056	5\\
1057	5\\
1058	5\\
1059	5\\
1060	5\\
1061	5\\
1062	5\\
1063	5\\
1064	5\\
1065	5\\
1066	5\\
1067	5\\
1068	5\\
1069	5\\
1070	5\\
1071	5\\
1072	5\\
1073	5\\
1074	5\\
1075	5\\
1076	5\\
1077	5\\
1078	5\\
1079	5\\
1080	5\\
1081	5\\
1082	5\\
1083	5\\
1084	5\\
1085	5\\
1086	5\\
1087	5\\
1088	5\\
1089	5\\
1090	5\\
1091	5\\
1092	5\\
1093	5\\
1094	5\\
1095	5\\
1096	5\\
1097	5\\
1098	5\\
1099	5\\
1100	5\\
1101	5\\
1102	5\\
1103	5\\
1104	5\\
1105	5\\
1106	5\\
1107	5\\
1108	5\\
1109	5\\
1110	5\\
1111	5\\
1112	5\\
1113	5\\
1114	5\\
1115	5\\
1116	5\\
1117	5\\
1118	5\\
1119	5\\
1120	5\\
1121	5\\
1122	5\\
1123	5\\
1124	5\\
1125	5\\
1126	5\\
1127	5\\
1128	5\\
1129	5\\
1130	5\\
1131	5\\
1132	5\\
1133	5\\
1134	5\\
1135	5\\
1136	5\\
1137	5\\
1138	5\\
1139	5\\
1140	5\\
1141	5\\
1142	5\\
1143	5\\
1144	5\\
1145	5\\
1146	5\\
1147	5\\
1148	5\\
1149	5\\
1150	5\\
1151	5\\
1152	5\\
1153	5\\
1154	5\\
1155	5\\
1156	5\\
1157	5\\
1158	5\\
1159	5\\
1160	5\\
1161	5\\
1162	5\\
1163	5\\
1164	5\\
1165	5\\
1166	5\\
1167	5\\
1168	5\\
1169	5\\
1170	5\\
1171	5\\
1172	5\\
1173	5\\
1174	5\\
1175	5\\
1176	5\\
1177	5\\
1178	5\\
1179	5\\
1180	5\\
1181	5\\
1182	5\\
1183	5\\
1184	5\\
1185	5\\
1186	5\\
1187	5\\
1188	5\\
1189	5\\
1190	5\\
1191	5\\
1192	5\\
1193	5\\
1194	5\\
1195	5\\
1196	5\\
1197	5\\
1198	5\\
1199	5\\
1200	5\\
};
\addlegendentry{$y^{zad}(k)$}

\end{axis}
\end{tikzpicture}%
	\caption{Przebiegi sygna��w dla rozmytego regulatora DMC z 2 regulatorami lokalnymi}
\end{figure}

\subsection{Liczba regulator�w lokalnych $n_r = 3$}

Parametry regulatora: $\lambda^{1} = 0,2, \lambda^{2} = 0,6, \lambda^{3} = 0,1$

\begin{figure}[h!]
	\centering
	% This file was created by matlab2tikz.
%
\definecolor{mycolor1}{rgb}{0.00000,0.44700,0.74100}%
\definecolor{mycolor2}{rgb}{0.85000,0.32500,0.09800}%
%
\begin{tikzpicture}[scale=0.8]

\begin{axis}[%
width=4.521in,
height=3.566in,
at={(0.758in,0.481in)},
scale only axis,
xmin=0,
xmax=1200,
xlabel style={font=\color{white!15!black}},
xlabel={$k$},
ymin=-1,
ymax=6,
ylabel style={font=\color{white!15!black}},
ylabel={$y(k)$, $y^{zad}(k)$},
axis background/.style={fill=white},
title style={font=\bfseries},
title={Sygna� wyj�ciowy i zadany},
xmajorgrids,
ymajorgrids,
legend style={at={(0.97,0.03)}, anchor=south east, legend cell align=left, align=left, draw=white!15!black}
]
\addplot [color=mycolor1]
  table[row sep=crcr]{%
1	0\\
2	0\\
3	0\\
4	0\\
5	0\\
6	0\\
7	0\\
8	0\\
9	0\\
10	0\\
11	0\\
12	0\\
13	0\\
14	0\\
15	0\\
16	0\\
17	0\\
18	0\\
19	0\\
20	0\\
21	0\\
22	0\\
23	0\\
24	0\\
25	0\\
26	0\\
27	0\\
28	0\\
29	0\\
30	0\\
31	0\\
32	0\\
33	0\\
34	0\\
35	0\\
36	0\\
37	0\\
38	0\\
39	0\\
40	0\\
41	0\\
42	0\\
43	0\\
44	0\\
45	0\\
46	0\\
47	0\\
48	0\\
49	0\\
50	0\\
51	0\\
52	0\\
53	0\\
54	0\\
55	0\\
56	0\\
57	0\\
58	0\\
59	0\\
60	0\\
61	0\\
62	0\\
63	0\\
64	0\\
65	0\\
66	0\\
67	0\\
68	0\\
69	0\\
70	0\\
71	0\\
72	0\\
73	0\\
74	0\\
75	0\\
76	0\\
77	0\\
78	0\\
79	0\\
80	0\\
81	0\\
82	0\\
83	0\\
84	0\\
85	0\\
86	0\\
87	0\\
88	0\\
89	0\\
90	0\\
91	0\\
92	0\\
93	0\\
94	0\\
95	0\\
96	0\\
97	0\\
98	0\\
99	0\\
100	0\\
101	0\\
102	0\\
103	0\\
104	0\\
105	0.0488873959250467\\
106	0.166324649066985\\
107	0.277816582148071\\
108	0.376147918689745\\
109	0.434284183613812\\
110	0.475257758175106\\
111	0.487685820617506\\
112	0.487836177077399\\
113	0.476198097497558\\
114	0.459434511745512\\
115	0.442283394962272\\
116	0.429271834081922\\
117	0.423723492891798\\
118	0.427623289879692\\
119	0.441245232618621\\
120	0.462808865581464\\
121	0.488602097594882\\
122	0.513849121272853\\
123	0.534060015828397\\
124	0.546105309938807\\
125	0.548604411912322\\
126	0.541838321640352\\
127	0.527510630719248\\
128	0.508450302414893\\
129	0.488212422515649\\
130	0.470564745795216\\
131	0.458906622116344\\
132	0.455666126599539\\
133	0.461706967221006\\
134	0.475864514518072\\
135	0.49493760535265\\
136	0.514485275133275\\
137	0.53022575645785\\
138	0.539203473062819\\
139	0.540175698339243\\
140	0.533446687810141\\
141	0.520570669656625\\
142	0.504051154050188\\
143	0.486972406905313\\
144	0.472532536990746\\
145	0.463521281497807\\
146	0.461796534288998\\
147	0.467813661512233\\
148	0.480334873587179\\
149	0.496558578007155\\
150	0.512834632861944\\
151	0.525727904793622\\
152	0.53286779610473\\
153	0.533262469910598\\
154	0.527217889390035\\
155	0.516114800577664\\
156	0.502136789805816\\
157	0.487929436143773\\
158	0.476184529398173\\
159	0.469184635753579\\
160	0.468361957269678\\
161	0.473931723751253\\
162	0.484714621926168\\
163	0.498318812586134\\
164	0.511746365003144\\
165	0.522213774301348\\
166	0.527799527289061\\
167	0.527711482784833\\
168	0.522253044078548\\
169	0.512648012657786\\
170	0.500795625631328\\
171	0.488959707066994\\
172	0.479400186600414\\
173	0.473980342021832\\
174	0.473796587800488\\
175	0.478895347726034\\
176	0.48817744438862\\
177	0.499600341683402\\
178	0.510689175196813\\
179	0.519177617291057\\
180	0.5235103813102\\
181	0.5230668392809\\
182	0.518150567994387\\
183	0.509847049591313\\
184	0.499805191444719\\
185	0.489957734182649\\
186	0.482196591655179\\
187	0.478035211769417\\
188	0.478302689818513\\
189	0.482930430383948\\
190	0.490912177022402\\
191	0.500506698795635\\
192	0.50966519626116\\
193	0.516537548270304\\
194	0.519867155664373\\
195	0.519176344699174\\
196	0.51476644649386\\
197	0.507599184554265\\
198	0.499102832370066\\
199	0.49092307916858\\
200	0.48463882150015\\
201	0.481474037720796\\
202	0.482047758002581\\
203	0.486216376952719\\
204	0.493070676823778\\
205	0.50112814611522\\
206	0.50868935987772\\
207	0.51424148174349\\
208	0.516771583894713\\
209	0.515919169105711\\
210	0.511979807646995\\
211	0.505804099743095\\
212	0.498626260227729\\
213	0.491843781085497\\
214	0.486770237635139\\
215	0.484391187956553\\
216	0.485161763563599\\
217	0.488892507735821\\
218	0.494769975550587\\
219	0.501533312291527\\
220	0.5077710578652\\
221	0.512245162278522\\
222	0.514141428932572\\
223	0.513194096117002\\
224	0.509689395186217\\
225	0.504378010813864\\
226	0.498323878246584\\
227	0.49271054560047\\
228	0.48862751111669\\
229	0.486864834557795\\
230	0.487750709672097\\
231	0.491070791311756\\
232	0.49610292007477\\
233	0.501775844995517\\
234	0.506916378539969\\
235	0.510511176827329\\
236	0.51190796886007\\
237	0.51091661798076\\
238	0.507810892398055\\
239	0.503251674676452\\
240	0.498153887879738\\
241	0.49351726451821\\
242	0.490242750260444\\
243	0.488960786975774\\
244	0.489902025032994\\
245	0.492842044080928\\
246	0.497143682542152\\
247	0.501897678895087\\
248	0.506128569575893\\
249	0.509007226800203\\
250	0.510013118405638\\
251	0.509015722018459\\
252	0.506273892203304\\
253	0.502367932888665\\
254	0.498082740466291\\
255	0.494260637602141\\
256	0.491644474182901\\
257	0.490734903507368\\
258	0.491688267185191\\
259	0.494280307848602\\
260	0.497951707293478\\
261	0.501931462158208\\
262	0.505408525046588\\
263	0.507705027086141\\
264	0.508407403712513\\
265	0.507431506852636\\
266	0.50501955050217\\
267	0.501679716080087\\
268	0.498083774265946\\
269	0.494939685369008\\
270	0.492858191545893\\
271	0.492234830890561\\
272	0.493169897025889\\
273	0.495446158625821\\
274	0.498574721776102\\
275	0.501902480200848\\
276	0.504755299559636\\
277	0.50657961070225\\
278	0.5070484939333\\
279	0.506113348959672\\
280	0.503998703949882\\
281	0.501148380630454\\
282	0.49813601838831\\
283	0.495555244280108\\
284	0.493906765245172\\
285	0.49350131864485\\
286	0.494397421730101\\
287	0.49638926439387\\
288	0.499051071719561\\
289	0.501830184737281\\
290	0.504166580306344\\
291	0.505608857570528\\
292	0.50590008252101\\
293	0.5050184401987\\
294	0.503170322643572\\
295	0.5007423084489\\
296	0.498223159455826\\
297	0.496109500552456\\
298	0.494810668621304\\
299	0.494569255500597\\
300	0.495413092485762\\
301	0.497150399657094\\
302	0.499411557988188\\
303	0.501729413835917\\
304	0.503639093173884\\
305	0.592911290798146\\
306	0.748552545830577\\
307	0.88674303033884\\
308	1.0031606162115\\
309	1.06954110666287\\
310	1.1184774011384\\
311	1.11728554417478\\
312	1.09787233691252\\
313	1.05824721618095\\
314	1.00548316981145\\
315	0.948219793748211\\
316	0.898805847223637\\
317	0.869379282743708\\
318	0.869585625108362\\
319	0.902462214677196\\
320	0.958524472519233\\
321	1.01779061815254\\
322	1.06439578521354\\
323	1.09279794734688\\
324	1.10213264871551\\
325	1.09282744544654\\
326	1.06706127443802\\
327	1.02949667039308\\
328	0.986962378444963\\
329	0.947610253530281\\
330	0.919843284504514\\
331	0.910728100168838\\
332	0.923461750544557\\
333	0.95446044751004\\
334	0.99339514895204\\
335	1.02884188691898\\
336	1.05385136692844\\
337	1.06583597764275\\
338	1.06442267903128\\
339	1.05076387147004\\
340	1.02773619537971\\
341	0.999839910209961\\
342	0.972613303264762\\
343	0.95176559364585\\
344	0.942060179531095\\
345	0.945918816245947\\
346	0.962126015163066\\
347	0.985754506195349\\
348	1.01017805059268\\
349	1.02984474647521\\
350	1.0415301241656\\
351	1.04403507694453\\
352	1.03768917185706\\
353	1.02416502731018\\
354	1.00631725362207\\
355	0.987804234159636\\
356	0.97249038509181\\
357	0.963687314380106\\
358	0.963314878049953\\
359	0.971205264512808\\
360	0.985001795433842\\
361	1.00095690997054\\
362	1.01527386917568\\
363	1.02514929421035\\
364	1.029104022369\\
365	1.02689883767605\\
366	1.01937310043093\\
367	1.00826821471759\\
368	0.995953169348888\\
369	0.985019336838991\\
370	0.977775270677068\\
371	0.975717558866412\\
372	0.979113468378257\\
373	0.986888695703025\\
374	0.996956173398576\\
375	1.00688339577551\\
376	1.01456685021335\\
377	1.01863876430952\\
378	1.01858825381934\\
379	1.01472028058598\\
380	1.00803312845053\\
381	1.00002210383426\\
382	0.992403711457111\\
383	0.986779805137328\\
384	0.984292585482249\\
385	0.985351267962916\\
386	0.989527891486906\\
387	0.99568874971695\\
388	1.00233114659901\\
389	1.00798990337748\\
390	1.01156176776255\\
391	1.01247820916136\\
392	1.01074345330888\\
393	1.00687835714638\\
394	1.00179414192687\\
395	0.996608113831818\\
396	0.992419672406459\\
397	0.990080385371522\\
398	0.990006895732278\\
399	0.992090599039817\\
400	0.995741027998281\\
401	1.00005587057356\\
402	1.00405931202123\\
403	1.00692840311316\\
404	1.00814919527561\\
405	1.00758541360613\\
406	1.0054694767609\\
407	1.00233180824224\\
408	0.998883018875328\\
409	0.995865793471039\\
410	0.993900140687321\\
411	0.993352649628856\\
412	0.994261861194964\\
413	0.99634236647983\\
414	0.999068366391642\\
415	1.00181088792069\\
416	1.00398636289049\\
417	1.00517722381125\\
418	1.00520230395777\\
419	1.00413266164631\\
420	1.0022588618852\\
421	1.00002007406197\\
422	0.997908241881782\\
423	0.996364207521355\\
424	0.995685919182784\\
425	0.995969025950237\\
426	0.997094696483015\\
427	0.998767857441174\\
428	1.00059452003565\\
429	1.00217584238984\\
430	1.00319427942778\\
431	1.00347341129282\\
432	1.00300287850827\\
433	1.0019283440525\\
434	1.00051193708572\\
435	0.999072196258419\\
436	0.99791525038604\\
437	0.997270791948361\\
438	0.997246269780609\\
439	0.997809507703281\\
440	0.998803436718196\\
441	0.999988309628234\\
442	1.00109961924403\\
443	1.00190679722438\\
444	1.00225943509654\\
445	1.00211265344784\\
446	1.00152886376511\\
447	1.00065787430648\\
448	0.999700726347811\\
449	0.998865030147273\\
450	0.998320928533178\\
451	0.998166783959141\\
452	0.998411805765072\\
453	0.998979060443007\\
454	0.99972737998688\\
455	1.00048606962395\\
456	1.00109357072927\\
457	1.00143114121166\\
458	1.00144480125194\\
459	1.00115213279561\\
460	1.00063391289999\\
461	1.00001338337032\\
462	0.99942800763615\\
463	0.99899974871396\\
464	0.998810059170632\\
465	0.998884734765892\\
466	0.999191572168798\\
467	0.999650788515834\\
468	1.00015517669549\\
469	1.00059485943902\\
470	1.00088084747761\\
471	1.00096242839931\\
472	1.00083525401454\\
473	1.00053922082134\\
474	1.00014733183875\\
475	0.999748378215433\\
476	0.99942731398996\\
477	0.999247497045894\\
478	0.999238458142084\\
479	0.999391577644754\\
480	0.999664228933004\\
481	0.999991015867875\\
482	1.00029919626775\\
483	1.00052463984807\\
484	1.00062484699342\\
485	1.00058649589165\\
486	1.00042635728407\\
487	1.00018586479205\\
488	0.999920877034654\\
489	0.999689030356115\\
490	0.999537442871759\\
491	0.999493337931713\\
492	0.999559436238424\\
493	0.999714859419333\\
494	0.999921041740195\\
495	1.00013106379968\\
496	1.00030016561747\\
497	1.00039510024379\\
498	1.00040042209667\\
499	1.00032060557358\\
500	1.00017784157348\\
501	1.00000625599101\\
502	0.999843974719268\\
503	0.99972481791019\\
504	0.999671390817358\\
505	0.651197994432466\\
506	0.340630705701158\\
507	0.133415402141414\\
508	0.0062380747918627\\
509	-0.070346336942648\\
510	-0.116252375887872\\
511	-0.143738092411026\\
512	-0.160190248368038\\
513	-0.170037353158282\\
514	-0.175841609073223\\
515	-0.177560592079202\\
516	-0.174699070649411\\
517	-0.168508730522578\\
518	-0.160930832897498\\
519	-0.153979996623975\\
520	-0.148974194951265\\
521	-0.14590037778831\\
522	-0.143981326641801\\
523	-0.14246260909147\\
524	-0.14087056499378\\
525	-0.139016789103525\\
526	-0.136931259492657\\
527	-0.134772398457662\\
528	-0.132729798034875\\
529	-0.130942031773228\\
530	-0.129453236299902\\
531	-0.128219607223183\\
532	-0.12715099947624\\
533	-0.126158058737767\\
534	-0.125182970367696\\
535	-0.124207969510829\\
536	-0.123246457140525\\
537	-0.12232540417998\\
538	-0.121467918452296\\
539	-0.120682695883253\\
540	-0.119962827001814\\
541	-0.119291808014293\\
542	-0.11865203727734\\
543	-0.118031412297061\\
544	-0.117425804757869\\
545	-0.116837575290266\\
546	-0.116271863257957\\
547	-0.115732822663962\\
548	-0.115221416443542\\
549	-0.114735284392782\\
550	-0.114270159257991\\
551	-0.113821773162061\\
552	-0.113387277911497\\
553	-0.112965688580774\\
554	-0.112557421577855\\
555	-0.112163373652618\\
556	-0.111784068591931\\
557	-0.111419227732613\\
558	-0.111067836989349\\
559	-0.110728543355391\\
560	-0.110400112241039\\
561	-0.110081726482435\\
562	-0.109773031851656\\
563	-0.109473976033171\\
564	-0.109184567949658\\
565	-0.108904685435637\\
566	-0.108634003617841\\
567	-0.108372043006138\\
568	-0.108118282845946\\
569	-0.107872270535023\\
570	-0.107633678261322\\
571	-0.107402294628922\\
572	-0.107177971907385\\
573	-0.106960564112574\\
574	-0.106749885974865\\
575	-0.106545705481142\\
576	-0.106347764106607\\
577	-0.106155807890538\\
578	-0.105969612107013\\
579	-0.105788989779194\\
580	-0.105613784216377\\
581	-0.105443852980454\\
582	-0.105279052644907\\
583	-0.105119230894074\\
584	-0.104964227459054\\
585	-0.104813880984318\\
586	-0.104668036997843\\
587	-0.104526552938601\\
588	-0.104389298611695\\
589	-0.104256152962816\\
590	-0.104126999514167\\
591	-0.10400172280478\\
592	-0.103880207110047\\
593	-0.103762337346179\\
594	-0.103648001098689\\
595	-0.103537090485018\\
596	-0.103429502980753\\
597	-0.103325141044986\\
598	-0.103223910975524\\
599	-0.103125721666547\\
600	-0.103030483812158\\
601	-0.10293810975562\\
602	-0.102848513839154\\
603	-0.102761612920924\\
604	-0.102677326739184\\
605	-0.102595577959448\\
606	-0.102516291930368\\
607	-0.102439396302147\\
608	-0.102364820685329\\
609	-0.102292496464102\\
610	-0.102222356779004\\
611	-0.102154336613766\\
612	-0.102088372891711\\
613	-0.102024404508666\\
614	-0.101962372278559\\
615	-0.10190221881504\\
616	-0.101843888396486\\
617	-0.101787326857786\\
618	-0.101732481529424\\
619	-0.101679301218133\\
620	-0.10162773620656\\
621	-0.101577738247266\\
622	-0.101529260536102\\
623	-0.101482257663963\\
624	-0.101436685556652\\
625	-0.101392501416027\\
626	-0.101349663672098\\
627	-0.101308131948678\\
628	-0.101267867038799\\
629	-0.101228830883095\\
630	-0.101190986545209\\
631	-0.101154298181651\\
632	-0.101118731007139\\
633	-0.101084251258681\\
634	-0.101050826161735\\
635	-0.101018423900352\\
636	-0.100987013591225\\
637	-0.100956565260177\\
638	-0.100927049819189\\
639	-0.100898439042684\\
640	-0.100870705542762\\
641	-0.100843822743951\\
642	-0.100817764858415\\
643	-0.100792506862407\\
644	-0.100768024474241\\
645	-0.100744294133592\\
646	-0.100721292981619\\
647	-0.100698998841447\\
648	-0.100677390198737\\
649	-0.100656446182365\\
650	-0.100636146545413\\
651	-0.100616471646714\\
652	-0.100597402433104\\
653	-0.100578920422403\\
654	-0.10056100768701\\
655	-0.100543646837973\\
656	-0.100526821009413\\
657	-0.100510513843253\\
658	-0.100494709474272\\
659	-0.10047939251555\\
660	-0.100464548044346\\
661	-0.10045016158843\\
662	-0.100436219112864\\
663	-0.100422707007173\\
664	-0.100409612072871\\
665	-0.100396921511316\\
666	-0.100384622911864\\
667	-0.100372704240345\\
668	-0.100361153827867\\
669	-0.100349960359949\\
670	-0.100339112865977\\
671	-0.100328600708985\\
672	-0.100318413575718\\
673	-0.100308541466991\\
674	-0.100298974688301\\
675	-0.10028970384071\\
676	-0.100280719811987\\
677	-0.100272013768009\\
678	-0.100263577144408\\
679	-0.100255401638478\\
680	-0.100247479201305\\
681	-0.100239802030138\\
682	-0.100232362560976\\
683	-0.100225153461369\\
684	-0.100218167623431\\
685	-0.100211398157051\\
686	-0.100204838383315\\
687	-0.100198481828114\\
688	-0.100192322215943\\
689	-0.100186353463888\\
690	-0.100180569675786\\
691	-0.100174965136557\\
692	-0.100169534306711\\
693	-0.100164271817004\\
694	-0.10015917246327\\
695	-0.100154231201389\\
696	-0.100149443142421\\
697	-0.100144803547878\\
698	-0.100140307825136\\
699	-0.100135951522988\\
700	-0.100131730327329\\
701	-0.100127640056968\\
702	-0.100123676659568\\
703	-0.100119836207709\\
704	-0.100116114895065\\
705	-0.015745704595541\\
706	0.252050153403803\\
707	0.639832606845997\\
708	1.09821932925756\\
709	1.58961529866445\\
710	2.08616842544729\\
711	2.49826786328515\\
712	2.87950976603842\\
713	3.20555185242966\\
714	3.5091117835975\\
715	3.8014414188884\\
716	4.07497613494337\\
717	4.32505500933149\\
718	4.54924661130245\\
719	4.74678879175193\\
720	4.91812999177367\\
721	5.06455946527316\\
722	5.18791419078438\\
723	5.29035096988686\\
724	5.37417314774199\\
725	5.42382529370421\\
726	5.42564660725584\\
727	5.38635042230456\\
728	5.32237494930975\\
729	5.24777474624313\\
730	5.17176680505126\\
731	5.10101080794973\\
732	5.04036539087333\\
733	4.99225826382566\\
734	4.95685815058004\\
735	4.93310197847743\\
736	4.91950487323964\\
737	4.91443218990242\\
738	4.91616668459322\\
739	4.92299775305589\\
740	4.93331851440696\\
741	4.9456874942018\\
742	4.95886160192866\\
743	4.97181778291094\\
744	4.98376586297186\\
745	4.99414856857048\\
746	5.00262843774313\\
747	5.00906392004082\\
748	5.01347836600264\\
749	5.01602466673928\\
750	5.01694839483799\\
751	5.01655216317978\\
752	5.01516346851006\\
753	5.0131076353556\\
754	5.01068080624297\\
755	5.00812171045045\\
756	5.00561763575411\\
757	5.00331423841476\\
758	5.00131612689048\\
759	4.99964571052976\\
760	4.99832562288768\\
761	4.99738354906513\\
762	4.99681666978216\\
763	4.99658714366697\\
764	4.99663617156413\\
765	4.99690639765606\\
766	4.99732558688521\\
767	4.99783167678974\\
768	4.99836776766944\\
769	4.99888948110072\\
770	4.99936595352941\\
771	4.99977664281932\\
772	5.00010905034765\\
773	5.00035791138537\\
774	5.00052455363373\\
775	5.0006162458715\\
776	5.00064452552798\\
777	5.00062274727528\\
778	5.00056420166945\\
779	5.00048296024496\\
780	5.00039166586206\\
781	5.00029875972537\\
782	5.00020919359158\\
783	5.00012662640225\\
784	5.00005425177531\\
785	4.99999431213058\\
786	4.99994777054086\\
787	4.99991456393463\\
788	4.99989393160918\\
789	4.99988454553112\\
790	4.99988460656948\\
791	4.99989204097438\\
792	4.99990473185679\\
793	4.99992069214566\\
794	4.99993816748922\\
795	4.9999556965142\\
796	4.99997214212903\\
797	4.99998669444443\\
798	4.99999884892059\\
799	5.00000836838548\\
800	5.00001523681619\\
801	5.00001961002976\\
802	5.00002176670524\\
803	5.00002206273275\\
804	5.00002089150588\\
805	5.00001865151596\\
806	5.00001572056293\\
807	5.00001243630722\\
808	5.00000908466615\\
809	5.00000589619897\\
810	5.00000304185515\\
811	5.00000063146932\\
812	4.99999871957158\\
813	4.99999731986757\\
814	4.99999640836365\\
815	4.9999959303238\\
816	4.99999581517461\\
817	4.99999598796013\\
818	4.99999637447817\\
819	4.99999690300092\\
820	4.9999975094736\\
821	4.99999813889477\\
822	4.9999987470055\\
823	4.99999930039786\\
824	4.99999977613531\\
825	5.00000016115806\\
826	5.00000045117007\\
827	5.00000064895241\\
828	5.00000076247353\\
829	5.00000080304366\\
830	5.0000007838514\\
831	5.00000071894884\\
832	5.00000062238049\\
833	5.00000050711818\\
834	5.00000038446653\\
835	5.00000026395351\\
836	5.00000015311321\\
837	5.00000005716647\\
838	4.99999997902498\\
839	4.99999991968863\\
840	4.9999998787293\\
841	4.9999998546616\\
842	4.99999984524646\\
843	4.99999984779992\\
844	4.99999985948294\\
845	4.99999987752316\\
846	4.99999989936649\\
847	4.99999992278157\\
848	4.99999994592866\\
849	4.99999996739178\\
850	4.99999998617558\\
851	5.00000000167533\\
852	5.00000001362918\\
853	5.00000002205953\\
854	5.0000000272092\\
855	5.00000002947709\\
856	5.00000002935809\\
857	5.00000002739014\\
858	5.00000002411008\\
859	5.00000002001931\\
860	5.0000000155593\\
861	5.00000001109687\\
862	5.00000000691738\\
863	5.0000000032242\\
864	5.00000000014425\\
865	4.99999999773765\\
866	4.99999999600895\\
867	4.99999999491826\\
868	4.99999999439381\\
869	4.99999999434406\\
870	4.99999999466751\\
871	4.99999999526078\\
872	4.99999999602569\\
873	4.99999999687492\\
874	4.99999999773523\\
875	4.99999999854904\\
876	4.99999999927458\\
877	4.99999999988503\\
878	5.00000000036696\\
879	5.00000000071818\\
880	5.00000000094531\\
881	5.00000000106144\\
882	5.00000000108384\\
883	5.00000000103199\\
884	5.00000000092579\\
885	5.00000000078419\\
886	5.00000000062412\\
887	5.00000000045989\\
888	5.00000000030289\\
889	5.0000000001615\\
890	5.00000000004122\\
891	4.99999999994498\\
892	4.99999999987358\\
893	4.99999999982608\\
894	4.99999999980022\\
895	4.99999999979286\\
896	4.99999999980035\\
897	4.99999999981893\\
898	4.99999999984494\\
899	4.99999999987507\\
900	4.99999999990648\\
901	4.99999999993688\\
902	4.99999999996456\\
903	4.99999999998835\\
904	5.00000000000759\\
905	5.00000000002208\\
906	5.00000000003194\\
907	5.00000000003757\\
908	5.00000000003956\\
909	5.00000000003857\\
910	5.00000000003535\\
911	5.00000000003058\\
912	5.00000000002492\\
913	5.00000000001893\\
914	5.00000000001306\\
915	5.00000000000766\\
916	5.00000000000297\\
917	4.99999999999913\\
918	4.9999999999962\\
919	4.99999999999416\\
920	4.99999999999295\\
921	4.99999999999246\\
922	4.99999999999256\\
923	4.99999999999311\\
924	4.99999999999398\\
925	4.99999999999504\\
926	4.99999999999618\\
927	4.99999999999731\\
928	4.99999999999836\\
929	4.99999999999928\\
930	5.00000000000005\\
931	5.00000000000064\\
932	5.00000000000105\\
933	5.00000000000131\\
934	5.00000000000143\\
935	5.00000000000143\\
936	5.00000000000134\\
937	5.00000000000118\\
938	5.00000000000099\\
939	5.00000000000077\\
940	5.00000000000056\\
941	5.00000000000036\\
942	5.00000000000018\\
943	5.00000000000002\\
944	4.9999999999999\\
945	4.99999999999982\\
946	4.99999999999976\\
947	4.99999999999973\\
948	4.99999999999972\\
949	4.99999999999973\\
950	4.99999999999976\\
951	4.99999999999979\\
952	4.99999999999983\\
953	4.99999999999988\\
954	4.99999999999992\\
955	4.99999999999995\\
956	4.99999999999998\\
957	5.00000000000001\\
958	5.00000000000003\\
959	5.00000000000004\\
960	5.00000000000005\\
961	5.00000000000005\\
962	5.00000000000005\\
963	5.00000000000005\\
964	5.00000000000005\\
965	5.00000000000004\\
966	5.00000000000004\\
967	5.00000000000003\\
968	5.00000000000002\\
969	5.00000000000002\\
970	5.00000000000001\\
971	5\\
972	5\\
973	5\\
974	4.99999999999999\\
975	4.99999999999999\\
976	4.99999999999999\\
977	4.99999999999999\\
978	4.99999999999999\\
979	4.99999999999999\\
980	4.99999999999999\\
981	4.99999999999999\\
982	4.99999999999999\\
983	4.99999999999999\\
984	4.99999999999999\\
985	4.99999999999999\\
986	4.99999999999999\\
987	4.99999999999999\\
988	4.99999999999999\\
989	4.99999999999999\\
990	4.99999999999999\\
991	4.99999999999999\\
992	4.99999999999999\\
993	4.99999999999999\\
994	4.99999999999999\\
995	4.99999999999999\\
996	4.99999999999999\\
997	4.99999999999999\\
998	4.99999999999999\\
999	4.99999999999999\\
1000	5\\
1001	5\\
1002	5\\
1003	5.00000000000001\\
1004	5.00000000000002\\
1005	5.00000000000002\\
1006	5.00000000000002\\
1007	5.00000000000002\\
1008	5.00000000000002\\
1009	5.00000000000002\\
1010	5.00000000000001\\
1011	5.00000000000001\\
1012	5.00000000000001\\
1013	5\\
1014	5\\
1015	5\\
1016	5\\
1017	5\\
1018	4.99999999999999\\
1019	4.99999999999999\\
1020	4.99999999999999\\
1021	4.99999999999999\\
1022	4.99999999999999\\
1023	4.99999999999999\\
1024	4.99999999999999\\
1025	4.99999999999999\\
1026	4.99999999999999\\
1027	4.99999999999999\\
1028	4.99999999999999\\
1029	4.99999999999999\\
1030	4.99999999999999\\
1031	4.99999999999999\\
1032	5\\
1033	5\\
1034	5\\
1035	5\\
1036	5\\
1037	5.00000000000001\\
1038	5.00000000000001\\
1039	5.00000000000001\\
1040	5.00000000000001\\
1041	5\\
1042	5\\
1043	5\\
1044	5\\
1045	5\\
1046	5\\
1047	5\\
1048	5\\
1049	5\\
1050	5\\
1051	5\\
1052	5\\
1053	5\\
1054	5\\
1055	5\\
1056	5\\
1057	5\\
1058	5\\
1059	5\\
1060	5\\
1061	5\\
1062	5\\
1063	5\\
1064	5\\
1065	5\\
1066	5\\
1067	5\\
1068	5\\
1069	5\\
1070	5\\
1071	5\\
1072	4.99999999999999\\
1073	4.99999999999999\\
1074	4.99999999999999\\
1075	4.99999999999999\\
1076	4.99999999999999\\
1077	5\\
1078	5\\
1079	5\\
1080	5\\
1081	5\\
1082	5\\
1083	5\\
1084	5\\
1085	5\\
1086	5\\
1087	5\\
1088	5\\
1089	5\\
1090	5.00000000000001\\
1091	5.00000000000001\\
1092	5.00000000000001\\
1093	5.00000000000001\\
1094	5.00000000000001\\
1095	5.00000000000001\\
1096	5.00000000000001\\
1097	5.00000000000001\\
1098	5\\
1099	5\\
1100	5\\
1101	5\\
1102	5\\
1103	5\\
1104	5\\
1105	5\\
1106	5\\
1107	5\\
1108	5\\
1109	5\\
1110	5\\
1111	5\\
1112	5\\
1113	5\\
1114	5\\
1115	5\\
1116	5\\
1117	5\\
1118	5\\
1119	5\\
1120	5\\
1121	5\\
1122	5\\
1123	5\\
1124	5\\
1125	5\\
1126	5\\
1127	5\\
1128	5\\
1129	5\\
1130	5\\
1131	5\\
1132	5\\
1133	5\\
1134	5\\
1135	5\\
1136	5\\
1137	5\\
1138	5\\
1139	5\\
1140	5\\
1141	5\\
1142	5\\
1143	5\\
1144	5\\
1145	5\\
1146	5\\
1147	5\\
1148	5\\
1149	5\\
1150	5\\
1151	5\\
1152	5\\
1153	5\\
1154	5\\
1155	5\\
1156	5\\
1157	5\\
1158	5\\
1159	5\\
1160	5\\
1161	5\\
1162	5\\
1163	5\\
1164	5\\
1165	5\\
1166	5\\
1167	5\\
1168	5\\
1169	5\\
1170	5\\
1171	5\\
1172	5\\
1173	5\\
1174	5\\
1175	5\\
1176	5\\
1177	5\\
1178	5\\
1179	5\\
1180	5\\
1181	5\\
1182	5\\
1183	5\\
1184	4.99999999999999\\
1185	4.99999999999999\\
1186	4.99999999999999\\
1187	4.99999999999999\\
1188	4.99999999999999\\
1189	4.99999999999999\\
1190	5\\
1191	5\\
1192	5\\
1193	5\\
1194	5.00000000000001\\
1195	5.00000000000001\\
1196	5.00000000000001\\
1197	5.00000000000001\\
1198	5\\
1199	5\\
1200	5\\
};
\addlegendentry{$y(k)$}

\addplot[const plot, color=mycolor2, dashed] table[row sep=crcr] {%
1	0\\
2	0\\
3	0\\
4	0\\
5	0\\
6	0\\
7	0\\
8	0\\
9	0\\
10	0\\
11	0\\
12	0\\
13	0\\
14	0\\
15	0\\
16	0\\
17	0\\
18	0\\
19	0\\
20	0\\
21	0\\
22	0\\
23	0\\
24	0\\
25	0\\
26	0\\
27	0\\
28	0\\
29	0\\
30	0\\
31	0\\
32	0\\
33	0\\
34	0\\
35	0\\
36	0\\
37	0\\
38	0\\
39	0\\
40	0\\
41	0\\
42	0\\
43	0\\
44	0\\
45	0\\
46	0\\
47	0\\
48	0\\
49	0\\
50	0\\
51	0\\
52	0\\
53	0\\
54	0\\
55	0\\
56	0\\
57	0\\
58	0\\
59	0\\
60	0\\
61	0\\
62	0\\
63	0\\
64	0\\
65	0\\
66	0\\
67	0\\
68	0\\
69	0\\
70	0\\
71	0\\
72	0\\
73	0\\
74	0\\
75	0\\
76	0\\
77	0\\
78	0\\
79	0\\
80	0\\
81	0\\
82	0\\
83	0\\
84	0\\
85	0\\
86	0\\
87	0\\
88	0\\
89	0\\
90	0\\
91	0\\
92	0\\
93	0\\
94	0\\
95	0\\
96	0\\
97	0\\
98	0\\
99	0\\
100	0.5\\
101	0.5\\
102	0.5\\
103	0.5\\
104	0.5\\
105	0.5\\
106	0.5\\
107	0.5\\
108	0.5\\
109	0.5\\
110	0.5\\
111	0.5\\
112	0.5\\
113	0.5\\
114	0.5\\
115	0.5\\
116	0.5\\
117	0.5\\
118	0.5\\
119	0.5\\
120	0.5\\
121	0.5\\
122	0.5\\
123	0.5\\
124	0.5\\
125	0.5\\
126	0.5\\
127	0.5\\
128	0.5\\
129	0.5\\
130	0.5\\
131	0.5\\
132	0.5\\
133	0.5\\
134	0.5\\
135	0.5\\
136	0.5\\
137	0.5\\
138	0.5\\
139	0.5\\
140	0.5\\
141	0.5\\
142	0.5\\
143	0.5\\
144	0.5\\
145	0.5\\
146	0.5\\
147	0.5\\
148	0.5\\
149	0.5\\
150	0.5\\
151	0.5\\
152	0.5\\
153	0.5\\
154	0.5\\
155	0.5\\
156	0.5\\
157	0.5\\
158	0.5\\
159	0.5\\
160	0.5\\
161	0.5\\
162	0.5\\
163	0.5\\
164	0.5\\
165	0.5\\
166	0.5\\
167	0.5\\
168	0.5\\
169	0.5\\
170	0.5\\
171	0.5\\
172	0.5\\
173	0.5\\
174	0.5\\
175	0.5\\
176	0.5\\
177	0.5\\
178	0.5\\
179	0.5\\
180	0.5\\
181	0.5\\
182	0.5\\
183	0.5\\
184	0.5\\
185	0.5\\
186	0.5\\
187	0.5\\
188	0.5\\
189	0.5\\
190	0.5\\
191	0.5\\
192	0.5\\
193	0.5\\
194	0.5\\
195	0.5\\
196	0.5\\
197	0.5\\
198	0.5\\
199	0.5\\
200	0.5\\
201	0.5\\
202	0.5\\
203	0.5\\
204	0.5\\
205	0.5\\
206	0.5\\
207	0.5\\
208	0.5\\
209	0.5\\
210	0.5\\
211	0.5\\
212	0.5\\
213	0.5\\
214	0.5\\
215	0.5\\
216	0.5\\
217	0.5\\
218	0.5\\
219	0.5\\
220	0.5\\
221	0.5\\
222	0.5\\
223	0.5\\
224	0.5\\
225	0.5\\
226	0.5\\
227	0.5\\
228	0.5\\
229	0.5\\
230	0.5\\
231	0.5\\
232	0.5\\
233	0.5\\
234	0.5\\
235	0.5\\
236	0.5\\
237	0.5\\
238	0.5\\
239	0.5\\
240	0.5\\
241	0.5\\
242	0.5\\
243	0.5\\
244	0.5\\
245	0.5\\
246	0.5\\
247	0.5\\
248	0.5\\
249	0.5\\
250	0.5\\
251	0.5\\
252	0.5\\
253	0.5\\
254	0.5\\
255	0.5\\
256	0.5\\
257	0.5\\
258	0.5\\
259	0.5\\
260	0.5\\
261	0.5\\
262	0.5\\
263	0.5\\
264	0.5\\
265	0.5\\
266	0.5\\
267	0.5\\
268	0.5\\
269	0.5\\
270	0.5\\
271	0.5\\
272	0.5\\
273	0.5\\
274	0.5\\
275	0.5\\
276	0.5\\
277	0.5\\
278	0.5\\
279	0.5\\
280	0.5\\
281	0.5\\
282	0.5\\
283	0.5\\
284	0.5\\
285	0.5\\
286	0.5\\
287	0.5\\
288	0.5\\
289	0.5\\
290	0.5\\
291	0.5\\
292	0.5\\
293	0.5\\
294	0.5\\
295	0.5\\
296	0.5\\
297	0.5\\
298	0.5\\
299	0.5\\
300	1\\
301	1\\
302	1\\
303	1\\
304	1\\
305	1\\
306	1\\
307	1\\
308	1\\
309	1\\
310	1\\
311	1\\
312	1\\
313	1\\
314	1\\
315	1\\
316	1\\
317	1\\
318	1\\
319	1\\
320	1\\
321	1\\
322	1\\
323	1\\
324	1\\
325	1\\
326	1\\
327	1\\
328	1\\
329	1\\
330	1\\
331	1\\
332	1\\
333	1\\
334	1\\
335	1\\
336	1\\
337	1\\
338	1\\
339	1\\
340	1\\
341	1\\
342	1\\
343	1\\
344	1\\
345	1\\
346	1\\
347	1\\
348	1\\
349	1\\
350	1\\
351	1\\
352	1\\
353	1\\
354	1\\
355	1\\
356	1\\
357	1\\
358	1\\
359	1\\
360	1\\
361	1\\
362	1\\
363	1\\
364	1\\
365	1\\
366	1\\
367	1\\
368	1\\
369	1\\
370	1\\
371	1\\
372	1\\
373	1\\
374	1\\
375	1\\
376	1\\
377	1\\
378	1\\
379	1\\
380	1\\
381	1\\
382	1\\
383	1\\
384	1\\
385	1\\
386	1\\
387	1\\
388	1\\
389	1\\
390	1\\
391	1\\
392	1\\
393	1\\
394	1\\
395	1\\
396	1\\
397	1\\
398	1\\
399	1\\
400	1\\
401	1\\
402	1\\
403	1\\
404	1\\
405	1\\
406	1\\
407	1\\
408	1\\
409	1\\
410	1\\
411	1\\
412	1\\
413	1\\
414	1\\
415	1\\
416	1\\
417	1\\
418	1\\
419	1\\
420	1\\
421	1\\
422	1\\
423	1\\
424	1\\
425	1\\
426	1\\
427	1\\
428	1\\
429	1\\
430	1\\
431	1\\
432	1\\
433	1\\
434	1\\
435	1\\
436	1\\
437	1\\
438	1\\
439	1\\
440	1\\
441	1\\
442	1\\
443	1\\
444	1\\
445	1\\
446	1\\
447	1\\
448	1\\
449	1\\
450	1\\
451	1\\
452	1\\
453	1\\
454	1\\
455	1\\
456	1\\
457	1\\
458	1\\
459	1\\
460	1\\
461	1\\
462	1\\
463	1\\
464	1\\
465	1\\
466	1\\
467	1\\
468	1\\
469	1\\
470	1\\
471	1\\
472	1\\
473	1\\
474	1\\
475	1\\
476	1\\
477	1\\
478	1\\
479	1\\
480	1\\
481	1\\
482	1\\
483	1\\
484	1\\
485	1\\
486	1\\
487	1\\
488	1\\
489	1\\
490	1\\
491	1\\
492	1\\
493	1\\
494	1\\
495	1\\
496	1\\
497	1\\
498	1\\
499	1\\
500	-0.1\\
501	-0.1\\
502	-0.1\\
503	-0.1\\
504	-0.1\\
505	-0.1\\
506	-0.1\\
507	-0.1\\
508	-0.1\\
509	-0.1\\
510	-0.1\\
511	-0.1\\
512	-0.1\\
513	-0.1\\
514	-0.1\\
515	-0.1\\
516	-0.1\\
517	-0.1\\
518	-0.1\\
519	-0.1\\
520	-0.1\\
521	-0.1\\
522	-0.1\\
523	-0.1\\
524	-0.1\\
525	-0.1\\
526	-0.1\\
527	-0.1\\
528	-0.1\\
529	-0.1\\
530	-0.1\\
531	-0.1\\
532	-0.1\\
533	-0.1\\
534	-0.1\\
535	-0.1\\
536	-0.1\\
537	-0.1\\
538	-0.1\\
539	-0.1\\
540	-0.1\\
541	-0.1\\
542	-0.1\\
543	-0.1\\
544	-0.1\\
545	-0.1\\
546	-0.1\\
547	-0.1\\
548	-0.1\\
549	-0.1\\
550	-0.1\\
551	-0.1\\
552	-0.1\\
553	-0.1\\
554	-0.1\\
555	-0.1\\
556	-0.1\\
557	-0.1\\
558	-0.1\\
559	-0.1\\
560	-0.1\\
561	-0.1\\
562	-0.1\\
563	-0.1\\
564	-0.1\\
565	-0.1\\
566	-0.1\\
567	-0.1\\
568	-0.1\\
569	-0.1\\
570	-0.1\\
571	-0.1\\
572	-0.1\\
573	-0.1\\
574	-0.1\\
575	-0.1\\
576	-0.1\\
577	-0.1\\
578	-0.1\\
579	-0.1\\
580	-0.1\\
581	-0.1\\
582	-0.1\\
583	-0.1\\
584	-0.1\\
585	-0.1\\
586	-0.1\\
587	-0.1\\
588	-0.1\\
589	-0.1\\
590	-0.1\\
591	-0.1\\
592	-0.1\\
593	-0.1\\
594	-0.1\\
595	-0.1\\
596	-0.1\\
597	-0.1\\
598	-0.1\\
599	-0.1\\
600	-0.1\\
601	-0.1\\
602	-0.1\\
603	-0.1\\
604	-0.1\\
605	-0.1\\
606	-0.1\\
607	-0.1\\
608	-0.1\\
609	-0.1\\
610	-0.1\\
611	-0.1\\
612	-0.1\\
613	-0.1\\
614	-0.1\\
615	-0.1\\
616	-0.1\\
617	-0.1\\
618	-0.1\\
619	-0.1\\
620	-0.1\\
621	-0.1\\
622	-0.1\\
623	-0.1\\
624	-0.1\\
625	-0.1\\
626	-0.1\\
627	-0.1\\
628	-0.1\\
629	-0.1\\
630	-0.1\\
631	-0.1\\
632	-0.1\\
633	-0.1\\
634	-0.1\\
635	-0.1\\
636	-0.1\\
637	-0.1\\
638	-0.1\\
639	-0.1\\
640	-0.1\\
641	-0.1\\
642	-0.1\\
643	-0.1\\
644	-0.1\\
645	-0.1\\
646	-0.1\\
647	-0.1\\
648	-0.1\\
649	-0.1\\
650	-0.1\\
651	-0.1\\
652	-0.1\\
653	-0.1\\
654	-0.1\\
655	-0.1\\
656	-0.1\\
657	-0.1\\
658	-0.1\\
659	-0.1\\
660	-0.1\\
661	-0.1\\
662	-0.1\\
663	-0.1\\
664	-0.1\\
665	-0.1\\
666	-0.1\\
667	-0.1\\
668	-0.1\\
669	-0.1\\
670	-0.1\\
671	-0.1\\
672	-0.1\\
673	-0.1\\
674	-0.1\\
675	-0.1\\
676	-0.1\\
677	-0.1\\
678	-0.1\\
679	-0.1\\
680	-0.1\\
681	-0.1\\
682	-0.1\\
683	-0.1\\
684	-0.1\\
685	-0.1\\
686	-0.1\\
687	-0.1\\
688	-0.1\\
689	-0.1\\
690	-0.1\\
691	-0.1\\
692	-0.1\\
693	-0.1\\
694	-0.1\\
695	-0.1\\
696	-0.1\\
697	-0.1\\
698	-0.1\\
699	-0.1\\
700	5\\
701	5\\
702	5\\
703	5\\
704	5\\
705	5\\
706	5\\
707	5\\
708	5\\
709	5\\
710	5\\
711	5\\
712	5\\
713	5\\
714	5\\
715	5\\
716	5\\
717	5\\
718	5\\
719	5\\
720	5\\
721	5\\
722	5\\
723	5\\
724	5\\
725	5\\
726	5\\
727	5\\
728	5\\
729	5\\
730	5\\
731	5\\
732	5\\
733	5\\
734	5\\
735	5\\
736	5\\
737	5\\
738	5\\
739	5\\
740	5\\
741	5\\
742	5\\
743	5\\
744	5\\
745	5\\
746	5\\
747	5\\
748	5\\
749	5\\
750	5\\
751	5\\
752	5\\
753	5\\
754	5\\
755	5\\
756	5\\
757	5\\
758	5\\
759	5\\
760	5\\
761	5\\
762	5\\
763	5\\
764	5\\
765	5\\
766	5\\
767	5\\
768	5\\
769	5\\
770	5\\
771	5\\
772	5\\
773	5\\
774	5\\
775	5\\
776	5\\
777	5\\
778	5\\
779	5\\
780	5\\
781	5\\
782	5\\
783	5\\
784	5\\
785	5\\
786	5\\
787	5\\
788	5\\
789	5\\
790	5\\
791	5\\
792	5\\
793	5\\
794	5\\
795	5\\
796	5\\
797	5\\
798	5\\
799	5\\
800	5\\
801	5\\
802	5\\
803	5\\
804	5\\
805	5\\
806	5\\
807	5\\
808	5\\
809	5\\
810	5\\
811	5\\
812	5\\
813	5\\
814	5\\
815	5\\
816	5\\
817	5\\
818	5\\
819	5\\
820	5\\
821	5\\
822	5\\
823	5\\
824	5\\
825	5\\
826	5\\
827	5\\
828	5\\
829	5\\
830	5\\
831	5\\
832	5\\
833	5\\
834	5\\
835	5\\
836	5\\
837	5\\
838	5\\
839	5\\
840	5\\
841	5\\
842	5\\
843	5\\
844	5\\
845	5\\
846	5\\
847	5\\
848	5\\
849	5\\
850	5\\
851	5\\
852	5\\
853	5\\
854	5\\
855	5\\
856	5\\
857	5\\
858	5\\
859	5\\
860	5\\
861	5\\
862	5\\
863	5\\
864	5\\
865	5\\
866	5\\
867	5\\
868	5\\
869	5\\
870	5\\
871	5\\
872	5\\
873	5\\
874	5\\
875	5\\
876	5\\
877	5\\
878	5\\
879	5\\
880	5\\
881	5\\
882	5\\
883	5\\
884	5\\
885	5\\
886	5\\
887	5\\
888	5\\
889	5\\
890	5\\
891	5\\
892	5\\
893	5\\
894	5\\
895	5\\
896	5\\
897	5\\
898	5\\
899	5\\
900	5\\
901	5\\
902	5\\
903	5\\
904	5\\
905	5\\
906	5\\
907	5\\
908	5\\
909	5\\
910	5\\
911	5\\
912	5\\
913	5\\
914	5\\
915	5\\
916	5\\
917	5\\
918	5\\
919	5\\
920	5\\
921	5\\
922	5\\
923	5\\
924	5\\
925	5\\
926	5\\
927	5\\
928	5\\
929	5\\
930	5\\
931	5\\
932	5\\
933	5\\
934	5\\
935	5\\
936	5\\
937	5\\
938	5\\
939	5\\
940	5\\
941	5\\
942	5\\
943	5\\
944	5\\
945	5\\
946	5\\
947	5\\
948	5\\
949	5\\
950	5\\
951	5\\
952	5\\
953	5\\
954	5\\
955	5\\
956	5\\
957	5\\
958	5\\
959	5\\
960	5\\
961	5\\
962	5\\
963	5\\
964	5\\
965	5\\
966	5\\
967	5\\
968	5\\
969	5\\
970	5\\
971	5\\
972	5\\
973	5\\
974	5\\
975	5\\
976	5\\
977	5\\
978	5\\
979	5\\
980	5\\
981	5\\
982	5\\
983	5\\
984	5\\
985	5\\
986	5\\
987	5\\
988	5\\
989	5\\
990	5\\
991	5\\
992	5\\
993	5\\
994	5\\
995	5\\
996	5\\
997	5\\
998	5\\
999	5\\
1000	5\\
1001	5\\
1002	5\\
1003	5\\
1004	5\\
1005	5\\
1006	5\\
1007	5\\
1008	5\\
1009	5\\
1010	5\\
1011	5\\
1012	5\\
1013	5\\
1014	5\\
1015	5\\
1016	5\\
1017	5\\
1018	5\\
1019	5\\
1020	5\\
1021	5\\
1022	5\\
1023	5\\
1024	5\\
1025	5\\
1026	5\\
1027	5\\
1028	5\\
1029	5\\
1030	5\\
1031	5\\
1032	5\\
1033	5\\
1034	5\\
1035	5\\
1036	5\\
1037	5\\
1038	5\\
1039	5\\
1040	5\\
1041	5\\
1042	5\\
1043	5\\
1044	5\\
1045	5\\
1046	5\\
1047	5\\
1048	5\\
1049	5\\
1050	5\\
1051	5\\
1052	5\\
1053	5\\
1054	5\\
1055	5\\
1056	5\\
1057	5\\
1058	5\\
1059	5\\
1060	5\\
1061	5\\
1062	5\\
1063	5\\
1064	5\\
1065	5\\
1066	5\\
1067	5\\
1068	5\\
1069	5\\
1070	5\\
1071	5\\
1072	5\\
1073	5\\
1074	5\\
1075	5\\
1076	5\\
1077	5\\
1078	5\\
1079	5\\
1080	5\\
1081	5\\
1082	5\\
1083	5\\
1084	5\\
1085	5\\
1086	5\\
1087	5\\
1088	5\\
1089	5\\
1090	5\\
1091	5\\
1092	5\\
1093	5\\
1094	5\\
1095	5\\
1096	5\\
1097	5\\
1098	5\\
1099	5\\
1100	5\\
1101	5\\
1102	5\\
1103	5\\
1104	5\\
1105	5\\
1106	5\\
1107	5\\
1108	5\\
1109	5\\
1110	5\\
1111	5\\
1112	5\\
1113	5\\
1114	5\\
1115	5\\
1116	5\\
1117	5\\
1118	5\\
1119	5\\
1120	5\\
1121	5\\
1122	5\\
1123	5\\
1124	5\\
1125	5\\
1126	5\\
1127	5\\
1128	5\\
1129	5\\
1130	5\\
1131	5\\
1132	5\\
1133	5\\
1134	5\\
1135	5\\
1136	5\\
1137	5\\
1138	5\\
1139	5\\
1140	5\\
1141	5\\
1142	5\\
1143	5\\
1144	5\\
1145	5\\
1146	5\\
1147	5\\
1148	5\\
1149	5\\
1150	5\\
1151	5\\
1152	5\\
1153	5\\
1154	5\\
1155	5\\
1156	5\\
1157	5\\
1158	5\\
1159	5\\
1160	5\\
1161	5\\
1162	5\\
1163	5\\
1164	5\\
1165	5\\
1166	5\\
1167	5\\
1168	5\\
1169	5\\
1170	5\\
1171	5\\
1172	5\\
1173	5\\
1174	5\\
1175	5\\
1176	5\\
1177	5\\
1178	5\\
1179	5\\
1180	5\\
1181	5\\
1182	5\\
1183	5\\
1184	5\\
1185	5\\
1186	5\\
1187	5\\
1188	5\\
1189	5\\
1190	5\\
1191	5\\
1192	5\\
1193	5\\
1194	5\\
1195	5\\
1196	5\\
1197	5\\
1198	5\\
1199	5\\
1200	5\\
};
\addlegendentry{$y^{zad}(k)$}

\end{axis}
\end{tikzpicture}%
	\caption{Przebiegi sygna��w dla rozmytego regulatora DMC z 3 regulatorami lokalnymi}
\end{figure}

\newpage
\subsection{Liczba regulator�w lokalnych $n_r = 4$}

Parametry regulatora: $\lambda^{1} = 0,2, \lambda^{2} = 0,6, \lambda^{3} = 0,1, \lambda^{4} = 10$

\begin{figure}[h!]
	\centering
	% This file was created by matlab2tikz.
%
\definecolor{mycolor1}{rgb}{0.00000,0.44700,0.74100}%
\definecolor{mycolor2}{rgb}{0.85000,0.32500,0.09800}%
%
\begin{tikzpicture}[scale=0.8]

\begin{axis}[%
width=4.521in,
height=3.566in,
at={(0.758in,0.481in)},
scale only axis,
xmin=0,
xmax=1200,
xlabel style={font=\color{white!15!black}},
xlabel={$k$},
ymin=-1,
ymax=6,
ylabel style={font=\color{white!15!black}},
ylabel={$y(k)$, $y^{zad}(k)$},
axis background/.style={fill=white},
title style={font=\bfseries},
title={Sygna� wyj�ciowy i zadany},
xmajorgrids,
ymajorgrids,
legend style={at={(0.97,0.03)}, anchor=south east, legend cell align=left, align=left, draw=white!15!black}
]
\addplot [color=mycolor1]
  table[row sep=crcr]{%
1	0\\
2	0\\
3	0\\
4	0\\
5	0\\
6	0\\
7	0\\
8	0\\
9	0\\
10	0\\
11	0\\
12	0\\
13	0\\
14	0\\
15	0\\
16	0\\
17	0\\
18	0\\
19	0\\
20	0\\
21	0\\
22	0\\
23	0\\
24	0\\
25	0\\
26	0\\
27	0\\
28	0\\
29	0\\
30	0\\
31	0\\
32	0\\
33	0\\
34	0\\
35	0\\
36	0\\
37	0\\
38	0\\
39	0\\
40	0\\
41	0\\
42	0\\
43	0\\
44	0\\
45	0\\
46	0\\
47	0\\
48	0\\
49	0\\
50	0\\
51	0\\
52	0\\
53	0\\
54	0\\
55	0\\
56	0\\
57	0\\
58	0\\
59	0\\
60	0\\
61	0\\
62	0\\
63	0\\
64	0\\
65	0\\
66	0\\
67	0\\
68	0\\
69	0\\
70	0\\
71	0\\
72	0\\
73	0\\
74	0\\
75	0\\
76	0\\
77	0\\
78	0\\
79	0\\
80	0\\
81	0\\
82	0\\
83	0\\
84	0\\
85	0\\
86	0\\
87	0\\
88	0\\
89	0\\
90	0\\
91	0\\
92	0\\
93	0\\
94	0\\
95	0\\
96	0\\
97	0\\
98	0\\
99	0\\
100	0\\
101	0\\
102	0\\
103	0\\
104	0\\
105	0.0687398882690996\\
106	0.193915208700101\\
107	0.270903410569045\\
108	0.29661136102804\\
109	0.293205616735853\\
110	0.287690094148441\\
111	0.292304845156861\\
112	0.313319648211752\\
113	0.350277618734353\\
114	0.395194931319366\\
115	0.437709788900537\\
116	0.4708233741733\\
117	0.492585349193204\\
118	0.504356117498128\\
119	0.508747952469061\\
120	0.508575583954737\\
121	0.506366093843246\\
122	0.503971149855566\\
123	0.502366973202307\\
124	0.501760877434505\\
125	0.501888904645785\\
126	0.50232140911257\\
127	0.502679463750197\\
128	0.502743581005295\\
129	0.502469497964365\\
130	0.501942187445701\\
131	0.50130603068145\\
132	0.500702036648693\\
133	0.500228140214511\\
134	0.499925377235918\\
135	0.499784632721976\\
136	0.499764992266446\\
137	0.499814540613894\\
138	0.49988689757669\\
139	0.499950418034441\\
140	0.499990186117023\\
141	0.500004931529882\\
142	0.500001587867467\\
143	0.499989826330015\\
144	0.499978039282216\\
145	0.499971305673989\\
146	0.4999711292386\\
147	0.499976329955501\\
148	0.499984386206415\\
149	0.499992668827219\\
150	0.499999252478321\\
151	0.500003229040434\\
152	0.500004617957572\\
153	0.500004047923442\\
154	0.500002430939403\\
155	0.500000699784763\\
156	0.499999455150218\\
157	0.499998860428179\\
158	0.499998794554986\\
159	0.499999373502237\\
160	0.500000225853706\\
161	0.500000781087849\\
162	0.500000856078823\\
163	0.500000608650738\\
164	0.500000271344395\\
165	0.499999991550151\\
166	0.499999846048358\\
167	0.499999850033794\\
168	0.49999995585375\\
169	0.500000084583681\\
170	0.50000017353737\\
171	0.500000198652237\\
172	0.500000167605058\\
173	0.50000010365914\\
174	0.500000033160887\\
175	0.499999977399151\\
176	0.499999947624182\\
177	0.499999944075137\\
178	0.499999959037267\\
179	0.49999998176347\\
180	0.500000002754075\\
181	0.500000016169207\\
182	0.500000020348706\\
183	0.500000016948238\\
184	0.500000009372038\\
185	0.500000001183452\\
186	0.499999994986975\\
187	0.499999991968137\\
188	0.499999992006896\\
189	0.499999994137103\\
190	0.499999997102488\\
191	0.499999999815017\\
192	0.500000001613396\\
193	0.500000002312064\\
194	0.500000002096253\\
195	0.500000001345407\\
196	0.50000000046022\\
197	0.499999999741851\\
198	0.499999999340338\\
199	0.499999999263593\\
200	0.499999999423824\\
201	0.499999999695365\\
202	0.499999999963469\\
203	0.50000000015327\\
204	0.500000000237496\\
205	0.500000000228039\\
206	0.500000000159188\\
207	0.500000000070187\\
208	0.499999999992634\\
209	0.499999999944583\\
210	0.499999999929679\\
211	0.499999999940525\\
212	0.499999999964683\\
213	0.499999999991074\\
214	0.500000000011537\\
215	0.50000000002158\\
216	0.500000000021344\\
217	0.500000000014532\\
218	0.500000000005763\\
219	0.499999999998467\\
220	0.499999999994319\\
221	0.49999999999348\\
222	0.499999999995038\\
223	0.499999999997609\\
224	0.499999999999967\\
225	0.500000000001416\\
226	0.500000000001829\\
227	0.50000000000146\\
228	0.50000000000072\\
229	0.499999999999993\\
230	0.499999999999522\\
231	0.499999999999379\\
232	0.499999999999499\\
233	0.49999999999975\\
234	0.500000000000003\\
235	0.50000000000017\\
236	0.500000000000224\\
237	0.500000000000183\\
238	0.500000000000092\\
239	0.499999999999997\\
240	0.499999999999932\\
241	0.499999999999909\\
242	0.499999999999922\\
243	0.499999999999957\\
244	0.499999999999995\\
245	0.500000000000024\\
246	0.500000000000037\\
247	0.500000000000035\\
248	0.500000000000023\\
249	0.500000000000007\\
250	0.499999999999994\\
251	0.499999999999987\\
252	0.499999999999986\\
253	0.499999999999989\\
254	0.499999999999995\\
255	0.5\\
256	0.500000000000004\\
257	0.500000000000005\\
258	0.500000000000005\\
259	0.500000000000003\\
260	0.500000000000001\\
261	0.499999999999999\\
262	0.499999999999998\\
263	0.499999999999998\\
264	0.499999999999999\\
265	0.5\\
266	0.500000000000001\\
267	0.500000000000001\\
268	0.500000000000001\\
269	0.500000000000001\\
270	0.5\\
271	0.5\\
272	0.5\\
273	0.499999999999999\\
274	0.499999999999999\\
275	0.5\\
276	0.5\\
277	0.5\\
278	0.500000000000001\\
279	0.500000000000001\\
280	0.500000000000001\\
281	0.500000000000001\\
282	0.5\\
283	0.5\\
284	0.5\\
285	0.499999999999999\\
286	0.499999999999999\\
287	0.5\\
288	0.5\\
289	0.5\\
290	0.5\\
291	0.5\\
292	0.5\\
293	0.5\\
294	0.5\\
295	0.5\\
296	0.5\\
297	0.5\\
298	0.5\\
299	0.5\\
300	0.5\\
301	0.5\\
302	0.5\\
303	0.5\\
304	0.5\\
305	0.614352222244396\\
306	0.806355699101332\\
307	0.960461755692768\\
308	1.05208740866432\\
309	1.10066173954215\\
310	1.13187585409848\\
311	1.14969040028682\\
312	1.15044325311431\\
313	1.13591256263971\\
314	1.11131001520676\\
315	1.08054505968424\\
316	1.0469706423249\\
317	1.01480956591347\\
318	0.988193152545355\\
319	0.969663800094995\\
320	0.959912482441841\\
321	0.958252353681343\\
322	0.963047122943037\\
323	0.972075226060695\\
324	0.982983259236827\\
325	0.993716113504413\\
326	1.00276606833716\\
327	1.00923900864606\\
328	1.01281758563623\\
329	1.01367007267466\\
330	1.01231873143927\\
331	1.00948487234785\\
332	1.00593806323816\\
333	1.00237236235874\\
334	0.999321266766736\\
335	0.997115036434835\\
336	0.995878302487306\\
337	0.995560057930271\\
338	0.995983827361647\\
339	0.996904950618464\\
340	0.998064162417831\\
341	0.999230076279788\\
342	1.00022690296201\\
343	1.00094698906799\\
344	1.00135015601757\\
345	1.00145323410944\\
346	1.00131372615392\\
347	1.00101140475642\\
348	1.00063103702531\\
349	1.00024851793689\\
350	0.999921660465689\\
351	0.999685908113005\\
352	0.999554436333488\\
353	0.999521573586747\\
354	0.999568283768034\\
355	0.999668352424685\\
356	0.999793930626639\\
357	0.999920033568934\\
358	1.00002763073747\\
359	1.00010517682734\\
360	1.00014826740475\\
361	1.00015867419597\\
362	1.00014272412718\\
363	1.00010922996824\\
364	1.00006748864266\\
365	1.00002574548999\\
366	0.999990241641459\\
367	0.999964783191408\\
368	0.999950754266832\\
369	0.999947488178344\\
370	0.999952866912758\\
371	0.999963994505747\\
372	0.999977818066587\\
373	0.999991619491439\\
374	1.00000334018421\\
375	1.00001173016951\\
376	1.00001633977316\\
377	1.00001739309604\\
378	1.00001559158776\\
379	1.00001189404518\\
380	1.00000731121563\\
381	1.00000274208058\\
382	0.999998866396469\\
383	0.999996096046998\\
384	0.999994578211285\\
385	0.999994237307825\\
386	0.999994840088737\\
387	0.999996068581083\\
388	0.999997588075785\\
389	0.999999101193474\\
390	1.00000038331307\\
391	1.00000129857413\\
392	1.0000017987523\\
393	1.00000190930855\\
394	1.00000170779271\\
395	1.00000129967549\\
396	1.00000079584209\\
397	1.0000002947107\\
398	0.999999870531904\\
399	0.999999568125815\\
400	0.999999403289115\\
401	0.999999367441993\\
402	0.999999434796529\\
403	0.999999570366212\\
404	0.999999737416425\\
405	0.999999903377266\\
406	1.0000000437061\\
407	1.00000014361758\\
408	1.00000019793764\\
409	1.00000020955614\\
410	1.00000018704762\\
411	1.00000014201684\\
412	1.00000008663268\\
413	1.00000003167328\\
414	0.999999985250672\\
415	0.999999952242104\\
416	0.999999934342262\\
417	0.999999930578443\\
418	0.9999999380999\\
419	0.999999953057372\\
420	0.999999971419583\\
421	0.999999989619791\\
422	1.00000000497673\\
423	1.00000001588154\\
424	1.00000002177932\\
425	1.00000002299758\\
426	1.00000002048418\\
427	1.0000000155159\\
428	1.00000000942815\\
429	1.00000000340118\\
430	0.999999998321184\\
431	0.999999994718804\\
432	0.99999999277569\\
433	0.999999992381616\\
434	0.999999993221431\\
435	0.999999994871657\\
436	0.999999996889939\\
437	0.999999998885726\\
438	1.00000000056613\\
439	1.00000000175615\\
440	1.00000000239631\\
441	1.00000000252371\\
442	1.00000000224312\\
443	1.00000000169501\\
444	1.00000000102589\\
445	1.00000000036501\\
446	0.999999999809152\\
447	0.999999999416047\\
448	0.999999999205151\\
449	0.999999999163989\\
450	0.999999999257726\\
451	0.999999999439777\\
452	0.999999999661605\\
453	0.999999999880448\\
454	1.00000000006432\\
455	1.00000000019417\\
456	1.00000000026365\\
457	1.00000000027694\\
458	1.00000000024562\\
459	1.00000000018516\\
460	1.00000000011162\\
461	1.00000000003915\\
462	0.999999999978333\\
463	0.999999999935438\\
464	0.999999999912552\\
465	0.999999999908263\\
466	0.999999999918722\\
467	0.999999999938804\\
468	0.999999999963184\\
469	0.99999999998718\\
470	1.0000000000073\\
471	1.00000000002147\\
472	1.00000000002901\\
473	1.00000000003039\\
474	1.0000000000269\\
475	1.00000000002023\\
476	1.00000000001214\\
477	1.0000000000042\\
478	0.999999999997543\\
479	0.999999999992862\\
480	0.999999999990379\\
481	0.999999999989933\\
482	0.9999999999911\\
483	0.999999999993315\\
484	0.999999999995995\\
485	0.999999999998626\\
486	1.00000000000083\\
487	1.00000000000237\\
488	1.00000000000319\\
489	1.00000000000333\\
490	1.00000000000294\\
491	1.00000000000221\\
492	1.00000000000132\\
493	1.00000000000045\\
494	0.999999999999722\\
495	0.999999999999212\\
496	0.999999999998942\\
497	0.999999999998896\\
498	0.999999999999026\\
499	0.99999999999927\\
500	0.999999999999564\\
501	0.999999999999853\\
502	1.00000000000009\\
503	1.00000000000026\\
504	1.00000000000035\\
505	0.62277093487233\\
506	0.313133489239668\\
507	0.115407385286916\\
508	-0.00476933596129012\\
509	-0.07696836757723\\
510	-0.120220784787066\\
511	-0.146113991272713\\
512	-0.161612372002514\\
513	-0.170888533386373\\
514	-0.17644048521764\\
515	-0.178633264919937\\
516	-0.17716483338573\\
517	-0.172784350000812\\
518	-0.166693536612458\\
519	-0.16017833663132\\
520	-0.154364560191781\\
521	-0.149861797782581\\
522	-0.146630879056123\\
523	-0.144274967261412\\
524	-0.142351382521463\\
525	-0.140530959563136\\
526	-0.138640213051488\\
527	-0.136643821440161\\
528	-0.134599825943187\\
529	-0.132605695202496\\
530	-0.130749706395686\\
531	-0.129080113148918\\
532	-0.12759777542028\\
533	-0.126268195951672\\
534	-0.125042493625679\\
535	-0.123876736902192\\
536	-0.122743350387551\\
537	-0.121633384550853\\
538	-0.120551954104588\\
539	-0.119510590032128\\
540	-0.118520044403802\\
541	-0.1175858145908\\
542	-0.116707040776785\\
543	-0.115878094798107\\
544	-0.115091404603554\\
545	-0.114340091520659\\
546	-0.113619540932186\\
547	-0.112927701324923\\
548	-0.112264423795712\\
549	-0.11163038783327\\
550	-0.111026124349463\\
551	-0.110451444215997\\
552	-0.109905335529393\\
553	-0.109386206039325\\
554	-0.108892266441551\\
555	-0.108421869528191\\
556	-0.107973698591471\\
557	-0.107546787839685\\
558	-0.107140423826534\\
559	-0.106754002397522\\
560	-0.106386906856886\\
561	-0.106038438639861\\
562	-0.105707810962184\\
563	-0.105394184490732\\
564	-0.105096717703923\\
565	-0.104814609147935\\
566	-0.10454711948192\\
567	-0.104293572432532\\
568	-0.104053341473986\\
569	-0.103825831702183\\
570	-0.103610464740477\\
571	-0.103406670929934\\
572	-0.103213888888752\\
573	-0.103031569920347\\
574	-0.102859183929352\\
575	-0.102696224111941\\
576	-0.102542209052827\\
577	-0.102396682248865\\
578	-0.102259209961266\\
579	-0.102129378554372\\
580	-0.102006792246744\\
581	-0.101891071729605\\
582	-0.101781853643009\\
583	-0.101678790600988\\
584	-0.101581551370307\\
585	-0.101489820889822\\
586	-0.101403299981129\\
587	-0.101321704760654\\
588	-0.101244765864496\\
589	-0.101172227625276\\
590	-0.101103847310768\\
591	-0.101039394477391\\
592	-0.100978650436867\\
593	-0.100921407799704\\
594	-0.100867470049622\\
595	-0.10081665111306\\
596	-0.100768774907073\\
597	-0.100723674867413\\
598	-0.100681193470136\\
599	-0.100641181763041\\
600	-0.100603498919635\\
601	-0.100568011821557\\
602	-0.100534594668964\\
603	-0.10050312861432\\
604	-0.100473501413923\\
605	-0.100445607092627\\
606	-0.100419345619455\\
607	-0.100394622593904\\
608	-0.100371348944065\\
609	-0.10034944063802\\
610	-0.100328818409548\\
611	-0.100309407498368\\
612	-0.10029113740442\\
613	-0.100273941655212\\
614	-0.100257757585141\\
615	-0.100242526125806\\
616	-0.10022819160664\\
617	-0.10021470156542\\
618	-0.100202006568359\\
619	-0.100190060039589\\
620	-0.100178818099744\\
621	-0.100168239413328\\
622	-0.100158285044444\\
623	-0.10014891832044\\
624	-0.100140104703011\\
625	-0.10013181166634\\
626	-0.100124008581894\\
627	-0.100116666609533\\
628	-0.100109758594638\\
629	-0.100103258970954\\
630	-0.100097143668883\\
631	-0.100091390028943\\
632	-0.100085976720125\\
633	-0.100080883662873\\
634	-0.100076091956436\\
635	-0.100071583810354\\
636	-0.100067342479839\\
637	-0.100063352204843\\
638	-0.10005959815261\\
639	-0.100056066363533\\
640	-0.10005274370011\\
641	-0.10004961779886\\
642	-0.100046677025002\\
643	-0.100043910429763\\
644	-0.10004130771016\\
645	-0.100038859171108\\
646	-0.100036555689731\\
647	-0.100034388681749\\
648	-0.100032350069814\\
649	-0.100030432253704\\
650	-0.10002862808224\\
651	-0.100026930826855\\
652	-0.100025334156706\\
653	-0.100023832115236\\
654	-0.100022419098114\\
655	-0.10002108983247\\
656	-0.100019839357338\\
657	-0.100018663005257\\
658	-0.100017556384945\\
659	-0.100016515364993\\
660	-0.10001553605852\\
661	-0.100014614808727\\
662	-0.100013748175304\\
663	-0.100012932921633\\
664	-0.10001216600275\\
665	-0.100011444554013\\
666	-0.100010765880439\\
667	-0.100010127446671\\
668	-0.100009526867532\\
669	-0.100008961899141\\
670	-0.100008430430557\\
671	-0.100007930475902\\
672	-0.100007460166969\\
673	-0.100007017746249\\
674	-0.10000660156038\\
675	-0.100006210053981\\
676	-0.100005841763849\\
677	-0.100005495313503\\
678	-0.100005169408042\\
679	-0.10000486282932\\
680	-0.100004574431392\\
681	-0.100004303136246\\
682	-0.100004047929767\\
683	-0.100003807857961\\
684	-0.100003582023389\\
685	-0.100003369581814\\
686	-0.100003169739051\\
687	-0.100002981747999\\
688	-0.100002804905854\\
689	-0.100002638551477\\
690	-0.100002482062933\\
691	-0.100002334855159\\
692	-0.100002196377786\\
693	-0.100002066113075\\
694	-0.100001943573989\\
695	-0.10000182830237\\
696	-0.100001719867228\\
697	-0.100001617863127\\
698	-0.100001521908677\\
699	-0.100001431645101\\
700	-0.100001346734898\\
701	-0.100001266860582\\
702	-0.100001191723495\\
703	-0.100001121042687\\
704	-0.100001054553872\\
705	-0.0155700480917355\\
706	0.252262215253552\\
707	0.640062590007399\\
708	1.07760786640881\\
709	1.50523194128787\\
710	1.93577505059502\\
711	2.26473650259237\\
712	2.58323619670949\\
713	2.83709316252282\\
714	3.09015272936012\\
715	3.31582475288716\\
716	3.54352602375018\\
717	3.76212855365971\\
718	3.97518580999551\\
719	4.18296688488507\\
720	4.38172838461817\\
721	4.56747100872797\\
722	4.73710453280942\\
723	4.88903758317553\\
724	5.02279941343009\\
725	5.13873045751939\\
726	5.23463925334622\\
727	5.30728533780463\\
728	5.35453978884249\\
729	5.37615786045763\\
730	5.37373896169899\\
731	5.35036689831476\\
732	5.31019342600423\\
733	5.25804999584457\\
734	5.19907516592946\\
735	5.13832142701563\\
736	5.08033624530344\\
737	5.02876844604673\\
738	4.98608998764225\\
739	4.95350794186672\\
740	4.93108368477396\\
741	4.91800334965706\\
742	4.91290696564703\\
743	4.91419194927124\\
744	4.92024131916135\\
745	4.92956595324775\\
746	4.94087362850287\\
747	4.95309015871367\\
748	4.96535395652064\\
749	4.97699999587956\\
750	4.98754053621789\\
751	4.99664585251414\\
752	5.00412505077949\\
753	5.0099064834505\\
754	5.01401750152073\\
755	5.01656451394687\\
756	5.01771469176129\\
757	5.0176783921697\\
758	5.01669174549448\\
759	5.0149797243038\\
760	5.01275941454\\
761	5.01024258128369\\
762	5.00762866147226\\
763	5.00509272472498\\
764	5.00277160349819\\
765	5.00076641648448\\
766	4.99913383831386\\
767	4.9978965234581\\
768	4.9970444874691\\
769	4.99654669659278\\
770	4.99635592299458\\
771	4.99641712063658\\
772	4.9966722053924\\
773	4.99706427159761\\
774	4.99754083124797\\
775	4.99805596922063\\
776	4.99857154074094\\
777	4.99905761429363\\
778	4.99949236910295\\
779	4.99986158749935\\
780	5.00015800471695\\
781	5.00038037897742\\
782	5.00053227424308\\
783	5.0006207223329\\
784	5.00065494013794\\
785	5.00064521485879\\
786	5.00060200463702\\
787	5.00053525931189\\
788	5.00045394564269\\
789	5.00036575415891\\
790	5.00027696225809\\
791	5.0001924251667\\
792	5.00011566207311\\
793	5.00004900208576\\
794	4.99999375670632\\
795	4.99995039384615\\
796	4.99991870007971\\
797	4.99989792807002\\
798	4.99988693174255\\
799	4.99988429240699\\
800	4.99988843682719\\
801	4.99989774521894\\
802	4.99991064541218\\
803	4.99992568925232\\
804	4.99994160864758\\
805	4.9999573506115\\
806	4.99997209261501\\
807	4.99998524105088\\
808	4.99999641644778\\
809	5.00000542921758\\
810	5.00001224940718\\
811	5.00001697361062\\
812	5.00001979179025\\
813	5.00002095716289\\
814	5.0000207588453\\
815	5.00001949773168\\
816	5.00001746672275\\
817	5.00001493607118\\
818	5.00001214395139\\
819	5.00000929098591\\
820	5.0000065384537\\
821	5.00000400870827\\
822	5.00000178748645\\
823	4.99999992723638\\
824	4.99999845131804\\
825	4.99999735863864\\
826	4.99999662855057\\
827	4.99999622576973\\
828	4.99999610508226\\
829	4.99999621564772\\
830	4.99999650475718\\
831	4.99999692095507\\
832	4.99999741648138\\
833	4.99999794903516\\
834	4.99999848289093\\
835	4.99999898943879\\
836	4.99999944724551\\
837	4.99999984173889\\
838	5.00000016460914\\
839	5.00000041301092\\
840	5.00000058864006\\
841	5.00000069675224\\
842	5.00000074518247\\
843	5.00000074341576\\
844	5.00000070174819\\
845	5.00000063056634\\
846	5.00000053976177\\
847	5.00000043828722\\
848	5.00000033385238\\
849	5.00000023274997\\
850	5.000000139797\\
851	5.00000005837182\\
852	4.9999999905254\\
853	4.99999993714436\\
854	4.99999989814515\\
855	4.99999987268093\\
856	4.99999985934665\\
857	4.99999985637098\\
858	4.99999986178756\\
859	4.99999987358099\\
860	4.99999988980502\\
861	4.9999999086728\\
862	4.99999992862026\\
863	4.9999999483447\\
864	4.99999996682169\\
865	4.9999999833037\\
866	4.99999999730421\\
867	5.00000000857097\\
868	5.00000001705197\\
869	5.00000002285748\\
870	5.00000002622103\\
871	5.00000002746156\\
872	5.00000002694861\\
873	5.00000002507174\\
874	5.0000000222151\\
875	5.00000001873732\\
876	5.00000001495707\\
877	5.00000001114366\\
878	5.00000000751242\\
879	5.00000000422404\\
880	5.00000000138704\\
881	4.99999999906269\\
882	4.99999999727144\\
883	4.99999999600037\\
884	4.99999999521077\\
885	4.9999999948457\\
886	4.99999999483687\\
887	4.99999999511075\\
888	4.99999999559368\\
889	4.99999999621593\\
890	4.99999999691469\\
891	4.99999999763603\\
892	4.99999999833604\\
893	4.99999999898111\\
894	4.99999999954761\\
895	5.00000000002112\\
896	5.00000000039524\\
897	5.00000000067033\\
898	5.00000000085203\\
899	5.00000000094986\\
900	5.00000000097594\\
901	5.00000000094372\\
902	5.00000000086705\\
903	5.00000000075929\\
904	5.00000000063272\\
905	5.0000000004981\\
906	5.00000000036438\\
907	5.00000000023863\\
908	5.000000000126\\
909	5.00000000002987\\
910	4.99999999995205\\
911	4.99999999989294\\
912	4.99999999985187\\
913	4.99999999982733\\
914	4.99999999981723\\
915	4.99999999981911\\
916	4.99999999983039\\
917	4.99999999984851\\
918	4.99999999987108\\
919	4.99999999989594\\
920	4.99999999992129\\
921	4.99999999994565\\
922	4.9999999999679\\
923	4.99999999998729\\
924	5.00000000000335\\
925	5.00000000001591\\
926	5.00000000002502\\
927	5.00000000003089\\
928	5.00000000003387\\
929	5.00000000003441\\
930	5.00000000003298\\
931	5.00000000003006\\
932	5.00000000002612\\
933	5.00000000002158\\
934	5.00000000001681\\
935	5.00000000001212\\
936	5.00000000000774\\
937	5.00000000000385\\
938	5.00000000000055\\
939	4.9999999999979\\
940	4.99999999999592\\
941	4.99999999999456\\
942	4.99999999999378\\
943	4.99999999999349\\
944	4.99999999999362\\
945	4.99999999999406\\
946	4.99999999999473\\
947	4.99999999999555\\
948	4.99999999999644\\
949	4.99999999999734\\
950	4.9999999999982\\
951	4.99999999999898\\
952	4.99999999999966\\
953	5.00000000000021\\
954	5.00000000000064\\
955	5.00000000000094\\
956	5.00000000000113\\
957	5.00000000000122\\
958	5.00000000000123\\
959	5.00000000000117\\
960	5.00000000000106\\
961	5.00000000000091\\
962	5.00000000000075\\
963	5.00000000000057\\
964	5.00000000000041\\
965	5.00000000000025\\
966	5.00000000000011\\
967	5\\
968	4.99999999999991\\
969	4.99999999999984\\
970	4.99999999999979\\
971	4.99999999999977\\
972	4.99999999999977\\
973	4.99999999999977\\
974	4.99999999999979\\
975	4.99999999999982\\
976	4.99999999999985\\
977	4.99999999999988\\
978	4.99999999999992\\
979	4.99999999999994\\
980	4.99999999999997\\
981	4.99999999999999\\
982	5.00000000000001\\
983	5.00000000000002\\
984	5.00000000000002\\
985	5.00000000000003\\
986	5.00000000000003\\
987	5.00000000000003\\
988	5.00000000000003\\
989	5.00000000000003\\
990	5.00000000000003\\
991	5.00000000000003\\
992	5.00000000000003\\
993	5.00000000000002\\
994	5.00000000000002\\
995	5.00000000000002\\
996	5.00000000000001\\
997	5.00000000000001\\
998	5.00000000000001\\
999	5\\
1000	5\\
1001	5\\
1002	5\\
1003	5\\
1004	5\\
1005	5\\
1006	5\\
1007	5\\
1008	5\\
1009	5\\
1010	5\\
1011	5\\
1012	5\\
1013	5\\
1014	5\\
1015	5\\
1016	5\\
1017	5\\
1018	5\\
1019	5\\
1020	5\\
1021	5\\
1022	5\\
1023	5\\
1024	5\\
1025	5\\
1026	5\\
1027	5\\
1028	5\\
1029	5\\
1030	5\\
1031	5\\
1032	5\\
1033	5\\
1034	5\\
1035	5\\
1036	5\\
1037	5\\
1038	5\\
1039	5\\
1040	5\\
1041	5\\
1042	5\\
1043	5\\
1044	5\\
1045	5\\
1046	5\\
1047	5\\
1048	5\\
1049	5\\
1050	5\\
1051	5\\
1052	5\\
1053	5\\
1054	5\\
1055	5\\
1056	5\\
1057	5\\
1058	5\\
1059	5\\
1060	5\\
1061	5\\
1062	5\\
1063	5\\
1064	5\\
1065	5\\
1066	5\\
1067	5\\
1068	5\\
1069	5\\
1070	5\\
1071	5\\
1072	5\\
1073	5\\
1074	5\\
1075	5\\
1076	5\\
1077	5\\
1078	5\\
1079	5\\
1080	5\\
1081	5\\
1082	5\\
1083	5\\
1084	5\\
1085	5.00000000000001\\
1086	5.00000000000001\\
1087	5.00000000000001\\
1088	5.00000000000001\\
1089	5\\
1090	5\\
1091	5\\
1092	5\\
1093	5\\
1094	5\\
1095	5\\
1096	5\\
1097	5\\
1098	5\\
1099	5\\
1100	5\\
1101	5\\
1102	5\\
1103	5\\
1104	5\\
1105	5\\
1106	5\\
1107	5\\
1108	5\\
1109	5\\
1110	5\\
1111	5\\
1112	5\\
1113	5\\
1114	5\\
1115	5\\
1116	5\\
1117	5\\
1118	5\\
1119	5.00000000000001\\
1120	5.00000000000001\\
1121	5.00000000000001\\
1122	5.00000000000001\\
1123	5\\
1124	5\\
1125	5\\
1126	5\\
1127	5\\
1128	5\\
1129	5\\
1130	5\\
1131	5\\
1132	5\\
1133	5\\
1134	5\\
1135	5\\
1136	5\\
1137	5\\
1138	5\\
1139	5\\
1140	5\\
1141	5\\
1142	5\\
1143	5\\
1144	5\\
1145	5\\
1146	5\\
1147	5\\
1148	5\\
1149	5\\
1150	5\\
1151	5\\
1152	5\\
1153	5\\
1154	5\\
1155	5\\
1156	5\\
1157	5.00000000000001\\
1158	5.00000000000001\\
1159	5.00000000000001\\
1160	5.00000000000001\\
1161	5\\
1162	5\\
1163	5\\
1164	5\\
1165	5\\
1166	5\\
1167	5\\
1168	5\\
1169	5\\
1170	5\\
1171	5\\
1172	5\\
1173	5\\
1174	5\\
1175	5\\
1176	5\\
1177	5\\
1178	5\\
1179	5\\
1180	5\\
1181	5\\
1182	5\\
1183	5\\
1184	5\\
1185	5\\
1186	5\\
1187	5\\
1188	5\\
1189	5\\
1190	5\\
1191	5.00000000000001\\
1192	5.00000000000001\\
1193	5.00000000000001\\
1194	5.00000000000001\\
1195	5\\
1196	5\\
1197	5\\
1198	5\\
1199	5\\
1200	5\\
};
\addlegendentry{$y(k)$}

\addplot[const plot, color=mycolor2, dashed] table[row sep=crcr] {%
1	0\\
2	0\\
3	0\\
4	0\\
5	0\\
6	0\\
7	0\\
8	0\\
9	0\\
10	0\\
11	0\\
12	0\\
13	0\\
14	0\\
15	0\\
16	0\\
17	0\\
18	0\\
19	0\\
20	0\\
21	0\\
22	0\\
23	0\\
24	0\\
25	0\\
26	0\\
27	0\\
28	0\\
29	0\\
30	0\\
31	0\\
32	0\\
33	0\\
34	0\\
35	0\\
36	0\\
37	0\\
38	0\\
39	0\\
40	0\\
41	0\\
42	0\\
43	0\\
44	0\\
45	0\\
46	0\\
47	0\\
48	0\\
49	0\\
50	0\\
51	0\\
52	0\\
53	0\\
54	0\\
55	0\\
56	0\\
57	0\\
58	0\\
59	0\\
60	0\\
61	0\\
62	0\\
63	0\\
64	0\\
65	0\\
66	0\\
67	0\\
68	0\\
69	0\\
70	0\\
71	0\\
72	0\\
73	0\\
74	0\\
75	0\\
76	0\\
77	0\\
78	0\\
79	0\\
80	0\\
81	0\\
82	0\\
83	0\\
84	0\\
85	0\\
86	0\\
87	0\\
88	0\\
89	0\\
90	0\\
91	0\\
92	0\\
93	0\\
94	0\\
95	0\\
96	0\\
97	0\\
98	0\\
99	0\\
100	0.5\\
101	0.5\\
102	0.5\\
103	0.5\\
104	0.5\\
105	0.5\\
106	0.5\\
107	0.5\\
108	0.5\\
109	0.5\\
110	0.5\\
111	0.5\\
112	0.5\\
113	0.5\\
114	0.5\\
115	0.5\\
116	0.5\\
117	0.5\\
118	0.5\\
119	0.5\\
120	0.5\\
121	0.5\\
122	0.5\\
123	0.5\\
124	0.5\\
125	0.5\\
126	0.5\\
127	0.5\\
128	0.5\\
129	0.5\\
130	0.5\\
131	0.5\\
132	0.5\\
133	0.5\\
134	0.5\\
135	0.5\\
136	0.5\\
137	0.5\\
138	0.5\\
139	0.5\\
140	0.5\\
141	0.5\\
142	0.5\\
143	0.5\\
144	0.5\\
145	0.5\\
146	0.5\\
147	0.5\\
148	0.5\\
149	0.5\\
150	0.5\\
151	0.5\\
152	0.5\\
153	0.5\\
154	0.5\\
155	0.5\\
156	0.5\\
157	0.5\\
158	0.5\\
159	0.5\\
160	0.5\\
161	0.5\\
162	0.5\\
163	0.5\\
164	0.5\\
165	0.5\\
166	0.5\\
167	0.5\\
168	0.5\\
169	0.5\\
170	0.5\\
171	0.5\\
172	0.5\\
173	0.5\\
174	0.5\\
175	0.5\\
176	0.5\\
177	0.5\\
178	0.5\\
179	0.5\\
180	0.5\\
181	0.5\\
182	0.5\\
183	0.5\\
184	0.5\\
185	0.5\\
186	0.5\\
187	0.5\\
188	0.5\\
189	0.5\\
190	0.5\\
191	0.5\\
192	0.5\\
193	0.5\\
194	0.5\\
195	0.5\\
196	0.5\\
197	0.5\\
198	0.5\\
199	0.5\\
200	0.5\\
201	0.5\\
202	0.5\\
203	0.5\\
204	0.5\\
205	0.5\\
206	0.5\\
207	0.5\\
208	0.5\\
209	0.5\\
210	0.5\\
211	0.5\\
212	0.5\\
213	0.5\\
214	0.5\\
215	0.5\\
216	0.5\\
217	0.5\\
218	0.5\\
219	0.5\\
220	0.5\\
221	0.5\\
222	0.5\\
223	0.5\\
224	0.5\\
225	0.5\\
226	0.5\\
227	0.5\\
228	0.5\\
229	0.5\\
230	0.5\\
231	0.5\\
232	0.5\\
233	0.5\\
234	0.5\\
235	0.5\\
236	0.5\\
237	0.5\\
238	0.5\\
239	0.5\\
240	0.5\\
241	0.5\\
242	0.5\\
243	0.5\\
244	0.5\\
245	0.5\\
246	0.5\\
247	0.5\\
248	0.5\\
249	0.5\\
250	0.5\\
251	0.5\\
252	0.5\\
253	0.5\\
254	0.5\\
255	0.5\\
256	0.5\\
257	0.5\\
258	0.5\\
259	0.5\\
260	0.5\\
261	0.5\\
262	0.5\\
263	0.5\\
264	0.5\\
265	0.5\\
266	0.5\\
267	0.5\\
268	0.5\\
269	0.5\\
270	0.5\\
271	0.5\\
272	0.5\\
273	0.5\\
274	0.5\\
275	0.5\\
276	0.5\\
277	0.5\\
278	0.5\\
279	0.5\\
280	0.5\\
281	0.5\\
282	0.5\\
283	0.5\\
284	0.5\\
285	0.5\\
286	0.5\\
287	0.5\\
288	0.5\\
289	0.5\\
290	0.5\\
291	0.5\\
292	0.5\\
293	0.5\\
294	0.5\\
295	0.5\\
296	0.5\\
297	0.5\\
298	0.5\\
299	0.5\\
300	1\\
301	1\\
302	1\\
303	1\\
304	1\\
305	1\\
306	1\\
307	1\\
308	1\\
309	1\\
310	1\\
311	1\\
312	1\\
313	1\\
314	1\\
315	1\\
316	1\\
317	1\\
318	1\\
319	1\\
320	1\\
321	1\\
322	1\\
323	1\\
324	1\\
325	1\\
326	1\\
327	1\\
328	1\\
329	1\\
330	1\\
331	1\\
332	1\\
333	1\\
334	1\\
335	1\\
336	1\\
337	1\\
338	1\\
339	1\\
340	1\\
341	1\\
342	1\\
343	1\\
344	1\\
345	1\\
346	1\\
347	1\\
348	1\\
349	1\\
350	1\\
351	1\\
352	1\\
353	1\\
354	1\\
355	1\\
356	1\\
357	1\\
358	1\\
359	1\\
360	1\\
361	1\\
362	1\\
363	1\\
364	1\\
365	1\\
366	1\\
367	1\\
368	1\\
369	1\\
370	1\\
371	1\\
372	1\\
373	1\\
374	1\\
375	1\\
376	1\\
377	1\\
378	1\\
379	1\\
380	1\\
381	1\\
382	1\\
383	1\\
384	1\\
385	1\\
386	1\\
387	1\\
388	1\\
389	1\\
390	1\\
391	1\\
392	1\\
393	1\\
394	1\\
395	1\\
396	1\\
397	1\\
398	1\\
399	1\\
400	1\\
401	1\\
402	1\\
403	1\\
404	1\\
405	1\\
406	1\\
407	1\\
408	1\\
409	1\\
410	1\\
411	1\\
412	1\\
413	1\\
414	1\\
415	1\\
416	1\\
417	1\\
418	1\\
419	1\\
420	1\\
421	1\\
422	1\\
423	1\\
424	1\\
425	1\\
426	1\\
427	1\\
428	1\\
429	1\\
430	1\\
431	1\\
432	1\\
433	1\\
434	1\\
435	1\\
436	1\\
437	1\\
438	1\\
439	1\\
440	1\\
441	1\\
442	1\\
443	1\\
444	1\\
445	1\\
446	1\\
447	1\\
448	1\\
449	1\\
450	1\\
451	1\\
452	1\\
453	1\\
454	1\\
455	1\\
456	1\\
457	1\\
458	1\\
459	1\\
460	1\\
461	1\\
462	1\\
463	1\\
464	1\\
465	1\\
466	1\\
467	1\\
468	1\\
469	1\\
470	1\\
471	1\\
472	1\\
473	1\\
474	1\\
475	1\\
476	1\\
477	1\\
478	1\\
479	1\\
480	1\\
481	1\\
482	1\\
483	1\\
484	1\\
485	1\\
486	1\\
487	1\\
488	1\\
489	1\\
490	1\\
491	1\\
492	1\\
493	1\\
494	1\\
495	1\\
496	1\\
497	1\\
498	1\\
499	1\\
500	-0.1\\
501	-0.1\\
502	-0.1\\
503	-0.1\\
504	-0.1\\
505	-0.1\\
506	-0.1\\
507	-0.1\\
508	-0.1\\
509	-0.1\\
510	-0.1\\
511	-0.1\\
512	-0.1\\
513	-0.1\\
514	-0.1\\
515	-0.1\\
516	-0.1\\
517	-0.1\\
518	-0.1\\
519	-0.1\\
520	-0.1\\
521	-0.1\\
522	-0.1\\
523	-0.1\\
524	-0.1\\
525	-0.1\\
526	-0.1\\
527	-0.1\\
528	-0.1\\
529	-0.1\\
530	-0.1\\
531	-0.1\\
532	-0.1\\
533	-0.1\\
534	-0.1\\
535	-0.1\\
536	-0.1\\
537	-0.1\\
538	-0.1\\
539	-0.1\\
540	-0.1\\
541	-0.1\\
542	-0.1\\
543	-0.1\\
544	-0.1\\
545	-0.1\\
546	-0.1\\
547	-0.1\\
548	-0.1\\
549	-0.1\\
550	-0.1\\
551	-0.1\\
552	-0.1\\
553	-0.1\\
554	-0.1\\
555	-0.1\\
556	-0.1\\
557	-0.1\\
558	-0.1\\
559	-0.1\\
560	-0.1\\
561	-0.1\\
562	-0.1\\
563	-0.1\\
564	-0.1\\
565	-0.1\\
566	-0.1\\
567	-0.1\\
568	-0.1\\
569	-0.1\\
570	-0.1\\
571	-0.1\\
572	-0.1\\
573	-0.1\\
574	-0.1\\
575	-0.1\\
576	-0.1\\
577	-0.1\\
578	-0.1\\
579	-0.1\\
580	-0.1\\
581	-0.1\\
582	-0.1\\
583	-0.1\\
584	-0.1\\
585	-0.1\\
586	-0.1\\
587	-0.1\\
588	-0.1\\
589	-0.1\\
590	-0.1\\
591	-0.1\\
592	-0.1\\
593	-0.1\\
594	-0.1\\
595	-0.1\\
596	-0.1\\
597	-0.1\\
598	-0.1\\
599	-0.1\\
600	-0.1\\
601	-0.1\\
602	-0.1\\
603	-0.1\\
604	-0.1\\
605	-0.1\\
606	-0.1\\
607	-0.1\\
608	-0.1\\
609	-0.1\\
610	-0.1\\
611	-0.1\\
612	-0.1\\
613	-0.1\\
614	-0.1\\
615	-0.1\\
616	-0.1\\
617	-0.1\\
618	-0.1\\
619	-0.1\\
620	-0.1\\
621	-0.1\\
622	-0.1\\
623	-0.1\\
624	-0.1\\
625	-0.1\\
626	-0.1\\
627	-0.1\\
628	-0.1\\
629	-0.1\\
630	-0.1\\
631	-0.1\\
632	-0.1\\
633	-0.1\\
634	-0.1\\
635	-0.1\\
636	-0.1\\
637	-0.1\\
638	-0.1\\
639	-0.1\\
640	-0.1\\
641	-0.1\\
642	-0.1\\
643	-0.1\\
644	-0.1\\
645	-0.1\\
646	-0.1\\
647	-0.1\\
648	-0.1\\
649	-0.1\\
650	-0.1\\
651	-0.1\\
652	-0.1\\
653	-0.1\\
654	-0.1\\
655	-0.1\\
656	-0.1\\
657	-0.1\\
658	-0.1\\
659	-0.1\\
660	-0.1\\
661	-0.1\\
662	-0.1\\
663	-0.1\\
664	-0.1\\
665	-0.1\\
666	-0.1\\
667	-0.1\\
668	-0.1\\
669	-0.1\\
670	-0.1\\
671	-0.1\\
672	-0.1\\
673	-0.1\\
674	-0.1\\
675	-0.1\\
676	-0.1\\
677	-0.1\\
678	-0.1\\
679	-0.1\\
680	-0.1\\
681	-0.1\\
682	-0.1\\
683	-0.1\\
684	-0.1\\
685	-0.1\\
686	-0.1\\
687	-0.1\\
688	-0.1\\
689	-0.1\\
690	-0.1\\
691	-0.1\\
692	-0.1\\
693	-0.1\\
694	-0.1\\
695	-0.1\\
696	-0.1\\
697	-0.1\\
698	-0.1\\
699	-0.1\\
700	5\\
701	5\\
702	5\\
703	5\\
704	5\\
705	5\\
706	5\\
707	5\\
708	5\\
709	5\\
710	5\\
711	5\\
712	5\\
713	5\\
714	5\\
715	5\\
716	5\\
717	5\\
718	5\\
719	5\\
720	5\\
721	5\\
722	5\\
723	5\\
724	5\\
725	5\\
726	5\\
727	5\\
728	5\\
729	5\\
730	5\\
731	5\\
732	5\\
733	5\\
734	5\\
735	5\\
736	5\\
737	5\\
738	5\\
739	5\\
740	5\\
741	5\\
742	5\\
743	5\\
744	5\\
745	5\\
746	5\\
747	5\\
748	5\\
749	5\\
750	5\\
751	5\\
752	5\\
753	5\\
754	5\\
755	5\\
756	5\\
757	5\\
758	5\\
759	5\\
760	5\\
761	5\\
762	5\\
763	5\\
764	5\\
765	5\\
766	5\\
767	5\\
768	5\\
769	5\\
770	5\\
771	5\\
772	5\\
773	5\\
774	5\\
775	5\\
776	5\\
777	5\\
778	5\\
779	5\\
780	5\\
781	5\\
782	5\\
783	5\\
784	5\\
785	5\\
786	5\\
787	5\\
788	5\\
789	5\\
790	5\\
791	5\\
792	5\\
793	5\\
794	5\\
795	5\\
796	5\\
797	5\\
798	5\\
799	5\\
800	5\\
801	5\\
802	5\\
803	5\\
804	5\\
805	5\\
806	5\\
807	5\\
808	5\\
809	5\\
810	5\\
811	5\\
812	5\\
813	5\\
814	5\\
815	5\\
816	5\\
817	5\\
818	5\\
819	5\\
820	5\\
821	5\\
822	5\\
823	5\\
824	5\\
825	5\\
826	5\\
827	5\\
828	5\\
829	5\\
830	5\\
831	5\\
832	5\\
833	5\\
834	5\\
835	5\\
836	5\\
837	5\\
838	5\\
839	5\\
840	5\\
841	5\\
842	5\\
843	5\\
844	5\\
845	5\\
846	5\\
847	5\\
848	5\\
849	5\\
850	5\\
851	5\\
852	5\\
853	5\\
854	5\\
855	5\\
856	5\\
857	5\\
858	5\\
859	5\\
860	5\\
861	5\\
862	5\\
863	5\\
864	5\\
865	5\\
866	5\\
867	5\\
868	5\\
869	5\\
870	5\\
871	5\\
872	5\\
873	5\\
874	5\\
875	5\\
876	5\\
877	5\\
878	5\\
879	5\\
880	5\\
881	5\\
882	5\\
883	5\\
884	5\\
885	5\\
886	5\\
887	5\\
888	5\\
889	5\\
890	5\\
891	5\\
892	5\\
893	5\\
894	5\\
895	5\\
896	5\\
897	5\\
898	5\\
899	5\\
900	5\\
901	5\\
902	5\\
903	5\\
904	5\\
905	5\\
906	5\\
907	5\\
908	5\\
909	5\\
910	5\\
911	5\\
912	5\\
913	5\\
914	5\\
915	5\\
916	5\\
917	5\\
918	5\\
919	5\\
920	5\\
921	5\\
922	5\\
923	5\\
924	5\\
925	5\\
926	5\\
927	5\\
928	5\\
929	5\\
930	5\\
931	5\\
932	5\\
933	5\\
934	5\\
935	5\\
936	5\\
937	5\\
938	5\\
939	5\\
940	5\\
941	5\\
942	5\\
943	5\\
944	5\\
945	5\\
946	5\\
947	5\\
948	5\\
949	5\\
950	5\\
951	5\\
952	5\\
953	5\\
954	5\\
955	5\\
956	5\\
957	5\\
958	5\\
959	5\\
960	5\\
961	5\\
962	5\\
963	5\\
964	5\\
965	5\\
966	5\\
967	5\\
968	5\\
969	5\\
970	5\\
971	5\\
972	5\\
973	5\\
974	5\\
975	5\\
976	5\\
977	5\\
978	5\\
979	5\\
980	5\\
981	5\\
982	5\\
983	5\\
984	5\\
985	5\\
986	5\\
987	5\\
988	5\\
989	5\\
990	5\\
991	5\\
992	5\\
993	5\\
994	5\\
995	5\\
996	5\\
997	5\\
998	5\\
999	5\\
1000	5\\
1001	5\\
1002	5\\
1003	5\\
1004	5\\
1005	5\\
1006	5\\
1007	5\\
1008	5\\
1009	5\\
1010	5\\
1011	5\\
1012	5\\
1013	5\\
1014	5\\
1015	5\\
1016	5\\
1017	5\\
1018	5\\
1019	5\\
1020	5\\
1021	5\\
1022	5\\
1023	5\\
1024	5\\
1025	5\\
1026	5\\
1027	5\\
1028	5\\
1029	5\\
1030	5\\
1031	5\\
1032	5\\
1033	5\\
1034	5\\
1035	5\\
1036	5\\
1037	5\\
1038	5\\
1039	5\\
1040	5\\
1041	5\\
1042	5\\
1043	5\\
1044	5\\
1045	5\\
1046	5\\
1047	5\\
1048	5\\
1049	5\\
1050	5\\
1051	5\\
1052	5\\
1053	5\\
1054	5\\
1055	5\\
1056	5\\
1057	5\\
1058	5\\
1059	5\\
1060	5\\
1061	5\\
1062	5\\
1063	5\\
1064	5\\
1065	5\\
1066	5\\
1067	5\\
1068	5\\
1069	5\\
1070	5\\
1071	5\\
1072	5\\
1073	5\\
1074	5\\
1075	5\\
1076	5\\
1077	5\\
1078	5\\
1079	5\\
1080	5\\
1081	5\\
1082	5\\
1083	5\\
1084	5\\
1085	5\\
1086	5\\
1087	5\\
1088	5\\
1089	5\\
1090	5\\
1091	5\\
1092	5\\
1093	5\\
1094	5\\
1095	5\\
1096	5\\
1097	5\\
1098	5\\
1099	5\\
1100	5\\
1101	5\\
1102	5\\
1103	5\\
1104	5\\
1105	5\\
1106	5\\
1107	5\\
1108	5\\
1109	5\\
1110	5\\
1111	5\\
1112	5\\
1113	5\\
1114	5\\
1115	5\\
1116	5\\
1117	5\\
1118	5\\
1119	5\\
1120	5\\
1121	5\\
1122	5\\
1123	5\\
1124	5\\
1125	5\\
1126	5\\
1127	5\\
1128	5\\
1129	5\\
1130	5\\
1131	5\\
1132	5\\
1133	5\\
1134	5\\
1135	5\\
1136	5\\
1137	5\\
1138	5\\
1139	5\\
1140	5\\
1141	5\\
1142	5\\
1143	5\\
1144	5\\
1145	5\\
1146	5\\
1147	5\\
1148	5\\
1149	5\\
1150	5\\
1151	5\\
1152	5\\
1153	5\\
1154	5\\
1155	5\\
1156	5\\
1157	5\\
1158	5\\
1159	5\\
1160	5\\
1161	5\\
1162	5\\
1163	5\\
1164	5\\
1165	5\\
1166	5\\
1167	5\\
1168	5\\
1169	5\\
1170	5\\
1171	5\\
1172	5\\
1173	5\\
1174	5\\
1175	5\\
1176	5\\
1177	5\\
1178	5\\
1179	5\\
1180	5\\
1181	5\\
1182	5\\
1183	5\\
1184	5\\
1185	5\\
1186	5\\
1187	5\\
1188	5\\
1189	5\\
1190	5\\
1191	5\\
1192	5\\
1193	5\\
1194	5\\
1195	5\\
1196	5\\
1197	5\\
1198	5\\
1199	5\\
1200	5\\
};
\addlegendentry{$y^{zad}(k)$}

\end{axis}
\end{tikzpicture}%
	\caption{Przebiegi sygna��w dla rozmytego regulatora DMC z 4 regulatorami lokalnymi}
\end{figure}

\section{Wnioski}

Dobranie r�nych warto�ci parametru $\lambda$ dla ka�dego regulatora lokalnego sprawi�o, �e jako�� regulacji uleg�a poprawie.