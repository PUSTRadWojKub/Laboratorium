\chapter{Odpowied� skokowa dla algorytmu DMC}

\section{Odpowied� skokowa}
Do wyznaczania odpowiedzi skokowej dla algorytmu DMC wybrana zosta�a odpowied� dla zmiany sygna�u steruj�cego o $\num{0.1}$ z punktu pracy $U_{\mathrm{pp}}$=1,1. Otrzymana odpowied� skokowa poddana zosta�a normalizacji, czyli przesuni�ciu o warto�� sygna�u wyj�ciowego w punkcie pracy, a nast�pnie podzielona przez d�ugo�� skoku. Nast�pnie, w celu wyznaczenia wsp�czynik�w odpowiedzi skokowej dla algorytmu DMC zastosowany zosta� wz�r:

\begin{equation}
S_i = \frac{S_i^0(k) - Y_{\mathrm{pp}}}{\Delta U}, \textrm{dla } i=1,\ldots
\end{equation}
gdzie $S_i^0$ to seria pomiar�w pozyskanych w celu wyznaczenia odpowiedzi skokowej, za� wielko�� $\Delta U$ jest to przyrost warto�ci sygna�u steruj�cego.
Poni�ej przedstawiono gotow� odpowied� skokow� dla algorytmu DMC.

\begin{figure}[h!]
\centering
% This file was created by matlab2tikz.
%
%The latest updates can be retrieved from
%  http://www.mathworks.com/matlabcentral/fileexchange/22022-matlab2tikz-matlab2tikz
%where you can also make suggestions and rate matlab2tikz.
%
\definecolor{mycolor1}{rgb}{0.00000,0.44700,0.74100}%
%
\begin{tikzpicture}

\begin{axis}[%
width=4.521in,
height=3.566in,
at={(0.758in,0.481in)},
scale only axis,
xmin=20,
xmax=190,
ymin=0,
ymax=3.5,
axis background/.style={fill=white},
title style={font=\bfseries},
title={Odpowiedz skokowa},
legend style={at={(0.97,0.03)}, anchor=south east, legend cell align=left, align=left, draw=white!15!black}
]
\addplot [color=mycolor1]
  table[row sep=crcr]{%
20	0\\
21	0\\
22	0\\
23	0\\
24	0\\
25	0\\
26	0\\
27	0\\
28	0\\
29	0\\
30	0.0100919999999993\\
31	0.0382113076000002\\
32	0.0814165932162815\\
33	0.137124848838672\\
34	0.203072341492092\\
35	0.277279588002939\\
36	0.358019950175584\\
37	0.443791488655397\\
38	0.5332917491479\\
39	0.625395186639862\\
40	0.719132962154624\\
41	0.813674872666242\\
42	0.908313198363039\\
43	1.00244827273298\\
44	1.09557560016016\\
45	1.18727436307172\\
46	1.27719717633732\\
47	1.36506096076062\\
48	1.45063882026176\\
49	1.53375281886306\\
50	1.61426756397852\\
51	1.69208451187846\\
52	1.7671369196534\\
53	1.83938537562327\\
54	1.90881384701129\\
55	1.97542618989729\\
56	2.03924307204987\\
57	2.10029926426912\\
58	2.15864126040568\\
59	2.21432519030606\\
60	2.26741499361203\\
61	2.31798082565395\\
62	2.36609766965825\\
63	2.41184413217252\\
64	2.45530140102478\\
65	2.4965523473045\\
66	2.53568075480502\\
67	2.57277066212137\\
68	2.6079058041746\\
69	2.64116914135002\\
70	2.67264246570872\\
71	2.7024060748738\\
72	2.73053850521713\\
73	2.75711631689192\\
74	2.78221392408\\
75	2.80590346456174\\
76	2.82825470337774\\
77	2.84933496594383\\
78	2.86920909651125\\
79	2.88793943833751\\
80	2.90558583235768\\
81	2.92220563152391\\
82	2.93785372831932\\
83	2.95258259325368\\
84	2.96644232241702\\
85	2.97948069240665\\
86	2.99174322115641\\
87	3.00327323338576\\
88	3.01411192955533\\
89	3.0242984573643\\
90	3.03386998495753\\
91	3.04286177512761\\
92	3.05130725990049\\
93	3.05923811498485\\
94	3.06668433364605\\
95	3.07367429963646\\
96	3.08023485887626\\
97	3.08639138963371\\
98	3.09216787100135\\
99	3.09758694950663\\
100	3.10267000373184\\
101	3.10743720684933\\
102	3.11190758700589\\
103	3.11609908551312\\
104	3.12002861282104\\
105	3.12371210226942\\
106	3.12716456162566\\
107	3.13040012243068\\
108	3.13343208718429\\
109	3.13627297441013\\
110	3.1389345616472\\
111	3.14142792642066\\
112	3.14376348524881\\
113	3.14595103074683\\
114	3.14799976688998\\
115	3.14991834250105\\
116	3.15171488302765\\
117	3.1533970206755\\
118	3.15497192296388\\
119	3.15644631976903\\
120	3.15782652892047\\
121	3.15911848041427\\
122	3.16032773930577\\
123	3.16145952734335\\
124	3.16251874340241\\
125	3.16350998277786\\
126	3.16443755539083\\
127	3.16530550296394\\
128	3.16611761521731\\
129	3.1668774451355\\
130	3.16758832335371\\
131	3.16825337170954\\
132	3.16887551600461\\
133	3.16945749801859\\
134	3.17000188681603\\
135	3.17051108938477\\
136	3.17098736064276\\
137	3.17143281284845\\
138	3.17184942444812\\
139	3.17223904839191\\
140	3.1726034199488\\
141	3.17294416404898\\
142	3.17326280218108\\
143	3.17356075886967\\
144	3.1738393677576\\
145	3.17409987731633\\
146	3.17434345620577\\
147	3.17457119830476\\
148	3.17478412743119\\
149	3.1749832017706\\
150	3.17516931803049\\
151	3.17534331533673\\
152	3.17550597888773\\
153	3.17565804338086\\
154	3.17580019622492\\
155	3.17593308055168\\
156	3.17605729803879\\
157	3.1761734115554\\
158	3.17628194764153\\
159	3.17638339883129\\
160	3.17647822582962\\
161	3.17656685955154\\
162	3.17664970303244\\
163	3.17672713321741\\
164	3.1767995026371\\
165	3.17686714097719\\
166	3.17693035654806\\
167	3.17698943766103\\
168	3.17704465391679\\
169	3.17709625741176\\
170	3.17714448386738\\
171	3.17718955368723\\
172	3.17723167294655\\
173	3.17727103431837\\
174	3.1773078179403\\
175	3.17734219222572\\
176	3.17737431462289\\
177	3.17740433232527\\
178	3.17743238293622\\
179	3.17745859509083\\
180	3.17748308903779\\
181	3.17750597718377\\
182	3.17752736460259\\
183	3.17754734951162\\
184	3.17756602371744\\
185	3.1775834730326\\
186	3.17759977766556\\
187	3.1776150125853\\
188	3.17762924786238\\
189	3.17764254898794\\
190	3.17765497717195\\
};
\addlegendentry{y(k)}

\end{axis}

\begin{axis}[%
width=5.833in,
height=4.375in,
at={(0in,0in)},
scale only axis,
xmin=0,
xmax=1,
ymin=0,
ymax=1,
axis line style={draw=none},
ticks=none,
axis x line*=bottom,
axis y line*=left,
legend style={legend cell align=left, align=left, draw=white!15!black}
]
\end{axis}
\end{tikzpicture}%
\caption{Odpowied� skokowa dla algorytmu DMC}
\end{figure}

\section{Implementacja}
Implementacja fukcji wykorzystanych do wykonania zadania zawarte s� w skrypcie \verb+podpunkt_3_v1.m+.