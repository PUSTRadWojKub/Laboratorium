\chapter{Regulacja procesu}

\section{Regulator PID}
Regulator PID sk�ada si� trzech cz�on�w: proporcjonalnego, ca�kuj�cego oraz r�niczkuj�cego. Dzia�a on w p�tli sprz�enia zwrotnego, maj�c na celu zredukowanie uchybu (r�nicy mi�dzy warto�ci� zadan� a zmierzon� warto�ci� sygna�u wyj�ciowego procesu) poprzez odpowiedni� zmian� sygna�u steruj�cego. Dyskretny regulator PID ma posta�:

\begin{equation}
u(k)=r_2*e(k-2)+r_1*e(k-1)+r_0*e(k)+u(k-1)
\end{equation}
gdzie:

\begin{equation}
r_2=K*T_d/T_p
\end{equation}
\begin{equation}
r_1=K*(T_p/(2*T_i)-2*T_d/T_p-1)
\end{equation}
\begin{equation}
r_0=K*(1+T_p/(2*T_i)+T_d/T_p)
\end{equation}
gdzie $K$ - wzmocnienie cz�onu proporcjonalnego, $T_i$ - czas zdwojenia cz�onu ca�kuj�cego, $T_d$ - czas wyprzedzenia cz�onu r�niczkuj�cego, $T_p$ - okres pr�bkowania

\section{Regulator DMC}
Regulator DMC jest to regulator predykcyjny - dzia�a on z wyprzedzeniem, zanim nast�pi� zmiany warto�ci sygna�u wyj�ciowego. Wektor przyrost�w sterowa� dany jest wzorem:

\begin{equation}
\triangle U(k)=K(Y^{zad}(k)-Y^0(k)=Y(k)+M^P \triangle U^P(k))
\label{dU1}
\end{equation}
gdzie:

\begin{equation}
K=(M^TM+\lambda*I)^{-1}M^T
\label{K}
\end{equation}

\begin{equation}
\triangle\boldsymbol{U^P}(k)=\left[
\begin{array}{c}
\triangle u(k-1)\\
\vdots\\
\triangle u(k-(D-1))
\end{array}
\right]_{\mathrm{(D-1)x1}}
\label{dUPm}
\end{equation}

\begin{equation}
\boldsymbol{M}=\left[
\begin{array}
{cccc}
s_{1} & 0 & \ldots & 0\\
s_{2} & s_{1} & \ldots & 0\\
\vdots & \vdots & \ddots & \vdots\\
s_{N} & s_{N-1} & \ldots &  s_{N-N_{\mathrm{u}}+1}
\end{array}
\right]_{\mathrm{NxN_u}}
\label{Mm}
\end{equation}

\begin{equation}
\boldsymbol{M^P}=\left[
\begin{array}
{cccc}
s_{2}-s_{1} & s_{3}-s_{2} & \ldots & s_{D}-s_{D-1}\\
s_{3}-s_{1} & s_{4}-s_{2} & \ldots & s_{D+1}-s_{D-1}\\
\vdots & \vdots & \ddots & \vdots\\
s_{N+1}-s_{1} & s_{N+2}-s_{2} & \ldots &  s_{N+D-1}-S_{D-1}
\end{array}
\right]_{\mathrm{NxD-1}}
\label{MPm}
\end{equation}
gdzie $N$ - horyzont predykcji, $N_u$ - horyzont sterowania, $D$ - horyzont dynamiki, $\lambda$ - kara za zmian� sterowania
\bigbreak
W tym przypadku nale�y wyznaczy� tylko pierwszy element macierzy $\Delta U(k)$ czyli $\Delta u(k|k)$. Aktualne sterowanie uzyskuje si� poprzez zsumowanie $\Delta u(k|k)$ z poprzednim sterowaniem.

\section{Implementacja}
Implementacja fukcji wykorzystanych do wykonania zadania zawarte s� w skrypcie \verb+podpunkt_4_v1.m+.