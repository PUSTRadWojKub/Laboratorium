\chapter{Regulacja procesu}

\section{Regulator PID}
Regulator PID składa się z trzech członów: proporcjonalnego, całkującego oraz różniczkującego. Działa on w pętli sprzężenia zwrotnego, mając na celu zredukowanie uchybu \(różnicy między wartością zadaną a zmierzoną wartością sygnału wyjściowego procesu\) poprzez odpowiednią zmianę sygnału sterującego.
Dyskretny regulator PID ma postać:

\begin{equation}
u(t)=r_2*e(k-2)+r_1*e(k-1)+r_0*e(k)+u(k-1)
\end{equation}

\section{Regulator DMC}
Hej