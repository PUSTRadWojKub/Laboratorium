\chapter{Regulacja procesu}

\section{Regulator PID}
Regulator PID sk�ada si� trzech cz�on�w: proporcjonalnego, ca�kuj�cego oraz r�niczkuj�cego. Dzia�a on w p�tli sprz�enia zwrotnego, maj�c na celu zredukowanie uchybu (r�nicy mi�dzy warto�ci� zadan� a zmierzon� warto�ci� sygna�u wyj�ciowego procesu) poprzez odpowiedni� zmian� sygna�u steruj�cego. Dyskretny regulator PID ma posta�:

\begin{equation}
u(k)=r_2e(k-2)+r_1e(k-1)+r_0e(k)+u(k-1)
\end{equation}
gdzie:

\begin{equation}
r_2=K\frac{T_\mathrm{d}}{T_\mathrm{p}}
\end{equation}
\begin{equation}
r_1=K(\frac{T_\mathrm{p}}{2T_\mathrm{i}}-\frac{2T_\mathrm{d}}{T_\mathrm{p}}-1)
\end{equation}
\begin{equation}
r_0=K(1+\frac{T_\mathrm{p}}{2T_\mathrm{i}}+\frac{T_\mathrm{d}}{T_\mathrm{p}})
\end{equation}
gdzie $K$ - wzmocnienie cz�onu proporcjonalnego, $T_\mathrm{i}$ - czas zdwojenia cz�onu ca�kuj�cego, $T_\mathrm{d}$ - czas wyprzedzenia cz�onu r�niczkuj�cego, $T_\mathrm{p}$ - okres pr�bkowania

\section{Regulator DMC}
Regulator DMC jest to regulator predykcyjny - dzia�a on z wyprzedzeniem, zanim nast�pi� zmiany warto�ci sygna�u wyj�ciowego. Wektor przyrost�w sterowa� dany jest wzorem:

\begin{align}
\triangle U(k)&=\boldsymbol{K}[Y^{\mathrm{zad}}(k)-Y^0(k)]\\
&=\boldsymbol{K}[Y^{\mathrm{zad}}(k)+Y(k)+\boldsymbol{M}^\mathrm{P} \triangle U^\mathrm{P}(k)]
\label{dU1}
\end{align}
gdzie:

\begin{equation}
\boldsymbol{K}=(\boldsymbol{M}^\mathrm{T}\boldsymbol{M}+\lambda I)^{-1}\boldsymbol{M}^\mathrm{T}
\label{K}
\end{equation}

\begin{equation}
\triangle U^\mathrm{P}(k)=\left[
\begin{array}{c}
\triangle u(k-1)\\
\vdots\\
\triangle u(k-(D-1))
\end{array}
\right]_{(D-1)\times 1}
\label{dUPm}
\end{equation}

\begin{equation}
\boldsymbol{M}=\left[
\begin{array}
{cccc}
s_{1} & 0 & \ldots & 0\\
s_{2} & s_{1} & \ldots & 0\\
\vdots & \vdots & \ddots & \vdots\\
s_{N} & s_{N-1} & \ldots &  s_{N-N_{\mathrm{u}}+1}
\end{array}
\right]_{N\times N_u}
\label{Mm}
\end{equation}

\begin{equation}
\boldsymbol{M}^\mathrm{P}=\left[
\begin{array}
{cccc}
s_{2}-s_{1} & s_{3}-s_{2} & \ldots & s_{D}-s_{D-1}\\
s_{3}-s_{1} & s_{4}-s_{2} & \ldots & s_{D+1}-s_{D-1}\\
\vdots & \vdots & \ddots & \vdots\\
s_{N+1}-s_{1} & s_{N+2}-s_{2} & \ldots &  s_{N+D-1}-S_{D-1}
\end{array}
\right]_{N\times (D-1)}
\label{MPm}
\end{equation}
gdzie $N$ - horyzont predykcji, $N_\mathrm{u}$ - horyzont sterowania, $D$ - horyzont dynamiki, $\lambda$ - kara za zmian� sterowania
\bigbreak
W tym przypadku nale�y wyznaczy� tylko pierwszy element macierzy $\Delta U(k)$ czyli $\Delta u(k|k)$. Aktualne sterowanie uzyskuje si� poprzez zsumowanie $\Delta u(k|k)$ z poprzednim sterowaniem.

\begin{equation}
\triangle u(k|k) = k_\mathrm{e}e(k) - \sum_{j=1}^{D-1}\boldsymbol{k}_j^u\triangle u(k-j)
\end{equation}

\begin{equation}
k_e=\sum_{i=1}^{N}k_{1,i}
\end{equation}

\begin{equation}
\boldsymbol{k}_j^u=\boldsymbol{\overline{K}}_1\boldsymbol{M}_j^\mathrm{P}, \quad j=1,\ldots ,D-1
\end{equation}

\section{Implementacja}
Implementacja fukcji wykorzystanych do wykonania zadania zawarte s� w skrypcie \verb+podpunkt_4_v1.m+.