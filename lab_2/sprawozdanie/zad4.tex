\chapter{Implementacja algorytmu DMC w najprostszej wersji analitycznej}

\section{Regulator DMC z uwzgl�dnieniem zak��ce�}
Regulator DMC jest to regulator predykcyjny - dzia�a on z wyprzedzeniem, zanim nast�pi� zmiany warto�ci sygna�u wyj�ciowego. Wektor przyrost�w sterowa� dany jest wzorem:

\begin{align}
\triangle U(k)&=\boldsymbol{K}[Y^{\mathrm{zad}}(k)-Y^0(k)]\\
&=\boldsymbol{K}[Y^{\mathrm{zad}}(k)-Y(k)-\boldsymbol{M}^\mathrm{P} \triangle U^\mathrm{P}(k)-\boldsymbol{M}^{\mathrm{zP}}\triangle Z^\mathrm{P}(k)]
\label{dU1}
\end{align}
gdzie:

\begin{equation}
\boldsymbol{K}=(\boldsymbol{M}^\mathrm{T}\boldsymbol{M}+\lambda I)^{-1}\boldsymbol{M}^\mathrm{T}
\label{K}
\end{equation}

\begin{equation}
\boldsymbol{M}=\left[
\begin{array}
{cccc}
s_{1} & 0 & \ldots & 0\\
s_{2} & s_{1} & \ldots & 0\\
\vdots & \vdots & \ddots & \vdots\\
s_{N} & s_{N-1} & \ldots &  s_{N-N_{\mathrm{u}}+1}
\end{array}
\right]_{N\times N_u}
\label{Mm}
\end{equation}

\begin{equation}
\boldsymbol{M}^\mathrm{P}=\left[
\begin{array}
{cccc}
s_{2}-s_{1} & s_{3}-s_{2} & \ldots & s_{D}-s_{D-1}\\
s_{3}-s_{1} & s_{4}-s_{2} & \ldots & s_{D+1}-s_{D-1}\\
\vdots & \vdots & \ddots & \vdots\\
s_{N+1}-s_{1} & s_{N+2}-s_{2} & \ldots &  s_{N+D-1}-S_{D-1}
\end{array}
\right]_{N\times (D-1)}
\label{MPm}
\end{equation}

\begin{equation}
\triangle U^\mathrm{P}(k)=\left[
\begin{array}{c}
\triangle u(k-1)\\
\vdots\\
\triangle u(k-(D-1))
\end{array}
\right]_{(D-1)\times 1}
\label{dUPm}
\end{equation}

\begin{equation}
\boldsymbol{M}^\mathrm{zP}=\left[
\begin{array}
{ccccc}
s_1^\mathrm{z} & s_{2}^\mathrm{z}-s_{1}^\mathrm{z} & s_{3}^\mathrm{z}-s_{2}^\mathrm{z} & \ldots & s_{D^\mathrm{z}}^\mathrm{z}-s_{D^\mathrm{z}-1}^\mathrm{z}\\
s_2^\mathrm{z} & s_{3}^\mathrm{z}-s_{1}^\mathrm{z} & s_{4}^\mathrm{z}-s_{2}^\mathrm{z} & \ldots & s_{D^\mathrm{z}+1}^\mathrm{z}-s_{D^\mathrm{z}-1}^\mathrm{z}\\
\vdots & \vdots & \ddots & \vdots\\
s_N^\mathrm{z} & s_{N+1}^\mathrm{z}-s_{1}^\mathrm{z} & s_{N+2}^\mathrm{z}-s_{2}^\mathrm{z} & \ldots &  s_{N+D^\mathrm{z}-1}^\mathrm{z}-s_{D^\mathrm{z}-1}^\mathrm{z}
\end{array}
\right]_{N\times (D^\mathrm{z}-1)}
\label{MPm}
\end{equation}

\begin{equation}
\triangle Z^\mathrm{P}(k)=\left[
\begin{array}{c}
\triangle z(k-1)\\
\vdots\\
\triangle z(k-(D^\mathrm{z}-1))
\end{array}
\right]_{(D^\mathrm{z}-1)\times 1}
\label{dUPm}
\end{equation}

gdzie $N$ - horyzont predykcji, $N_\mathrm{u}$ - horyzont sterowania, $D$ - horyzont dynamiki, $D^\mathrm{z}$ - horyzont dynamiki zak��ce�, $\lambda$ - kara za zmian� sterowania
\bigbreak
W tym przypadku nale�y wyznaczy� tylko pierwszy element macierzy $\Delta U(k)$ czyli $\Delta u(k|k)$. Aktualne sterowanie uzyskuje si� poprzez zsumowanie $\Delta u(k|k)$ z poprzednim sterowaniem.

\begin{equation}
\triangle u(k|k) = k_\mathrm{e}e(k) - \sum_{j=1}^{D-1}\boldsymbol{k}_j^\mathrm{u}\triangle u(k-j) -  \sum_{j=1}^{D^\mathrm{z}}\boldsymbol{k}_j^\mathrm{z}\triangle u(k-j-1)
\end{equation}

\begin{equation}
k_\mathrm{e}=\sum_{i=1}^{N}k_{1,i}
\end{equation}

\begin{equation}
\boldsymbol{k}_j^\mathrm{u}=\boldsymbol{\overline{K}}_1\boldsymbol{M}_j^\mathrm{P}, \quad j=1,\ldots ,D-1
\end{equation}

\begin{equation}
\boldsymbol{k}_j^\mathrm{z}=\boldsymbol{\overline{K}}_1\boldsymbol{M}_j^\mathrm{zP}, \quad j=1,\ldots ,D^\mathrm{z}
\end{equation}