\chapter{Regulacja za pomoc� DMC z uwzgl�dnieniem zak��ce�}

$D^\mathrm{z} = 50$, poniewa� tyle wynosi horyzont dynamiki zak��ce�.

Parametry:
$D = 175$, $D^\mathrm{z} = 50$, $N = 10$, $N_\mathrm{u} = 10$ i dla $\lambda = 1$

\begin{figure}[h!]
	\centering
	% This file was created by matlab2tikz.
%
\definecolor{mycolor1}{rgb}{0.00000,0.44700,0.74100}%
\definecolor{mycolor2}{rgb}{0.85000,0.32500,0.09800}%
%
\begin{tikzpicture}

\begin{axis}[%
width=4.521in,
height=1.493in,
at={(0.758in,2.554in)},
scale only axis,
xmin=0,
xmax=600,
xlabel style={font=\color{white!15!black}},
xlabel={$k$},
ymin=-4,
ymax=3,
axis background/.style={fill=white},
title style={font=\bfseries},
title={Sygna� wej�ciowy sterowania i zak��cenia},
xmajorgrids,
ymajorgrids,
legend style={legend cell align=left, align=left, draw=white!15!black}
]
\addplot[const plot, color=mycolor1] table[row sep=crcr] {%
1	0\\
2	0\\
3	0\\
4	0\\
5	0\\
6	0\\
7	0\\
8	0\\
9	0\\
10	0\\
11	0\\
12	0\\
13	0\\
14	0\\
15	0\\
16	0\\
17	0\\
18	0\\
19	0\\
20	0\\
21	0\\
22	0\\
23	0\\
24	0\\
25	0\\
26	0\\
27	0\\
28	0\\
29	0\\
30	0\\
31	0\\
32	0\\
33	0\\
34	0\\
35	0\\
36	0\\
37	0\\
38	0\\
39	0\\
40	0\\
41	0\\
42	0\\
43	0\\
44	0\\
45	0\\
46	0\\
47	0\\
48	0\\
49	0\\
50	0\\
51	0\\
52	0\\
53	0\\
54	0\\
55	0\\
56	0\\
57	0\\
58	0\\
59	0\\
60	0\\
61	0\\
62	0\\
63	0\\
64	0\\
65	0\\
66	0\\
67	0\\
68	0\\
69	0\\
70	0\\
71	0\\
72	0\\
73	0\\
74	0\\
75	0\\
76	0\\
77	0\\
78	0\\
79	0\\
80	0\\
81	0\\
82	0\\
83	0\\
84	0\\
85	0\\
86	0\\
87	0\\
88	0\\
89	0\\
90	0\\
91	0\\
92	0\\
93	0\\
94	0\\
95	0\\
96	0\\
97	0\\
98	0\\
99	0\\
100	1.06438689272261\\
101	1.67330799970846\\
102	1.94630786721308\\
103	1.9873867877016\\
104	1.88160065495093\\
105	1.69466404103899\\
106	1.47444334123278\\
107	1.2534827815949\\
108	1.05194116533132\\
109	0.88051830203168\\
110	0.743112059095731\\
111	0.639070284192932\\
112	0.564990187660185\\
113	0.516076581765213\\
114	0.487105661396056\\
115	0.47305860386581\\
116	0.469494351726138\\
117	0.472727848975606\\
118	0.479872134885331\\
119	0.488792593525579\\
120	0.498011076122221\\
121	0.506587690199236\\
122	0.513999424293346\\
123	0.520027728514665\\
124	0.524661729730118\\
125	0.528019799977936\\
126	0.530289507139993\\
127	0.531684313584522\\
128	0.532414508349938\\
129	0.532669540981234\\
130	0.53260898363768\\
131	0.532359633484115\\
132	0.532016667214764\\
133	0.531647194123235\\
134	0.531294971082772\\
135	0.53098541140078\\
136	0.530730325128121\\
137	0.530532067662125\\
138	0.530386950263108\\
139	0.530287888418633\\
140	0.530226341679766\\
141	0.53019364174679\\
142	0.530181823531268\\
143	0.530184074694741\\
144	0.530194909262603\\
145	0.530210155308029\\
146	0.530226828978867\\
147	0.530242949702892\\
148	0.530257335712222\\
149	0.530269405819655\\
150	0.530279002893992\\
151	0.530286246619096\\
152	0.530291417585385\\
153	0.530294871160459\\
154	0.53029697750129\\
155	0.530298083110692\\
156	0.5302984891616\\
157	0.530298442131522\\
158	0.53029813288721\\
159	0.530297701075738\\
160	0.53029724240299\\
161	0.5302968170457\\
162	0.530296458011551\\
163	0.530296178719471\\
164	0.530295979420275\\
165	0.530295852326937\\
166	0.530295785489372\\
167	0.530295765547805\\
168	0.530295779548015\\
169	0.530295816015713\\
170	0.53029586547824\\
171	0.530295920599273\\
172	0.530295976063461\\
173	0.530296028317908\\
174	0.530296075249234\\
175	0.530296115850542\\
176	0.530296149912657\\
177	0.530296177758576\\
178	0.530296200028913\\
179	0.530296217518459\\
180	0.530296231059335\\
181	0.530296241443675\\
182	0.530296249377924\\
183	0.530296255460952\\
184	0.530296260179045\\
185	0.530296263911903\\
186	0.530296266945044\\
187	0.530296269485158\\
188	0.530296271676013\\
189	0.530296273613349\\
190	0.5302962753579\\
191	0.530296276946142\\
192	0.53029627839874\\
193	0.530296279726851\\
194	0.530296280936599\\
195	0.530296282032038\\
196	0.53029628301696\\
197	0.530296283895852\\
198	0.530296284674259\\
199	0.530296285358785\\
200	0.53029628595687\\
201	0.530296286476478\\
202	0.530296286925765\\
203	0.530296287312782\\
204	0.53029628764523\\
205	0.530296287930287\\
206	0.530296288174488\\
207	0.530296288383675\\
208	0.530296288562982\\
209	0.530296288716863\\
210	0.530296288849134\\
211	0.530296288963043\\
212	0.530296289061329\\
213	0.530296289146296\\
214	0.530296289219876\\
215	0.530296289283692\\
216	0.530296289339104\\
217	0.530296289387264\\
218	0.530296289429144\\
219	0.530296289465574\\
220	0.530296289497263\\
221	0.530296289524822\\
222	0.530296289548782\\
223	0.530296289569602\\
224	0.530296289587684\\
225	0.530296289603381\\
226	0.530296289617\\
227	0.53029628962881\\
228	0.530296289639049\\
229	0.530296289647923\\
230	0.530296289655612\\
231	0.530296289662274\\
232	0.530296289668046\\
233	0.530296289673046\\
234	0.530296289677378\\
235	0.53029628968113\\
236	0.53029628968438\\
237	0.530296289687193\\
238	0.530296289689628\\
239	0.530296289691737\\
240	0.530296289693563\\
241	0.530296289695144\\
242	0.530296289696515\\
243	0.530296289697705\\
244	0.530296289698736\\
245	0.530296289699632\\
246	0.53029628970041\\
247	0.530296289701086\\
248	0.530296289701672\\
249	0.53029628970218\\
250	0.530296289702619\\
251	0.530296289702997\\
252	0.530296289703324\\
253	0.530296289703606\\
254	0.530296289703851\\
255	0.530296289704065\\
256	0.530296289704251\\
257	0.530296289704414\\
258	0.530296289704558\\
259	0.530296289704683\\
260	0.530296289704794\\
261	0.530296289704891\\
262	0.530296289704976\\
263	0.530296289705049\\
264	0.530296289705111\\
265	0.530296289705163\\
266	0.530306508427582\\
267	0.53033666381662\\
268	0.530391127427623\\
269	0.530467847736317\\
270	0.530559386950657\\
271	0.530654876638281\\
272	0.530742108994647\\
273	0.530811305149986\\
274	0.530856665729264\\
275	0.530876482260541\\
276	0.530835314532807\\
277	0.530733501687177\\
278	0.530589437566782\\
279	0.530427392420687\\
280	0.530270030302657\\
281	0.530134556578039\\
282	0.530031433039188\\
283	0.529964727279454\\
284	0.529933350393245\\
285	0.529932636565899\\
286	0.529955901281582\\
287	0.529995766900072\\
288	0.530045160082298\\
289	0.530097966103479\\
290	0.530149375192492\\
291	0.530195981986413\\
292	0.53023570754184\\
293	0.530267610057856\\
294	0.530291640471443\\
295	0.530308386143677\\
296	0.530318832592561\\
297	0.530324161292871\\
298	0.530325591820423\\
299	0.530324269357558\\
300	0.530321193702817\\
301	0.530317183116816\\
302	0.374454692457114\\
303	0.150189920242165\\
304	-0.0841687007393884\\
305	-0.293305990857416\\
306	-0.459065905145216\\
307	-0.575335251323347\\
308	-0.643826712635386\\
309	-0.67079290099526\\
310	-0.664607787100226\\
311	-0.634098624214425\\
312	-0.587489069582044\\
313	-0.531812844233788\\
314	-0.472669034401636\\
315	-0.414208952682258\\
316	-0.359265960009701\\
317	-0.309560852787542\\
318	-0.265934547766166\\
319	-0.228575927884438\\
320	-0.197225544720392\\
321	-0.171345517900598\\
322	-0.15025278323979\\
323	-0.133217292770863\\
324	-0.11952936529187\\
325	-0.108541603489323\\
326	-0.0996910518668214\\
327	-0.0925069132661727\\
328	-0.08660843864353\\
329	-0.08169675144588\\
330	-0.0775434991579077\\
331	-0.0739784248603501\\
332	-0.0708772668636958\\
333	-0.0681508426101222\\
334	-0.06573575316227\\
335	-0.0635868442507005\\
336	-0.0616713605524981\\
337	-0.0599646110957391\\
338	-0.0584469054888079\\
339	-0.0571015052779895\\
340	-0.0559133472934445\\
341	-0.0548683246534904\\
342	-0.0539529474593137\\
343	-0.0531542430614944\\
344	-0.0524597912250879\\
345	-0.051857820364377\\
346	-0.0513373163095013\\
347	-0.0508881147081177\\
348	-0.0505009625971523\\
349	-0.0501675446371043\\
350	-0.0498804758347786\\
351	-0.049633266136124\\
352	-0.0494202638173879\\
353	-0.0492365847916994\\
354	-0.0490780343008994\\
355	-0.0489410263735499\\
356	-0.0488225051806927\\
357	-0.0487198711972087\\
358	-0.0486309139894885\\
359	-0.048553752554376\\
360	-0.0484867834453776\\
361	-0.0484286364297771\\
362	-0.0483781370996903\\
363	-0.048334275679621\\
364	-0.048296181199933\\
365	-0.048263100209306\\
366	-0.0482343792534446\\
367	-0.0482094504308044\\
368	-0.0481878194327893\\
369	-0.0481690555742383\\
370	-0.0481527834125318\\
371	-0.0481386756359103\\
372	-0.0481264469715935\\
373	-0.0481158489216952\\
374	-0.0481066651804795\\
375	-0.0480987076216145\\
376	-0.0480918127703918\\
377	-0.0480858386951204\\
378	-0.0480806622656638\\
379	-0.0480761767367655\\
380	-0.0480722896205588\\
381	-0.048068920817397\\
382	-0.0480660009775655\\
383	-0.0480634700690414\\
384	-0.0480612761285963\\
385	-0.0480593741754101\\
386	-0.0480577252681013\\
387	-0.0480562956877621\\
388	-0.0480550562312143\\
389	-0.0480539816002934\\
390	-0.0480530498744979\\
391	-0.0480522420557965\\
392	-0.0480515416757386\\
393	-0.0480509344562728\\
394	-0.0480504080168046\\
395	-0.0480499516210484\\
396	-0.0480495559581212\\
397	-0.0480492129531169\\
398	-0.0480489156030768\\
399	-0.0480486578348652\\
400	-0.0480484343819546\\
401	-0.0480482406775588\\
402	-0.0480480727619125\\
403	-0.0480479272018081\\
404	-0.0480478010207588\\
405	-0.0480476916383848\\
406	-0.0480475968178056\\
407	-0.0480475146199866\\
408	-0.0480474433641271\\
409	-0.048047381593294\\
410	-0.048047328044616\\
411	-0.0480472816234347\\
412	-0.0480472413808927\\
413	-0.0480472064945048\\
414	-0.0480471762513166\\
415	-0.048047150033307\\
416	-0.0480471273047385\\
417	-0.0480471076011945\\
418	-0.0480470905200769\\
419	-0.0480470757123741\\
420	-0.0480470628755247\\
421	-0.0480470517472329\\
422	-0.0480470421001093\\
423	-0.0480470337370243\\
424	-0.0480470264870814\\
425	-0.0480470202021267\\
426	-0.0480470147537222\\
427	-0.0480470100305231\\
428	-0.0480470059360034\\
429	-0.0480470023864836\\
430	-0.048046999309422\\
431	-0.0480469966419318\\
432	-0.0480469942313911\\
433	-0.0480469917038581\\
434	-0.0480469883902438\\
435	-0.0480469833639741\\
436	-0.0480469755945327\\
437	-0.0480469641826706\\
438	-0.048046948623438\\
439	-0.0480469290034347\\
440	-0.0480469060682631\\
441	-0.0480468811454065\\
442	-0.0480468566628945\\
443	-0.0480468362314308\\
444	-0.0480468241226477\\
445	-0.0480468243718501\\
446	-0.0480468398314283\\
447	-0.0480468714423476\\
448	-0.0480469178727783\\
449	-0.0480469756928455\\
450	-0.048047039980181\\
451	-0.0480471051465833\\
452	-0.0480471644810588\\
453	-0.0480472113247945\\
454	-0.0480472404854955\\
455	-0.0480472492083757\\
456	-0.0480472374947136\\
457	-0.048047207833957\\
458	-0.0480471645415345\\
459	-0.0480471129209979\\
460	-0.0480470584378195\\
461	-0.0480470060349797\\
462	-0.0480469596591017\\
463	-0.0480469220130897\\
464	-0.0480468945129078\\
465	-0.0480468774033707\\
466	-0.0480468699787434\\
467	-0.0480468708552877\\
468	-0.0480483745785011\\
469	-0.0480540992217485\\
470	-0.0480670406454224\\
471	-0.0480894949951434\\
472	-0.0481223631885147\\
473	-0.0481648344130962\\
474	-0.0482144180429708\\
475	-0.0482675231473114\\
476	-0.0483201879035611\\
477	-0.0483687281884207\\
478	-0.0484047614006875\\
479	-0.0484210716459429\\
480	-0.0484140984119962\\
481	-0.0483843103920886\\
482	-0.048335433654712\\
483	-0.0482732183208783\\
484	-0.04820418124392\\
485	-0.0481345672567406\\
486	-0.0480696300103641\\
487	-0.0480132399424159\\
488	-0.047967772698902\\
489	-0.047934206600592\\
490	-0.047912353256996\\
491	-0.0479011533407365\\
492	-0.0478989837313573\\
493	-0.0479039383461056\\
494	-0.0479140601635711\\
495	-0.0479275146316354\\
496	-0.047942704184391\\
497	-0.0479583299611326\\
498	-0.0479734103984245\\
499	-0.0479872677194033\\
500	-0.0479994930901083\\
501	-0.0480098999283745\\
502	-0.0480184730233832\\
503	-0.0480253191278683\\
504	-0.0480306227819229\\
505	-0.0480346094784174\\
506	-0.0480375169643461\\
507	-0.0480395745078958\\
508	-0.0480409893241144\\
509	-0.0480419389947116\\
510	-0.0480425685806159\\
511	-0.0480429911490522\\
512	-0.0480432905648485\\
513	-0.0480435255819388\\
514	-0.0480437344794114\\
515	-0.0480439396909306\\
516	-0.0480441520600924\\
517	-0.0480443745080098\\
518	-0.0480446050197774\\
519	-0.048044838944186\\
520	-0.048045070659579\\
521	-0.0480452946929066\\
522	-0.0480455063942001\\
523	-0.0480457022700352\\
524	-0.0480458800717182\\
525	-0.0480460387207321\\
526	-0.0480461781384141\\
527	-0.0480462990310457\\
528	-0.0480464026669569\\
529	-0.0480464906697001\\
530	-0.0480465648411939\\
531	-0.0480466270210109\\
532	-0.0480466789824836\\
533	-0.0480467223627223\\
534	-0.0480467586215949\\
535	-0.0480467890238541\\
536	-0.0480468146385528\\
537	-0.0480468363503834\\
538	-0.0480468548783673\\
539	-0.0480468707982409\\
540	-0.0480468845657986\\
541	-0.0480468965392825\\
542	-0.0480469069996234\\
543	-0.0480469161678972\\
544	-0.0480469242197849\\
545	-0.0480469312971228\\
546	-0.0480469375168129\\
547	-0.0480469429774659\\
548	-0.0480469477641852\\
549	-0.0480469519518904\\
550	-0.0480469556075431\\
551	-0.0480469587915844\\
552	-0.0480469615588359\\
553	-0.0480469639590571\\
554	-0.0480469660373044\\
555	-0.0480469678341911\\
556	-0.0480469693861104\\
557	-0.0480469707254603\\
558	-0.0480469718808879\\
559	-0.0480469728775564\\
560	-0.0480469737374336\\
561	-0.0480469744795914\\
562	-0.0480469751205093\\
563	-0.0480469756743717\\
564	-0.0480469761533513\\
565	-0.0480469765678735\\
566	-0.0480469769268577\\
567	-0.0480469772379327\\
568	-0.0480469775076284\\
569	-0.0480469777415399\\
570	-0.0480469779444696\\
571	-0.0480469781205475\\
572	-0.0480469782733336\\
573	-0.0480469784059023\\
574	-0.0480469785209147\\
575	-0.0480469786206779\\
576	-0.0480469787071953\\
577	-0.0480469787822082\\
578	-0.0480469788472315\\
579	-0.0480469789035832\\
580	-0.0480469789524105\\
581	-0.0480469789947108\\
582	-0.0480469790313518\\
583	-0.0480469790630875\\
584	-0.0480469790905732\\
585	-0.0480469791143774\\
586	-0.0480469791349935\\
587	-0.048046979152849\\
588	-0.0480469791683146\\
589	-0.0480469791817109\\
590	-0.0480469791933157\\
591	-0.0480469792033694\\
592	-0.0480469792120798\\
593	-0.0480469792196266\\
594	-0.0480469792261658\\
595	-0.0480469792318322\\
596	-0.0480469792367428\\
597	-0.0480469792409983\\
598	-0.0480469792446851\\
599	-0.0480469792478738\\
600	-0.0480469792506128\\
};
\addlegendentry{$u(k)$}

\addplot[const plot, color=mycolor2] table[row sep=crcr] {%
1	0\\
2	0\\
3	0\\
4	0\\
5	0\\
6	0\\
7	0\\
8	0\\
9	0\\
10	0\\
11	0\\
12	0\\
13	0\\
14	0\\
15	0\\
16	0\\
17	0\\
18	0\\
19	0\\
20	0\\
21	0\\
22	0\\
23	0\\
24	0\\
25	0\\
26	0\\
27	0\\
28	0\\
29	0\\
30	0\\
31	0\\
32	0\\
33	0\\
34	0\\
35	0\\
36	0\\
37	0\\
38	0\\
39	0\\
40	0\\
41	0\\
42	0\\
43	0\\
44	0\\
45	0\\
46	0\\
47	0\\
48	0\\
49	0\\
50	0\\
51	0\\
52	0\\
53	0\\
54	0\\
55	0\\
56	0\\
57	0\\
58	0\\
59	0\\
60	0\\
61	0\\
62	0\\
63	0\\
64	0\\
65	0\\
66	0\\
67	0\\
68	0\\
69	0\\
70	0\\
71	0\\
72	0\\
73	0\\
74	0\\
75	0\\
76	0\\
77	0\\
78	0\\
79	0\\
80	0\\
81	0\\
82	0\\
83	0\\
84	0\\
85	0\\
86	0\\
87	0\\
88	0\\
89	0\\
90	0\\
91	0\\
92	0\\
93	0\\
94	0\\
95	0\\
96	0\\
97	0\\
98	0\\
99	0\\
100	0\\
101	0\\
102	0\\
103	0\\
104	0\\
105	0\\
106	0\\
107	0\\
108	0\\
109	0\\
110	0\\
111	0\\
112	0\\
113	0\\
114	0\\
115	0\\
116	0\\
117	0\\
118	0\\
119	0\\
120	0\\
121	0\\
122	0\\
123	0\\
124	0\\
125	0\\
126	0\\
127	0\\
128	0\\
129	0\\
130	0\\
131	0\\
132	0\\
133	0\\
134	0\\
135	0\\
136	0\\
137	0\\
138	0\\
139	0\\
140	0\\
141	0\\
142	0\\
143	0\\
144	0\\
145	0\\
146	0\\
147	0\\
148	0\\
149	0\\
150	0\\
151	0\\
152	0\\
153	0\\
154	0\\
155	0\\
156	0\\
157	0\\
158	0\\
159	0\\
160	0\\
161	0\\
162	0\\
163	0\\
164	0\\
165	0\\
166	0\\
167	0\\
168	0\\
169	0\\
170	0\\
171	0\\
172	0\\
173	0\\
174	0\\
175	0\\
176	0\\
177	0\\
178	0\\
179	0\\
180	0\\
181	0\\
182	0\\
183	0\\
184	0\\
185	0\\
186	0\\
187	0\\
188	0\\
189	0\\
190	0\\
191	0\\
192	0\\
193	0\\
194	0\\
195	0\\
196	0\\
197	0\\
198	0\\
199	0\\
200	0\\
201	0\\
202	0\\
203	0\\
204	0\\
205	0\\
206	0\\
207	0\\
208	0\\
209	0\\
210	0\\
211	0\\
212	0\\
213	0\\
214	0\\
215	0\\
216	0\\
217	0\\
218	0\\
219	0\\
220	0\\
221	0\\
222	0\\
223	0\\
224	0\\
225	0\\
226	0\\
227	0\\
228	0\\
229	0\\
230	0\\
231	0\\
232	0\\
233	0\\
234	0\\
235	0\\
236	0\\
237	0\\
238	0\\
239	0\\
240	0\\
241	0\\
242	0\\
243	0\\
244	0\\
245	0\\
246	0\\
247	0\\
248	0\\
249	0\\
250	0\\
251	0\\
252	0\\
253	0\\
254	0\\
255	0\\
256	0\\
257	0\\
258	0\\
259	0\\
260	0\\
261	0\\
262	0\\
263	0\\
264	0\\
265	0\\
266	0\\
267	0\\
268	0\\
269	0\\
270	0\\
271	0\\
272	0\\
273	0\\
274	0\\
275	0\\
276	0\\
277	0\\
278	0\\
279	0\\
280	0\\
281	0\\
282	0\\
283	0\\
284	0\\
285	0\\
286	0\\
287	0\\
288	0\\
289	0\\
290	0\\
291	0\\
292	0\\
293	0\\
294	0\\
295	0\\
296	0\\
297	0\\
298	0\\
299	0\\
300	1\\
301	1\\
302	1\\
303	1\\
304	1\\
305	1\\
306	1\\
307	1\\
308	1\\
309	1\\
310	1\\
311	1\\
312	1\\
313	1\\
314	1\\
315	1\\
316	1\\
317	1\\
318	1\\
319	1\\
320	1\\
321	1\\
322	1\\
323	1\\
324	1\\
325	1\\
326	1\\
327	1\\
328	1\\
329	1\\
330	1\\
331	1\\
332	1\\
333	1\\
334	1\\
335	1\\
336	1\\
337	1\\
338	1\\
339	1\\
340	1\\
341	1\\
342	1\\
343	1\\
344	1\\
345	1\\
346	1\\
347	1\\
348	1\\
349	1\\
350	1\\
351	1\\
352	1\\
353	1\\
354	1\\
355	1\\
356	1\\
357	1\\
358	1\\
359	1\\
360	1\\
361	1\\
362	1\\
363	1\\
364	1\\
365	1\\
366	1\\
367	1\\
368	1\\
369	1\\
370	1\\
371	1\\
372	1\\
373	1\\
374	1\\
375	1\\
376	1\\
377	1\\
378	1\\
379	1\\
380	1\\
381	1\\
382	1\\
383	1\\
384	1\\
385	1\\
386	1\\
387	1\\
388	1\\
389	1\\
390	1\\
391	1\\
392	1\\
393	1\\
394	1\\
395	1\\
396	1\\
397	1\\
398	1\\
399	1\\
400	1\\
401	1\\
402	1\\
403	1\\
404	1\\
405	1\\
406	1\\
407	1\\
408	1\\
409	1\\
410	1\\
411	1\\
412	1\\
413	1\\
414	1\\
415	1\\
416	1\\
417	1\\
418	1\\
419	1\\
420	1\\
421	1\\
422	1\\
423	1\\
424	1\\
425	1\\
426	1\\
427	1\\
428	1\\
429	1\\
430	1\\
431	1\\
432	1\\
433	1\\
434	1\\
435	1\\
436	1\\
437	1\\
438	1\\
439	1\\
440	1\\
441	1\\
442	1\\
443	1\\
444	1\\
445	1\\
446	1\\
447	1\\
448	1\\
449	1\\
450	1\\
451	1\\
452	1\\
453	1\\
454	1\\
455	1\\
456	1\\
457	1\\
458	1\\
459	1\\
460	1\\
461	1\\
462	1\\
463	1\\
464	1\\
465	1\\
466	1\\
467	1\\
468	1\\
469	1\\
470	1\\
471	1\\
472	1\\
473	1\\
474	1\\
475	1\\
476	1\\
477	1\\
478	1\\
479	1\\
480	1\\
481	1\\
482	1\\
483	1\\
484	1\\
485	1\\
486	1\\
487	1\\
488	1\\
489	1\\
490	1\\
491	1\\
492	1\\
493	1\\
494	1\\
495	1\\
496	1\\
497	1\\
498	1\\
499	1\\
500	1\\
501	1\\
502	1\\
503	1\\
504	1\\
505	1\\
506	1\\
507	1\\
508	1\\
509	1\\
510	1\\
511	1\\
512	1\\
513	1\\
514	1\\
515	1\\
516	1\\
517	1\\
518	1\\
519	1\\
520	1\\
521	1\\
522	1\\
523	1\\
524	1\\
525	1\\
526	1\\
527	1\\
528	1\\
529	1\\
530	1\\
531	1\\
532	1\\
533	1\\
534	1\\
535	1\\
536	1\\
537	1\\
538	1\\
539	1\\
540	1\\
541	1\\
542	1\\
543	1\\
544	1\\
545	1\\
546	1\\
547	1\\
548	1\\
549	1\\
550	1\\
551	1\\
552	1\\
553	1\\
554	1\\
555	1\\
556	1\\
557	1\\
558	1\\
559	1\\
560	1\\
561	1\\
562	1\\
563	1\\
564	1\\
565	1\\
566	1\\
567	1\\
568	1\\
569	1\\
570	1\\
571	1\\
572	1\\
573	1\\
574	1\\
575	1\\
576	1\\
577	1\\
578	1\\
579	1\\
580	1\\
581	1\\
582	1\\
583	1\\
584	1\\
585	1\\
586	1\\
587	1\\
588	1\\
589	1\\
590	1\\
591	1\\
592	1\\
593	1\\
594	1\\
595	1\\
596	1\\
597	1\\
598	1\\
599	1\\
600	1\\
};
\addlegendentry{$z(k)$}

\end{axis}

\begin{axis}[%
width=4.521in,
height=1.493in,
at={(0.758in,0.481in)},
scale only axis,
xmin=0,
xmax=600,
xlabel style={font=\color{white!15!black}},
xlabel={$k$},
ymin=-2,
ymax=3,
axis background/.style={fill=white},
title style={font=\bfseries},
title={Sygna� wyj�ciowy i warto�� zadana},
xmajorgrids,
ymajorgrids,
legend style={legend cell align=left, align=left, draw=white!15!black}
]
\addplot [color=mycolor1]
  table[row sep=crcr]{%
1	0\\
2	0\\
3	0\\
4	0\\
5	0\\
6	0\\
7	0\\
8	0\\
9	0\\
10	0\\
11	0\\
12	0\\
13	0\\
14	0\\
15	0\\
16	0\\
17	0\\
18	0\\
19	0\\
20	0\\
21	0\\
22	0\\
23	0\\
24	0\\
25	0\\
26	0\\
27	0\\
28	0\\
29	0\\
30	0\\
31	0\\
32	0\\
33	0\\
34	0\\
35	0\\
36	0\\
37	0\\
38	0\\
39	0\\
40	0\\
41	0\\
42	0\\
43	0\\
44	0\\
45	0\\
46	0\\
47	0\\
48	0\\
49	0\\
50	0\\
51	0\\
52	0\\
53	0\\
54	0\\
55	0\\
56	0\\
57	0\\
58	0\\
59	0\\
60	0\\
61	0\\
62	0\\
63	0\\
64	0\\
65	0\\
66	0\\
67	0\\
68	0\\
69	0\\
70	0\\
71	0\\
72	0\\
73	0\\
74	0\\
75	0\\
76	0\\
77	0\\
78	0\\
79	0\\
80	0\\
81	0\\
82	0\\
83	0\\
84	0\\
85	0\\
86	0\\
87	0\\
88	0\\
89	0\\
90	0\\
91	0\\
92	0\\
93	0\\
94	0\\
95	0\\
96	0\\
97	0\\
98	0\\
99	0\\
100	0\\
101	0\\
102	0\\
103	0\\
104	0\\
105	0.0898661853525702\\
106	0.227145989536648\\
107	0.381364985455071\\
108	0.532183744977198\\
109	0.667348874413722\\
110	0.780701113190663\\
111	0.870397690541912\\
112	0.937426745782853\\
113	0.984435604898583\\
114	1.01485816163926\\
115	1.03230537220306\\
116	1.04017299290448\\
117	1.04141871252479\\
118	1.03846391027622\\
119	1.03318119527604\\
120	1.0269360330158\\
121	1.02065802748983\\
122	1.01492410818031\\
123	1.01004159145033\\
124	1.0061236992538\\
125	1.00315363327061\\
126	1.00103582376953\\
127	0.999634653341271\\
128	0.998801966526442\\
129	0.99839518271941\\
130	0.998287979047469\\
131	0.998375425366301\\
132	0.998575232142644\\
133	0.998826486044746\\
134	0.99908694793467\\
135	0.999329706050688\\
136	0.99953973184925\\
137	0.999710685296531\\
138	0.999842161319463\\
139	0.9999374561398\\
140	1.00000185537463\\
141	1.00004139817149\\
142	1.00006204635586\\
143	1.00006917833604\\
144	1.00006732902273\\
145	1.00006010503136\\
146	1.00005021573868\\
147	1.00003957308605\\
148	1.00002942486313\\
149	1.00002049668706\\
150	1.00001312659169\\
151	1.00000738295854\\
152	1.00000316156378\\
153	1.00000026100877\\
154	0.999998438024651\\
155	0.999997445379893\\
156	0.999997055639453\\
157	0.999997074052887\\
158	0.99999734357207\\
159	0.999997744558619\\
160	0.999998191239006\\
161	0.999998626470347\\
162	0.999999015933754\\
163	0.999999342496316\\
164	0.999999601184079\\
165	0.999999794984267\\
166	0.999999931536935\\
167	1.00000002067327\\
168	1.00000007269799\\
169	1.0000000972857\\
170	1.00000010285557\\
171	1.00000009629762\\
172	1.00000008294079\\
173	1.00000006667322\\
174	1.00000005014576\\
175	1.00000003500852\\
176	1.00000002214677\\
177	1.00000001189507\\
178	1.0000000042189\\
179	0.999999998859796\\
180	0.999999995445021\\
181	0.999999993565414\\
182	0.999999992826672\\
183	0.999999992879594\\
184	0.999999993434638\\
185	0.99999999426549\\
186	0.999999995205538\\
187	0.999999996140285\\
188	0.999999996997944\\
189	0.999999997739762\\
190	0.999999998351042\\
191	0.999999998833424\\
192	0.99999999919862\\
193	0.999999999463627\\
194	0.999999999647282\\
195	0.999999999767949\\
196	0.999999999842131\\
197	0.999999999883761\\
198	0.999999999903995\\
199	0.999999999911317\\
200	0.999999999911846\\
201	0.999999999909724\\
202	0.999999999907537\\
203	0.999999999906703\\
204	0.999999999907817\\
205	0.999999999910932\\
206	0.999999999915784\\
207	0.999999999921947\\
208	0.999999999928953\\
209	0.999999999936356\\
210	0.99999999994378\\
211	0.999999999950931\\
212	0.999999999957603\\
213	0.999999999963665\\
214	0.999999999969054\\
215	0.999999999973759\\
216	0.999999999977806\\
217	0.999999999981245\\
218	0.999999999984139\\
219	0.999999999986558\\
220	0.999999999988573\\
221	0.999999999990247\\
222	0.999999999991641\\
223	0.999999999992805\\
224	0.999999999993781\\
225	0.999999999994607\\
226	0.999999999995309\\
227	0.999999999995912\\
228	0.999999999996432\\
229	0.999999999996883\\
230	0.999999999997277\\
231	0.999999999997622\\
232	0.999999999997925\\
233	0.999999999998191\\
234	0.999999999998424\\
235	0.999999999998629\\
236	0.999999999998809\\
237	0.999999999998967\\
238	0.999999999999105\\
239	0.999999999999226\\
240	0.999999999999331\\
241	0.999999999999423\\
242	0.999999999999503\\
243	0.999999999999573\\
244	0.999999999999632\\
245	0.999999999999683\\
246	0.999999999999727\\
247	0.999999999999764\\
248	0.999999999999796\\
249	0.999999999999823\\
250	0.999999999999846\\
251	0.999999999999867\\
252	0.999999999999885\\
253	0.9999999999999\\
254	0.999999999999913\\
255	0.999999999999924\\
256	0.999999999999934\\
257	0.999999999999942\\
258	0.999999999999948\\
259	0.999999999999954\\
260	0.999999999999958\\
261	0.999999999999961\\
262	0.999999999999964\\
263	0.999999999999967\\
264	0.999999999999969\\
265	0.999999999999972\\
266	0.999999999999975\\
267	0.999999999999979\\
268	0.999999999999982\\
269	0.999999999999985\\
270	0.999999999999987\\
271	1.00000086276672\\
272	1.00000423317379\\
273	1.0000120519818\\
274	1.0000260003166\\
275	1.00004705639391\\
276	1.00007523709427\\
277	1.00010952763252\\
278	1.00014813255248\\
279	1.00018884650625\\
280	1.00022941796769\\
281	1.00026470340528\\
282	1.00028981685954\\
283	1.0003016430238\\
284	1.00029925476994\\
285	1.00028368025733\\
286	1.00025735504373\\
287	1.00022349001474\\
288	1.00018549688746\\
289	1.00014654375849\\
290	1.0001092638004\\
291	1.00007560850438\\
292	1.00004681940536\\
293	1.0000234853142\\
294	1.0000056522805\\
295	0.999992957931546\\
296	0.99998476825077\\
297	0.999980301698218\\
298	0.999978731833114\\
299	0.999979264735191\\
300	0.999981191336499\\
301	0.999983917301141\\
302	1.14641697449577\\
303	1.27334405061373\\
304	1.38335645051608\\
305	1.47870157243989\\
306	1.56132853754721\\
307	1.61976865701438\\
308	1.65029811475081\\
309	1.65415280574158\\
310	1.6353836934162\\
311	1.59929764912302\\
312	1.5513975368778\\
313	1.49672374361036\\
314	1.43950096300089\\
315	1.38300364246762\\
316	1.3295671820248\\
317	1.28068694550759\\
318	1.23716157264408\\
319	1.1992498791137\\
320	1.16682127680328\\
321	1.13948800743582\\
322	1.11671368397649\\
323	1.09789694634769\\
324	1.08243179765954\\
325	1.06974774157522\\
326	1.05933351310471\\
327	1.0507482623825\\
328	1.04362373987991\\
329	1.03766051549358\\
330	1.03262066846105\\
331	1.02831879458103\\
332	1.02461264340819\\
333	1.02139424768214\\
334	1.01858204919051\\
335	1.01611425673597\\
336	1.0139434830441\\
337	1.0120325850424\\
338	1.01035156176432\\
339	1.00887533269555\\
340	1.0075822148509\\
341	1.0064529294987\\
342	1.00546999165992\\
343	1.00461736174178\\
344	1.00388026511582\\
345	1.00324510973547\\
346	1.00269945272184\\
347	1.00223198373553\\
348	1.00183250593878\\
349	1.00149190480645\\
350	1.00120210150601\\
351	1.00095599163337\\
352	1.00074737234233\\
353	1.00057086185877\\
354	1.00042181545767\\
355	1.00029624154553\\
356	1.00019072078701\\
357	1.00010233042326\\
358	1.00002857516731\\
359	0.999967325396964\\
360	0.999916762829632\\
361	0.999875333463792\\
362	0.999841707298361\\
363	0.999814744177129\\
364	0.999793465028368\\
365	0.999777027758045\\
366	0.999764707088747\\
367	0.999755877698484\\
368	0.999750000090515\\
369	0.999746608707196\\
370	0.999745301880547\\
371	0.999745733285467\\
372	0.999747604625732\\
373	0.999750659337355\\
374	0.999754677138596\\
375	0.9997594692917\\
376	0.999764874469474\\
377	0.999770755141353\\
378	0.999776994409907\\
379	0.999783493240964\\
380	0.999790168039673\\
381	0.999796948531745\\
382	0.999803775914357\\
383	0.999810601245413\\
384	0.999817384043225\\
385	0.999824091071599\\
386	0.999830695287842\\
387	0.999837174933504\\
388	0.999843512749801\\
389	0.999849695301606\\
390	0.999855712395719\\
391	0.999861556580807\\
392	0.999867222717925\\
393	0.99987270761195\\
394	0.999878009695504\\
395	0.999883128758055\\
396	0.999888065713905\\
397	0.999892822403621\\
398	0.999897401424235\\
399	0.999901805984193\\
400	0.999906039779606\\
401	0.999910106888841\\
402	0.999914011682888\\
403	0.999917758749339\\
404	0.999921352828065\\
405	0.999924798756987\\
406	0.99992810142652\\
407	0.999931265741495\\
408	0.999934296589498\\
409	0.999937198814721\\
410	0.999939977196552\\
411	0.999942636432213\\
412	0.999945181122855\\
413	0.999947615762621\\
414	0.999949944730215\\
415	0.999952172282607\\
416	0.999954302550548\\
417	0.999956339535612\\
418	0.999958287108504\\
419	0.99996014900846\\
420	0.999961928843514\\
421	0.999963630091521\\
422	0.999965256101768\\
423	0.999966810097078\\
424	0.999968295176313\\
425	0.999969714317182\\
426	0.999971070379285\\
427	0.999972366107347\\
428	0.999973604134578\\
429	0.999974786986114\\
430	0.999975917082526\\
431	0.999976996743336\\
432	0.999978028190536\\
433	0.999979013552091\\
434	0.999979954865389\\
435	0.99998085408065\\
436	0.99998171306426\\
437	0.999982533610324\\
438	0.999983317462777\\
439	0.999984066342726\\
440	0.999984781971367\\
441	0.999985466078906\\
442	0.999986120393193\\
443	0.99998674660665\\
444	0.999987346327991\\
445	0.999987921030393\\
446	0.99998847200847\\
447	0.999989000293364\\
448	0.999989506564685\\
449	0.99998999111041\\
450	0.999990453864334\\
451	0.999990894521776\\
452	0.99999131271165\\
453	0.99999170819139\\
454	0.999992081018447\\
455	0.999992431663006\\
456	0.999992761045885\\
457	0.999993070613301\\
458	0.999993362309092\\
459	0.999993638428737\\
460	0.999993901397882\\
461	0.999994153533803\\
462	0.999994396840108\\
463	0.999994632866485\\
464	0.99999486264545\\
465	0.999995086701682\\
466	0.99999530511875\\
467	0.999995517642893\\
468	0.999995723803084\\
469	0.999995923029409\\
470	0.9999961147564\\
471	0.999996298503147\\
472	0.999996473926825\\
473	0.999996514515286\\
474	0.999996069897115\\
475	0.999994552351066\\
476	0.999991206454048\\
477	0.999985234360511\\
478	0.999975942234953\\
479	0.999962877474267\\
480	0.999945910891014\\
481	0.999925253645177\\
482	0.999901418649191\\
483	0.999875603816546\\
484	0.99984956305564\\
485	0.999825272861038\\
486	0.999804582159654\\
487	0.999788942899025\\
488	0.999779256763201\\
489	0.999775834885122\\
490	0.999778447078218\\
491	0.999786429620217\\
492	0.999798821359922\\
493	0.999814503201776\\
494	0.999832323161529\\
495	0.999851196387164\\
496	0.999870175752557\\
497	0.999888493369912\\
498	0.99990557653466\\
499	0.999921043360744\\
500	0.999934683959451\\
501	0.999946432778514\\
502	0.999956336954255\\
503	0.99996452449727\\
504	0.99997117503132\\
505	0.999976494774572\\
506	0.999980696576041\\
507	0.999983985137006\\
508	0.99998654706304\\
509	0.999988545090403\\
510	0.99999011568138\\
511	0.999991369151696\\
512	0.999992391544658\\
513	0.999993247569659\\
514	0.999993984051312\\
515	0.999994633470008\\
516	0.999995217301267\\
517	0.999995748971414\\
518	0.999996236336676\\
519	0.999996683660768\\
520	0.999997093113713\\
521	0.999997465844681\\
522	0.999997802697391\\
523	0.999998104641421\\
524	0.99999837299006\\
525	0.999998609467694\\
526	0.999998816179602\\
527	0.999998995526054\\
528	0.999999150091985\\
529	0.999999282534062\\
530	0.999999395479016\\
531	0.999999491440825\\
532	0.9999995727597\\
533	0.999999641562565\\
534	0.999999699742748\\
535	0.999999748955481\\
536	0.999999790625475\\
537	0.999999825962906\\
538	0.999999855984604\\
539	0.999999881537738\\
540	0.999999903323957\\
541	0.99999992192249\\
542	0.99999993781126\\
543	0.999999951385468\\
544	0.999999962973448\\
545	0.999999972849849\\
546	0.999999981246323\\
547	0.999999988360027\\
548	0.999999994360263\\
549	0.999999999393599\\
550	1.00000000358778\\
551	1.00000000705472\\
552	1.00000000989274\\
553	1.00000001218843\\
554	1.00000001401794\\
555	1.00000001544826\\
556	1.00000001653811\\
557	1.00000001733883\\
558	1.00000001789516\\
559	1.00000001824595\\
560	1.00000001842478\\
561	1.00000001846059\\
562	1.00000001837821\\
563	1.0000000181989\\
564	1.00000001794079\\
565	1.00000001761928\\
566	1.00000001724745\\
567	1.00000001683635\\
568	1.00000001639533\\
569	1.00000001593225\\
570	1.0000000154537\\
571	1.00000001496524\\
572	1.00000001447148\\
573	1.00000001397627\\
574	1.0000000134828\\
575	1.0000000129937\\
576	1.00000001251113\\
577	1.00000001203682\\
578	1.00000001157217\\
579	1.0000000111183\\
580	1.00000001067604\\
581	1.00000001024604\\
582	1.00000000982875\\
583	1.00000000942449\\
584	1.00000000903342\\
585	1.00000000865561\\
586	1.00000000829105\\
587	1.00000000793963\\
588	1.00000000760119\\
589	1.00000000727553\\
590	1.0000000069624\\
591	1.00000000666151\\
592	1.00000000637255\\
593	1.00000000609521\\
594	1.00000000582914\\
595	1.00000000557399\\
596	1.00000000532942\\
597	1.00000000509505\\
598	1.00000000487055\\
599	1.00000000465555\\
600	1.0000000044497\\
};
\addlegendentry{$y(k)$}

\addplot[const plot, color=mycolor2, dashed] table[row sep=crcr] {%
1	0\\
2	0\\
3	0\\
4	0\\
5	0\\
6	0\\
7	0\\
8	0\\
9	0\\
10	0\\
11	0\\
12	0\\
13	0\\
14	0\\
15	0\\
16	0\\
17	0\\
18	0\\
19	0\\
20	0\\
21	0\\
22	0\\
23	0\\
24	0\\
25	0\\
26	0\\
27	0\\
28	0\\
29	0\\
30	0\\
31	0\\
32	0\\
33	0\\
34	0\\
35	0\\
36	0\\
37	0\\
38	0\\
39	0\\
40	0\\
41	0\\
42	0\\
43	0\\
44	0\\
45	0\\
46	0\\
47	0\\
48	0\\
49	0\\
50	0\\
51	0\\
52	0\\
53	0\\
54	0\\
55	0\\
56	0\\
57	0\\
58	0\\
59	0\\
60	0\\
61	0\\
62	0\\
63	0\\
64	0\\
65	0\\
66	0\\
67	0\\
68	0\\
69	0\\
70	0\\
71	0\\
72	0\\
73	0\\
74	0\\
75	0\\
76	0\\
77	0\\
78	0\\
79	0\\
80	0\\
81	0\\
82	0\\
83	0\\
84	0\\
85	0\\
86	0\\
87	0\\
88	0\\
89	0\\
90	0\\
91	0\\
92	0\\
93	0\\
94	0\\
95	0\\
96	0\\
97	0\\
98	0\\
99	0\\
100	1\\
101	1\\
102	1\\
103	1\\
104	1\\
105	1\\
106	1\\
107	1\\
108	1\\
109	1\\
110	1\\
111	1\\
112	1\\
113	1\\
114	1\\
115	1\\
116	1\\
117	1\\
118	1\\
119	1\\
120	1\\
121	1\\
122	1\\
123	1\\
124	1\\
125	1\\
126	1\\
127	1\\
128	1\\
129	1\\
130	1\\
131	1\\
132	1\\
133	1\\
134	1\\
135	1\\
136	1\\
137	1\\
138	1\\
139	1\\
140	1\\
141	1\\
142	1\\
143	1\\
144	1\\
145	1\\
146	1\\
147	1\\
148	1\\
149	1\\
150	1\\
151	1\\
152	1\\
153	1\\
154	1\\
155	1\\
156	1\\
157	1\\
158	1\\
159	1\\
160	1\\
161	1\\
162	1\\
163	1\\
164	1\\
165	1\\
166	1\\
167	1\\
168	1\\
169	1\\
170	1\\
171	1\\
172	1\\
173	1\\
174	1\\
175	1\\
176	1\\
177	1\\
178	1\\
179	1\\
180	1\\
181	1\\
182	1\\
183	1\\
184	1\\
185	1\\
186	1\\
187	1\\
188	1\\
189	1\\
190	1\\
191	1\\
192	1\\
193	1\\
194	1\\
195	1\\
196	1\\
197	1\\
198	1\\
199	1\\
200	1\\
201	1\\
202	1\\
203	1\\
204	1\\
205	1\\
206	1\\
207	1\\
208	1\\
209	1\\
210	1\\
211	1\\
212	1\\
213	1\\
214	1\\
215	1\\
216	1\\
217	1\\
218	1\\
219	1\\
220	1\\
221	1\\
222	1\\
223	1\\
224	1\\
225	1\\
226	1\\
227	1\\
228	1\\
229	1\\
230	1\\
231	1\\
232	1\\
233	1\\
234	1\\
235	1\\
236	1\\
237	1\\
238	1\\
239	1\\
240	1\\
241	1\\
242	1\\
243	1\\
244	1\\
245	1\\
246	1\\
247	1\\
248	1\\
249	1\\
250	1\\
251	1\\
252	1\\
253	1\\
254	1\\
255	1\\
256	1\\
257	1\\
258	1\\
259	1\\
260	1\\
261	1\\
262	1\\
263	1\\
264	1\\
265	1\\
266	1\\
267	1\\
268	1\\
269	1\\
270	1\\
271	1\\
272	1\\
273	1\\
274	1\\
275	1\\
276	1\\
277	1\\
278	1\\
279	1\\
280	1\\
281	1\\
282	1\\
283	1\\
284	1\\
285	1\\
286	1\\
287	1\\
288	1\\
289	1\\
290	1\\
291	1\\
292	1\\
293	1\\
294	1\\
295	1\\
296	1\\
297	1\\
298	1\\
299	1\\
300	1\\
301	1\\
302	1\\
303	1\\
304	1\\
305	1\\
306	1\\
307	1\\
308	1\\
309	1\\
310	1\\
311	1\\
312	1\\
313	1\\
314	1\\
315	1\\
316	1\\
317	1\\
318	1\\
319	1\\
320	1\\
321	1\\
322	1\\
323	1\\
324	1\\
325	1\\
326	1\\
327	1\\
328	1\\
329	1\\
330	1\\
331	1\\
332	1\\
333	1\\
334	1\\
335	1\\
336	1\\
337	1\\
338	1\\
339	1\\
340	1\\
341	1\\
342	1\\
343	1\\
344	1\\
345	1\\
346	1\\
347	1\\
348	1\\
349	1\\
350	1\\
351	1\\
352	1\\
353	1\\
354	1\\
355	1\\
356	1\\
357	1\\
358	1\\
359	1\\
360	1\\
361	1\\
362	1\\
363	1\\
364	1\\
365	1\\
366	1\\
367	1\\
368	1\\
369	1\\
370	1\\
371	1\\
372	1\\
373	1\\
374	1\\
375	1\\
376	1\\
377	1\\
378	1\\
379	1\\
380	1\\
381	1\\
382	1\\
383	1\\
384	1\\
385	1\\
386	1\\
387	1\\
388	1\\
389	1\\
390	1\\
391	1\\
392	1\\
393	1\\
394	1\\
395	1\\
396	1\\
397	1\\
398	1\\
399	1\\
400	1\\
401	1\\
402	1\\
403	1\\
404	1\\
405	1\\
406	1\\
407	1\\
408	1\\
409	1\\
410	1\\
411	1\\
412	1\\
413	1\\
414	1\\
415	1\\
416	1\\
417	1\\
418	1\\
419	1\\
420	1\\
421	1\\
422	1\\
423	1\\
424	1\\
425	1\\
426	1\\
427	1\\
428	1\\
429	1\\
430	1\\
431	1\\
432	1\\
433	1\\
434	1\\
435	1\\
436	1\\
437	1\\
438	1\\
439	1\\
440	1\\
441	1\\
442	1\\
443	1\\
444	1\\
445	1\\
446	1\\
447	1\\
448	1\\
449	1\\
450	1\\
451	1\\
452	1\\
453	1\\
454	1\\
455	1\\
456	1\\
457	1\\
458	1\\
459	1\\
460	1\\
461	1\\
462	1\\
463	1\\
464	1\\
465	1\\
466	1\\
467	1\\
468	1\\
469	1\\
470	1\\
471	1\\
472	1\\
473	1\\
474	1\\
475	1\\
476	1\\
477	1\\
478	1\\
479	1\\
480	1\\
481	1\\
482	1\\
483	1\\
484	1\\
485	1\\
486	1\\
487	1\\
488	1\\
489	1\\
490	1\\
491	1\\
492	1\\
493	1\\
494	1\\
495	1\\
496	1\\
497	1\\
498	1\\
499	1\\
500	1\\
501	1\\
502	1\\
503	1\\
504	1\\
505	1\\
506	1\\
507	1\\
508	1\\
509	1\\
510	1\\
511	1\\
512	1\\
513	1\\
514	1\\
515	1\\
516	1\\
517	1\\
518	1\\
519	1\\
520	1\\
521	1\\
522	1\\
523	1\\
524	1\\
525	1\\
526	1\\
527	1\\
528	1\\
529	1\\
530	1\\
531	1\\
532	1\\
533	1\\
534	1\\
535	1\\
536	1\\
537	1\\
538	1\\
539	1\\
540	1\\
541	1\\
542	1\\
543	1\\
544	1\\
545	1\\
546	1\\
547	1\\
548	1\\
549	1\\
550	1\\
551	1\\
552	1\\
553	1\\
554	1\\
555	1\\
556	1\\
557	1\\
558	1\\
559	1\\
560	1\\
561	1\\
562	1\\
563	1\\
564	1\\
565	1\\
566	1\\
567	1\\
568	1\\
569	1\\
570	1\\
571	1\\
572	1\\
573	1\\
574	1\\
575	1\\
576	1\\
577	1\\
578	1\\
579	1\\
580	1\\
581	1\\
582	1\\
583	1\\
584	1\\
585	1\\
586	1\\
587	1\\
588	1\\
589	1\\
590	1\\
591	1\\
592	1\\
593	1\\
594	1\\
595	1\\
596	1\\
597	1\\
598	1\\
599	1\\
600	1\\
};
\addlegendentry{$y^{\mathrm{zad}}(k)$}

\end{axis}
\end{tikzpicture}%
	\caption{Przebiegi sygna��w dla regulacji przy pomocy algorytmu DMC bez uwzgl�dnienia zak��ce�}
\end{figure}

Wska�nik ilo�ciowy: 11.2706

\begin{figure}[h!]
	\centering
	% This file was created by matlab2tikz.
%
\definecolor{mycolor1}{rgb}{0.00000,0.44700,0.74100}%
\definecolor{mycolor2}{rgb}{0.85000,0.32500,0.09800}%
%
\begin{tikzpicture}

\begin{axis}[%
width=4.521in,
height=1.493in,
at={(0.758in,2.554in)},
scale only axis,
xmin=0,
xmax=600,
xlabel style={font=\color{white!15!black}},
xlabel={$k$},
ymin=-4,
ymax=3,
axis background/.style={fill=white},
title style={font=\bfseries},
title={Sygna� wej�ciowy sterowania i zak��cenia},
xmajorgrids,
ymajorgrids,
legend style={legend cell align=left, align=left, draw=white!15!black}
]
\addplot[const plot, color=mycolor1] table[row sep=crcr] {%
1	0\\
2	0\\
3	0\\
4	0\\
5	0\\
6	0\\
7	0\\
8	0\\
9	0\\
10	0\\
11	0\\
12	0\\
13	0\\
14	0\\
15	0\\
16	0\\
17	0\\
18	0\\
19	0\\
20	0\\
21	0\\
22	0\\
23	0\\
24	0\\
25	0\\
26	0\\
27	0\\
28	0\\
29	0\\
30	0\\
31	0\\
32	0\\
33	0\\
34	0\\
35	0\\
36	0\\
37	0\\
38	0\\
39	0\\
40	0\\
41	0\\
42	0\\
43	0\\
44	0\\
45	0\\
46	0\\
47	0\\
48	0\\
49	0\\
50	0\\
51	0\\
52	0\\
53	0\\
54	0\\
55	0\\
56	0\\
57	0\\
58	0\\
59	0\\
60	0\\
61	0\\
62	0\\
63	0\\
64	0\\
65	0\\
66	0\\
67	0\\
68	0\\
69	0\\
70	0\\
71	0\\
72	0\\
73	0\\
74	0\\
75	0\\
76	0\\
77	0\\
78	0\\
79	0\\
80	0\\
81	0\\
82	0\\
83	0\\
84	0\\
85	0\\
86	0\\
87	0\\
88	0\\
89	0\\
90	0\\
91	0\\
92	0\\
93	0\\
94	0\\
95	0\\
96	0\\
97	0\\
98	0\\
99	0\\
100	1.06438689272261\\
101	1.67330799970846\\
102	1.94630786721308\\
103	1.9873867877016\\
104	1.88160065495093\\
105	1.69466404103899\\
106	1.47444334123278\\
107	1.2534827815949\\
108	1.05194116533132\\
109	0.88051830203168\\
110	0.743112059095731\\
111	0.639070284192932\\
112	0.564990187660185\\
113	0.516076581765213\\
114	0.487105661396056\\
115	0.47305860386581\\
116	0.469494351726138\\
117	0.472727848975606\\
118	0.479872134885331\\
119	0.488792593525579\\
120	0.498011076122221\\
121	0.506587690199236\\
122	0.513999424293346\\
123	0.520027728514665\\
124	0.524661729730118\\
125	0.528019799977936\\
126	0.530289507139993\\
127	0.531684313584522\\
128	0.532414508349938\\
129	0.532669540981234\\
130	0.53260898363768\\
131	0.532359633484115\\
132	0.532016667214764\\
133	0.531647194123235\\
134	0.531294971082772\\
135	0.53098541140078\\
136	0.530730325128121\\
137	0.530532067662125\\
138	0.530386950263108\\
139	0.530287888418633\\
140	0.530226341679766\\
141	0.53019364174679\\
142	0.530181823531268\\
143	0.530184074694741\\
144	0.530194909262603\\
145	0.530210155308029\\
146	0.530226828978867\\
147	0.530242949702892\\
148	0.530257335712222\\
149	0.530269405819655\\
150	0.530279002893992\\
151	0.530286246619096\\
152	0.530291417585385\\
153	0.530294871160459\\
154	0.53029697750129\\
155	0.530298083110692\\
156	0.5302984891616\\
157	0.530298442131522\\
158	0.53029813288721\\
159	0.530297701075738\\
160	0.53029724240299\\
161	0.5302968170457\\
162	0.530296458011551\\
163	0.530296178719471\\
164	0.530295979420275\\
165	0.530295852326937\\
166	0.530295785489372\\
167	0.530295765547805\\
168	0.530295779548015\\
169	0.530295816015713\\
170	0.53029586547824\\
171	0.530295920599273\\
172	0.530295976063461\\
173	0.530296028317908\\
174	0.530296075249234\\
175	0.530296115850542\\
176	0.530296149912657\\
177	0.530296177758576\\
178	0.530296200028913\\
179	0.530296217518459\\
180	0.530296231059335\\
181	0.530296241443675\\
182	0.530296249377924\\
183	0.530296255460952\\
184	0.530296260179045\\
185	0.530296263911903\\
186	0.530296266945044\\
187	0.530296269485158\\
188	0.530296271676013\\
189	0.530296273613349\\
190	0.5302962753579\\
191	0.530296276946142\\
192	0.53029627839874\\
193	0.530296279726851\\
194	0.530296280936599\\
195	0.530296282032038\\
196	0.53029628301696\\
197	0.530296283895852\\
198	0.530296284674259\\
199	0.530296285358785\\
200	0.53029628595687\\
201	0.530296286476478\\
202	0.530296286925765\\
203	0.530296287312782\\
204	0.53029628764523\\
205	0.530296287930287\\
206	0.530296288174488\\
207	0.530296288383675\\
208	0.530296288562982\\
209	0.530296288716863\\
210	0.530296288849134\\
211	0.530296288963043\\
212	0.530296289061329\\
213	0.530296289146296\\
214	0.530296289219876\\
215	0.530296289283692\\
216	0.530296289339104\\
217	0.530296289387264\\
218	0.530296289429144\\
219	0.530296289465574\\
220	0.530296289497263\\
221	0.530296289524822\\
222	0.530296289548782\\
223	0.530296289569602\\
224	0.530296289587684\\
225	0.530296289603381\\
226	0.530296289617\\
227	0.53029628962881\\
228	0.530296289639049\\
229	0.530296289647923\\
230	0.530296289655612\\
231	0.530296289662274\\
232	0.530296289668046\\
233	0.530296289673046\\
234	0.530296289677378\\
235	0.53029628968113\\
236	0.53029628968438\\
237	0.530296289687193\\
238	0.530296289689628\\
239	0.530296289691737\\
240	0.530296289693563\\
241	0.530296289695144\\
242	0.530296289696515\\
243	0.530296289697705\\
244	0.530296289698736\\
245	0.530296289699632\\
246	0.53029628970041\\
247	0.530296289701086\\
248	0.530296289701672\\
249	0.53029628970218\\
250	0.530296289702619\\
251	0.530296289702997\\
252	0.530296289703324\\
253	0.530296289703606\\
254	0.530296289703851\\
255	0.530296289704065\\
256	0.530296289704251\\
257	0.530296289704414\\
258	0.530296289704558\\
259	0.530296289704683\\
260	0.530296289704794\\
261	0.530296289704891\\
262	0.530296289704976\\
263	0.530296289705049\\
264	0.530296289705111\\
265	0.530296289705163\\
266	0.530306508427582\\
267	0.53033666381662\\
268	0.530391127427623\\
269	0.530467847736317\\
270	0.530559386950657\\
271	0.530654876638281\\
272	0.530742108994647\\
273	0.530811305149986\\
274	0.530856665729264\\
275	0.530876482260541\\
276	0.530835314532807\\
277	0.530733501687177\\
278	0.530589437566782\\
279	0.530427392420687\\
280	0.530270030302657\\
281	0.530134556578039\\
282	0.530031433039188\\
283	0.529964727279454\\
284	0.529933350393245\\
285	0.529932636565899\\
286	0.529955901281582\\
287	0.529995766900072\\
288	0.530045160082298\\
289	0.530097966103479\\
290	0.530149375192492\\
291	0.530195981986413\\
292	0.53023570754184\\
293	0.530267610057856\\
294	0.530291640471443\\
295	0.530308386143677\\
296	0.530318832592561\\
297	0.530324161292871\\
298	0.530325591820423\\
299	0.530324269357558\\
300	0.530321193702817\\
301	0.530317183116816\\
302	-0.372294877081476\\
303	-1.08002405564981\\
304	-1.43097561558473\\
305	-1.5291955829421\\
306	-1.46176178421793\\
307	-1.29743282195591\\
308	-1.08746241872302\\
309	-0.867677328802277\\
310	-0.66114846155498\\
311	-0.480988012759194\\
312	-0.332972825381776\\
313	-0.217824519280358\\
314	-0.133072623160248\\
315	-0.0744926566323954\\
316	-0.0371524078411757\\
317	-0.0161221618736558\\
318	-0.00691352126912582\\
319	-0.00571113554804939\\
320	-0.00945567448849594\\
321	-0.0158274420385783\\
322	-0.0231700683578626\\
323	-0.0303840225536514\\
324	-0.0368110285059847\\
325	-0.0421232233675043\\
326	-0.0462251817285526\\
327	-0.0491726680825173\\
328	-0.051109002015993\\
329	-0.0522180017427602\\
330	-0.0526913794435688\\
331	-0.0527079783444793\\
332	-0.0524221772384589\\
333	-0.0519589886357501\\
334	-0.051413722667788\\
335	-0.0508544935573386\\
336	-0.0503262500805057\\
337	-0.0498553796667761\\
338	-0.0494542484548538\\
339	-0.0491252899065846\\
340	-0.048864443861755\\
341	-0.0486638825279862\\
342	-0.0485140486660657\\
343	-0.0483919996883374\\
344	-0.0482870808504886\\
345	-0.0481962054567241\\
346	-0.0481203820566373\\
347	-0.0480623071448293\\
348	-0.048024817727156\\
349	-0.048010001396777\\
350	-0.0480162318811517\\
351	-0.0480398526923105\\
352	-0.0480763836407007\\
353	-0.0481688576397127\\
354	-0.04827790538555\\
355	-0.0483790136639778\\
356	-0.0484588743993195\\
357	-0.0485121460294292\\
358	-0.0485387575190394\\
359	-0.0485417871082756\\
360	-0.0485258843100861\\
361	-0.0484961671933859\\
362	-0.048457510533373\\
363	-0.0484141378020655\\
364	-0.048369436152772\\
365	-0.0483259246083653\\
366	-0.0482853187396293\\
367	-0.0482486482715858\\
368	-0.0482163960744009\\
369	-0.0481886372367196\\
370	-0.0481651651445453\\
371	-0.0481455977317221\\
372	-0.0481294615274074\\
373	-0.0481162540844175\\
374	-0.0481054871393597\\
375	-0.0480967137289445\\
376	-0.0480895427309549\\
377	-0.0480836441325986\\
378	-0.0480787479256835\\
379	-0.0480746390148894\\
380	-0.0480711499909998\\
381	-0.0480681531218575\\
382	-0.0480655524817439\\
383	-0.0480632767883314\\
384	-0.0480612732463943\\
385	-0.0480595025023705\\
386	-0.048057934682741\\
387	-0.0480565464095204\\
388	-0.0480553186455604\\
389	-0.0480542352096962\\
390	-0.0480532818075707\\
391	-0.0480524454407791\\
392	-0.048051714079136\\
393	-0.0480510765044236\\
394	-0.0480505222563485\\
395	-0.0480500416311027\\
396	-0.0480496256992204\\
397	-0.0480492663221965\\
398	-0.0480489561568428\\
399	-0.0480486886430028\\
400	-0.0480484579745756\\
401	-0.0480482590563171\\
402	-0.0480480874500899\\
403	-0.0480479393145299\\
404	-0.0480478113418209\\
405	-0.048047700694687\\
406	-0.0480476049459863\\
407	-0.0480475220225759\\
408	-0.0480474501544579\\
409	-0.0480473878296771\\
410	-0.0480473337550245\\
411	-0.048047286822291\\
412	-0.0480472460796277\\
413	-0.0480472107074583\\
414	-0.0480471799983487\\
415	-0.048047153340248\\
416	-0.0480471302025516\\
417	-0.0480471101244954\\
418	-0.0480470927054576\\
419	-0.0480470775968129\\
420	-0.048047064495043\\
421	-0.0480470531358682\\
422	-0.0480470432892138\\
423	-0.0480470347548605\\
424	-0.0480470273586655\\
425	-0.0480470209492642\\
426	-0.0480470153951797\\
427	-0.0480470105822873\\
428	-0.0480470064115863\\
429	-0.0480470027972433\\
430	-0.0480469996648756\\
431	-0.0480469969500468\\
432	-0.0480469944988457\\
433	-0.0480469919362663\\
434	-0.0480469885923458\\
435	-0.0480469835397945\\
436	-0.0480469757475086\\
437	-0.0480469643157548\\
438	-0.0480469487391809\\
439	-0.04804692910405\\
440	-0.04804690615568\\
441	-0.0480468812213116\\
442	-0.0480468567287653\\
443	-0.0480468362885626\\
444	-0.0480468241721764\\
445	-0.0480468244147711\\
446	-0.0480468398686124\\
447	-0.0480468714745555\\
448	-0.0480469179006733\\
449	-0.0480469757170046\\
450	-0.0480470400011052\\
451	-0.0480471051647061\\
452	-0.0480471644967559\\
453	-0.0480472113383911\\
454	-0.0480472404972731\\
455	-0.0480472492185785\\
456	-0.0480472375035532\\
457	-0.0480472078416173\\
458	-0.0480471645481744\\
459	-0.0480471129267551\\
460	-0.048047058442813\\
461	-0.0480470060393124\\
462	-0.0480469596628626\\
463	-0.048046922016355\\
464	-0.0480468945157434\\
465	-0.0480468774058331\\
466	-0.048046869980881\\
467	-0.0480468708571422\\
468	-0.0480555438026794\\
469	-0.0480829648998093\\
470	-0.048134682205252\\
471	-0.0482099138497231\\
472	-0.0483021729480712\\
473	-0.0484010498395163\\
474	-0.0484942928155748\\
475	-0.0485713597784899\\
476	-0.0486254130746451\\
477	-0.0486537302012708\\
478	-0.0486256968397673\\
479	-0.0485349944986556\\
480	-0.0483967480952487\\
481	-0.0482343885746553\\
482	-0.0480712769381486\\
483	-0.0479261148613076\\
484	-0.0478111119491141\\
485	-0.0477319712141301\\
486	-0.0476889205267906\\
487	-0.0476782080566769\\
488	-0.0476936617917344\\
489	-0.0477280690630911\\
490	-0.0477742540074655\\
491	-0.0478258184554178\\
492	-0.047877568438591\\
493	-0.0479256800523871\\
494	-0.0479676710880148\\
495	-0.0480022446087504\\
496	-0.0480290625615232\\
497	-0.0480484955602727\\
498	-0.0480613820158236\\
499	-0.048068817687936\\
500	-0.0480719865535584\\
501	-0.0480720360597749\\
502	-0.0480699943715776\\
503	-0.0480667238790333\\
504	-0.0480629036043758\\
505	-0.0480590328128582\\
506	-0.0480554486676769\\
507	-0.0480523518215963\\
508	-0.0480498351225735\\
509	-0.0480479117952373\\
510	-0.0480465409748008\\
511	-0.0480456493358259\\
512	-0.0480451482764288\\
513	-0.0480449466639503\\
514	-0.0480449595225782\\
515	-0.0480451132658068\\
516	-0.0480453481260021\\
517	-0.0480456185031388\\
518	-0.0480458919174768\\
519	-0.0480461480649655\\
520	-0.0480463762779526\\
521	-0.048046572595876\\
522	-0.0480467371262878\\
523	-0.0480468720007338\\
524	-0.0480469799898032\\
525	-0.0480470637055994\\
526	-0.0480471254358842\\
527	-0.0480471672549414\\
528	-0.0480471912006894\\
529	-0.048047197753486\\
530	-0.0480471879098904\\
531	-0.0480471638815885\\
532	-0.0480471290028124\\
533	-0.0480470872683302\\
534	-0.0480470427802137\\
535	-0.0480469992652189\\
536	-0.0480469597382747\\
537	-0.0480469263291239\\
538	-0.0480469002540332\\
539	-0.0480468818972129\\
540	-0.0480468709618108\\
541	-0.0480468666534392\\
542	-0.0480468678664719\\
543	-0.048046873352128\\
544	-0.0480468818558728\\
545	-0.0480468922188852\\
546	-0.0480469034438219\\
547	-0.0480469147287842\\
548	-0.0480469254754531\\
549	-0.048046935278123\\
550	-0.0480469439001749\\
551	-0.0480469512437364\\
552	-0.0480469573171564\\
553	-0.0480469622037059\\
554	-0.0480469660337571\\
555	-0.0480469689616889\\
556	-0.0480469711479658\\
557	-0.0480469727462494\\
558	-0.0480469738950161\\
559	-0.0480469747129382\\
560	-0.0480469752972071\\
561	-0.0480469757239918\\
562	-0.0480469760503099\\
563	-0.048046976316707\\
564	-0.0480469765502672\\
565	-0.0480469767676111\\
566	-0.0480469769776487\\
567	-0.0480469771839546\\
568	-0.0480469773867053\\
569	-0.0480469775841783\\
570	-0.0480469777738434\\
571	-0.0480469779531038\\
572	-0.0480469781197502\\
573	-0.0480469782721957\\
574	-0.048046978409551\\
575	-0.0480469785315937\\
576	-0.0480469786386723\\
577	-0.0480469787315812\\
578	-0.0480469788114269\\
579	-0.0480469788795042\\
580	-0.0480469789371893\\
581	-0.0480469789858544\\
582	-0.0480469790268054\\
583	-0.0480469790612402\\
584	-0.0480469790902254\\
585	-0.0480469791146866\\
586	-0.0480469791354098\\
587	-0.0480469791530503\\
588	-0.0480469791681465\\
589	-0.0480469791811354\\
590	-0.0480469791923681\\
591	-0.0480469792021254\\
592	-0.0480469792106318\\
593	-0.0480469792180673\\
594	-0.0480469792245776\\
595	-0.0480469792302822\\
596	-0.0480469792352806\\
597	-0.0480469792396572\\
598	-0.0480469792434841\\
599	-0.0480469792468194\\
600	-0.0480469792497025\\
};
\addlegendentry{$u(k)$}

\addplot[const plot, color=mycolor2] table[row sep=crcr] {%
1	0\\
2	0\\
3	0\\
4	0\\
5	0\\
6	0\\
7	0\\
8	0\\
9	0\\
10	0\\
11	0\\
12	0\\
13	0\\
14	0\\
15	0\\
16	0\\
17	0\\
18	0\\
19	0\\
20	0\\
21	0\\
22	0\\
23	0\\
24	0\\
25	0\\
26	0\\
27	0\\
28	0\\
29	0\\
30	0\\
31	0\\
32	0\\
33	0\\
34	0\\
35	0\\
36	0\\
37	0\\
38	0\\
39	0\\
40	0\\
41	0\\
42	0\\
43	0\\
44	0\\
45	0\\
46	0\\
47	0\\
48	0\\
49	0\\
50	0\\
51	0\\
52	0\\
53	0\\
54	0\\
55	0\\
56	0\\
57	0\\
58	0\\
59	0\\
60	0\\
61	0\\
62	0\\
63	0\\
64	0\\
65	0\\
66	0\\
67	0\\
68	0\\
69	0\\
70	0\\
71	0\\
72	0\\
73	0\\
74	0\\
75	0\\
76	0\\
77	0\\
78	0\\
79	0\\
80	0\\
81	0\\
82	0\\
83	0\\
84	0\\
85	0\\
86	0\\
87	0\\
88	0\\
89	0\\
90	0\\
91	0\\
92	0\\
93	0\\
94	0\\
95	0\\
96	0\\
97	0\\
98	0\\
99	0\\
100	0\\
101	0\\
102	0\\
103	0\\
104	0\\
105	0\\
106	0\\
107	0\\
108	0\\
109	0\\
110	0\\
111	0\\
112	0\\
113	0\\
114	0\\
115	0\\
116	0\\
117	0\\
118	0\\
119	0\\
120	0\\
121	0\\
122	0\\
123	0\\
124	0\\
125	0\\
126	0\\
127	0\\
128	0\\
129	0\\
130	0\\
131	0\\
132	0\\
133	0\\
134	0\\
135	0\\
136	0\\
137	0\\
138	0\\
139	0\\
140	0\\
141	0\\
142	0\\
143	0\\
144	0\\
145	0\\
146	0\\
147	0\\
148	0\\
149	0\\
150	0\\
151	0\\
152	0\\
153	0\\
154	0\\
155	0\\
156	0\\
157	0\\
158	0\\
159	0\\
160	0\\
161	0\\
162	0\\
163	0\\
164	0\\
165	0\\
166	0\\
167	0\\
168	0\\
169	0\\
170	0\\
171	0\\
172	0\\
173	0\\
174	0\\
175	0\\
176	0\\
177	0\\
178	0\\
179	0\\
180	0\\
181	0\\
182	0\\
183	0\\
184	0\\
185	0\\
186	0\\
187	0\\
188	0\\
189	0\\
190	0\\
191	0\\
192	0\\
193	0\\
194	0\\
195	0\\
196	0\\
197	0\\
198	0\\
199	0\\
200	0\\
201	0\\
202	0\\
203	0\\
204	0\\
205	0\\
206	0\\
207	0\\
208	0\\
209	0\\
210	0\\
211	0\\
212	0\\
213	0\\
214	0\\
215	0\\
216	0\\
217	0\\
218	0\\
219	0\\
220	0\\
221	0\\
222	0\\
223	0\\
224	0\\
225	0\\
226	0\\
227	0\\
228	0\\
229	0\\
230	0\\
231	0\\
232	0\\
233	0\\
234	0\\
235	0\\
236	0\\
237	0\\
238	0\\
239	0\\
240	0\\
241	0\\
242	0\\
243	0\\
244	0\\
245	0\\
246	0\\
247	0\\
248	0\\
249	0\\
250	0\\
251	0\\
252	0\\
253	0\\
254	0\\
255	0\\
256	0\\
257	0\\
258	0\\
259	0\\
260	0\\
261	0\\
262	0\\
263	0\\
264	0\\
265	0\\
266	0\\
267	0\\
268	0\\
269	0\\
270	0\\
271	0\\
272	0\\
273	0\\
274	0\\
275	0\\
276	0\\
277	0\\
278	0\\
279	0\\
280	0\\
281	0\\
282	0\\
283	0\\
284	0\\
285	0\\
286	0\\
287	0\\
288	0\\
289	0\\
290	0\\
291	0\\
292	0\\
293	0\\
294	0\\
295	0\\
296	0\\
297	0\\
298	0\\
299	0\\
300	1\\
301	1\\
302	1\\
303	1\\
304	1\\
305	1\\
306	1\\
307	1\\
308	1\\
309	1\\
310	1\\
311	1\\
312	1\\
313	1\\
314	1\\
315	1\\
316	1\\
317	1\\
318	1\\
319	1\\
320	1\\
321	1\\
322	1\\
323	1\\
324	1\\
325	1\\
326	1\\
327	1\\
328	1\\
329	1\\
330	1\\
331	1\\
332	1\\
333	1\\
334	1\\
335	1\\
336	1\\
337	1\\
338	1\\
339	1\\
340	1\\
341	1\\
342	1\\
343	1\\
344	1\\
345	1\\
346	1\\
347	1\\
348	1\\
349	1\\
350	1\\
351	1\\
352	1\\
353	1\\
354	1\\
355	1\\
356	1\\
357	1\\
358	1\\
359	1\\
360	1\\
361	1\\
362	1\\
363	1\\
364	1\\
365	1\\
366	1\\
367	1\\
368	1\\
369	1\\
370	1\\
371	1\\
372	1\\
373	1\\
374	1\\
375	1\\
376	1\\
377	1\\
378	1\\
379	1\\
380	1\\
381	1\\
382	1\\
383	1\\
384	1\\
385	1\\
386	1\\
387	1\\
388	1\\
389	1\\
390	1\\
391	1\\
392	1\\
393	1\\
394	1\\
395	1\\
396	1\\
397	1\\
398	1\\
399	1\\
400	1\\
401	1\\
402	1\\
403	1\\
404	1\\
405	1\\
406	1\\
407	1\\
408	1\\
409	1\\
410	1\\
411	1\\
412	1\\
413	1\\
414	1\\
415	1\\
416	1\\
417	1\\
418	1\\
419	1\\
420	1\\
421	1\\
422	1\\
423	1\\
424	1\\
425	1\\
426	1\\
427	1\\
428	1\\
429	1\\
430	1\\
431	1\\
432	1\\
433	1\\
434	1\\
435	1\\
436	1\\
437	1\\
438	1\\
439	1\\
440	1\\
441	1\\
442	1\\
443	1\\
444	1\\
445	1\\
446	1\\
447	1\\
448	1\\
449	1\\
450	1\\
451	1\\
452	1\\
453	1\\
454	1\\
455	1\\
456	1\\
457	1\\
458	1\\
459	1\\
460	1\\
461	1\\
462	1\\
463	1\\
464	1\\
465	1\\
466	1\\
467	1\\
468	1\\
469	1\\
470	1\\
471	1\\
472	1\\
473	1\\
474	1\\
475	1\\
476	1\\
477	1\\
478	1\\
479	1\\
480	1\\
481	1\\
482	1\\
483	1\\
484	1\\
485	1\\
486	1\\
487	1\\
488	1\\
489	1\\
490	1\\
491	1\\
492	1\\
493	1\\
494	1\\
495	1\\
496	1\\
497	1\\
498	1\\
499	1\\
500	1\\
501	1\\
502	1\\
503	1\\
504	1\\
505	1\\
506	1\\
507	1\\
508	1\\
509	1\\
510	1\\
511	1\\
512	1\\
513	1\\
514	1\\
515	1\\
516	1\\
517	1\\
518	1\\
519	1\\
520	1\\
521	1\\
522	1\\
523	1\\
524	1\\
525	1\\
526	1\\
527	1\\
528	1\\
529	1\\
530	1\\
531	1\\
532	1\\
533	1\\
534	1\\
535	1\\
536	1\\
537	1\\
538	1\\
539	1\\
540	1\\
541	1\\
542	1\\
543	1\\
544	1\\
545	1\\
546	1\\
547	1\\
548	1\\
549	1\\
550	1\\
551	1\\
552	1\\
553	1\\
554	1\\
555	1\\
556	1\\
557	1\\
558	1\\
559	1\\
560	1\\
561	1\\
562	1\\
563	1\\
564	1\\
565	1\\
566	1\\
567	1\\
568	1\\
569	1\\
570	1\\
571	1\\
572	1\\
573	1\\
574	1\\
575	1\\
576	1\\
577	1\\
578	1\\
579	1\\
580	1\\
581	1\\
582	1\\
583	1\\
584	1\\
585	1\\
586	1\\
587	1\\
588	1\\
589	1\\
590	1\\
591	1\\
592	1\\
593	1\\
594	1\\
595	1\\
596	1\\
597	1\\
598	1\\
599	1\\
600	1\\
};
\addlegendentry{$z(k)$}

\end{axis}

\begin{axis}[%
width=4.521in,
height=1.493in,
at={(0.758in,0.481in)},
scale only axis,
xmin=0,
xmax=600,
xlabel style={font=\color{white!15!black}},
xlabel={$k$},
ymin=-2,
ymax=3,
axis background/.style={fill=white},
title style={font=\bfseries},
title={Sygna� wyj�ciowy i warto�� zadana},
xmajorgrids,
ymajorgrids,
legend style={legend cell align=left, align=left, draw=white!15!black}
]
\addplot [color=mycolor1]
  table[row sep=crcr]{%
1	0\\
2	0\\
3	0\\
4	0\\
5	0\\
6	0\\
7	0\\
8	0\\
9	0\\
10	0\\
11	0\\
12	0\\
13	0\\
14	0\\
15	0\\
16	0\\
17	0\\
18	0\\
19	0\\
20	0\\
21	0\\
22	0\\
23	0\\
24	0\\
25	0\\
26	0\\
27	0\\
28	0\\
29	0\\
30	0\\
31	0\\
32	0\\
33	0\\
34	0\\
35	0\\
36	0\\
37	0\\
38	0\\
39	0\\
40	0\\
41	0\\
42	0\\
43	0\\
44	0\\
45	0\\
46	0\\
47	0\\
48	0\\
49	0\\
50	0\\
51	0\\
52	0\\
53	0\\
54	0\\
55	0\\
56	0\\
57	0\\
58	0\\
59	0\\
60	0\\
61	0\\
62	0\\
63	0\\
64	0\\
65	0\\
66	0\\
67	0\\
68	0\\
69	0\\
70	0\\
71	0\\
72	0\\
73	0\\
74	0\\
75	0\\
76	0\\
77	0\\
78	0\\
79	0\\
80	0\\
81	0\\
82	0\\
83	0\\
84	0\\
85	0\\
86	0\\
87	0\\
88	0\\
89	0\\
90	0\\
91	0\\
92	0\\
93	0\\
94	0\\
95	0\\
96	0\\
97	0\\
98	0\\
99	0\\
100	0\\
101	0\\
102	0\\
103	0\\
104	0\\
105	0.0898661853525702\\
106	0.227145989536648\\
107	0.381364985455071\\
108	0.532183744977198\\
109	0.667348874413722\\
110	0.780701113190663\\
111	0.870397690541912\\
112	0.937426745782853\\
113	0.984435604898583\\
114	1.01485816163926\\
115	1.03230537220306\\
116	1.04017299290448\\
117	1.04141871252479\\
118	1.03846391027622\\
119	1.03318119527604\\
120	1.0269360330158\\
121	1.02065802748983\\
122	1.01492410818031\\
123	1.01004159145033\\
124	1.0061236992538\\
125	1.00315363327061\\
126	1.00103582376953\\
127	0.999634653341271\\
128	0.998801966526442\\
129	0.99839518271941\\
130	0.998287979047469\\
131	0.998375425366301\\
132	0.998575232142644\\
133	0.998826486044746\\
134	0.99908694793467\\
135	0.999329706050688\\
136	0.99953973184925\\
137	0.999710685296531\\
138	0.999842161319463\\
139	0.9999374561398\\
140	1.00000185537463\\
141	1.00004139817149\\
142	1.00006204635586\\
143	1.00006917833604\\
144	1.00006732902273\\
145	1.00006010503136\\
146	1.00005021573868\\
147	1.00003957308605\\
148	1.00002942486313\\
149	1.00002049668706\\
150	1.00001312659169\\
151	1.00000738295854\\
152	1.00000316156378\\
153	1.00000026100877\\
154	0.999998438024651\\
155	0.999997445379893\\
156	0.999997055639453\\
157	0.999997074052887\\
158	0.99999734357207\\
159	0.999997744558619\\
160	0.999998191239006\\
161	0.999998626470347\\
162	0.999999015933754\\
163	0.999999342496316\\
164	0.999999601184079\\
165	0.999999794984267\\
166	0.999999931536935\\
167	1.00000002067327\\
168	1.00000007269799\\
169	1.0000000972857\\
170	1.00000010285557\\
171	1.00000009629762\\
172	1.00000008294079\\
173	1.00000006667322\\
174	1.00000005014576\\
175	1.00000003500852\\
176	1.00000002214677\\
177	1.00000001189507\\
178	1.0000000042189\\
179	0.999999998859796\\
180	0.999999995445021\\
181	0.999999993565414\\
182	0.999999992826672\\
183	0.999999992879594\\
184	0.999999993434638\\
185	0.99999999426549\\
186	0.999999995205538\\
187	0.999999996140285\\
188	0.999999996997944\\
189	0.999999997739762\\
190	0.999999998351042\\
191	0.999999998833424\\
192	0.99999999919862\\
193	0.999999999463627\\
194	0.999999999647282\\
195	0.999999999767949\\
196	0.999999999842131\\
197	0.999999999883761\\
198	0.999999999903995\\
199	0.999999999911317\\
200	0.999999999911846\\
201	0.999999999909724\\
202	0.999999999907537\\
203	0.999999999906703\\
204	0.999999999907817\\
205	0.999999999910932\\
206	0.999999999915784\\
207	0.999999999921947\\
208	0.999999999928953\\
209	0.999999999936356\\
210	0.99999999994378\\
211	0.999999999950931\\
212	0.999999999957603\\
213	0.999999999963665\\
214	0.999999999969054\\
215	0.999999999973759\\
216	0.999999999977806\\
217	0.999999999981245\\
218	0.999999999984139\\
219	0.999999999986558\\
220	0.999999999988573\\
221	0.999999999990247\\
222	0.999999999991641\\
223	0.999999999992805\\
224	0.999999999993781\\
225	0.999999999994607\\
226	0.999999999995309\\
227	0.999999999995912\\
228	0.999999999996432\\
229	0.999999999996883\\
230	0.999999999997277\\
231	0.999999999997622\\
232	0.999999999997925\\
233	0.999999999998191\\
234	0.999999999998424\\
235	0.999999999998629\\
236	0.999999999998809\\
237	0.999999999998967\\
238	0.999999999999105\\
239	0.999999999999226\\
240	0.999999999999331\\
241	0.999999999999423\\
242	0.999999999999503\\
243	0.999999999999573\\
244	0.999999999999632\\
245	0.999999999999683\\
246	0.999999999999727\\
247	0.999999999999764\\
248	0.999999999999796\\
249	0.999999999999823\\
250	0.999999999999846\\
251	0.999999999999867\\
252	0.999999999999885\\
253	0.9999999999999\\
254	0.999999999999913\\
255	0.999999999999924\\
256	0.999999999999934\\
257	0.999999999999942\\
258	0.999999999999948\\
259	0.999999999999954\\
260	0.999999999999958\\
261	0.999999999999961\\
262	0.999999999999964\\
263	0.999999999999967\\
264	0.999999999999969\\
265	0.999999999999972\\
266	0.999999999999975\\
267	0.999999999999979\\
268	0.999999999999982\\
269	0.999999999999985\\
270	0.999999999999987\\
271	1.00000086276672\\
272	1.00000423317379\\
273	1.0000120519818\\
274	1.0000260003166\\
275	1.00004705639391\\
276	1.00007523709427\\
277	1.00010952763252\\
278	1.00014813255248\\
279	1.00018884650625\\
280	1.00022941796769\\
281	1.00026470340528\\
282	1.00028981685954\\
283	1.0003016430238\\
284	1.00029925476994\\
285	1.00028368025733\\
286	1.00025735504373\\
287	1.00022349001474\\
288	1.00018549688746\\
289	1.00014654375849\\
290	1.0001092638004\\
291	1.00007560850438\\
292	1.00004681940536\\
293	1.0000234853142\\
294	1.0000056522805\\
295	0.999992957931546\\
296	0.99998476825077\\
297	0.999980301698218\\
298	0.999978731833114\\
299	0.999979264735191\\
300	0.999981191336499\\
301	0.999983917301141\\
302	1.14641697449577\\
303	1.27334405061373\\
304	1.38335645051608\\
305	1.47870157243989\\
306	1.56132853754721\\
307	1.55672059085823\\
308	1.48618770074739\\
309	1.38363416663197\\
310	1.27256101469146\\
311	1.16797480137204\\
312	1.07832418948155\\
313	1.00728035010563\\
314	0.955261549451539\\
315	0.920663303670266\\
316	0.900796646637428\\
317	0.892562252598622\\
318	0.892901228826826\\
319	0.899067804668642\\
320	0.908767808679392\\
321	0.920202062099559\\
322	0.932047363727886\\
323	0.943400824045852\\
324	0.953706723884415\\
325	0.96267928812517\\
326	0.970229992222585\\
327	0.976404300153865\\
328	0.981329997691879\\
329	0.985177403576189\\
330	0.988130555627665\\
331	0.990367819297634\\
332	0.992050105737963\\
333	0.993314889612893\\
334	0.994274382320848\\
335	0.995016466980912\\
336	0.995607282227946\\
337	0.996094615762669\\
338	0.996511513623289\\
339	0.996879716246316\\
340	0.997212694347031\\
341	0.997518178371889\\
342	0.99780015964\\
343	0.998060395587945\\
344	0.998299482351928\\
345	0.998517571447228\\
346	0.998714808913233\\
347	0.99889156937668\\
348	0.999049652122927\\
349	0.999191246232939\\
350	0.999318331410176\\
351	0.99943239104893\\
352	0.999534334731506\\
353	0.999624549126938\\
354	0.99970301682708\\
355	0.999769675905386\\
356	0.999824617615898\\
357	0.999868170197845\\
358	0.999896895064773\\
359	0.999910727408762\\
360	0.999911585683297\\
361	0.999902349216463\\
362	0.999886153082363\\
363	0.999865940563057\\
364	0.999844213430832\\
365	0.999822925559476\\
366	0.999803473504428\\
367	0.99978674685674\\
368	0.999773210158414\\
369	0.999762996231938\\
370	0.999755997553032\\
371	0.999751947668782\\
372	0.999750488688861\\
373	0.999751223710752\\
374	0.999753754883355\\
375	0.999757708881227\\
376	0.999762752055577\\
377	0.999768597624053\\
378	0.999775007103623\\
379	0.999781787891298\\
380	0.999788788537826\\
381	0.999795892895459\\
382	0.999803013987398\\
383	0.999810088162094\\
384	0.999817069867402\\
385	0.999823927206799\\
386	0.999830638317223\\
387	0.999837188527392\\
388	0.999843568208316\\
389	0.999849771205506\\
390	0.999855793737776\\
391	0.999861633654317\\
392	0.999867289955012\\
393	0.999872762495139\\
394	0.999878051812239\\
395	0.999883159028371\\
396	0.99988808579433\\
397	0.999892834253401\\
398	0.999897407010703\\
399	0.999901807100491\\
400	0.999906037948098\\
401	0.999910103326089\\
402	0.999914007305746\\
403	0.999917754205859\\
404	0.999921348540916\\
405	0.999924794970677\\
406	0.999928098252701\\
407	0.999931263198992\\
408	0.999934294637469\\
409	0.999937197378587\\
410	0.999939976187109\\
411	0.999942635758817\\
412	0.99994518070175\\
413	0.999947615521509\\
414	0.999949944610092\\
415	0.999952172237739\\
416	0.999954302547298\\
417	0.99995633955065\\
418	0.999958287126808\\
419	0.999960149021347\\
420	0.999961928846882\\
421	0.99996363008436\\
422	0.999965256084978\\
423	0.999966810072584\\
424	0.999968295146431\\
425	0.999969714284202\\
426	0.999971070345227\\
427	0.99997236607384\\
428	0.999973604102821\\
429	0.999974786956906\\
430	0.999975917056315\\
431	0.999976996720287\\
432	0.999978028170609\\
433	0.999979013535099\\
434	0.999979954851057\\
435	0.99998085406866\\
436	0.999981713054284\\
437	0.999982533602045\\
438	0.999983317455907\\
439	0.99998406633701\\
440	0.99998478196659\\
441	0.999985466074886\\
442	0.999986120389785\\
443	0.999986746603737\\
444	0.999987346325483\\
445	0.999987921028219\\
446	0.999988472006574\\
447	0.999989000291704\\
448	0.999989506563228\\
449	0.99998999110913\\
450	0.999990453863209\\
451	0.999990894520788\\
452	0.999991312710783\\
453	0.999991708190632\\
454	0.999992081017785\\
455	0.999992431662429\\
456	0.999992761045383\\
457	0.999993070612865\\
458	0.999993362308714\\
459	0.99999363842841\\
460	0.999993901397601\\
461	0.999994153533561\\
462	0.9999943968399\\
463	0.999994632866307\\
464	0.999994862645297\\
465	0.999995086701552\\
466	0.999995305118638\\
467	0.999995517642796\\
468	0.999995723803001\\
469	0.999995923029337\\
470	0.999996114756337\\
471	0.999996298503092\\
472	0.999996473926777\\
473	0.999995909217782\\
474	0.999993054396472\\
475	0.999985960037882\\
476	0.999972829532126\\
477	0.999952494024937\\
478	0.999924715742105\\
479	0.9998903021143\\
480	0.999850914920338\\
481	0.999808719459077\\
482	0.999766014626264\\
483	0.999727581786762\\
484	0.999698522964261\\
485	0.999682435941857\\
486	0.999680779740245\\
487	0.999692975584107\\
488	0.999716890951486\\
489	0.999749456906088\\
490	0.99978725924214\\
491	0.999827016265205\\
492	0.999865909051942\\
493	0.999901765323614\\
494	0.99993311863269\\
495	0.999959173909691\\
496	0.999979711955316\\
497	0.999994962142836\\
498	1.00000546675856\\
499	1.00001195375892\\
500	1.00001522838842\\
501	1.00001608873241\\
502	1.00001526615589\\
503	1.00001338873508\\
504	1.00001096409164\\
505	1.0000083772787\\
506	1.00000589930106\\
507	1.00000370225013\\
508	1.00000187769655\\
509	1.00000045575439\\
510	0.999999422995406\\
511	0.999998738077714\\
512	0.999998344519491\\
513	0.99999818048104\\
514	0.999998185732067\\
515	0.999998306152547\\
516	0.999998496206449\\
517	0.999998719853669\\
518	0.999998950344415\\
519	0.999999169288467\\
520	0.999999365323388\\
521	0.999999532636306\\
522	0.999999669521594\\
523	0.999999777090812\\
524	0.999999858119596\\
525	0.999999916160081\\
526	0.999999954940074\\
527	0.999999978011718\\
528	0.999999988588383\\
529	0.99999998950575\\
530	0.999999983251975\\
531	0.999999972010542\\
532	0.999999957691848\\
533	0.999999941948454\\
534	0.999999926318086\\
535	0.999999912185201\\
536	0.999999900685252\\
537	0.99999989262102\\
538	0.999999888421691\\
539	0.999999888150464\\
540	0.999999891552579\\
541	0.99999989812962\\
542	0.999999907225161\\
543	0.999999918108988\\
544	0.999999930050697\\
545	0.999999942377262\\
546	0.999999954512542\\
547	0.999999965999294\\
548	0.999999976505999\\
549	0.999999985821781\\
550	0.999999993842978\\
551	1.00000000055474\\
552	1.0000000060106\\
553	1.00000001031221\\
554	1.00000001359098\\
555	1.00000001599242\\
556	1.00000001766386\\
557	1.00000001874532\\
558	1.00000001936358\\
559	1.00000001962874\\
560	1.00000001963294\\
561	1.00000001945066\\
562	1.00000001914007\\
563	1.00000001874502\\
564	1.0000000182974\\
565	1.00000001781943\\
566	1.00000001732589\\
567	1.00000001682609\\
568	1.00000001632547\\
569	1.00000001582693\\
570	1.00000001533184\\
571	1.00000001484072\\
572	1.00000001435377\\
573	1.00000001387117\\
574	1.00000001339321\\
575	1.00000001292038\\
576	1.00000001245336\\
577	1.000000011993\\
578	1.00000001154022\\
579	1.00000001109597\\
580	1.0000000106612\\
581	1.00000001023675\\
582	1.00000000982337\\
583	1.0000000094217\\
584	1.00000000903221\\
585	1.00000000865524\\
586	1.000000008291\\
587	1.00000000793959\\
588	1.00000000760098\\
589	1.00000000727508\\
590	1.00000000696169\\
591	1.00000000666058\\
592	1.00000000637147\\
593	1.00000000609403\\
594	1.00000000582792\\
595	1.00000000557279\\
596	1.00000000532826\\
597	1.00000000509398\\
598	1.00000000486958\\
599	1.00000000465469\\
600	1.00000000444895\\
};
\addlegendentry{$y(k)$}

\addplot[const plot, color=mycolor2, dashed] table[row sep=crcr] {%
1	0\\
2	0\\
3	0\\
4	0\\
5	0\\
6	0\\
7	0\\
8	0\\
9	0\\
10	0\\
11	0\\
12	0\\
13	0\\
14	0\\
15	0\\
16	0\\
17	0\\
18	0\\
19	0\\
20	0\\
21	0\\
22	0\\
23	0\\
24	0\\
25	0\\
26	0\\
27	0\\
28	0\\
29	0\\
30	0\\
31	0\\
32	0\\
33	0\\
34	0\\
35	0\\
36	0\\
37	0\\
38	0\\
39	0\\
40	0\\
41	0\\
42	0\\
43	0\\
44	0\\
45	0\\
46	0\\
47	0\\
48	0\\
49	0\\
50	0\\
51	0\\
52	0\\
53	0\\
54	0\\
55	0\\
56	0\\
57	0\\
58	0\\
59	0\\
60	0\\
61	0\\
62	0\\
63	0\\
64	0\\
65	0\\
66	0\\
67	0\\
68	0\\
69	0\\
70	0\\
71	0\\
72	0\\
73	0\\
74	0\\
75	0\\
76	0\\
77	0\\
78	0\\
79	0\\
80	0\\
81	0\\
82	0\\
83	0\\
84	0\\
85	0\\
86	0\\
87	0\\
88	0\\
89	0\\
90	0\\
91	0\\
92	0\\
93	0\\
94	0\\
95	0\\
96	0\\
97	0\\
98	0\\
99	0\\
100	1\\
101	1\\
102	1\\
103	1\\
104	1\\
105	1\\
106	1\\
107	1\\
108	1\\
109	1\\
110	1\\
111	1\\
112	1\\
113	1\\
114	1\\
115	1\\
116	1\\
117	1\\
118	1\\
119	1\\
120	1\\
121	1\\
122	1\\
123	1\\
124	1\\
125	1\\
126	1\\
127	1\\
128	1\\
129	1\\
130	1\\
131	1\\
132	1\\
133	1\\
134	1\\
135	1\\
136	1\\
137	1\\
138	1\\
139	1\\
140	1\\
141	1\\
142	1\\
143	1\\
144	1\\
145	1\\
146	1\\
147	1\\
148	1\\
149	1\\
150	1\\
151	1\\
152	1\\
153	1\\
154	1\\
155	1\\
156	1\\
157	1\\
158	1\\
159	1\\
160	1\\
161	1\\
162	1\\
163	1\\
164	1\\
165	1\\
166	1\\
167	1\\
168	1\\
169	1\\
170	1\\
171	1\\
172	1\\
173	1\\
174	1\\
175	1\\
176	1\\
177	1\\
178	1\\
179	1\\
180	1\\
181	1\\
182	1\\
183	1\\
184	1\\
185	1\\
186	1\\
187	1\\
188	1\\
189	1\\
190	1\\
191	1\\
192	1\\
193	1\\
194	1\\
195	1\\
196	1\\
197	1\\
198	1\\
199	1\\
200	1\\
201	1\\
202	1\\
203	1\\
204	1\\
205	1\\
206	1\\
207	1\\
208	1\\
209	1\\
210	1\\
211	1\\
212	1\\
213	1\\
214	1\\
215	1\\
216	1\\
217	1\\
218	1\\
219	1\\
220	1\\
221	1\\
222	1\\
223	1\\
224	1\\
225	1\\
226	1\\
227	1\\
228	1\\
229	1\\
230	1\\
231	1\\
232	1\\
233	1\\
234	1\\
235	1\\
236	1\\
237	1\\
238	1\\
239	1\\
240	1\\
241	1\\
242	1\\
243	1\\
244	1\\
245	1\\
246	1\\
247	1\\
248	1\\
249	1\\
250	1\\
251	1\\
252	1\\
253	1\\
254	1\\
255	1\\
256	1\\
257	1\\
258	1\\
259	1\\
260	1\\
261	1\\
262	1\\
263	1\\
264	1\\
265	1\\
266	1\\
267	1\\
268	1\\
269	1\\
270	1\\
271	1\\
272	1\\
273	1\\
274	1\\
275	1\\
276	1\\
277	1\\
278	1\\
279	1\\
280	1\\
281	1\\
282	1\\
283	1\\
284	1\\
285	1\\
286	1\\
287	1\\
288	1\\
289	1\\
290	1\\
291	1\\
292	1\\
293	1\\
294	1\\
295	1\\
296	1\\
297	1\\
298	1\\
299	1\\
300	1\\
301	1\\
302	1\\
303	1\\
304	1\\
305	1\\
306	1\\
307	1\\
308	1\\
309	1\\
310	1\\
311	1\\
312	1\\
313	1\\
314	1\\
315	1\\
316	1\\
317	1\\
318	1\\
319	1\\
320	1\\
321	1\\
322	1\\
323	1\\
324	1\\
325	1\\
326	1\\
327	1\\
328	1\\
329	1\\
330	1\\
331	1\\
332	1\\
333	1\\
334	1\\
335	1\\
336	1\\
337	1\\
338	1\\
339	1\\
340	1\\
341	1\\
342	1\\
343	1\\
344	1\\
345	1\\
346	1\\
347	1\\
348	1\\
349	1\\
350	1\\
351	1\\
352	1\\
353	1\\
354	1\\
355	1\\
356	1\\
357	1\\
358	1\\
359	1\\
360	1\\
361	1\\
362	1\\
363	1\\
364	1\\
365	1\\
366	1\\
367	1\\
368	1\\
369	1\\
370	1\\
371	1\\
372	1\\
373	1\\
374	1\\
375	1\\
376	1\\
377	1\\
378	1\\
379	1\\
380	1\\
381	1\\
382	1\\
383	1\\
384	1\\
385	1\\
386	1\\
387	1\\
388	1\\
389	1\\
390	1\\
391	1\\
392	1\\
393	1\\
394	1\\
395	1\\
396	1\\
397	1\\
398	1\\
399	1\\
400	1\\
401	1\\
402	1\\
403	1\\
404	1\\
405	1\\
406	1\\
407	1\\
408	1\\
409	1\\
410	1\\
411	1\\
412	1\\
413	1\\
414	1\\
415	1\\
416	1\\
417	1\\
418	1\\
419	1\\
420	1\\
421	1\\
422	1\\
423	1\\
424	1\\
425	1\\
426	1\\
427	1\\
428	1\\
429	1\\
430	1\\
431	1\\
432	1\\
433	1\\
434	1\\
435	1\\
436	1\\
437	1\\
438	1\\
439	1\\
440	1\\
441	1\\
442	1\\
443	1\\
444	1\\
445	1\\
446	1\\
447	1\\
448	1\\
449	1\\
450	1\\
451	1\\
452	1\\
453	1\\
454	1\\
455	1\\
456	1\\
457	1\\
458	1\\
459	1\\
460	1\\
461	1\\
462	1\\
463	1\\
464	1\\
465	1\\
466	1\\
467	1\\
468	1\\
469	1\\
470	1\\
471	1\\
472	1\\
473	1\\
474	1\\
475	1\\
476	1\\
477	1\\
478	1\\
479	1\\
480	1\\
481	1\\
482	1\\
483	1\\
484	1\\
485	1\\
486	1\\
487	1\\
488	1\\
489	1\\
490	1\\
491	1\\
492	1\\
493	1\\
494	1\\
495	1\\
496	1\\
497	1\\
498	1\\
499	1\\
500	1\\
501	1\\
502	1\\
503	1\\
504	1\\
505	1\\
506	1\\
507	1\\
508	1\\
509	1\\
510	1\\
511	1\\
512	1\\
513	1\\
514	1\\
515	1\\
516	1\\
517	1\\
518	1\\
519	1\\
520	1\\
521	1\\
522	1\\
523	1\\
524	1\\
525	1\\
526	1\\
527	1\\
528	1\\
529	1\\
530	1\\
531	1\\
532	1\\
533	1\\
534	1\\
535	1\\
536	1\\
537	1\\
538	1\\
539	1\\
540	1\\
541	1\\
542	1\\
543	1\\
544	1\\
545	1\\
546	1\\
547	1\\
548	1\\
549	1\\
550	1\\
551	1\\
552	1\\
553	1\\
554	1\\
555	1\\
556	1\\
557	1\\
558	1\\
559	1\\
560	1\\
561	1\\
562	1\\
563	1\\
564	1\\
565	1\\
566	1\\
567	1\\
568	1\\
569	1\\
570	1\\
571	1\\
572	1\\
573	1\\
574	1\\
575	1\\
576	1\\
577	1\\
578	1\\
579	1\\
580	1\\
581	1\\
582	1\\
583	1\\
584	1\\
585	1\\
586	1\\
587	1\\
588	1\\
589	1\\
590	1\\
591	1\\
592	1\\
593	1\\
594	1\\
595	1\\
596	1\\
597	1\\
598	1\\
599	1\\
600	1\\
};
\addlegendentry{$y^{\mathrm{zad}}(k)$}

\end{axis}
\end{tikzpicture}%
	\caption{Przebiegi sygna��w dla regulacji przy pomocy algorytmu DMC z uwzgl�dnieniem zak��ce�}
\end{figure}

Wska�nik ilo�ciowy: 8.8851

\section{Wnioski}
Mo�emy zauwa�y� znaczn� popraw� wska�nika ilo�ciowego. Nast�puje szybszy powr�t do warto�ci zadanej co wskazuje na to, �e pomiar zak��cenia i jego uwzgl�dnienie prowadzi do lepszej regulacji ni� gdy brak jest tego pomiaru.