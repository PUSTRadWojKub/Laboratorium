\chapter{Regulacja za pomoc� DMC z uwzgl�dnieniem zak��ce�}

\begin{figure}[h!]
	\centering
	% This file was created by matlab2tikz.
%
\definecolor{mycolor1}{rgb}{0.00000,0.44700,0.74100}%
\definecolor{mycolor2}{rgb}{0.85000,0.32500,0.09800}%
%
\begin{tikzpicture}

\begin{axis}[%
width=4.521in,
height=1.493in,
at={(0.758in,2.554in)},
scale only axis,
xmin=0,
xmax=600,
xlabel style={font=\color{white!15!black}},
xlabel={$k$},
ymin=-4,
ymax=3,
axis background/.style={fill=white},
title style={font=\bfseries},
title={Sygna� wej�ciowy sterowania i zak��cenia},
xmajorgrids,
ymajorgrids,
legend style={legend cell align=left, align=left, draw=white!15!black}
]
\addplot[const plot, color=mycolor1] table[row sep=crcr] {%
1	0\\
2	0\\
3	0\\
4	0\\
5	0\\
6	0\\
7	0\\
8	0\\
9	0\\
10	0\\
11	0\\
12	0\\
13	0\\
14	0\\
15	0\\
16	0\\
17	0\\
18	0\\
19	0\\
20	0\\
21	0\\
22	0\\
23	0\\
24	0\\
25	0\\
26	0\\
27	0\\
28	0\\
29	0\\
30	0\\
31	0\\
32	0\\
33	0\\
34	0\\
35	0\\
36	0\\
37	0\\
38	0\\
39	0\\
40	0\\
41	0\\
42	0\\
43	0\\
44	0\\
45	0\\
46	0\\
47	0\\
48	0\\
49	0\\
50	0\\
51	0\\
52	0\\
53	0\\
54	0\\
55	0\\
56	0\\
57	0\\
58	0\\
59	0\\
60	0\\
61	0\\
62	0\\
63	0\\
64	0\\
65	0\\
66	0\\
67	0\\
68	0\\
69	0\\
70	0\\
71	0\\
72	0\\
73	0\\
74	0\\
75	0\\
76	0\\
77	0\\
78	0\\
79	0\\
80	0\\
81	0\\
82	0\\
83	0\\
84	0\\
85	0\\
86	0\\
87	0\\
88	0\\
89	0\\
90	0\\
91	0\\
92	0\\
93	0\\
94	0\\
95	0\\
96	0\\
97	0\\
98	0\\
99	0\\
100	1.06438689272261\\
101	1.67330799970846\\
102	1.94630786721308\\
103	1.9873867877016\\
104	1.88160065495093\\
105	1.69466404103899\\
106	1.47444334123278\\
107	1.2534827815949\\
108	1.05194116533132\\
109	0.88051830203168\\
110	0.743112059095731\\
111	0.639070284192932\\
112	0.564990187660185\\
113	0.516076581765213\\
114	0.487105661396056\\
115	0.47305860386581\\
116	0.469494351726138\\
117	0.472727848975606\\
118	0.479872134885331\\
119	0.488792593525579\\
120	0.498011076122221\\
121	0.506587690199236\\
122	0.513999424293346\\
123	0.520027728514665\\
124	0.524661729730118\\
125	0.528019799977936\\
126	0.530289507139993\\
127	0.531684313584522\\
128	0.532414508349938\\
129	0.532669540981234\\
130	0.53260898363768\\
131	0.532359633484115\\
132	0.532016667214764\\
133	0.531647194123235\\
134	0.531294971082772\\
135	0.53098541140078\\
136	0.530730325128121\\
137	0.530532067662125\\
138	0.530386950263108\\
139	0.530287888418633\\
140	0.530226341679766\\
141	0.53019364174679\\
142	0.530181823531268\\
143	0.530184074694741\\
144	0.530194909262603\\
145	0.530210155308029\\
146	0.530226828978867\\
147	0.530242949702892\\
148	0.530257335712222\\
149	0.530269405819655\\
150	0.530279002893992\\
151	0.530286246619096\\
152	0.530291417585385\\
153	0.530294871160459\\
154	0.53029697750129\\
155	0.530298083110692\\
156	0.5302984891616\\
157	0.530298442131522\\
158	0.53029813288721\\
159	0.530297701075738\\
160	0.53029724240299\\
161	0.5302968170457\\
162	0.530296458011551\\
163	0.530296178719471\\
164	0.530295979420275\\
165	0.530295852326937\\
166	0.530295785489372\\
167	0.530295765547805\\
168	0.530295779548015\\
169	0.530295816015713\\
170	0.53029586547824\\
171	0.530295920599273\\
172	0.530295976063461\\
173	0.530296028317908\\
174	0.530296075249234\\
175	0.530296115850542\\
176	0.530296149912657\\
177	0.530296177758576\\
178	0.530296200028913\\
179	0.530296217518459\\
180	0.530296231059335\\
181	0.530296241443675\\
182	0.530296249377924\\
183	0.530296255460952\\
184	0.530296260179045\\
185	0.530296263911903\\
186	0.530296266945044\\
187	0.530296269485158\\
188	0.530296271676013\\
189	0.530296273613349\\
190	0.5302962753579\\
191	0.530296276946142\\
192	0.53029627839874\\
193	0.530296279726851\\
194	0.530296280936599\\
195	0.530296282032038\\
196	0.53029628301696\\
197	0.530296283895852\\
198	0.530296284674259\\
199	0.530296285358785\\
200	0.53029628595687\\
201	0.530296286476478\\
202	0.530296286925765\\
203	0.530296287312782\\
204	0.53029628764523\\
205	0.530296287930287\\
206	0.530296288174488\\
207	0.530296288383675\\
208	0.530296288562982\\
209	0.530296288716863\\
210	0.530296288849134\\
211	0.530296288963043\\
212	0.530296289061329\\
213	0.530296289146296\\
214	0.530296289219876\\
215	0.530296289283692\\
216	0.530296289339104\\
217	0.530296289387264\\
218	0.530296289429144\\
219	0.530296289465574\\
220	0.530296289497263\\
221	0.530296289524822\\
222	0.530296289548782\\
223	0.530296289569602\\
224	0.530296289587684\\
225	0.530296289603381\\
226	0.530296289617\\
227	0.53029628962881\\
228	0.530296289639049\\
229	0.530296289647923\\
230	0.530296289655612\\
231	0.530296289662274\\
232	0.530296289668046\\
233	0.530296289673046\\
234	0.530296289677378\\
235	0.53029628968113\\
236	0.53029628968438\\
237	0.530296289687193\\
238	0.530296289689628\\
239	0.530296289691737\\
240	0.530296289693563\\
241	0.530296289695144\\
242	0.530296289696515\\
243	0.530296289697705\\
244	0.530296289698736\\
245	0.530296289699632\\
246	0.53029628970041\\
247	0.530296289701086\\
248	0.530296289701672\\
249	0.53029628970218\\
250	0.530296289702619\\
251	0.530296289702997\\
252	0.530296289703324\\
253	0.530296289703606\\
254	0.530296289703851\\
255	0.530296289704065\\
256	0.530296289704251\\
257	0.530296289704414\\
258	0.530296289704558\\
259	0.530296289704683\\
260	0.530296289704794\\
261	0.530296289704891\\
262	0.530296289704976\\
263	0.530296289705049\\
264	0.530296289705111\\
265	0.530296289705163\\
266	0.530306508427582\\
267	0.53033666381662\\
268	0.530391127427623\\
269	0.530467847736317\\
270	0.530559386950657\\
271	0.530654876638281\\
272	0.530742108994647\\
273	0.530811305149986\\
274	0.530856665729264\\
275	0.530876482260541\\
276	0.530835314532807\\
277	0.530733501687177\\
278	0.530589437566782\\
279	0.530427392420687\\
280	0.530270030302657\\
281	0.530134556578039\\
282	0.530031433039188\\
283	0.529964727279454\\
284	0.529933350393245\\
285	0.529932636565899\\
286	0.529955901281582\\
287	0.529995766900072\\
288	0.530045160082298\\
289	0.530097966103479\\
290	0.530149375192492\\
291	0.530195981986413\\
292	0.53023570754184\\
293	0.530267610057856\\
294	0.530291640471443\\
295	0.530308386143677\\
296	0.530318832592561\\
297	0.530324161292871\\
298	0.530325591820423\\
299	0.530324269357558\\
300	0.530321193702817\\
301	0.530317183116816\\
302	0.600883759528613\\
303	0.699554994427609\\
304	0.79823839296497\\
305	0.88015612172364\\
306	0.936954334048936\\
307	0.966174070187183\\
308	0.969170946909345\\
309	0.94949736714915\\
310	0.911714344408782\\
311	0.86057417407796\\
312	0.800504475639192\\
313	0.735323744165137\\
314	0.668124574934206\\
315	0.601270169584827\\
316	0.536460441066096\\
317	0.474834563250558\\
318	0.417086283519744\\
319	0.363576286787009\\
320	0.314432226668253\\
321	0.269631784534059\\
322	0.22906745952303\\
323	0.192593969084479\\
324	0.16006040116183\\
325	0.131329842803198\\
326	0.106289323029759\\
327	0.0848527204026409\\
328	0.0669589299421534\\
329	0.0525671564849458\\
330	0.0416507682823088\\
331	0.0341907469187706\\
332	0.0301694297226505\\
333	0.0295649672709539\\
334	0.0323467106376653\\
335	0.0384715943354051\\
336	0.0478814821866279\\
337	0.0605013844035554\\
338	0.0762384250567114\\
339	0.0949814310662051\\
340	0.11660101952307\\
341	0.14095007377343\\
342	0.167864515997486\\
343	0.197164302014405\\
344	0.228654580843186\\
345	0.262126976083645\\
346	0.297360957997805\\
347	0.334125284237031\\
348	0.372179493703366\\
349	0.411275442426316\\
350	0.451158873004972\\
351	0.491571010532393\\
352	0.532250178366536\\
353	0.572933426963262\\
354	0.613358168499996\\
355	0.653263809388318\\
356	0.692393372138159\\
357	0.730495097486358\\
358	0.767324017291234\\
359	0.802643488448084\\
360	0.836226678003558\\
361	0.867857989732454\\
362	0.897334422673734\\
363	0.924466852485027\\
364	0.949081226947198\\
365	0.971019667514258\\
366	0.990141469442203\\
367	1.00632399372895\\
368	1.01946344484401\\
369	1.02947552901078\\
370	1.036295988618\\
371	1.03988100917282\\
372	1.04020749605949\\
373	1.03727321923102\\
374	1.03109682482867\\
375	1.02171771359442\\
376	1.00919578680648\\
377	0.993611061325871\\
378	0.975063156186323\\
379	0.953670653986743\\
380	0.929570341149968\\
381	0.902916331889115\\
382	0.873879081468723\\
383	0.842644295057618\\
384	0.809411739139656\\
385	0.774393963072971\\
386	0.737814938964233\\
387	0.699908628547961\\
388	0.66091748622889\\
389	0.62109090785465\\
390	0.580683635134128\\
391	0.539954125901395\\
392	0.499162900644393\\
393	0.458570875870068\\
394	0.418437694962543\\
395	0.379020067207484\\
396	0.340570125604108\\
397	0.303333813966561\\
398	0.26754931362938\\
399	0.233445519818802\\
400	0.201240577434251\\
401	0.17114048560452\\
402	0.143337779943523\\
403	0.118010300933561\\
404	0.095320056313358\\
405	0.0754121847468737\\
406	0.0584140274011966\\
407	0.0444343133716435\\
408	0.0335624641640441\\
409	0.025868021682704\\
410	0.0214002033825679\\
411	0.0201875874307563\\
412	0.0222379298910657\\
413	0.0275381151005721\\
414	0.0360542395555314\\
415	0.0477318287697948\\
416	0.0624961857184236\\
417	0.0802528686374966\\
418	0.100888295123699\\
419	0.124270468669403\\
420	0.150249822985808\\
421	0.178660178713256\\
422	0.209319806398975\\
423	0.242032588942766\\
424	0.276589276074878\\
425	0.312768822841659\\
426	0.3503398035372\\
427	0.38906189203659\\
428	0.428687399061689\\
429	0.468962856546115\\
430	0.50963063896491\\
431	0.550430611257884\\
432	0.591101792903632\\
433	0.631384027434718\\
434	0.671019646625648\\
435	0.709755118655244\\
436	0.747342669696106\\
437	0.783541868591755\\
438	0.818121164537232\\
439	0.850859368019144\\
440	0.881547065604941\\
441	0.909987959545922\\
442	0.936000122885206\\
443	0.959417163071245\\
444	0.980089286800538\\
445	0.997884259053141\\
446	1.01268824986598\\
447	1.02440656316613\\
448	1.03296424285764\\
449	1.0383065521183\\
450	1.0403993229447\\
451	1.03922917391553\\
452	1.03480359627959\\
453	1.02715090490514\\
454	1.01632005578961\\
455	1.00238033197032\\
456	0.985420900048557\\
457	0.965550240095353\\
458	0.94289545239549\\
459	0.917601445249606\\
460	0.889830008847812\\
461	0.85975878101403\\
462	0.827580111372326\\
463	0.79349983118783\\
464	0.757735936774959\\
465	0.720517194940075\\
466	0.682081679432426\\
467	0.642675247816557\\
468	0.602550646072939\\
469	0.561965744994272\\
470	0.521181443215329\\
471	0.480459551239077\\
472	0.440060812547962\\
473	0.400243114008321\\
474	0.36125987727306\\
475	0.323358726237714\\
476	0.286780250500199\\
477	0.251756765925012\\
478	0.218508571242555\\
479	0.187245267494891\\
480	0.158165526133715\\
481	0.131455930855242\\
482	0.107289372205134\\
483	0.0858233327632011\\
484	0.0671982791258384\\
485	0.0515362811775642\\
486	0.0389399098409771\\
487	0.0294914188178521\\
488	0.0232521893521564\\
489	0.0202624048879773\\
490	0.0205409200989511\\
491	0.0240852922799671\\
492	0.0308719495347327\\
493	0.0408564774450177\\
494	0.0539740126105515\\
495	0.0701397368573305\\
496	0.0892494697366949\\
497	0.111180359196593\\
498	0.135791671204097\\
499	0.16292567892811\\
500	0.192408651168337\\
501	0.224051938333398\\
502	0.257653152672525\\
503	0.292997437839956\\
504	0.329858821350245\\
505	0.368001642147704\\
506	0.407182044404257\\
507	0.447149527787321\\
508	0.48764854379242\\
509	0.52842012729123\\
510	0.569203552176606\\
511	0.609737999862611\\
512	0.64976422939312\\
513	0.689026238004389\\
514	0.727272901157402\\
515	0.764259581291357\\
516	0.799749694841514\\
517	0.833516227406879\\
518	0.865343187342637\\
519	0.895026988487354\\
520	0.922377753214765\\
521	0.947220527524334\\
522	0.969396400452888\\
523	0.988763520700715\\
524	1.0051980040178\\
525	1.01859472558722\\
526	1.02886799236996\\
527	1.03595209113513\\
528	1.03980170868732\\
529	1.04039222161401\\
530	1.03771985370589\\
531	1.03180170004512\\
532	1.02267561760764\\
533	1.01039998307696\\
534	0.995053319416209\\
535	0.976733793583741\\
536	0.955558588602405\\
537	0.931663153996252\\
538	0.905200339387133\\
539	0.876339416791445\\
540	0.845264997869669\\
541	0.812175853053752\\
542	0.777283640105338\\
543	0.740811550237506\\
544	0.702992880460141\\
545	0.664069541281154\\
546	0.624290509309341\\
547	0.583910234657131\\
548	0.543187013330665\\
549	0.502381335018522\\
550	0.461754216847833\\
551	0.421565533766191\\
552	0.382072356229339\\
553	0.34352730582788\\
554	0.306176939371434\\
555	0.270260171766646\\
556	0.236006747777277\\
557	0.20363577244191\\
558	0.173354309549598\\
559	0.145356057138418\\
560	0.119820108489225\\
561	0.0969098065400207\\
562	0.0767716990487473\\
563	0.059534601187931\\
564	0.0453087715673311\\
565	0.0341852069552598\\
566	0.0262350602099379\\
567	0.0215091851441382\\
568	0.0200378112344348\\
569	0.0218303502558059\\
570	0.0268753360784724\\
571	0.0351404980120642\\
572	0.0465729672279696\\
573	0.0610996149394637\\
574	0.0786275201764263\\
575	0.0990445641624784\\
576	0.122220147492564\\
577	0.148006025523481\\
578	0.176237256633719\\
579	0.20673325728697\\
580	0.23929895715052\\
581	0.273726046879688\\
582	0.309794310586768\\
583	0.347273034471209\\
584	0.385922482600614\\
585	0.425495430402587\\
586	0.465738746058324\\
587	0.506395009682423\\
588	0.547204159931676\\
589	0.587905157510192\\
590	0.628237654930062\\
591	0.667943661846829\\
592	0.70676919531727\\
593	0.744465904423497\\
594	0.780792658871305\\
595	0.815517091401176\\
596	0.848417084145767\\
597	0.879282189426257\\
598	0.907914975899322\\
599	0.934132291443979\\
600	0.957766434710145\\
};
\addlegendentry{$u(k)$}

\addplot[const plot, color=mycolor2] table[row sep=crcr] {%
1	0\\
2	0\\
3	0\\
4	0\\
5	0\\
6	0\\
7	0\\
8	0\\
9	0\\
10	0\\
11	0\\
12	0\\
13	0\\
14	0\\
15	0\\
16	0\\
17	0\\
18	0\\
19	0\\
20	0\\
21	0\\
22	0\\
23	0\\
24	0\\
25	0\\
26	0\\
27	0\\
28	0\\
29	0\\
30	0\\
31	0\\
32	0\\
33	0\\
34	0\\
35	0\\
36	0\\
37	0\\
38	0\\
39	0\\
40	0\\
41	0\\
42	0\\
43	0\\
44	0\\
45	0\\
46	0\\
47	0\\
48	0\\
49	0\\
50	0\\
51	0\\
52	0\\
53	0\\
54	0\\
55	0\\
56	0\\
57	0\\
58	0\\
59	0\\
60	0\\
61	0\\
62	0\\
63	0\\
64	0\\
65	0\\
66	0\\
67	0\\
68	0\\
69	0\\
70	0\\
71	0\\
72	0\\
73	0\\
74	0\\
75	0\\
76	0\\
77	0\\
78	0\\
79	0\\
80	0\\
81	0\\
82	0\\
83	0\\
84	0\\
85	0\\
86	0\\
87	0\\
88	0\\
89	0\\
90	0\\
91	0\\
92	0\\
93	0\\
94	0\\
95	0\\
96	0\\
97	0\\
98	0\\
99	0\\
100	0\\
101	0\\
102	0\\
103	0\\
104	0\\
105	0\\
106	0\\
107	0\\
108	0\\
109	0\\
110	0\\
111	0\\
112	0\\
113	0\\
114	0\\
115	0\\
116	0\\
117	0\\
118	0\\
119	0\\
120	0\\
121	0\\
122	0\\
123	0\\
124	0\\
125	0\\
126	0\\
127	0\\
128	0\\
129	0\\
130	0\\
131	0\\
132	0\\
133	0\\
134	0\\
135	0\\
136	0\\
137	0\\
138	0\\
139	0\\
140	0\\
141	0\\
142	0\\
143	0\\
144	0\\
145	0\\
146	0\\
147	0\\
148	0\\
149	0\\
150	0\\
151	0\\
152	0\\
153	0\\
154	0\\
155	0\\
156	0\\
157	0\\
158	0\\
159	0\\
160	0\\
161	0\\
162	0\\
163	0\\
164	0\\
165	0\\
166	0\\
167	0\\
168	0\\
169	0\\
170	0\\
171	0\\
172	0\\
173	0\\
174	0\\
175	0\\
176	0\\
177	0\\
178	0\\
179	0\\
180	0\\
181	0\\
182	0\\
183	0\\
184	0\\
185	0\\
186	0\\
187	0\\
188	0\\
189	0\\
190	0\\
191	0\\
192	0\\
193	0\\
194	0\\
195	0\\
196	0\\
197	0\\
198	0\\
199	0\\
200	0\\
201	0\\
202	0\\
203	0\\
204	0\\
205	0\\
206	0\\
207	0\\
208	0\\
209	0\\
210	0\\
211	0\\
212	0\\
213	0\\
214	0\\
215	0\\
216	0\\
217	0\\
218	0\\
219	0\\
220	0\\
221	0\\
222	0\\
223	0\\
224	0\\
225	0\\
226	0\\
227	0\\
228	0\\
229	0\\
230	0\\
231	0\\
232	0\\
233	0\\
234	0\\
235	0\\
236	0\\
237	0\\
238	0\\
239	0\\
240	0\\
241	0\\
242	0\\
243	0\\
244	0\\
245	0\\
246	0\\
247	0\\
248	0\\
249	0\\
250	0\\
251	0\\
252	0\\
253	0\\
254	0\\
255	0\\
256	0\\
257	0\\
258	0\\
259	0\\
260	0\\
261	0\\
262	0\\
263	0\\
264	0\\
265	0\\
266	0\\
267	0\\
268	0\\
269	0\\
270	0\\
271	0\\
272	0\\
273	0\\
274	0\\
275	0\\
276	0\\
277	0\\
278	0\\
279	0\\
280	0\\
281	0\\
282	0\\
283	0\\
284	0\\
285	0\\
286	0\\
287	0\\
288	0\\
289	0\\
290	0\\
291	0\\
292	0\\
293	0\\
294	0\\
295	0\\
296	0\\
297	0\\
298	0\\
299	0\\
300	-0.452789181003312\\
301	-0.434391960439725\\
302	-0.413216113737604\\
303	-0.389397094051036\\
304	-0.363087261821106\\
305	-0.334454910189011\\
306	-0.303683188494286\\
307	-0.270968930744037\\
308	-0.23652139654702\\
309	-0.200560932566259\\
310	-0.163317563052361\\
311	-0.125029518473301\\
312	-0.0859417116524005\\
313	-0.0463041711620614\\
314	-0.00637044199414048\\
315	0.0336040362627375\\
316	0.0733635633648459\\
317	0.112653814022298\\
318	0.151223464713005\\
319	0.188825801295536\\
320	0.225220297137695\\
321	0.260174151666137\\
322	0.29346377949559\\
323	0.324876240612302\\
324	0.354210602463456\\
325	0.381279225239801\\
326	0.405908962130094\\
327	0.427942266869804\\
328	0.44723820149957\\
329	0.463673337887167\\
330	0.477142547246349\\
331	0.48755967260231\\
332	0.49485807990231\\
333	0.49899108424621\\
334	0.499932248510524\\
335	0.49767555245578\\
336	0.492235431235516\\
337	0.483646683060557\\
338	0.471964246609214\\
339	0.457262849607233\\
340	0.439636530825362\\
341	0.419198038552136\\
342	0.3960781093896\\
343	0.370424631985201\\
344	0.342401701049137\\
345	0.312188567708196\\
346	0.279978492910265\\
347	0.245977511213779\\
348	0.21040311286967\\
349	0.173482852625959\\
350	0.135452894153935\\
351	0.0965564994065796\\
352	0.05704247257224\\
353	0.0171635685768771\\
354	-0.0228251236848921\\
355	-0.0626678130482163\\
356	-0.102109642268537\\
357	-0.140898318238242\\
358	-0.178785725801558\\
359	-0.21552951484485\\
360	-0.250894650510287\\
361	-0.284654916616856\\
362	-0.316594362671938\\
363	-0.346508685217499\\
364	-0.374206534675023\\
365	-0.399510739329807\\
366	-0.422259438625323\\
367	-0.442307118518408\\
368	-0.459525542272567\\
369	-0.473804570735453\\
370	-0.485052866853593\\
371	-0.493198479917824\\
372	-0.498189305802297\\
373	-0.499993420253071\\
374	-0.49859928309439\\
375	-0.494015812046431\\
376	-0.486272325682322\\
377	-0.475418355889334\\
378	-0.461523331033825\\
379	-0.444676131856637\\
380	-0.424984522939663\\
381	-0.402574463380286\\
382	-0.377589301082996\\
383	-0.350188855822004\\
384	-0.320548396940074\\
385	-0.288857522222866\\
386	-0.255318945120126\\
387	-0.220147198071386\\
388	-0.183567260230471\\
389	-0.145813118366674\\
390	-0.107126270147944\\
391	-0.0677541793799356\\
392	-0.0279486930821649\\
393	0.0120355694734769\\
394	0.0519428454571341\\
395	0.091517864490294\\
396	0.130507481505058\\
397	0.168662296008706\\
398	0.205738247395733\\
399	0.241498176102848\\
400	0.275713340620845\\
401	0.308164880659659\\
402	0.338645217107341\\
403	0.366959379827974\\
404	0.392926254805211\\
405	0.41637974265389\\
406	0.437169821089277\\
407	0.455163504557712\\
408	0.470245694890339\\
409	0.482319917538589\\
410	0.49130893868207\\
411	0.497155259261444\\
412	0.499821482776186\\
413	0.499290554494572\\
414	0.495565870545754\\
415	0.488671256196129\\
416	0.478650813448939\\
417	0.465568638941948\\
418	0.449508413947706\\
419	0.430572869098965\\
420	0.408883127263221\\
421	0.384577928769695\\
422	0.357812743944697\\
423	0.328758778632095\\
424	0.29760187906018\\
425	0.264541343060012\\
426	0.229788645239433\\
427	0.193566084267253\\
428	0.156105360920399\\
429	0.117646095989627\\
430	0.078434297524205\\
431	0.0387207872199191\\
432	-0.0012404039838094\\
433	-0.0411936608350507\\
434	-0.0808834188346504\\
435	-0.120055798976891\\
436	-0.158460231708724\\
437	-0.195851059719838\\
438	-0.231989109311098\\
439	-0.266643220290017\\
440	-0.299591724607133\\
441	-0.330623864275049\\
442	-0.359541139500336\\
443	-0.386158578404798\\
444	-0.410305920214258\\
445	-0.431828704346478\\
446	-0.450589258431779\\
447	-0.466467578946381\\
448	-0.479362098825431\\
449	-0.489190337145607\\
450	-0.495889426721558\\
451	-0.499416516241347\\
452	-0.499749044368611\\
453	-0.496884884058122\\
454	-0.49084235616162\\
455	-0.48166011223688\\
456	-0.469396887309646\\
457	-0.454131124169907\\
458	-0.435960471605734\\
459	-0.415001159784274\\
460	-0.391387256775325\\
461	-0.365269810973218\\
462	-0.336815884902554\\
463	-0.306207486588167\\
464	-0.273640405324888\\
465	-0.239322959294207\\
466	-0.203474663038832\\
467	-0.166324823318709\\
468	-0.12811107233031\\
469	-0.0890778476715095\\
470	-0.0494748287751448\\
471	-0.00955533981267899\\
472	0.0304252707159852\\
473	0.0702112633415089\\
474	0.109548143484968\\
475	0.148184289354693\\
476	0.185872561467029\\
477	0.222371883495657\\
478	0.257448784337363\\
479	0.290878891530381\\
480	0.322448366472434\\
481	0.35195527225805\\
482	0.379210865385618\\
483	0.404040803071704\\
484	0.426286258449892\\
485	0.445804936520722\\
486	0.462471984354079\\
487	0.476180789721887\\
488	0.486843663052532\\
489	0.494392398344876\\
490	0.498778709453903\\
491	0.49997453895726\\
492	0.497972237627022\\
493	0.492784613358639\\
494	0.484444849244112\\
495	0.473006291313454\\
496	0.458542107302138\\
497	0.441144818627286\\
498	0.420925708566336\\
499	0.398014110423842\\
500	0.372556580239674\\
501	0.34471595933051\\
502	0.314670332661116\\
503	0.28261188970832\\
504	0.248745695104205\\
505	0.213288376922287\\
506	0.176466740997094\\
507	0.138516320140849\\
508	0.0996798675372212\\
509	0.0602058039493939\\
510	0.0203466286749289\\
511	-0.0196426955880406\\
512	-0.0595063736320124\\
513	-0.0989894139565018\\
514	-0.137839259843462\\
515	-0.175807404858408\\
516	-0.212650982443528\\
517	-0.248134319434865\\
518	-0.282030443566299\\
519	-0.314122535317479\\
520	-0.344205314818835\\
521	-0.372086354942211\\
522	-0.397587312177801\\
523	-0.420545067424\\
524	-0.440812769392983\\
525	-0.458260773957817\\
526	-0.472777473432439\\
527	-0.484270010479977\\
528	-0.492664872082818\\
529	-0.497908359775051\\
530	-0.499966933129402\\
531	-0.498827424301499\\
532	-0.494497122259149\\
533	-0.48700372615781\\
534	-0.476395168160528\\
535	-0.462739306835664\\
536	-0.446123493093636\\
537	-0.426654011439178\\
538	-0.404455400113225\\
539	-0.379669654473164\\
540	-0.352455318707124\\
541	-0.322986471692228\\
542	-0.291451613483828\\
543	-0.258052459558433\\
544	-0.223002650523038\\
545	-0.186526385544311\\
546	-0.148856988239021\\
547	-0.11023541419914\\
548	-0.0709087096983323\\
549	-0.0311284314388931\\
550	0.00885096255270679\\
551	0.0487737405888105\\
552	0.088384533130672\\
553	0.127429966282015\\
554	0.165660282517323\\
555	0.202830938277668\\
556	0.238704168214962\\
557	0.273050506078772\\
558	0.305650252517265\\
559	0.336294880403301\\
560	0.364788368696429\\
561	0.390948456308596\\
562	0.414607807953137\\
563	0.435615084519556\\
564	0.45383591112737\\
565	0.469153736666777\\
566	0.481470579328016\\
567	0.490707653350576\\
568	0.496805872983202\\
569	0.49972623043106\\
570	0.499450045372511\\
571	0.495979084449419\\
572	0.489335549966673\\
573	0.479561937873201\\
574	0.466720765932923\\
575	0.450894173824405\\
576	0.43218339772723\\
577	0.410708122755927\\
578	0.386605717383679\\
579	0.360030354752872\\
580	0.331152026493121\\
581	0.300155455354953\\
582	0.267238913614594\\
583	0.232612954808019\\
584	0.196499066906861\\
585	0.159128255551241\\
586	0.120739566402014\\
587	0.081578556064292\\
588	0.041895721363166\\
589	0.00194489701887388\\
590	-0.0380183680291768\\
591	-0.0777384452635191\\
592	-0.1169612617388\\
593	-0.155435925281568\\
594	-0.19291632934399\\
595	-0.229162727245883\\
596	-0.263943265735326\\
597	-0.297035468058234\\
598	-0.328227657050367\\
599	-0.357320309148797\\
600	-0.384127330661833\\
};
\addlegendentry{$z(k)$}

\end{axis}

\begin{axis}[%
width=4.521in,
height=1.493in,
at={(0.758in,0.481in)},
scale only axis,
xmin=0,
xmax=600,
xlabel style={font=\color{white!15!black}},
xlabel={$k$},
ymin=-2,
ymax=3,
axis background/.style={fill=white},
title style={font=\bfseries},
title={Sygna� wyj�ciowy i warto�� zadana},
xmajorgrids,
ymajorgrids,
legend style={legend cell align=left, align=left, draw=white!15!black}
]
\addplot [color=mycolor1]
  table[row sep=crcr]{%
1	0\\
2	0\\
3	0\\
4	0\\
5	0\\
6	0\\
7	0\\
8	0\\
9	0\\
10	0\\
11	0\\
12	0\\
13	0\\
14	0\\
15	0\\
16	0\\
17	0\\
18	0\\
19	0\\
20	0\\
21	0\\
22	0\\
23	0\\
24	0\\
25	0\\
26	0\\
27	0\\
28	0\\
29	0\\
30	0\\
31	0\\
32	0\\
33	0\\
34	0\\
35	0\\
36	0\\
37	0\\
38	0\\
39	0\\
40	0\\
41	0\\
42	0\\
43	0\\
44	0\\
45	0\\
46	0\\
47	0\\
48	0\\
49	0\\
50	0\\
51	0\\
52	0\\
53	0\\
54	0\\
55	0\\
56	0\\
57	0\\
58	0\\
59	0\\
60	0\\
61	0\\
62	0\\
63	0\\
64	0\\
65	0\\
66	0\\
67	0\\
68	0\\
69	0\\
70	0\\
71	0\\
72	0\\
73	0\\
74	0\\
75	0\\
76	0\\
77	0\\
78	0\\
79	0\\
80	0\\
81	0\\
82	0\\
83	0\\
84	0\\
85	0\\
86	0\\
87	0\\
88	0\\
89	0\\
90	0\\
91	0\\
92	0\\
93	0\\
94	0\\
95	0\\
96	0\\
97	0\\
98	0\\
99	0\\
100	0\\
101	0\\
102	0\\
103	0\\
104	0\\
105	0.0898661853525702\\
106	0.227145989536648\\
107	0.381364985455071\\
108	0.532183744977198\\
109	0.667348874413722\\
110	0.780701113190663\\
111	0.870397690541912\\
112	0.937426745782853\\
113	0.984435604898583\\
114	1.01485816163926\\
115	1.03230537220306\\
116	1.04017299290448\\
117	1.04141871252479\\
118	1.03846391027622\\
119	1.03318119527604\\
120	1.0269360330158\\
121	1.02065802748983\\
122	1.01492410818031\\
123	1.01004159145033\\
124	1.0061236992538\\
125	1.00315363327061\\
126	1.00103582376953\\
127	0.999634653341271\\
128	0.998801966526442\\
129	0.99839518271941\\
130	0.998287979047469\\
131	0.998375425366301\\
132	0.998575232142644\\
133	0.998826486044746\\
134	0.99908694793467\\
135	0.999329706050688\\
136	0.99953973184925\\
137	0.999710685296531\\
138	0.999842161319463\\
139	0.9999374561398\\
140	1.00000185537463\\
141	1.00004139817149\\
142	1.00006204635586\\
143	1.00006917833604\\
144	1.00006732902273\\
145	1.00006010503136\\
146	1.00005021573868\\
147	1.00003957308605\\
148	1.00002942486313\\
149	1.00002049668706\\
150	1.00001312659169\\
151	1.00000738295854\\
152	1.00000316156378\\
153	1.00000026100877\\
154	0.999998438024651\\
155	0.999997445379893\\
156	0.999997055639453\\
157	0.999997074052887\\
158	0.99999734357207\\
159	0.999997744558619\\
160	0.999998191239006\\
161	0.999998626470347\\
162	0.999999015933754\\
163	0.999999342496316\\
164	0.999999601184079\\
165	0.999999794984267\\
166	0.999999931536935\\
167	1.00000002067327\\
168	1.00000007269799\\
169	1.0000000972857\\
170	1.00000010285557\\
171	1.00000009629762\\
172	1.00000008294079\\
173	1.00000006667322\\
174	1.00000005014576\\
175	1.00000003500852\\
176	1.00000002214677\\
177	1.00000001189507\\
178	1.0000000042189\\
179	0.999999998859796\\
180	0.999999995445021\\
181	0.999999993565414\\
182	0.999999992826672\\
183	0.999999992879594\\
184	0.999999993434638\\
185	0.99999999426549\\
186	0.999999995205538\\
187	0.999999996140285\\
188	0.999999996997944\\
189	0.999999997739762\\
190	0.999999998351042\\
191	0.999999998833424\\
192	0.99999999919862\\
193	0.999999999463627\\
194	0.999999999647282\\
195	0.999999999767949\\
196	0.999999999842131\\
197	0.999999999883761\\
198	0.999999999903995\\
199	0.999999999911317\\
200	0.999999999911846\\
201	0.999999999909724\\
202	0.999999999907537\\
203	0.999999999906703\\
204	0.999999999907817\\
205	0.999999999910932\\
206	0.999999999915784\\
207	0.999999999921947\\
208	0.999999999928953\\
209	0.999999999936356\\
210	0.99999999994378\\
211	0.999999999950931\\
212	0.999999999957603\\
213	0.999999999963665\\
214	0.999999999969054\\
215	0.999999999973759\\
216	0.999999999977806\\
217	0.999999999981245\\
218	0.999999999984139\\
219	0.999999999986558\\
220	0.999999999988573\\
221	0.999999999990247\\
222	0.999999999991641\\
223	0.999999999992805\\
224	0.999999999993781\\
225	0.999999999994607\\
226	0.999999999995309\\
227	0.999999999995912\\
228	0.999999999996432\\
229	0.999999999996883\\
230	0.999999999997277\\
231	0.999999999997622\\
232	0.999999999997925\\
233	0.999999999998191\\
234	0.999999999998424\\
235	0.999999999998629\\
236	0.999999999998809\\
237	0.999999999998967\\
238	0.999999999999105\\
239	0.999999999999226\\
240	0.999999999999331\\
241	0.999999999999423\\
242	0.999999999999503\\
243	0.999999999999573\\
244	0.999999999999632\\
245	0.999999999999683\\
246	0.999999999999727\\
247	0.999999999999764\\
248	0.999999999999796\\
249	0.999999999999823\\
250	0.999999999999846\\
251	0.999999999999867\\
252	0.999999999999885\\
253	0.9999999999999\\
254	0.999999999999913\\
255	0.999999999999924\\
256	0.999999999999934\\
257	0.999999999999942\\
258	0.999999999999948\\
259	0.999999999999954\\
260	0.999999999999958\\
261	0.999999999999961\\
262	0.999999999999964\\
263	0.999999999999967\\
264	0.999999999999969\\
265	0.999999999999972\\
266	0.999999999999975\\
267	0.999999999999979\\
268	0.999999999999982\\
269	0.999999999999985\\
270	0.999999999999987\\
271	1.00000086276672\\
272	1.00000423317379\\
273	1.0000120519818\\
274	1.0000260003166\\
275	1.00004705639391\\
276	1.00007523709427\\
277	1.00010952763252\\
278	1.00014813255248\\
279	1.00018884650625\\
280	1.00022941796769\\
281	1.00026470340528\\
282	1.00028981685954\\
283	1.0003016430238\\
284	1.00029925476994\\
285	1.00028368025733\\
286	1.00025735504373\\
287	1.00022349001474\\
288	1.00018549688746\\
289	1.00014654375849\\
290	1.0001092638004\\
291	1.00007560850438\\
292	1.00004681940536\\
293	1.0000234853142\\
294	1.0000056522805\\
295	0.999992957931546\\
296	0.99998476825077\\
297	0.999980301698218\\
298	0.999978731833114\\
299	0.999979264735191\\
300	0.999981191336499\\
301	0.999983917301141\\
302	0.933685054721457\\
303	0.878912175347626\\
304	0.834539645585111\\
305	0.799571329642537\\
306	0.773120939228023\\
307	0.760353431941606\\
308	0.762661724378852\\
309	0.779038903582909\\
310	0.807132601403288\\
311	0.844009636007237\\
312	0.88667415157993\\
313	0.932387962640666\\
314	0.978840830529468\\
315	1.02421352105156\\
316	1.06716964217226\\
317	1.10680480707792\\
318	1.14257451006162\\
319	1.17421576941337\\
320	1.20167233542011\\
321	1.22502914204189\\
322	1.2444586324563\\
323	1.26017947493524\\
324	1.27242683824571\\
325	1.28143264230104\\
326	1.28741388023617\\
327	1.29056708478077\\
328	1.29106717318879\\
329	1.28906916561597\\
330	1.2847115700848\\
331	1.27812052164268\\
332	1.26941402881062\\
333	1.25870590401555\\
334	1.24610913220586\\
335	1.23173856484286\\
336	1.21571291992741\\
337	1.19815612930825\\
338	1.17919810929682\\
339	1.15897504627613\\
340	1.13762929139829\\
341	1.11530895243091\\
342	1.0921672600731\\
343	1.06836177335\\
344	1.0440534758788\\
345	1.0194058030412\\
346	0.994583630023024\\
347	0.969752242524576\\
348	0.945076305677061\\
349	0.920718842134172\\
350	0.896840227172275\\
351	0.873597206631874\\
352	0.851141942379827\\
353	0.829621089411346\\
354	0.809174908530684\\
355	0.7899364185824\\
356	0.772030592327091\\
357	0.75557360018016\\
358	0.740672106104392\\
359	0.727422619936005\\
360	0.715910910316668\\
361	0.706211482200272\\
362	0.698387122610488\\
363	0.692488517955409\\
364	0.688553945772568\\
365	0.686609043295508\\
366	0.686666654714943\\
367	0.688726758465092\\
368	0.692776475309026\\
369	0.698790157434136\\
370	0.70672955820703\\
371	0.716544081681727\\
372	0.728171110410636\\
373	0.741536409578089\\
374	0.756554604964426\\
375	0.773129731757513\\
376	0.791155850760581\\
377	0.810517728102804\\
378	0.831091574144095\\
379	0.852745836880358\\
380	0.875342044801816\\
381	0.898735693836793\\
382	0.922777172728375\\
383	0.947312720943025\\
384	0.97218541300018\\
385	0.99723616294104\\
386	1.02230474252445\\
387	1.04723080664856\\
388	1.07185491944942\\
389	1.09601957452229\\
390	1.11957020274776\\
391	1.14235616128333\\
392	1.16423169740047\\
393	1.1850568810074\\
394	1.20469849989743\\
395	1.22303091200037\\
396	1.23993684918905\\
397	1.25530816750255\\
398	1.26904653898972\\
399	1.28106408074988\\
400	1.29128391714904\\
401	1.29964067161687\\
402	1.30608088488015\\
403	1.31056335695866\\
404	1.31305941073691\\
405	1.31355307542677\\
406	1.31204118874811\\
407	1.30853341717471\\
408	1.30305219411632\\
409	1.29563257643301\\
410	1.28632202019991\\
411	1.27518007715711\\
412	1.2622780137868\\
413	1.24769835545447\\
414	1.23153435853029\\
415	1.21388941386771\\
416	1.19487638545497\\
417	1.17461688847006\\
418	1.15324051135745\\
419	1.13088398690247\\
420	1.10769031760594\\
421	1.08380786095371\\
422	1.05938938043227\\
423	1.0345910683608\\
424	1.00957154679029\\
425	0.984490852860497\\
426	0.959509415105162\\
427	0.934787027253577\\
428	0.91048182609272\\
429	0.886749279928191\\
430	0.863741194114353\\
431	0.841604740014891\\
432	0.820481513605172\\
433	0.800506629738109\\
434	0.781807857867182\\
435	0.76450480475501\\
436	0.748708149395386\\
437	0.734518935050915\\
438	0.722027922937257\\
439	0.711315011682936\\
440	0.702448726268734\\
441	0.695485779706244\\
442	0.690470710253125\\
443	0.687435596484091\\
444	0.686399852046514\\
445	0.687370101424809\\
446	0.690340137520302\\
447	0.695290961266942\\
448	0.702190903067609\\
449	0.710995825324789\\
450	0.721649404799409\\
451	0.734083492992638\\
452	0.748218552224268\\
453	0.76396416458583\\
454	0.781219610467832\\
455	0.799874512926259\\
456	0.81980954375123\\
457	0.840897186833329\\
458	0.863002553805687\\
459	0.885984246728671\\
460	0.909695262340677\\
461	0.933983932147925\\
462	0.958694892388656\\
463	0.983670077697726\\
464	1.00874973212662\\
465	1.03377343104692\\
466	1.0585811073855\\
467	1.08301407560694\\
468	1.10691604687323\\
469	1.13013412886992\\
470	1.15251980389058\\
471	1.17392987891577\\
472	1.19422740160624\\
473	1.21328259355528\\
474	1.23097372270467\\
475	1.24718788462814\\
476	1.26182168947855\\
477	1.27478186395619\\
478	1.28598578092804\\
479	1.29536192739149\\
480	1.30285032840042\\
481	1.30840292801505\\
482	1.31198391942426\\
483	1.31356980511967\\
484	1.31314961622393\\
485	1.3107251422362\\
486	1.30631108158584\\
487	1.29993506762915\\
488	1.29163755515067\\
489	1.28147157131192\\
490	1.26950234502157\\
491	1.25580683250626\\
492	1.24047315671191\\
493	1.22359997585482\\
494	1.20529579327721\\
495	1.18567821760004\\
496	1.16487317950698\\
497	1.1430141095566\\
498	1.1202410802406\\
499	1.096699914999\\
500	1.07254126692681\\
501	1.04791967029807\\
502	1.02299256864006\\
503	0.997919323783174\\
504	0.972860210991027\\
505	0.947975405873452\\
506	0.923423969260302\\
507	0.899362836549022\\
508	0.875945818231516\\
509	0.853322618365525\\
510	0.831637877698325\\
511	0.811030247994607\\
512	0.791631503884509\\
513	0.77356569824899\\
514	0.756948366812373\\
515	0.741885787227711\\
516	0.728474297528181\\
517	0.716799678383605\\
518	0.706936603149669\\
519	0.698948159231529\\
520	0.692885443805404\\
521	0.688787236452806\\
522	0.68667975076382\\
523	0.686576466459553\\
524	0.688478043071231\\
525	0.692372315696265\\
526	0.698234372832151\\
527	0.706026715769875\\
528	0.715699498512199\\
529	0.727190846671933\\
530	0.740427253304004\\
531	0.755324049136084\\
532	0.771785944188869\\
533	0.789707637321955\\
534	0.808974489807679\\
535	0.829463258626171\\
536	0.851042884792984\\
537	0.873575331678525\\
538	0.896916467958509\\
539	0.920916989548842\\
540	0.945423374628687\\
541	0.970278865643492\\
542	0.995324472006932\\
543	1.0203999870881\\
544	1.04534501297877\\
545	1.06999998648561\\
546	1.09420719978445\\
547	1.11781180920764\\
548	1.14066282571154\\
549	1.16261408068839\\
550	1.18352516094447\\
551	1.20326230686374\\
552	1.2216992680116\\
553	1.23871811070583\\
554	1.25420997238888\\
555	1.26807575797603\\
556	1.28022677372521\\
557	1.2905852945737\\
558	1.2990850613129\\
559	1.30567170442062\\
560	1.31030309184022\\
561	1.31294959848152\\
562	1.31359429572005\\
563	1.31223305968212\\
564	1.30887459762331\\
565	1.30354039223147\\
566	1.29626456421069\\
567	1.28709365402498\\
568	1.276086324198\\
569	1.26331298407294\\
570	1.2488553394329\\
571	1.2328058698626\\
572	1.21526723719454\\
573	1.19635162882353\\
574	1.17618004009019\\
575	1.15488150032357\\
576	1.13259224749376\\
577	1.10945485675377\\
578	1.08561732844505\\
579	1.06123214140034\\
580	1.0364552775995\\
581	1.01144522441709\\
582	0.986361960844022\\
583	0.961365934167942\\
584	0.936617033658131\\
585	0.912273567819798\\
586	0.888491251759883\\
587	0.865422211141764\\
588	0.843214009100144\\
589	0.822008702340559\\
590	0.801941932461241\\
591	0.783142058309774\\
592	0.765729334924507\\
593	0.749815144312688\\
594	0.735501282985716\\
595	0.722879310808857\\
596	0.712029965330565\\
597	0.703022645337693\\
598	0.695914966940086\\
599	0.690752395024103\\
600	0.687567952432491\\
};
\addlegendentry{$y(k)$}

\addplot[const plot, color=mycolor2, dashed] table[row sep=crcr] {%
1	0\\
2	0\\
3	0\\
4	0\\
5	0\\
6	0\\
7	0\\
8	0\\
9	0\\
10	0\\
11	0\\
12	0\\
13	0\\
14	0\\
15	0\\
16	0\\
17	0\\
18	0\\
19	0\\
20	0\\
21	0\\
22	0\\
23	0\\
24	0\\
25	0\\
26	0\\
27	0\\
28	0\\
29	0\\
30	0\\
31	0\\
32	0\\
33	0\\
34	0\\
35	0\\
36	0\\
37	0\\
38	0\\
39	0\\
40	0\\
41	0\\
42	0\\
43	0\\
44	0\\
45	0\\
46	0\\
47	0\\
48	0\\
49	0\\
50	0\\
51	0\\
52	0\\
53	0\\
54	0\\
55	0\\
56	0\\
57	0\\
58	0\\
59	0\\
60	0\\
61	0\\
62	0\\
63	0\\
64	0\\
65	0\\
66	0\\
67	0\\
68	0\\
69	0\\
70	0\\
71	0\\
72	0\\
73	0\\
74	0\\
75	0\\
76	0\\
77	0\\
78	0\\
79	0\\
80	0\\
81	0\\
82	0\\
83	0\\
84	0\\
85	0\\
86	0\\
87	0\\
88	0\\
89	0\\
90	0\\
91	0\\
92	0\\
93	0\\
94	0\\
95	0\\
96	0\\
97	0\\
98	0\\
99	0\\
100	1\\
101	1\\
102	1\\
103	1\\
104	1\\
105	1\\
106	1\\
107	1\\
108	1\\
109	1\\
110	1\\
111	1\\
112	1\\
113	1\\
114	1\\
115	1\\
116	1\\
117	1\\
118	1\\
119	1\\
120	1\\
121	1\\
122	1\\
123	1\\
124	1\\
125	1\\
126	1\\
127	1\\
128	1\\
129	1\\
130	1\\
131	1\\
132	1\\
133	1\\
134	1\\
135	1\\
136	1\\
137	1\\
138	1\\
139	1\\
140	1\\
141	1\\
142	1\\
143	1\\
144	1\\
145	1\\
146	1\\
147	1\\
148	1\\
149	1\\
150	1\\
151	1\\
152	1\\
153	1\\
154	1\\
155	1\\
156	1\\
157	1\\
158	1\\
159	1\\
160	1\\
161	1\\
162	1\\
163	1\\
164	1\\
165	1\\
166	1\\
167	1\\
168	1\\
169	1\\
170	1\\
171	1\\
172	1\\
173	1\\
174	1\\
175	1\\
176	1\\
177	1\\
178	1\\
179	1\\
180	1\\
181	1\\
182	1\\
183	1\\
184	1\\
185	1\\
186	1\\
187	1\\
188	1\\
189	1\\
190	1\\
191	1\\
192	1\\
193	1\\
194	1\\
195	1\\
196	1\\
197	1\\
198	1\\
199	1\\
200	1\\
201	1\\
202	1\\
203	1\\
204	1\\
205	1\\
206	1\\
207	1\\
208	1\\
209	1\\
210	1\\
211	1\\
212	1\\
213	1\\
214	1\\
215	1\\
216	1\\
217	1\\
218	1\\
219	1\\
220	1\\
221	1\\
222	1\\
223	1\\
224	1\\
225	1\\
226	1\\
227	1\\
228	1\\
229	1\\
230	1\\
231	1\\
232	1\\
233	1\\
234	1\\
235	1\\
236	1\\
237	1\\
238	1\\
239	1\\
240	1\\
241	1\\
242	1\\
243	1\\
244	1\\
245	1\\
246	1\\
247	1\\
248	1\\
249	1\\
250	1\\
251	1\\
252	1\\
253	1\\
254	1\\
255	1\\
256	1\\
257	1\\
258	1\\
259	1\\
260	1\\
261	1\\
262	1\\
263	1\\
264	1\\
265	1\\
266	1\\
267	1\\
268	1\\
269	1\\
270	1\\
271	1\\
272	1\\
273	1\\
274	1\\
275	1\\
276	1\\
277	1\\
278	1\\
279	1\\
280	1\\
281	1\\
282	1\\
283	1\\
284	1\\
285	1\\
286	1\\
287	1\\
288	1\\
289	1\\
290	1\\
291	1\\
292	1\\
293	1\\
294	1\\
295	1\\
296	1\\
297	1\\
298	1\\
299	1\\
300	1\\
301	1\\
302	1\\
303	1\\
304	1\\
305	1\\
306	1\\
307	1\\
308	1\\
309	1\\
310	1\\
311	1\\
312	1\\
313	1\\
314	1\\
315	1\\
316	1\\
317	1\\
318	1\\
319	1\\
320	1\\
321	1\\
322	1\\
323	1\\
324	1\\
325	1\\
326	1\\
327	1\\
328	1\\
329	1\\
330	1\\
331	1\\
332	1\\
333	1\\
334	1\\
335	1\\
336	1\\
337	1\\
338	1\\
339	1\\
340	1\\
341	1\\
342	1\\
343	1\\
344	1\\
345	1\\
346	1\\
347	1\\
348	1\\
349	1\\
350	1\\
351	1\\
352	1\\
353	1\\
354	1\\
355	1\\
356	1\\
357	1\\
358	1\\
359	1\\
360	1\\
361	1\\
362	1\\
363	1\\
364	1\\
365	1\\
366	1\\
367	1\\
368	1\\
369	1\\
370	1\\
371	1\\
372	1\\
373	1\\
374	1\\
375	1\\
376	1\\
377	1\\
378	1\\
379	1\\
380	1\\
381	1\\
382	1\\
383	1\\
384	1\\
385	1\\
386	1\\
387	1\\
388	1\\
389	1\\
390	1\\
391	1\\
392	1\\
393	1\\
394	1\\
395	1\\
396	1\\
397	1\\
398	1\\
399	1\\
400	1\\
401	1\\
402	1\\
403	1\\
404	1\\
405	1\\
406	1\\
407	1\\
408	1\\
409	1\\
410	1\\
411	1\\
412	1\\
413	1\\
414	1\\
415	1\\
416	1\\
417	1\\
418	1\\
419	1\\
420	1\\
421	1\\
422	1\\
423	1\\
424	1\\
425	1\\
426	1\\
427	1\\
428	1\\
429	1\\
430	1\\
431	1\\
432	1\\
433	1\\
434	1\\
435	1\\
436	1\\
437	1\\
438	1\\
439	1\\
440	1\\
441	1\\
442	1\\
443	1\\
444	1\\
445	1\\
446	1\\
447	1\\
448	1\\
449	1\\
450	1\\
451	1\\
452	1\\
453	1\\
454	1\\
455	1\\
456	1\\
457	1\\
458	1\\
459	1\\
460	1\\
461	1\\
462	1\\
463	1\\
464	1\\
465	1\\
466	1\\
467	1\\
468	1\\
469	1\\
470	1\\
471	1\\
472	1\\
473	1\\
474	1\\
475	1\\
476	1\\
477	1\\
478	1\\
479	1\\
480	1\\
481	1\\
482	1\\
483	1\\
484	1\\
485	1\\
486	1\\
487	1\\
488	1\\
489	1\\
490	1\\
491	1\\
492	1\\
493	1\\
494	1\\
495	1\\
496	1\\
497	1\\
498	1\\
499	1\\
500	1\\
501	1\\
502	1\\
503	1\\
504	1\\
505	1\\
506	1\\
507	1\\
508	1\\
509	1\\
510	1\\
511	1\\
512	1\\
513	1\\
514	1\\
515	1\\
516	1\\
517	1\\
518	1\\
519	1\\
520	1\\
521	1\\
522	1\\
523	1\\
524	1\\
525	1\\
526	1\\
527	1\\
528	1\\
529	1\\
530	1\\
531	1\\
532	1\\
533	1\\
534	1\\
535	1\\
536	1\\
537	1\\
538	1\\
539	1\\
540	1\\
541	1\\
542	1\\
543	1\\
544	1\\
545	1\\
546	1\\
547	1\\
548	1\\
549	1\\
550	1\\
551	1\\
552	1\\
553	1\\
554	1\\
555	1\\
556	1\\
557	1\\
558	1\\
559	1\\
560	1\\
561	1\\
562	1\\
563	1\\
564	1\\
565	1\\
566	1\\
567	1\\
568	1\\
569	1\\
570	1\\
571	1\\
572	1\\
573	1\\
574	1\\
575	1\\
576	1\\
577	1\\
578	1\\
579	1\\
580	1\\
581	1\\
582	1\\
583	1\\
584	1\\
585	1\\
586	1\\
587	1\\
588	1\\
589	1\\
590	1\\
591	1\\
592	1\\
593	1\\
594	1\\
595	1\\
596	1\\
597	1\\
598	1\\
599	1\\
600	1\\
};
\addlegendentry{$y^{\mathrm{zad}}(k)$}

\end{axis}
\end{tikzpicture}%
	\caption{Przebiegi sygna��w dla regulacji przy pomocy algorytmu DMC bez uwzgl�dnienia zak��ce� sinusoidalnych}
\end{figure}

Wska�nik ilo�ciowy: 21.4699

\begin{figure}[h!]
	\centering
	% This file was created by matlab2tikz.
%
\definecolor{mycolor1}{rgb}{0.00000,0.44700,0.74100}%
\definecolor{mycolor2}{rgb}{0.85000,0.32500,0.09800}%
%
\begin{tikzpicture}

\begin{axis}[%
width=4.521in,
height=1.493in,
at={(0.758in,2.554in)},
scale only axis,
xmin=0,
xmax=600,
xlabel style={font=\color{white!15!black}},
xlabel={$k$},
ymin=-4,
ymax=3,
axis background/.style={fill=white},
title style={font=\bfseries},
title={Sygna� wej�ciowy sterowania i zak��cenia},
xmajorgrids,
ymajorgrids,
legend style={legend cell align=left, align=left, draw=white!15!black}
]
\addplot[const plot, color=mycolor1] table[row sep=crcr] {%
1	0\\
2	0\\
3	0\\
4	0\\
5	0\\
6	0\\
7	0\\
8	0\\
9	0\\
10	0\\
11	0\\
12	0\\
13	0\\
14	0\\
15	0\\
16	0\\
17	0\\
18	0\\
19	0\\
20	0\\
21	0\\
22	0\\
23	0\\
24	0\\
25	0\\
26	0\\
27	0\\
28	0\\
29	0\\
30	0\\
31	0\\
32	0\\
33	0\\
34	0\\
35	0\\
36	0\\
37	0\\
38	0\\
39	0\\
40	0\\
41	0\\
42	0\\
43	0\\
44	0\\
45	0\\
46	0\\
47	0\\
48	0\\
49	0\\
50	0\\
51	0\\
52	0\\
53	0\\
54	0\\
55	0\\
56	0\\
57	0\\
58	0\\
59	0\\
60	0\\
61	0\\
62	0\\
63	0\\
64	0\\
65	0\\
66	0\\
67	0\\
68	0\\
69	0\\
70	0\\
71	0\\
72	0\\
73	0\\
74	0\\
75	0\\
76	0\\
77	0\\
78	0\\
79	0\\
80	0\\
81	0\\
82	0\\
83	0\\
84	0\\
85	0\\
86	0\\
87	0\\
88	0\\
89	0\\
90	0\\
91	0\\
92	0\\
93	0\\
94	0\\
95	0\\
96	0\\
97	0\\
98	0\\
99	0\\
100	1.06438689272261\\
101	1.67330799970846\\
102	1.94630786721308\\
103	1.9873867877016\\
104	1.88160065495093\\
105	1.69466404103899\\
106	1.47444334123278\\
107	1.2534827815949\\
108	1.05194116533132\\
109	0.88051830203168\\
110	0.743112059095731\\
111	0.639070284192932\\
112	0.564990187660185\\
113	0.516076581765213\\
114	0.487105661396056\\
115	0.47305860386581\\
116	0.469494351726138\\
117	0.472727848975606\\
118	0.479872134885331\\
119	0.488792593525579\\
120	0.498011076122221\\
121	0.506587690199236\\
122	0.513999424293346\\
123	0.520027728514665\\
124	0.524661729730118\\
125	0.528019799977936\\
126	0.530289507139993\\
127	0.531684313584522\\
128	0.532414508349938\\
129	0.532669540981234\\
130	0.53260898363768\\
131	0.532359633484115\\
132	0.532016667214764\\
133	0.531647194123235\\
134	0.531294971082772\\
135	0.53098541140078\\
136	0.530730325128121\\
137	0.530532067662125\\
138	0.530386950263108\\
139	0.530287888418633\\
140	0.530226341679766\\
141	0.53019364174679\\
142	0.530181823531268\\
143	0.530184074694741\\
144	0.530194909262603\\
145	0.530210155308029\\
146	0.530226828978867\\
147	0.530242949702892\\
148	0.530257335712222\\
149	0.530269405819655\\
150	0.530279002893992\\
151	0.530286246619096\\
152	0.530291417585385\\
153	0.530294871160459\\
154	0.53029697750129\\
155	0.530298083110692\\
156	0.5302984891616\\
157	0.530298442131522\\
158	0.53029813288721\\
159	0.530297701075738\\
160	0.53029724240299\\
161	0.5302968170457\\
162	0.530296458011551\\
163	0.530296178719471\\
164	0.530295979420275\\
165	0.530295852326937\\
166	0.530295785489372\\
167	0.530295765547805\\
168	0.530295779548015\\
169	0.530295816015713\\
170	0.53029586547824\\
171	0.530295920599273\\
172	0.530295976063461\\
173	0.530296028317908\\
174	0.530296075249234\\
175	0.530296115850542\\
176	0.530296149912657\\
177	0.530296177758576\\
178	0.530296200028913\\
179	0.530296217518459\\
180	0.530296231059335\\
181	0.530296241443675\\
182	0.530296249377924\\
183	0.530296255460952\\
184	0.530296260179045\\
185	0.530296263911903\\
186	0.530296266945044\\
187	0.530296269485158\\
188	0.530296271676013\\
189	0.530296273613349\\
190	0.5302962753579\\
191	0.530296276946142\\
192	0.53029627839874\\
193	0.530296279726851\\
194	0.530296280936599\\
195	0.530296282032038\\
196	0.53029628301696\\
197	0.530296283895852\\
198	0.530296284674259\\
199	0.530296285358785\\
200	0.53029628595687\\
201	0.530296286476478\\
202	0.530296286925765\\
203	0.530296287312782\\
204	0.53029628764523\\
205	0.530296287930287\\
206	0.530296288174488\\
207	0.530296288383675\\
208	0.530296288562982\\
209	0.530296288716863\\
210	0.530296288849134\\
211	0.530296288963043\\
212	0.530296289061329\\
213	0.530296289146296\\
214	0.530296289219876\\
215	0.530296289283692\\
216	0.530296289339104\\
217	0.530296289387264\\
218	0.530296289429144\\
219	0.530296289465574\\
220	0.530296289497263\\
221	0.530296289524822\\
222	0.530296289548782\\
223	0.530296289569602\\
224	0.530296289587684\\
225	0.530296289603381\\
226	0.530296289617\\
227	0.53029628962881\\
228	0.530296289639049\\
229	0.530296289647923\\
230	0.530296289655612\\
231	0.530296289662274\\
232	0.530296289668046\\
233	0.530296289673046\\
234	0.530296289677378\\
235	0.53029628968113\\
236	0.53029628968438\\
237	0.530296289687193\\
238	0.530296289689628\\
239	0.530296289691737\\
240	0.530296289693563\\
241	0.530296289695144\\
242	0.530296289696515\\
243	0.530296289697705\\
244	0.530296289698736\\
245	0.530296289699632\\
246	0.53029628970041\\
247	0.530296289701086\\
248	0.530296289701672\\
249	0.53029628970218\\
250	0.530296289702619\\
251	0.530296289702997\\
252	0.530296289703324\\
253	0.530296289703606\\
254	0.530296289703851\\
255	0.530296289704065\\
256	0.530296289704251\\
257	0.530296289704414\\
258	0.530296289704558\\
259	0.530296289704683\\
260	0.530296289704794\\
261	0.530296289704891\\
262	0.530296289704976\\
263	0.530296289705049\\
264	0.530296289705111\\
265	0.530296289705163\\
266	0.530306508427582\\
267	0.53033666381662\\
268	0.530391127427623\\
269	0.530467847736317\\
270	0.530559386950657\\
271	0.530654876638281\\
272	0.530742108994647\\
273	0.530811305149986\\
274	0.530856665729264\\
275	0.530876482260541\\
276	0.530835314532807\\
277	0.530733501687177\\
278	0.530589437566782\\
279	0.530427392420687\\
280	0.530270030302657\\
281	0.530134556578039\\
282	0.530031433039188\\
283	0.529964727279454\\
284	0.529933350393245\\
285	0.529932636565899\\
286	0.529955901281582\\
287	0.529995766900072\\
288	0.530045160082298\\
289	0.530097966103479\\
290	0.530149375192492\\
291	0.530195981986413\\
292	0.53023570754184\\
293	0.530267610057856\\
294	0.530291640471443\\
295	0.530308386143677\\
296	0.530318832592561\\
297	0.530324161292871\\
298	0.530325591820423\\
299	0.530324269357558\\
300	0.530321193702817\\
301	0.530317183116816\\
302	0.939003885534567\\
303	1.242844456494\\
304	1.36961242064307\\
305	1.37113838880314\\
306	1.29075812294293\\
307	1.16268667202267\\
308	1.01245224086321\\
309	0.857945124717722\\
310	0.710751883306419\\
311	0.577545480557067\\
312	0.461384809118545\\
313	0.362841251985164\\
314	0.280916995896937\\
315	0.21375204145139\\
316	0.15913703026406\\
317	0.11485992827404\\
318	0.0789188261349202\\
319	0.0496328249690404\\
320	0.0256799218786773\\
321	0.00608632524312788\\
322	-0.00981333477831559\\
323	-0.0224302500832234\\
324	-0.0319797721979923\\
325	-0.0385370630758046\\
326	-0.0420860987431023\\
327	-0.0425601628896694\\
328	-0.0398734302995896\\
329	-0.0339441795266506\\
330	-0.024710706232096\\
331	-0.0121412412545088\\
332	0.00376079555681341\\
333	0.0229549771171638\\
334	0.0453644308774876\\
335	0.0708779259819524\\
336	0.0993528313643932\\
337	0.130618471932471\\
338	0.164479563748674\\
339	0.200719531823053\\
340	0.239103606592341\\
341	0.279381660596797\\
342	0.321290789703933\\
343	0.364551743928446\\
344	0.40888254569579\\
345	0.453997965980058\\
346	0.499609659026843\\
347	0.545426853294006\\
348	0.591157484564669\\
349	0.636509659030629\\
350	0.681192189757563\\
351	0.724917279192552\\
352	0.767402937851751\\
353	0.808396687240636\\
354	0.847617614106866\\
355	0.884803482145685\\
356	0.919710526023809\\
357	0.952113370542026\\
358	0.9818051280661\\
359	1.00859767972442\\
360	1.03232211540811\\
361	1.0528292905652\\
362	1.06999045063616\\
363	1.08369787367293\\
364	1.09386548571457\\
365	1.10042940992669\\
366	1.10334841793768\\
367	1.10260425926473\\
368	1.09820185161937\\
369	1.09016932089859\\
370	1.07855788468317\\
371	1.06344157709561\\
372	1.04491681600985\\
373	1.02310181598963\\
374	0.998135852110243\\
375	0.970178381131491\\
376	0.939408027462349\\
377	0.906021442094192\\
378	0.870232043258028\\
379	0.83226864804053\\
380	0.792374004612138\\
381	0.750803235101584\\
382	0.707822199506141\\
383	0.663705791358537\\
384	0.618736176176949\\
385	0.573200983997528\\
386	0.527391467521595\\
387	0.48160063759398\\
388	0.436121387857241\\
389	0.391244620492262\\
390	0.347257384953936\\
391	0.304441041537453\\
392	0.263069461464105\\
393	0.223407274954322\\
394	0.185708178460288\\
395	0.150213311862012\\
396	0.117149715991366\\
397	0.0867288803410859\\
398	0.0591453902434635\\
399	0.0345756821703637\\
400	0.0131769151166641\\
401	-0.00491403471224719\\
402	-0.0195814494764311\\
403	-0.0307315099381844\\
404	-0.0382928956228689\\
405	-0.0422172410633553\\
406	-0.0424794452086119\\
407	-0.0390778320165451\\
408	-0.0320341612034618\\
409	-0.0213934890812607\\
410	-0.00722388037259267\\
411	0.0103840271524187\\
412	0.0313176024327568\\
413	0.0554429415576085\\
414	0.0826057242832674\\
415	0.112632201146733\\
416	0.145330304861543\\
417	0.180490878886447\\
418	0.217889015308018\\
419	0.257285493479163\\
420	0.298428310210972\\
421	0.341054291729784\\
422	0.384890777088337\\
423	0.429657362262852\\
424	0.475067693779706\\
425	0.520831300398548\\
426	0.566655451135311\\
427	0.612247027740085\\
428	0.657314399652336\\
429	0.70156928944016\\
430	0.744728616791076\\
431	0.786516309259139\\
432	0.826665068283814\\
433	0.864918079119037\\
434	0.901030653645542\\
435	0.934771795504843\\
436	0.965925677538625\\
437	0.994293022116215\\
438	1.01969237557303\\
439	1.04196126869989\\
440	1.06095725592273\\
441	1.07655882653985\\
442	1.08866618144797\\
443	1.09720187141778\\
444	1.10211129300289\\
445	1.10336303868512\\
446	1.10094909869587\\
447	1.0948849129613\\
448	1.08520927269456\\
449	1.071984072098\\
450	1.05529391186768\\
451	1.03524555724228\\
452	1.01196725556252\\
453	0.985607914793611\\
454	0.956336149650663\\
455	0.924339202102004\\
456	0.88982174335857\\
457	0.853004564944436\\
458	0.814123167030739\\
459	0.773426252848457\\
460	0.731174138628685\\
461	0.68763708911653\\
462	0.643093589241259\\
463	0.597828562985207\\
464	0.55213155086852\\
465	0.506294857753049\\
466	0.460611682866567\\
467	0.415374244060238\\
468	0.370875832008041\\
469	0.327405515459901\\
470	0.285243580284315\\
471	0.24465899069806\\
472	0.205908049764974\\
473	0.16923365529734\\
474	0.134864749090198\\
475	0.103016489237043\\
476	0.0738897506722455\\
477	0.0476699688907016\\
478	0.024511397688479\\
479	0.0045601848966666\\
480	-0.0120482305486445\\
481	-0.0251962657569326\\
482	-0.0347888672762811\\
483	-0.0407561960276713\\
484	-0.0430548316508357\\
485	-0.0416679561892253\\
486	-0.0366048937872629\\
487	-0.0279002890699304\\
488	-0.0156131177266119\\
489	0.000174352867054584\\
490	0.0193575938610962\\
491	0.0418108184435396\\
492	0.0673879675005172\\
493	0.0959237003238692\\
494	0.127234418140702\\
495	0.161119346891562\\
496	0.1973617006188\\
497	0.235729939997665\\
498	0.275979133350006\\
499	0.317852420832379\\
500	0.361082576893635\\
501	0.405393661756343\\
502	0.450502749587952\\
503	0.496121719059313\\
504	0.5419590909427\\
505	0.587721897062248\\
506	0.633117565071409\\
507	0.677855804018249\\
508	0.721650476330744\\
509	0.764221442552194\\
510	0.80529636607347\\
511	0.84461246578959\\
512	0.881918205285922\\
513	0.916974907820636\\
514	0.949558287014127\\
515	0.979459883790113\\
516	1.00648840072578\\
517	1.03047092564609\\
518	1.05125403697773\\
519	1.06870478449254\\
520	1.08271153839665\\
521	1.09318470190459\\
522	1.10005728305397\\
523	1.10328532223687\\
524	1.10284817273286\\
525	1.09874863240648\\
526	1.09101292573789\\
527	1.07969053613088\\
528	1.06485388947149\\
529	1.04659789016125\\
530	1.02503931501607\\
531	1.00031606660152\\
532	0.972586291066335\\
533	0.942027366320128\\
534	0.908834767159352\\
535	0.873220814675507\\
536	0.83541331797816\\
537	0.795654116926613\\
538	0.754197535181014\\
539	0.711308753449866\\
540	0.667262113319714\\
541	0.622339362498867\\
542	0.576827852685622\\
543	0.531018701578949\\
544	0.485204930783113\\
545	0.439679591515409\\
546	0.394733890106731\\
547	0.350655325287711\\
548	0.307725849178608\\
549	0.26622006374981\\
550	0.226403464292691\\
551	0.18853074113943\\
552	0.152844150497154\\
553	0.119571964819119\\
554	0.0889270126262544\\
555	0.0611053171197516\\
556	0.0362848422930288\\
557	0.0146243545635668\\
558	-0.00373759279404734\\
559	-0.018683545918098\\
560	-0.0301179016640944\\
561	-0.0379675191380944\\
562	-0.0421821875476543\\
563	-0.0427349473780335\\
564	-0.0396222628394768\\
565	-0.0328640444826332\\
566	-0.0225035218375702\\
567	-0.00860696689111001\\
568	0.00873672982868348\\
569	0.0294166278492703\\
570	0.0533004463945827\\
571	0.0802354105309201\\
572	0.11004922840537\\
573	0.142551193325806\\
574	0.177533403632928\\
575	0.21477209256138\\
576	0.254029059583358\\
577	0.295053194079045\\
578	0.337582081587564\\
579	0.381343682363957\\
580	0.426058071505127\\
581	0.47143922951391\\
582	0.517196871847759\\
583	0.563038305749172\\
584	0.60867030248048\\
585	0.653800972987096\\
586	0.698141634991365\\
587	0.741408659573997\\
588	0.783325285431222\\
589	0.82362338920269\\
590	0.862045200545975\\
591	0.898344950987108\\
592	0.932290446000114\\
593	0.963664550259582\\
594	0.992266576565764\\
595	1.0179135695578\\
596	1.04044147600365\\
597	1.05970619418097\\
598	1.07558449563635\\
599	1.08797481342688\\
600	1.09679789180203\\
};
\addlegendentry{$u(k)$}

\addplot[const plot, color=mycolor2] table[row sep=crcr] {%
1	0\\
2	0\\
3	0\\
4	0\\
5	0\\
6	0\\
7	0\\
8	0\\
9	0\\
10	0\\
11	0\\
12	0\\
13	0\\
14	0\\
15	0\\
16	0\\
17	0\\
18	0\\
19	0\\
20	0\\
21	0\\
22	0\\
23	0\\
24	0\\
25	0\\
26	0\\
27	0\\
28	0\\
29	0\\
30	0\\
31	0\\
32	0\\
33	0\\
34	0\\
35	0\\
36	0\\
37	0\\
38	0\\
39	0\\
40	0\\
41	0\\
42	0\\
43	0\\
44	0\\
45	0\\
46	0\\
47	0\\
48	0\\
49	0\\
50	0\\
51	0\\
52	0\\
53	0\\
54	0\\
55	0\\
56	0\\
57	0\\
58	0\\
59	0\\
60	0\\
61	0\\
62	0\\
63	0\\
64	0\\
65	0\\
66	0\\
67	0\\
68	0\\
69	0\\
70	0\\
71	0\\
72	0\\
73	0\\
74	0\\
75	0\\
76	0\\
77	0\\
78	0\\
79	0\\
80	0\\
81	0\\
82	0\\
83	0\\
84	0\\
85	0\\
86	0\\
87	0\\
88	0\\
89	0\\
90	0\\
91	0\\
92	0\\
93	0\\
94	0\\
95	0\\
96	0\\
97	0\\
98	0\\
99	0\\
100	0\\
101	0\\
102	0\\
103	0\\
104	0\\
105	0\\
106	0\\
107	0\\
108	0\\
109	0\\
110	0\\
111	0\\
112	0\\
113	0\\
114	0\\
115	0\\
116	0\\
117	0\\
118	0\\
119	0\\
120	0\\
121	0\\
122	0\\
123	0\\
124	0\\
125	0\\
126	0\\
127	0\\
128	0\\
129	0\\
130	0\\
131	0\\
132	0\\
133	0\\
134	0\\
135	0\\
136	0\\
137	0\\
138	0\\
139	0\\
140	0\\
141	0\\
142	0\\
143	0\\
144	0\\
145	0\\
146	0\\
147	0\\
148	0\\
149	0\\
150	0\\
151	0\\
152	0\\
153	0\\
154	0\\
155	0\\
156	0\\
157	0\\
158	0\\
159	0\\
160	0\\
161	0\\
162	0\\
163	0\\
164	0\\
165	0\\
166	0\\
167	0\\
168	0\\
169	0\\
170	0\\
171	0\\
172	0\\
173	0\\
174	0\\
175	0\\
176	0\\
177	0\\
178	0\\
179	0\\
180	0\\
181	0\\
182	0\\
183	0\\
184	0\\
185	0\\
186	0\\
187	0\\
188	0\\
189	0\\
190	0\\
191	0\\
192	0\\
193	0\\
194	0\\
195	0\\
196	0\\
197	0\\
198	0\\
199	0\\
200	0\\
201	0\\
202	0\\
203	0\\
204	0\\
205	0\\
206	0\\
207	0\\
208	0\\
209	0\\
210	0\\
211	0\\
212	0\\
213	0\\
214	0\\
215	0\\
216	0\\
217	0\\
218	0\\
219	0\\
220	0\\
221	0\\
222	0\\
223	0\\
224	0\\
225	0\\
226	0\\
227	0\\
228	0\\
229	0\\
230	0\\
231	0\\
232	0\\
233	0\\
234	0\\
235	0\\
236	0\\
237	0\\
238	0\\
239	0\\
240	0\\
241	0\\
242	0\\
243	0\\
244	0\\
245	0\\
246	0\\
247	0\\
248	0\\
249	0\\
250	0\\
251	0\\
252	0\\
253	0\\
254	0\\
255	0\\
256	0\\
257	0\\
258	0\\
259	0\\
260	0\\
261	0\\
262	0\\
263	0\\
264	0\\
265	0\\
266	0\\
267	0\\
268	0\\
269	0\\
270	0\\
271	0\\
272	0\\
273	0\\
274	0\\
275	0\\
276	0\\
277	0\\
278	0\\
279	0\\
280	0\\
281	0\\
282	0\\
283	0\\
284	0\\
285	0\\
286	0\\
287	0\\
288	0\\
289	0\\
290	0\\
291	0\\
292	0\\
293	0\\
294	0\\
295	0\\
296	0\\
297	0\\
298	0\\
299	0\\
300	-0.452789181003312\\
301	-0.434391960439725\\
302	-0.413216113737604\\
303	-0.389397094051036\\
304	-0.363087261821106\\
305	-0.334454910189011\\
306	-0.303683188494286\\
307	-0.270968930744037\\
308	-0.23652139654702\\
309	-0.200560932566259\\
310	-0.163317563052361\\
311	-0.125029518473301\\
312	-0.0859417116524005\\
313	-0.0463041711620614\\
314	-0.00637044199414048\\
315	0.0336040362627375\\
316	0.0733635633648459\\
317	0.112653814022298\\
318	0.151223464713005\\
319	0.188825801295536\\
320	0.225220297137695\\
321	0.260174151666137\\
322	0.29346377949559\\
323	0.324876240612302\\
324	0.354210602463456\\
325	0.381279225239801\\
326	0.405908962130094\\
327	0.427942266869804\\
328	0.44723820149957\\
329	0.463673337887167\\
330	0.477142547246349\\
331	0.48755967260231\\
332	0.49485807990231\\
333	0.49899108424621\\
334	0.499932248510524\\
335	0.49767555245578\\
336	0.492235431235516\\
337	0.483646683060557\\
338	0.471964246609214\\
339	0.457262849607233\\
340	0.439636530825362\\
341	0.419198038552136\\
342	0.3960781093896\\
343	0.370424631985201\\
344	0.342401701049137\\
345	0.312188567708196\\
346	0.279978492910265\\
347	0.245977511213779\\
348	0.21040311286967\\
349	0.173482852625959\\
350	0.135452894153935\\
351	0.0965564994065796\\
352	0.05704247257224\\
353	0.0171635685768771\\
354	-0.0228251236848921\\
355	-0.0626678130482163\\
356	-0.102109642268537\\
357	-0.140898318238242\\
358	-0.178785725801558\\
359	-0.21552951484485\\
360	-0.250894650510287\\
361	-0.284654916616856\\
362	-0.316594362671938\\
363	-0.346508685217499\\
364	-0.374206534675023\\
365	-0.399510739329807\\
366	-0.422259438625323\\
367	-0.442307118518408\\
368	-0.459525542272567\\
369	-0.473804570735453\\
370	-0.485052866853593\\
371	-0.493198479917824\\
372	-0.498189305802297\\
373	-0.499993420253071\\
374	-0.49859928309439\\
375	-0.494015812046431\\
376	-0.486272325682322\\
377	-0.475418355889334\\
378	-0.461523331033825\\
379	-0.444676131856637\\
380	-0.424984522939663\\
381	-0.402574463380286\\
382	-0.377589301082996\\
383	-0.350188855822004\\
384	-0.320548396940074\\
385	-0.288857522222866\\
386	-0.255318945120126\\
387	-0.220147198071386\\
388	-0.183567260230471\\
389	-0.145813118366674\\
390	-0.107126270147944\\
391	-0.0677541793799356\\
392	-0.0279486930821649\\
393	0.0120355694734769\\
394	0.0519428454571341\\
395	0.091517864490294\\
396	0.130507481505058\\
397	0.168662296008706\\
398	0.205738247395733\\
399	0.241498176102848\\
400	0.275713340620845\\
401	0.308164880659659\\
402	0.338645217107341\\
403	0.366959379827974\\
404	0.392926254805211\\
405	0.41637974265389\\
406	0.437169821089277\\
407	0.455163504557712\\
408	0.470245694890339\\
409	0.482319917538589\\
410	0.49130893868207\\
411	0.497155259261444\\
412	0.499821482776186\\
413	0.499290554494572\\
414	0.495565870545754\\
415	0.488671256196129\\
416	0.478650813448939\\
417	0.465568638941948\\
418	0.449508413947706\\
419	0.430572869098965\\
420	0.408883127263221\\
421	0.384577928769695\\
422	0.357812743944697\\
423	0.328758778632095\\
424	0.29760187906018\\
425	0.264541343060012\\
426	0.229788645239433\\
427	0.193566084267253\\
428	0.156105360920399\\
429	0.117646095989627\\
430	0.078434297524205\\
431	0.0387207872199191\\
432	-0.0012404039838094\\
433	-0.0411936608350507\\
434	-0.0808834188346504\\
435	-0.120055798976891\\
436	-0.158460231708724\\
437	-0.195851059719838\\
438	-0.231989109311098\\
439	-0.266643220290017\\
440	-0.299591724607133\\
441	-0.330623864275049\\
442	-0.359541139500336\\
443	-0.386158578404798\\
444	-0.410305920214258\\
445	-0.431828704346478\\
446	-0.450589258431779\\
447	-0.466467578946381\\
448	-0.479362098825431\\
449	-0.489190337145607\\
450	-0.495889426721558\\
451	-0.499416516241347\\
452	-0.499749044368611\\
453	-0.496884884058122\\
454	-0.49084235616162\\
455	-0.48166011223688\\
456	-0.469396887309646\\
457	-0.454131124169907\\
458	-0.435960471605734\\
459	-0.415001159784274\\
460	-0.391387256775325\\
461	-0.365269810973218\\
462	-0.336815884902554\\
463	-0.306207486588167\\
464	-0.273640405324888\\
465	-0.239322959294207\\
466	-0.203474663038832\\
467	-0.166324823318709\\
468	-0.12811107233031\\
469	-0.0890778476715095\\
470	-0.0494748287751448\\
471	-0.00955533981267899\\
472	0.0304252707159852\\
473	0.0702112633415089\\
474	0.109548143484968\\
475	0.148184289354693\\
476	0.185872561467029\\
477	0.222371883495657\\
478	0.257448784337363\\
479	0.290878891530381\\
480	0.322448366472434\\
481	0.35195527225805\\
482	0.379210865385618\\
483	0.404040803071704\\
484	0.426286258449892\\
485	0.445804936520722\\
486	0.462471984354079\\
487	0.476180789721887\\
488	0.486843663052532\\
489	0.494392398344876\\
490	0.498778709453903\\
491	0.49997453895726\\
492	0.497972237627022\\
493	0.492784613358639\\
494	0.484444849244112\\
495	0.473006291313454\\
496	0.458542107302138\\
497	0.441144818627286\\
498	0.420925708566336\\
499	0.398014110423842\\
500	0.372556580239674\\
501	0.34471595933051\\
502	0.314670332661116\\
503	0.28261188970832\\
504	0.248745695104205\\
505	0.213288376922287\\
506	0.176466740997094\\
507	0.138516320140849\\
508	0.0996798675372212\\
509	0.0602058039493939\\
510	0.0203466286749289\\
511	-0.0196426955880406\\
512	-0.0595063736320124\\
513	-0.0989894139565018\\
514	-0.137839259843462\\
515	-0.175807404858408\\
516	-0.212650982443528\\
517	-0.248134319434865\\
518	-0.282030443566299\\
519	-0.314122535317479\\
520	-0.344205314818835\\
521	-0.372086354942211\\
522	-0.397587312177801\\
523	-0.420545067424\\
524	-0.440812769392983\\
525	-0.458260773957817\\
526	-0.472777473432439\\
527	-0.484270010479977\\
528	-0.492664872082818\\
529	-0.497908359775051\\
530	-0.499966933129402\\
531	-0.498827424301499\\
532	-0.494497122259149\\
533	-0.48700372615781\\
534	-0.476395168160528\\
535	-0.462739306835664\\
536	-0.446123493093636\\
537	-0.426654011439178\\
538	-0.404455400113225\\
539	-0.379669654473164\\
540	-0.352455318707124\\
541	-0.322986471692228\\
542	-0.291451613483828\\
543	-0.258052459558433\\
544	-0.223002650523038\\
545	-0.186526385544311\\
546	-0.148856988239021\\
547	-0.11023541419914\\
548	-0.0709087096983323\\
549	-0.0311284314388931\\
550	0.00885096255270679\\
551	0.0487737405888105\\
552	0.088384533130672\\
553	0.127429966282015\\
554	0.165660282517323\\
555	0.202830938277668\\
556	0.238704168214962\\
557	0.273050506078772\\
558	0.305650252517265\\
559	0.336294880403301\\
560	0.364788368696429\\
561	0.390948456308596\\
562	0.414607807953137\\
563	0.435615084519556\\
564	0.45383591112737\\
565	0.469153736666777\\
566	0.481470579328016\\
567	0.490707653350576\\
568	0.496805872983202\\
569	0.49972623043106\\
570	0.499450045372511\\
571	0.495979084449419\\
572	0.489335549966673\\
573	0.479561937873201\\
574	0.466720765932923\\
575	0.450894173824405\\
576	0.43218339772723\\
577	0.410708122755927\\
578	0.386605717383679\\
579	0.360030354752872\\
580	0.331152026493121\\
581	0.300155455354953\\
582	0.267238913614594\\
583	0.232612954808019\\
584	0.196499066906861\\
585	0.159128255551241\\
586	0.120739566402014\\
587	0.081578556064292\\
588	0.041895721363166\\
589	0.00194489701887388\\
590	-0.0380183680291768\\
591	-0.0777384452635191\\
592	-0.1169612617388\\
593	-0.155435925281568\\
594	-0.19291632934399\\
595	-0.229162727245883\\
596	-0.263943265735326\\
597	-0.297035468058234\\
598	-0.328227657050367\\
599	-0.357320309148797\\
600	-0.384127330661833\\
};
\addlegendentry{$z(k)$}

\end{axis}

\begin{axis}[%
width=4.521in,
height=1.493in,
at={(0.758in,0.481in)},
scale only axis,
xmin=0,
xmax=600,
xlabel style={font=\color{white!15!black}},
xlabel={$k$},
ymin=-2,
ymax=3,
axis background/.style={fill=white},
title style={font=\bfseries},
title={Sygna� wyj�ciowy i warto�� zadana},
xmajorgrids,
ymajorgrids,
legend style={legend cell align=left, align=left, draw=white!15!black}
]
\addplot [color=mycolor1]
  table[row sep=crcr]{%
1	0\\
2	0\\
3	0\\
4	0\\
5	0\\
6	0\\
7	0\\
8	0\\
9	0\\
10	0\\
11	0\\
12	0\\
13	0\\
14	0\\
15	0\\
16	0\\
17	0\\
18	0\\
19	0\\
20	0\\
21	0\\
22	0\\
23	0\\
24	0\\
25	0\\
26	0\\
27	0\\
28	0\\
29	0\\
30	0\\
31	0\\
32	0\\
33	0\\
34	0\\
35	0\\
36	0\\
37	0\\
38	0\\
39	0\\
40	0\\
41	0\\
42	0\\
43	0\\
44	0\\
45	0\\
46	0\\
47	0\\
48	0\\
49	0\\
50	0\\
51	0\\
52	0\\
53	0\\
54	0\\
55	0\\
56	0\\
57	0\\
58	0\\
59	0\\
60	0\\
61	0\\
62	0\\
63	0\\
64	0\\
65	0\\
66	0\\
67	0\\
68	0\\
69	0\\
70	0\\
71	0\\
72	0\\
73	0\\
74	0\\
75	0\\
76	0\\
77	0\\
78	0\\
79	0\\
80	0\\
81	0\\
82	0\\
83	0\\
84	0\\
85	0\\
86	0\\
87	0\\
88	0\\
89	0\\
90	0\\
91	0\\
92	0\\
93	0\\
94	0\\
95	0\\
96	0\\
97	0\\
98	0\\
99	0\\
100	0\\
101	0\\
102	0\\
103	0\\
104	0\\
105	0.0898661853525702\\
106	0.227145989536648\\
107	0.381364985455071\\
108	0.532183744977198\\
109	0.667348874413722\\
110	0.780701113190663\\
111	0.870397690541912\\
112	0.937426745782853\\
113	0.984435604898583\\
114	1.01485816163926\\
115	1.03230537220306\\
116	1.04017299290448\\
117	1.04141871252479\\
118	1.03846391027622\\
119	1.03318119527604\\
120	1.0269360330158\\
121	1.02065802748983\\
122	1.01492410818031\\
123	1.01004159145033\\
124	1.0061236992538\\
125	1.00315363327061\\
126	1.00103582376953\\
127	0.999634653341271\\
128	0.998801966526442\\
129	0.99839518271941\\
130	0.998287979047469\\
131	0.998375425366301\\
132	0.998575232142644\\
133	0.998826486044746\\
134	0.99908694793467\\
135	0.999329706050688\\
136	0.99953973184925\\
137	0.999710685296531\\
138	0.999842161319463\\
139	0.9999374561398\\
140	1.00000185537463\\
141	1.00004139817149\\
142	1.00006204635586\\
143	1.00006917833604\\
144	1.00006732902273\\
145	1.00006010503136\\
146	1.00005021573868\\
147	1.00003957308605\\
148	1.00002942486313\\
149	1.00002049668706\\
150	1.00001312659169\\
151	1.00000738295854\\
152	1.00000316156378\\
153	1.00000026100877\\
154	0.999998438024651\\
155	0.999997445379893\\
156	0.999997055639453\\
157	0.999997074052887\\
158	0.99999734357207\\
159	0.999997744558619\\
160	0.999998191239006\\
161	0.999998626470347\\
162	0.999999015933754\\
163	0.999999342496316\\
164	0.999999601184079\\
165	0.999999794984267\\
166	0.999999931536935\\
167	1.00000002067327\\
168	1.00000007269799\\
169	1.0000000972857\\
170	1.00000010285557\\
171	1.00000009629762\\
172	1.00000008294079\\
173	1.00000006667322\\
174	1.00000005014576\\
175	1.00000003500852\\
176	1.00000002214677\\
177	1.00000001189507\\
178	1.0000000042189\\
179	0.999999998859796\\
180	0.999999995445021\\
181	0.999999993565414\\
182	0.999999992826672\\
183	0.999999992879594\\
184	0.999999993434638\\
185	0.99999999426549\\
186	0.999999995205538\\
187	0.999999996140285\\
188	0.999999996997944\\
189	0.999999997739762\\
190	0.999999998351042\\
191	0.999999998833424\\
192	0.99999999919862\\
193	0.999999999463627\\
194	0.999999999647282\\
195	0.999999999767949\\
196	0.999999999842131\\
197	0.999999999883761\\
198	0.999999999903995\\
199	0.999999999911317\\
200	0.999999999911846\\
201	0.999999999909724\\
202	0.999999999907537\\
203	0.999999999906703\\
204	0.999999999907817\\
205	0.999999999910932\\
206	0.999999999915784\\
207	0.999999999921947\\
208	0.999999999928953\\
209	0.999999999936356\\
210	0.99999999994378\\
211	0.999999999950931\\
212	0.999999999957603\\
213	0.999999999963665\\
214	0.999999999969054\\
215	0.999999999973759\\
216	0.999999999977806\\
217	0.999999999981245\\
218	0.999999999984139\\
219	0.999999999986558\\
220	0.999999999988573\\
221	0.999999999990247\\
222	0.999999999991641\\
223	0.999999999992805\\
224	0.999999999993781\\
225	0.999999999994607\\
226	0.999999999995309\\
227	0.999999999995912\\
228	0.999999999996432\\
229	0.999999999996883\\
230	0.999999999997277\\
231	0.999999999997622\\
232	0.999999999997925\\
233	0.999999999998191\\
234	0.999999999998424\\
235	0.999999999998629\\
236	0.999999999998809\\
237	0.999999999998967\\
238	0.999999999999105\\
239	0.999999999999226\\
240	0.999999999999331\\
241	0.999999999999423\\
242	0.999999999999503\\
243	0.999999999999573\\
244	0.999999999999632\\
245	0.999999999999683\\
246	0.999999999999727\\
247	0.999999999999764\\
248	0.999999999999796\\
249	0.999999999999823\\
250	0.999999999999846\\
251	0.999999999999867\\
252	0.999999999999885\\
253	0.9999999999999\\
254	0.999999999999913\\
255	0.999999999999924\\
256	0.999999999999934\\
257	0.999999999999942\\
258	0.999999999999948\\
259	0.999999999999954\\
260	0.999999999999958\\
261	0.999999999999961\\
262	0.999999999999964\\
263	0.999999999999967\\
264	0.999999999999969\\
265	0.999999999999972\\
266	0.999999999999975\\
267	0.999999999999979\\
268	0.999999999999982\\
269	0.999999999999985\\
270	0.999999999999987\\
271	1.00000086276672\\
272	1.00000423317379\\
273	1.0000120519818\\
274	1.0000260003166\\
275	1.00004705639391\\
276	1.00007523709427\\
277	1.00010952763252\\
278	1.00014813255248\\
279	1.00018884650625\\
280	1.00022941796769\\
281	1.00026470340528\\
282	1.00028981685954\\
283	1.0003016430238\\
284	1.00029925476994\\
285	1.00028368025733\\
286	1.00025735504373\\
287	1.00022349001474\\
288	1.00018549688746\\
289	1.00014654375849\\
290	1.0001092638004\\
291	1.00007560850438\\
292	1.00004681940536\\
293	1.0000234853142\\
294	1.0000056522805\\
295	0.999992957931546\\
296	0.99998476825077\\
297	0.999980301698218\\
298	0.999978731833114\\
299	0.999979264735191\\
300	0.999981191336499\\
301	0.999983917301141\\
302	0.933685054721457\\
303	0.878912175347626\\
304	0.834539645585111\\
305	0.799571329642537\\
306	0.773120939228023\\
307	0.788900914180289\\
308	0.835809235150391\\
309	0.897172544964487\\
310	0.961461073784098\\
311	1.02133683168569\\
312	1.072692005463\\
313	1.11376742955255\\
314	1.14439946457285\\
315	1.16541362010601\\
316	1.17816308348949\\
317	1.18419802014485\\
318	1.1850451852028\\
319	1.18207531600157\\
320	1.17643651902151\\
321	1.16903427996747\\
322	1.16054195601277\\
323	1.15142905281408\\
324	1.14199785496304\\
325	1.13242184165551\\
326	1.12278167654373\\
327	1.11309639384248\\
328	1.1033487476649\\
329	1.09350461427521\\
330	1.08352691615339\\
331	1.07338485191461\\
332	1.06305933998743\\
333	1.05254557864594\\
334	1.04185354046485\\
335	1.03100709347799\\
336	1.02004230140637\\
337	1.00900531929908\\
338	0.997950179568747\\
339	0.986936662028075\\
340	0.976028361633021\\
341	0.965291008193953\\
342	0.954791050858415\\
343	0.94459449355172\\
344	0.934765952504387\\
345	0.925367900431129\\
346	0.91646006119418\\
347	0.908098921714811\\
348	0.900336832630361\\
349	0.893222164091219\\
350	0.886799273701188\\
351	0.8811083494562\\
352	0.876185180078652\\
353	0.872060894151296\\
354	0.868761699087344\\
355	0.866308544424406\\
356	0.864716897612587\\
357	0.863996609503794\\
358	0.864153667895227\\
359	0.865186867649922\\
360	0.867088500312948\\
361	0.869844897086973\\
362	0.873436850434739\\
363	0.877839943546282\\
364	0.883024816619615\\
365	0.888957395991071\\
366	0.895599107862709\\
367	0.902907093604303\\
368	0.910834438953537\\
369	0.919330425259941\\
370	0.92834080739665\\
371	0.937808120153272\\
372	0.947672012794318\\
373	0.957869609943036\\
374	0.968335895929753\\
375	0.979004119121138\\
376	0.989806212422829\\
377	1.00067322603612\\
378	1.01153576857908\\
379	1.02232445279865\\
380	1.03297034226354\\
381	1.04340539560984\\
382	1.0535629050966\\
383	1.0633779264068\\
384	1.07278769679713\\
385	1.08173203885794\\
386	1.09015374729514\\
387	1.09799895629101\\
388	1.10521748514639\\
389	1.11176316005393\\
390	1.11759411000534\\
391	1.12267303499616\\
392	1.12696744486095\\
393	1.13044986725102\\
394	1.13309802345536\\
395	1.13489497096319\\
396	1.13582921187254\\
397	1.1358947664622\\
398	1.135091211463\\
399	1.13342368278733\\
400	1.13090284270061\\
401	1.12754481164519\\
402	1.12337106515184\\
403	1.11840829649743\\
404	1.11268824598639\\
405	1.10624749794694\\
406	1.09912724673996\\
407	1.09137303327681\\
408	1.08303445373125\\
409	1.07416484230845\\
410	1.0648209301007\\
411	1.05506248221176\\
412	1.04495191547164\\
413	1.03455389918717\\
414	1.02393494148258\\
415	1.01316296387643\\
416	1.00230686681632\\
417	0.991436088950794\\
418	0.980620162957818\\
419	0.969928270771101\\
420	0.959428801049534\\
421	0.94918891172051\\
422	0.939274100395463\\
423	0.929747785405571\\
424	0.920670900137642\\
425	0.912101503265076\\
426	0.904094407367145\\
427	0.896700828312186\\
428	0.889968057647483\\
429	0.883939160091431\\
430	0.878652698063031\\
431	0.874142485010784\\
432	0.870437369118851\\
433	0.867561048774035\\
434	0.865531920973956\\
435	0.864362963646111\\
436	0.864061652630572\\
437	0.864629913865624\\
438	0.866064111084641\\
439	0.868355069097638\\
440	0.871488132499139\\
441	0.875443259417337\\
442	0.880195149698671\\
443	0.885713406706316\\
444	0.891962731703939\\
445	0.898903149592625\\
446	0.906490264569052\\
447	0.914675544018595\\
448	0.923406628865549\\
449	0.932627668445869\\
450	0.942279677789655\\
451	0.952300915028949\\
452	0.962627276495538\\
453	0.973192706949066\\
454	0.983929622266337\\
455	0.994769341853777\\
456	1.00564252800174\\
457	1.01647962948226\\
458	1.02721132641365\\
459	1.03776897353069\\
460	1.04808503906658\\
461	1.05809353649635\\
462	1.06773044642886\\
463	1.07693412597915\\
464	1.08564570301365\\
465	1.09380945274148\\
466	1.10137315422799\\
467	1.10828842452992\\
468	1.11451102829498\\
469	1.12000116082801\\
470	1.12472370280067\\
471	1.12864844496771\\
472	1.13175028144964\\
473	1.13400970162153\\
474	1.13541294283417\\
475	1.13595187614696\\
476	1.13562380203529\\
477	1.13443125492984\\
478	1.13238185934282\\
479	1.12948824380767\\
480	1.12576806267848\\
481	1.12124405526471\\
482	1.11594407565677\\
483	1.10989985306783\\
484	1.10314780619374\\
485	1.09572964152258\\
486	1.08769235054733\\
487	1.07908783031852\\
488	1.06997230072892\\
489	1.06040564130122\\
490	1.05045072599283\\
491	1.04017279933875\\
492	1.02963891163145\\
493	1.0189174139502\\
494	1.00807750417535\\
495	0.997188810901324\\
496	0.986321001704321\\
497	0.975543404073402\\
498	0.96492463032935\\
499	0.954532201203986\\
500	0.944432165889294\\
501	0.934688718981896\\
502	0.925363816713452\\
503	0.916516796164952\\
504	0.90820400188256\\
505	0.900478424554284\\
506	0.893389356292179\\
507	0.886982066709982\\
508	0.881297503490637\\
509	0.876372020580579\\
510	0.872237136584488\\
511	0.868919325402365\\
512	0.866439840670494\\
513	0.864814575146871\\
514	0.864053955814104\\
515	0.864162875169383\\
516	0.8651406589039\\
517	0.866981069936321\\
518	0.869672348549161\\
519	0.873197288176319\\
520	0.877533346198997\\
521	0.882652788919805\\
522	0.888522869702883\\
523	0.895106039088404\\
524	0.902360185546762\\
525	0.910238905276329\\
526	0.918691799313507\\
527	0.9276647960759\\
528	0.937100497307044\\
529	0.94693854524174\\
530	0.957116008670248\\
531	0.967567785458488\\
532	0.978227018960495\\
533	0.989025525661664\\
534	0.99989423125181\\
535	1.01076361248065\\
536	1.02156414191033\\
537	1.03222673268777\\
538	1.04268318047814\\
539	1.05286659973059\\
540	1.06271185149017\\
541	1.07215596002616\\
542	1.0811385156194\\
543	1.08960206093768\\
544	1.09749245853213\\
545	1.10475923710611\\
546	1.11135591434222\\
547	1.11724029422207\\
548	1.12237473693555\\
549	1.12672639965141\\
550	1.13026744660726\\
551	1.13297522717342\\
552	1.13483242075033\\
553	1.13582714757147\\
554	1.1359530447024\\
555	1.13520930674936\\
556	1.13360069101689\\
557	1.13113748708136\\
558	1.12783545097549\\
559	1.12371570440477\\
560	1.11880459964099\\
561	1.11313355095721\\
562	1.10673883368272\\
563	1.09966135216359\\
564	1.09194637811315\\
565	1.08364326102628\\
566	1.07480511250991\\
567	1.06548846654898\\
568	1.05575291788105\\
569	1.04566074079267\\
570	1.03527649077602\\
571	1.02466659159367\\
572	1.01389891039295\\
573	1.00304232358769\\
574	0.992166276284146\\
575	0.981340338069241\\
576	0.97063375800267\\
577	0.96011502165924\\
578	0.949851413054938\\
579	0.939908584258868\\
580	0.930350135444088\\
581	0.921237208063558\\
582	0.912628093753511\\
583	0.90457786146591\\
584	0.897138005215097\\
585	0.890356114691817\\
586	0.884275570851602\\
587	0.878935268424691\\
588	0.87436936712248\\
589	0.87060707313193\\
590	0.867672452295635\\
591	0.865584276172549\\
592	0.864355901964056\\
593	0.863995187073467\\
594	0.864504438845439\\
595	0.865880399806843\\
596	0.868114268503474\\
597	0.871191755799317\\
598	0.875093176278252\\
599	0.879793574163539\\
600	0.885262882949625\\
};
\addlegendentry{$y(k)$}

\addplot[const plot, color=mycolor2, dashed] table[row sep=crcr] {%
1	0\\
2	0\\
3	0\\
4	0\\
5	0\\
6	0\\
7	0\\
8	0\\
9	0\\
10	0\\
11	0\\
12	0\\
13	0\\
14	0\\
15	0\\
16	0\\
17	0\\
18	0\\
19	0\\
20	0\\
21	0\\
22	0\\
23	0\\
24	0\\
25	0\\
26	0\\
27	0\\
28	0\\
29	0\\
30	0\\
31	0\\
32	0\\
33	0\\
34	0\\
35	0\\
36	0\\
37	0\\
38	0\\
39	0\\
40	0\\
41	0\\
42	0\\
43	0\\
44	0\\
45	0\\
46	0\\
47	0\\
48	0\\
49	0\\
50	0\\
51	0\\
52	0\\
53	0\\
54	0\\
55	0\\
56	0\\
57	0\\
58	0\\
59	0\\
60	0\\
61	0\\
62	0\\
63	0\\
64	0\\
65	0\\
66	0\\
67	0\\
68	0\\
69	0\\
70	0\\
71	0\\
72	0\\
73	0\\
74	0\\
75	0\\
76	0\\
77	0\\
78	0\\
79	0\\
80	0\\
81	0\\
82	0\\
83	0\\
84	0\\
85	0\\
86	0\\
87	0\\
88	0\\
89	0\\
90	0\\
91	0\\
92	0\\
93	0\\
94	0\\
95	0\\
96	0\\
97	0\\
98	0\\
99	0\\
100	1\\
101	1\\
102	1\\
103	1\\
104	1\\
105	1\\
106	1\\
107	1\\
108	1\\
109	1\\
110	1\\
111	1\\
112	1\\
113	1\\
114	1\\
115	1\\
116	1\\
117	1\\
118	1\\
119	1\\
120	1\\
121	1\\
122	1\\
123	1\\
124	1\\
125	1\\
126	1\\
127	1\\
128	1\\
129	1\\
130	1\\
131	1\\
132	1\\
133	1\\
134	1\\
135	1\\
136	1\\
137	1\\
138	1\\
139	1\\
140	1\\
141	1\\
142	1\\
143	1\\
144	1\\
145	1\\
146	1\\
147	1\\
148	1\\
149	1\\
150	1\\
151	1\\
152	1\\
153	1\\
154	1\\
155	1\\
156	1\\
157	1\\
158	1\\
159	1\\
160	1\\
161	1\\
162	1\\
163	1\\
164	1\\
165	1\\
166	1\\
167	1\\
168	1\\
169	1\\
170	1\\
171	1\\
172	1\\
173	1\\
174	1\\
175	1\\
176	1\\
177	1\\
178	1\\
179	1\\
180	1\\
181	1\\
182	1\\
183	1\\
184	1\\
185	1\\
186	1\\
187	1\\
188	1\\
189	1\\
190	1\\
191	1\\
192	1\\
193	1\\
194	1\\
195	1\\
196	1\\
197	1\\
198	1\\
199	1\\
200	1\\
201	1\\
202	1\\
203	1\\
204	1\\
205	1\\
206	1\\
207	1\\
208	1\\
209	1\\
210	1\\
211	1\\
212	1\\
213	1\\
214	1\\
215	1\\
216	1\\
217	1\\
218	1\\
219	1\\
220	1\\
221	1\\
222	1\\
223	1\\
224	1\\
225	1\\
226	1\\
227	1\\
228	1\\
229	1\\
230	1\\
231	1\\
232	1\\
233	1\\
234	1\\
235	1\\
236	1\\
237	1\\
238	1\\
239	1\\
240	1\\
241	1\\
242	1\\
243	1\\
244	1\\
245	1\\
246	1\\
247	1\\
248	1\\
249	1\\
250	1\\
251	1\\
252	1\\
253	1\\
254	1\\
255	1\\
256	1\\
257	1\\
258	1\\
259	1\\
260	1\\
261	1\\
262	1\\
263	1\\
264	1\\
265	1\\
266	1\\
267	1\\
268	1\\
269	1\\
270	1\\
271	1\\
272	1\\
273	1\\
274	1\\
275	1\\
276	1\\
277	1\\
278	1\\
279	1\\
280	1\\
281	1\\
282	1\\
283	1\\
284	1\\
285	1\\
286	1\\
287	1\\
288	1\\
289	1\\
290	1\\
291	1\\
292	1\\
293	1\\
294	1\\
295	1\\
296	1\\
297	1\\
298	1\\
299	1\\
300	1\\
301	1\\
302	1\\
303	1\\
304	1\\
305	1\\
306	1\\
307	1\\
308	1\\
309	1\\
310	1\\
311	1\\
312	1\\
313	1\\
314	1\\
315	1\\
316	1\\
317	1\\
318	1\\
319	1\\
320	1\\
321	1\\
322	1\\
323	1\\
324	1\\
325	1\\
326	1\\
327	1\\
328	1\\
329	1\\
330	1\\
331	1\\
332	1\\
333	1\\
334	1\\
335	1\\
336	1\\
337	1\\
338	1\\
339	1\\
340	1\\
341	1\\
342	1\\
343	1\\
344	1\\
345	1\\
346	1\\
347	1\\
348	1\\
349	1\\
350	1\\
351	1\\
352	1\\
353	1\\
354	1\\
355	1\\
356	1\\
357	1\\
358	1\\
359	1\\
360	1\\
361	1\\
362	1\\
363	1\\
364	1\\
365	1\\
366	1\\
367	1\\
368	1\\
369	1\\
370	1\\
371	1\\
372	1\\
373	1\\
374	1\\
375	1\\
376	1\\
377	1\\
378	1\\
379	1\\
380	1\\
381	1\\
382	1\\
383	1\\
384	1\\
385	1\\
386	1\\
387	1\\
388	1\\
389	1\\
390	1\\
391	1\\
392	1\\
393	1\\
394	1\\
395	1\\
396	1\\
397	1\\
398	1\\
399	1\\
400	1\\
401	1\\
402	1\\
403	1\\
404	1\\
405	1\\
406	1\\
407	1\\
408	1\\
409	1\\
410	1\\
411	1\\
412	1\\
413	1\\
414	1\\
415	1\\
416	1\\
417	1\\
418	1\\
419	1\\
420	1\\
421	1\\
422	1\\
423	1\\
424	1\\
425	1\\
426	1\\
427	1\\
428	1\\
429	1\\
430	1\\
431	1\\
432	1\\
433	1\\
434	1\\
435	1\\
436	1\\
437	1\\
438	1\\
439	1\\
440	1\\
441	1\\
442	1\\
443	1\\
444	1\\
445	1\\
446	1\\
447	1\\
448	1\\
449	1\\
450	1\\
451	1\\
452	1\\
453	1\\
454	1\\
455	1\\
456	1\\
457	1\\
458	1\\
459	1\\
460	1\\
461	1\\
462	1\\
463	1\\
464	1\\
465	1\\
466	1\\
467	1\\
468	1\\
469	1\\
470	1\\
471	1\\
472	1\\
473	1\\
474	1\\
475	1\\
476	1\\
477	1\\
478	1\\
479	1\\
480	1\\
481	1\\
482	1\\
483	1\\
484	1\\
485	1\\
486	1\\
487	1\\
488	1\\
489	1\\
490	1\\
491	1\\
492	1\\
493	1\\
494	1\\
495	1\\
496	1\\
497	1\\
498	1\\
499	1\\
500	1\\
501	1\\
502	1\\
503	1\\
504	1\\
505	1\\
506	1\\
507	1\\
508	1\\
509	1\\
510	1\\
511	1\\
512	1\\
513	1\\
514	1\\
515	1\\
516	1\\
517	1\\
518	1\\
519	1\\
520	1\\
521	1\\
522	1\\
523	1\\
524	1\\
525	1\\
526	1\\
527	1\\
528	1\\
529	1\\
530	1\\
531	1\\
532	1\\
533	1\\
534	1\\
535	1\\
536	1\\
537	1\\
538	1\\
539	1\\
540	1\\
541	1\\
542	1\\
543	1\\
544	1\\
545	1\\
546	1\\
547	1\\
548	1\\
549	1\\
550	1\\
551	1\\
552	1\\
553	1\\
554	1\\
555	1\\
556	1\\
557	1\\
558	1\\
559	1\\
560	1\\
561	1\\
562	1\\
563	1\\
564	1\\
565	1\\
566	1\\
567	1\\
568	1\\
569	1\\
570	1\\
571	1\\
572	1\\
573	1\\
574	1\\
575	1\\
576	1\\
577	1\\
578	1\\
579	1\\
580	1\\
581	1\\
582	1\\
583	1\\
584	1\\
585	1\\
586	1\\
587	1\\
588	1\\
589	1\\
590	1\\
591	1\\
592	1\\
593	1\\
594	1\\
595	1\\
596	1\\
597	1\\
598	1\\
599	1\\
600	1\\
};
\addlegendentry{$y^{\mathrm{zad}}(k)$}

\end{axis}
\end{tikzpicture}%
	\caption{Przebiegi sygna��w dla regulacji przy pomocy algorytmu DMC z uwzgl�dnieniem zak��ce� sinusoidalnych}
\end{figure}

Wska�nik ilo�ciowy: 10.3366

Mo�emy zauwa�y� znaczn� popraw� wska�nika ilo�ciowego. Nie nast�puje szybszy powr�t do warto�ci zadanej co wskazuje na to, �e zak��ce� okresowych nie mo�na w pe�ni kompensowa�. Jednak�e w przypadku brania pod uwag� zak��ce� mo�na ograniczy� zakres wywo�anych oscylacji sygna�u wyj�ciowego.