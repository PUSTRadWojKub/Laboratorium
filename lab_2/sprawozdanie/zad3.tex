\chapter{Odpowiedzi skokowe dla algorytmu DMC}

\section{Odpowied� skokowa toru wej�cie-wyj�cie procesu}
Do wyznaczania odpowiedzi skokowej toru $U-Y$ dla algorytmu DMC wybrana zosta�a odpowied� procesu dla jednostkowej zmiany sygna�u steruj�cego: 

\begin{equation}
u(k)=
\begin{cases}
0 & \textrm{dla } k<0 \\
1 & \textrm{dla } k\ge0
\end{cases}
\nonumber
\end{equation}

W trakcie symulacji sygna� zak��cenia mia� caly czas warto�� zerow�, odpowiadaj�c� warto�ci z punktu pracy. Otrzyman� odpowied� skokow� przedstawiono na rys.~\ref{odptor_uy}.

\begin{figure}[h!]
\centering
% This file was created by matlab2tikz.
%
\definecolor{mycolor1}{rgb}{0.00000,0.44700,0.74100}%
%
\begin{tikzpicture}

\begin{axis}[%
width=4.521in,
height=3.566in,
at={(0.758in,0.481in)},
scale only axis,
xmin=0,
xmax=180,
xlabel style={font=\color{white!15!black}},
xlabel={k},
ymin=0,
ymax=2,
ylabel style={font=\color{white!15!black}},
ylabel={y(k)},
axis background/.style={fill=white},
title style={font=\bfseries},
title={Odpowiedz skokowa toru wej�cie-wyj�cie procesu},
xmajorgrids,
ymajorgrids,
legend style={at={(0.97,0.03)}, anchor=south east, legend cell align=left, align=left, draw=white!15!black}
]
\addplot [color=mycolor1]
  table[row sep=crcr]{%
1	0\\
2	0\\
3	0\\
4	0\\
5	0.08443\\
6	0.165104232\\
7	0.2421865175968\\
8	0.315833968048888\\
9	0.386197199655809\\
10	0.453420550879769\\
11	0.517642299616405\\
12	0.578994879294171\\
13	0.637605092715453\\
14	0.693594322752521\\
15	0.747078739182422\\
16	0.798169501090768\\
17	0.846972954398664\\
18	0.893590824172577\\
19	0.938120401466427\\
20	0.980654724520821\\
21	1.02128275420801\\
22	1.06008954366462\\
23	1.09715640209865\\
24	1.13256105279456\\
25	1.16637778537036\\
26	1.19867760236605\\
27	1.22952836026249\\
28	1.25899490504643\\
29	1.28713920244957\\
30	1.31402046299984\\
31	1.33969526202979\\
32	1.36421765479253\\
33	1.38763928683892\\
34	1.4100094998113\\
35	1.43137543281023\\
36	1.45178211948967\\
37	1.47127258103529\\
38	1.48988791517826\\
39	1.50766738139446\\
40	1.5246484824362\\
41	1.54086704233982\\
42	1.55635728104924\\
43	1.57115188579152\\
44	1.58528207933652\\
45	1.5987776852686\\
46	1.61166719039431\\
47	1.62397780440549\\
48	1.63573551691353\\
49	1.64696515196589\\
50	1.65769042015214\\
51	1.6679339684028\\
52	1.67771742758004\\
53	1.68706145795571\\
54	1.69598579266829\\
55	1.70450927924651\\
56	1.71264991928412\\
57	1.72042490634675\\
58	1.72785066218812\\
59	1.73494287135024\\
60	1.74171651421852\\
61	1.74818589860015\\
62	1.75436468989086\\
63	1.76026593989268\\
64	1.76590211434243\\
65	1.77128511920814\\
66	1.77642632580822\\
67	1.78133659480565\\
68	1.78602629912732\\
69	1.79050534585641\\
70	1.79478319714362\\
71	1.79886889018092\\
72	1.80277105627982\\
73	1.80649793909407\\
74	1.81005741202509\\
75	1.81345699484656\\
76	1.81670386958326\\
77	1.81980489567744\\
78	1.82276662447458\\
79	1.82559531305911\\
80	1.82829693746916\\
81	1.83087720531815\\
82	1.83334156784985\\
83	1.83569523145222\\
84	1.83794316865442\\
85	1.84009012863006\\
86	1.84214064722884\\
87	1.84409905655785\\
88	1.84596949413253\\
89	1.84775591161687\\
90	1.84946208317096\\
91	1.85109161342381\\
92	1.85264794508806\\
93	1.85413436623272\\
94	1.85555401722937\\
95	1.85690989738627\\
96	1.85820487128462\\
97	1.85944167483019\\
98	1.86062292103322\\
99	1.8617511055286\\
100	1.86282861184825\\
101	1.86385771645657\\
102	1.86484059355984\\
103	1.86577931969949\\
104	1.86667587813912\\
105	1.86753216305447\\
106	1.86834998353513\\
107	1.86913106740663\\
108	1.86987706488068\\
109	1.8705895520416\\
110	1.87127003417605\\
111	1.87191994895325\\
112	1.87254066946233\\
113	1.87313350711324\\
114	1.87369971440735\\
115	1.87424048758363\\
116	1.87475696914594\\
117	1.87525025027678\\
118	1.87572137314268\\
119	1.87617133309594\\
120	1.87660108077757\\
121	1.87701152412569\\
122	1.87740353029376\\
123	1.87777792748258\\
124	1.87813550669012\\
125	1.87847702338261\\
126	1.87880319909063\\
127	1.87911472293355\\
128	1.87941225307539\\
129	1.87969641811535\\
130	1.87996781841589\\
131	1.88022702737108\\
132	1.88047459261808\\
133	1.88071103719407\\
134	1.88093686064137\\
135	1.88115254006275\\
136	1.88135853112947\\
137	1.88155526904396\\
138	1.88174316945931\\
139	1.88192262935739\\
140	1.8820940278876\\
141	1.88225772716787\\
142	1.88241407304975\\
143	1.88256339584909\\
144	1.88270601104389\\
145	1.88284221994084\\
146	1.88297231031186\\
147	1.88309655700215\\
148	1.88321522251078\\
149	1.88332855754534\\
150	1.8834368015516\\
151	1.88354018321938\\
152	1.88363892096572\\
153	1.88373322339631\\
154	1.88382328974628\\
155	1.88390931030105\\
156	1.88399146679841\\
157	1.88406993281244\\
158	1.88414487412026\\
159	1.88421644905225\\
160	1.88428480882657\\
161	1.88435009786861\\
162	1.88441245411618\\
163	1.88447200931088\\
164	1.88452888927648\\
165	1.88458321418469\\
166	1.88463509880911\\
167	1.8846846527677\\
168	1.88473198075434\\
169	1.88477718276005\\
170	1.88482035428426\\
171	1.88486158653651\\
172	1.88490096662924\\
173	1.88493857776173\\
174	1.88497449939599\\
175	1.88500880742454\\
};
\addlegendentry{y(k)}

\end{axis}
\end{tikzpicture}%
\caption{Odpowied� skokowa toru wej�cie-wyj�cie procesu dla algorytmu DMC}
\label{odptor_uy}
\end{figure}


\section{Odpowied� skokowa toru zak��cenie-wyj�cie procesu}
Do wyznaczania odpowiedzi skokowej toru $Z-Y$ dla algorytmu DMC wybrana zosta�a odpowied� procesu dla jednostkowej zmiany sygna�u zak��cenia: 

\begin{equation}
z(k)=
\begin{cases}
0 & \textrm{dla } k<0 \\
1 & \textrm{dla } k\ge0
\end{cases}
\nonumber
\end{equation}

W trakcie symulacji sygna� steruj�cy mia� caly czas warto�� zerow�, odpowiadaj�c� warto�ci z punktu pracy. Otrzyman� odpowied� skokow� przedstawiono na rys.~\ref{odptor_zy}.

\begin{figure}[h!]
\centering
% This file was created by matlab2tikz.
%
\definecolor{mycolor1}{rgb}{0.00000,0.44700,0.74100}%
%
\begin{tikzpicture}

\begin{axis}[%
width=4.521in,
height=3.566in,
at={(0.758in,0.481in)},
scale only axis,
xmin=0,
xmax=50,
xlabel style={font=\color{white!15!black}},
xlabel={k},
ymin=0,
ymax=1.2,
ylabel style={font=\color{white!15!black}},
ylabel={y(k)},
axis background/.style={fill=white},
title style={font=\bfseries},
title={Odpowiedz skokowa toru zak��cenie-wyj�cie procesu},
xmajorgrids,
ymajorgrids,
legend style={at={(0.97,0.03)}, anchor=south east, legend cell align=left, align=left, draw=white!15!black}
]
\addplot [color=mycolor1]
  table[row sep=crcr]{%
1	0\\
2	0.14643\\
3	0.273354032\\
4	0.3833636331168\\
5	0.478706339044537\\
6	0.561331333146131\\
7	0.632929038514577\\
8	0.694965456664036\\
9	0.748711949880604\\
10	0.795271071780191\\
11	0.835598970409124\\
12	0.870524818653748\\
13	0.900767666386489\\
14	0.926951056442718\\
15	0.949615701133298\\
16	0.969230476630432\\
17	0.986201958420521\\
18	1.00088269140397\\
19	1.01357836253738\\
20	1.02455402163672\\
21	1.0340394766393\\
22	1.04223397286446\\
23	1.04931025127927\\
24	1.05541806816933\\
25	1.0606872476821\\
26	1.06523032922711\\
27	1.06914486349353\\
28	1.07251540371205\\
29	1.07541523260134\\
30	1.07790786007377\\
31	1.08004832212079\\
32	1.08188430726221\\
33	1.08345713344269\\
34	1.08480259522222\\
35	1.0859516984744\\
36	1.08693129752146\\
37	1.08776464765486\\
38	1.08847188427133\\
39	1.0890704383647\\
40	1.08957539682083\\
41	1.0899998148425\\
42	1.09035498685846\\
43	1.09065068142793\\
44	1.09089534492019\\
45	1.09109627811491\\
46	1.09125978931852\\
47	1.09139132711481\\
48	1.09149559545414\\
49	1.0915766534268\\
50	1.09163800175461\\
};
\addlegendentry{y(k)}

\end{axis}
\end{tikzpicture}%
\caption{Odpowied� skokowa toru zak��cenie-wyj�cie procesu dla algorytmu DMC}
\label{odptor_zy}
\end{figure}

\section{Implementacja}
Implementacje fukcji wykorzystanych do wykonania zadania s� zawarte w skryptach \verb+zad_skokU.m+ oraz \verb+zad_skokZ.m+.