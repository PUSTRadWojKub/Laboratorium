\documentclass{beamer}
\usetheme{Boadilla}
\usepackage{polski}
\usepackage[utf8]{inputenc}
\usepackage{hyperref}

\title{Kształtowanie umiejętności inżynierskich na przykładach}
\subtitle{Realizacja przedmiotu PUST}
\author{W. Rokicki, R. Pietkun, J. Gruszecki}
\date{Maj 2020}

\begin{document}

\begin{frame}
\titlepage
\end{frame}

\begin{frame}[allowframebreaks]{Plan prezentacji}
\tableofcontents    
\end{frame}

\section{Rozwiązywanie problemów}

	\subsection{Problem regulacji układu grzejąco - chłodzącego}
	\begin{frame}{Problem regulacji układu grzejąco - chłodzącego}
		
	\end{frame}
	
	\subsection{Problem regulacji poziomu cieczy w zbiornikach}
	\begin{frame}{Problem regulacji poziomu cieczy w zbiornikach}
		
	\end{frame}


\section{Tworzenie własnych i modyfikowanie gotowych rozwiązań}

	\subsection{Implementacja algorytmów reglacji PID i DMC}
	\begin{frame}{Implementacja algorytmów reglacji PID i DMC}
		\begin{itemize}
			\item PID \\
			opis
			\item DMC
		\end{itemize}
	\end{frame}
	
	\subsection{Modyfikacja wcześniej zaimplementowanych algorytmów}
	\begin{frame}{Modyfikacja wcześniej zaimplementowanych algorytmów}
		\begin{itemize}
			\item MIMO 
			\item Zakłócenia
		\end{itemize}
	\end{frame}

\section{Czytania dokumentacji}

	\subsection{Korzystanie z opracowanych dokumentacji}
	\begin{frame}{Korzystanie z opracowanych dokumentacji}
		\begin{itemize}
			\item Instrukcje prowadzących
			\item GxWorks
			\item GtDesigner
		\end{itemize}
	\end{frame}

\section{Przeprowadzanie eksperymentów/symulacji i wyciąganie wniosków}

	\subsection{Symulowanie obiektów o zmiennej dynamice podczas projektów - MATLAB}
	\begin{frame}{Symulowanie obiektów o zmiennej dynamice podczas projektów}
		
	\end{frame}
	
	\subsection{Symulowacja regulacji stanowisk za pomocą PLC - środowisko GxWorks}
	\begin{frame}{Symulowacja regulacji stanowisk za pomocą PLC - środowisko GxWorks}
		
	\end{frame}

\section{Testowanie, wykrywanie błędów, wprowadzanie zabezpieczeń}

	\subsection{Testowanie}
	\begin{frame}{Testowanie}
		Porównywanie różnych metod regulacji, wybranie najlepszego
		Symulacja modelu obiektu
	\end{frame}
	
	\subsection{Wykrywanie błędów}
	\begin{frame}{Wykrywanie błędów}
		? //może pominąć?
	\end{frame}
	
	\subsection{Wprowadzanie zabezpieczeń}
	\begin{frame}{Wprowadzanie zabezpieczeń}
		Zabezpieczenia na stanowiskach grzejąco-chłodzącym, zbiornikach
	\end{frame}

\section{Rozwiązywanie problemów skali i ograniczeń}

	\subsection{Odpowiednie skalowanie wykresów}
	\begin{frame}{Odpowiednie skalowanie wykresów}
		
	\end{frame}
	
	\subsection{Ograniczenia uwzględniane podczas regulacji}
	\begin{frame}{Ograniczenia uwzględniane podczas regulacji}
		
	\end{frame}

\section{Przygotowywanie raportów}

	\subsection{Sporządzanie dokumentacji}
	\begin{frame}{Sporządzanie dokumentacji}
		
	\end{frame}

\section{Wygłaszanie prezentacji, zainteresowanie tematyką innych}

	\subsection{Aktualna prezentacja}
	\begin{frame}{Aktualna prezentacja}
		
	\end{frame}

\section{Organizacja pracy}

	\subsection{Podział zadań w zespole}
	\begin{frame}{Podział zadań w zespole}
		\begin{itemize}
			\item Projekty - zamiennie sprawko, kod
			\item Laboratorium - strojenie, testy, hmi, kod
		\end{itemize}
	\end{frame}

\end{document}

