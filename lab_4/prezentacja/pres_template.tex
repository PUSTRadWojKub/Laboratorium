\documentclass{beamer}
\usetheme{Boadilla}
\usepackage{polski}
\usepackage[utf8]{inputenc}
\usepackage{hyperref}

\title{Kształtowanie umiejętności inżynierskich}
\subtitle{Na podstawie realizacji przedmiotu PUST}
\author{W. Rokicki, R. Pietkun, J. Gruszecki}
\date{Czerwiec 2020}

\begin{document}

\begin{frame}
\titlepage
\end{frame}

\begin{frame}{Plan prezentacji}
\tableofcontents    
\end{frame}

\section{Organizacja pracy}

	\subsection{Podział zadań w zespole}
	\begin{frame}{Podział zadań w zespole}
		\begin{block}{Podział zadań podczas projektów}
			\begin{itemize}
				\item Tworzenie kodu
				\item Testowanie kodu i przeprowadzanie symulacji
				\item Optymalizacja
				\item Tworzenie sprawozdania
			\end{itemize}
		\end{block}
		\begin{block}{Podział zadań podczas laboratorium}
			\begin{itemize}
				\item Tworzenie kodu
				\item Testowanie kodu i przeprowadzanie symulacji
				\item Tworzenie HMI
				\item Tworzenie sprawozdania
			\end{itemize}
		\end{block}
	\end{frame}

\section{Rozwiązywanie problemów - tworzenie własnych algorytmów i modyfikacja gotowych}

	\subsection{Implementacja algorytmów regulacji, wykorzystanie algorytmów optymalizacji}
	\begin{frame}{Implementacja algorytmów regulacji, wykorzystanie algorytmów optymalizacji}
		\begin{block}{Implementacja dyskretnych algorytmów PID i DMC}
			\begin{itemize}
				\item Implementacja algorytmów regulacji na podstawie wiedzy teorytycznej
				\item Strojenie regulatorów - metoda inżynierska, metoda Zieglera-Nicholsa
				\item Implementacja algorytmów optymalizacji w celu doboru optymalnych parametrów algorytmów
			\end{itemize}
		\end{block}
	\end{frame}
	
	\subsection{Modyfikacja wcześniej zaimplementowanych algorytmów regulacji}
	\begin{frame}{Modyfikacja wcześniej zaimplementowanych algorytmów regulacji}
		\begin{itemize}
			\item Zmiana jednowymiarowych algorytmów sterowania na algorytmy wielowymiarowe - wykorzystanie gotowego kodu z poprzednich projektów do implementacji
			\item Wykorzystanie gotowych algorytmów regulacji na innej platformie - wykorzystanie kodu z projektów podczas laboratorium w programie GX Works
		\end{itemize}
	\end{frame}

\section{Przeprowadzanie eksperymentów/symulacji i wyciąganie wniosków, wprowadzanie zabezpieczeń}

	\subsection{Testowanie i symulacja rozwiązań, wyciąganie wniosków}
	\begin{frame}{Testowanie i symulacja rozwiązań, wyciąganie wniosków}
		\begin{columns}
			\begin{column}{0.45\textwidth}
				\begin{enumerate}
					\item Analiza składni
					\item Kompilacja
					\item Symulacja
					\item Debugowanie
					\item Śledzenie zmian wartości
					\item Pamięć i złożoność obliczeniowa
				\end{enumerate}
			\end{column}
			\begin{column}{0.45\textwidth}
				\begin{block}{Dzięki testom mogliśmy wyciągać wnioski dotyczące:}
					\begin{itemize}
						\item Dynamiki obiektu
						\item Poprawności implementacji algorytmu
						\item Optymalnych wartości nastaw regulatora
						\item Jakości sterowania
						\item Odporności na zakłócenia
						\item Ograniczeń
						\item Najlepszego algorytmu regulacji przy zakładanych ograniczeniach
					\end{itemize}
				\end{block}
			\end{column}
		\end{columns}
	\end{frame}
	
	\subsection{Zabezpieczenia}
	\begin{frame}{Zabezpieczenia}
		\begin{itemize}
			\item Ochrona zdrowia człowieka
			\item Ochrona środowiska
			\item Zabezpieczenie przed uszkodzeniem urządzenia
			\item Zabezpieczenie i odporność na sytuacje nadzwyczajne
		\end{itemize}
		\begin{block}{Wprowadzone zabezpieczenia}
			\begin{itemize}
				\item Zabezpieczenie stanowiska grzejąco-chłodzącego - uszkodzenie czujnika (przekroczenie temperatury powyżej $250^\circ$) $\rightarrow$ wyłączenie najbliższej grzałki
				\item Zabezpieczenie stanowiska ze zbiornikami - limit poziomu wody w zbiorniku (osiągnięcie 20cm wysokości) $\rightarrow$ otworzenie zaworu tego zbiornika
			\end{itemize}
		\end{block}
	\end{frame}

\section{Czytanie dokumentacji}

	\subsection{Korzystanie z opracowanych dokumentacji}
	\begin{frame}{Korzystanie z opracowanych dokumentacji}
		\begin{enumerate}
			\item Instrukcje do projektu/laboratorium
			\item Materiały wykładowe
			\item Obsługa stanowisk laboratoryjnych
			\begin{itemize}
				\item Stanowisko grzewczo-chłodzące
				\item Stanowisko INTECO
			\end{itemize}
			\item Obsługa oprogramowania Mitsubishi Electric
			\begin{itemize}
				\item GX Works
				\item GT Designer
				\item GT Simulator
				\item LogViewer
			\end{itemize}
		\end{enumerate}
	\end{frame}


\section{Przygotowywanie raportów}
\subsection{Sporządzanie dokumentacji}
\begin{frame}{Sporządzanie dokumentacji}
	\begin{itemize}
		\item{Wstęp teoretyczny}
		\item{Opis implementacji}
		\item{Planowane eksperymenty}
		\item{Przeprowadzone eksperymenty}
		\item{Uzyskane wyniki}
		\item{Wnioski}
	\end{itemize}
\end{frame}

\end{document}

